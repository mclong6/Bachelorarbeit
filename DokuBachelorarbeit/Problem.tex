
%Kapitel des Hauptteils

\chapter{Problembeschreibung}  %Name des Kapitels
\label{cha:Problemspezifikation} %Label des Kapitels
Persönliche Daten sind im Internet oft frei zugänglich. Das heißt, dass unterschiedlichste Webseiten persönliche Information von Menschen öffentlich bereitstellen. Die bekanntesten Webseiten sind Social-Media-Seiten wie Twitter, Facebook und Instagram. Allerdings wird auch auf anderen Webseiten personenbezogene Daten in großen Mengen bereitgestellt. Ein Beispiel dafür ist das Berufsportal LinkedIn oder XING. Diese Art von Webseiten sind perfekte Informationsquellen für Phisher, da im Bereich von Social Engineering diese Informationen oft genutzt werden, um ein Opfer zu täuschen oder zu manipulieren.\\
Dass hier beschriebene Problem zeigt, dass der Zugang für persönliche Information durch das Internet für die Öffentlichkeit einfacher gemacht wird. Es wird mit einem kritischen Blick aufgezeigt, wie diese Daten für einen böswilligen Social-Engineering-Angriff missbraucht werden können.
%TODO Werden Persönliche Informationen werden im Internet immer leichter zugänglich gemacht?????.Möglicherweise Geschichte zur Verdeutlichung