%Kapitel der Umsetzung

\chapter{Generierung der Phishing-E-Mail}  %Name des Kapitels
\label{cha:ErstellungeinerPhishing-Mail} %Label des Kapitels
In diesem Kapitel wird die Umsetzung zur Erstellung einer Phishing-Mail beschrieben. Dabei wir zuerst auf die Implementierung des Algorithmus zur Adressgenerierung eingegangen. Anschließend werden die E-Mail-Muster vorgestellt. Im letzten Abschnitt wird die Umsetzung zum Versenden einer Phishing-E-Mail aufgezeigt. 

\section{Implementierung der Methode zur Generierung der E-Mail-Adressen}	
Für den zu entwickelnden Algorithmus wird eine eigene Klasse erstellt. Diese Klasse ist ausschließlich für die Generierung der E-Mail-Adressen zuständig. Diese Methode zeigt, wie aus personenbezogenen Daten eine mögliche E-Mail-Adresse generiert werden kann. Jedoch wird in der Anwendung keine Phishing-Mail an eine dieser Adressen, welche mit der Methode erzeugt wurden, versendet. Es dient ausschließlich als Beweis für die Durchführbarkeit dieser Methode.

	\subsection{Funktion des eigenen Algorithmus}
	Der lokale Teil einer E-Mail-Adresse befindet sich vor dem At-Zeichen. Dieser kann aus verschiedensten Daten bestehen. Allerdings wird in den meisten Fällen der bürgerlichen Namen verwendet. \cite{NameAlsEMail} Aus diesem Grund verwendet der Algorithmus die Personenattribute Vorname, Nachname und zusätzlich das Geburtsjahr.\\
	Im ersten Schritt wird kontrolliert, welche Daten bekannt sind. Im Idealfall sind das alle drei Attribute. Im zweiten Schritt wird festgelegt, aus welchen Daten der lokale Teil bestehen kann. Im Folgenden sind möglichen Kombinationen aufgezeigt.
	
	\textit{Vorname;}\\
	\textit{Nachname;}\\
	\textit{Vorname, Nachname;}\\
	\textit{Vorname, Nachname, vollständiges Geburtsjahr;}\\
	\textit{Vorname, Nachname, Kurzform des Geburtsjahrs;}
	
	Ein lokaler Teil kann somit aus mehreren Daten bestehen. Es kann vorkommen, dass anstatt "'Max Mustermann"' "'Mustermann Max"' als lokaler Namen verwendet wird. Aus diesem Grund wird für jeden lokalen Teil, der aus mehreren Daten besteht, eine Permutation ohne Wiederholung angewendet. Dadurch werden alle möglichen Kombinationen von Anordnungen der bekannten Daten generiert. Dabei wird zusätzlich auf die Reihenfolge der Anordnungen geachtet. Somit ist das Element "'Max Mustermann"' und "'Mustermann Max"' eine eigene Kombination. Es werden zusätzlich die selben Kombinationen mit den Trennzeichen "'."', "'\_"' und "'-"' erstellt und in einer Liste gespeichert. Das Ergebnis einer Permutation über die Attribute "'Vorname"' und "'Nachname"' werden nachstehen dargestellt.
	
	\textit{[Vorname], [Nachname], [Vorname \& Nachname], Vorname \& '.'\& Nachname], [Vorname \& '\_' \& Nachname], [Vorname \& '-' \& Nachname], [Nachname \& Vorname], [Nachname \& '.' \& Vorname], [Nachname \& '\_' \& Vorname], [Nachname \& '-' \& Vorname]}
	
	Für den Domainteil werden die bekannte Mailprovider in Deutschland verwendet. Dazu gehören die Provider GMX, WEB.DE, Gmail, T-Online, Freenet und 1\&1.\cite{AnbieterMail}. Das bedeutet, es wird für jeden lokalen Namen eine E-Mail-Adresse mit den jeweiligen Mailprovidern und der Landeskennung "'de"' erzeugt. Die folgende Tabelle  zeigt die erzeugten E-Mail-Adressen des Algorithmus für die Daten "'Marco"', "'Lang"' und "'1995"'. Es sind nur die Mailadressen für die Provider WEB.DE, Gmail und Freenet aufgelistet.
	
	\begin{center}
		%\begin{table}[h!]
		\scriptsize
		\begin{longtable}{c|c|c}
			\label{EMailAdressen}

			%\centering
			%\scriptsize
			%\begin{tabular}{c|c|c}
				marco@web.de& marco@gmail.com& marco@freenet.de\\ 
				lang@web.de& lang@gmail.com& lang@freenet.de\\
				marcolang@web.de& marcolang@gmail.com& marcolang@freenet.de\\
				marco.lang@web.de& marco.lang@gmail.com& marco.lang@freenet.de\\ 
				marco\_lang@web.de& marco\_lang@gmail.com& marco\_lang@freenet.de\\ 
				marco-lang@web.de& marco-lang@gmail.com& marco-lang@freenet.de\\
				langmarco@web.de& langmarco@gmail.com& langmarco@freenet.de\\
				lang.marco@web.de& lang.marco@gmail.com& lang.marco@freenet.de\\
				lang\_marco@web.de& lang\_marco@gmail.com& lang\_marco@freenet.de\\
				lang-marco@web.de& lang-marco@gmail.com& lang-marco@freenet.de\\
				marcolang1995@web.de& marcolang1995@gmail.com& marcolang1995@freenet.de\\
				marco.lang.1995@web.de& marco.lang.1995@gmail.com& marco.lang.1995@freenet.de\\
				marco\_lang\_1995@web.de& marco\_lang\_1995@gmail.com& marco\_lang\_1995@freenet.de\\
				marco-lang-1995@web.de& marco-lang-1995@gmail.com& marco-lang-1995@freenet.de\\ 
				marco1995lang@web.de& marco1995lang@gmail.com& marco1995lang@freenet.de\\
				marco.1995.lang@web.de& marco.1995.lang@gmail.com& marco.1995.lang@freenet.de\\ 
				marco\_1995\_lang@web.de& marco\_1995\_lang@gmail.com& marco\_1995\_lang@freenet.de\\
				marco-1995-lang@web.de& marco-1995-lang@gmail.com& marco-1995-lang@freenet.de\\
				langmarco1995@web.de& langmarco1995@gmail.com& langmarco1995@freenet.de\\
				lang.marco.1995@web.de& lang.marco.1995@gmail.com& lang.marco.1995@freenet.de\\ 
				lang\_marco\_1995@web.de& lang\_marco\_1995@gmail.com& lang\_marco\_1995@freenet.de\\
				lang-marco-1995@web.de& lang-marco-1995@gmail.com& lang-marco-1995@freenet.de\\
				lang1995marco@web.de& lang1995marco@gmail.com& lang1995marco@freenet.de\\
				lang.1995.marco@web.de& lang.1995.marco@gmail.com& lang.1995.marco@freenet.de\\ 
				lang\_1995\_marco@web.de& lang\_1995\_marco@gmail.com& lang\_1995\_marco@freenet.de\\ 
				lang-1995-marco@web.de& lang-1995-marco@gmail.com& lang-1995-marco@freenet.de\\ 
				1995marcolang@web.de& 1995marcolang@gmail.com& 1995marcolang@freenet.de\\
				1995.marco.lang@web.de& 1995.marco.lang@gmail.com& 1995.marco.lang@freenet.de\\ 
				1995\_marco\_lang@web.de& 1995\_marco\_lang@gmail.com& 1995\_marco\_lang@freenet.de\\ 
				1995-marco-lang@web.de& 1995-marco-lang@gmail.com& 1995-marco-lang@freenet.de\\
				1995langmarco@web.de& 1995langmarco@gmail.com& 1995langmarco@freenet.de\\
				1995.lang.marco@web.de& 1995.lang.marco@gmail.com& 1995.lang.marco@freenet.de\\ 
				1995\_lang\_marco@web.de& 1995\_lang\_marco@gmail.com& 1995\_lang\_marco@freenet.de\\
				1995-lang-marco@web.de& 1995-lang-marco@gmail.com& 1995-lang-marco@freenet.de\\
				marcolang95@web.de& marcolang95@gmail.com& marcolang95@freenet.de\\
				marco.lang.95@web.de& marco.lang.95@gmail.com& marco.lang.95@freenet.de\\ 
				marco\_lang\_95@web.de& marco\_lang\_95@gmail.com& marco\_lang\_95@freenet.de\\ 
				marco-lang-95@web.de& marco-lang-95@gmail.com& marco-lang-95@freenet.de\\ 
				marco95lang@web.de& marco95lang@gmail.com& marco95lang@freenet.de\\ 
				marco.95.lang@web.de& marco.95.lang@gmail.com& marco.95.lang@freenet.de\\ 
				marco\_95\_lang@web.de& marco\_95\_lang@gmail.com& marco\_95\_lang@freenet.de\\
				marco-95-lang@web.de& marco-95-lang@gmail.com& marco-95-lang@freenet.de\\
				langmarco95@web.de& langmarco95@gmail.com& langmarco95@freenet.de\\ 
				lang.marco.95@web.de& lang.marco.95@gmail.com& lang.marco.95@freenet.de\\ 
				lang\_marco\_95@web.de& lang\_marco\_95@gmail.com& lang\_marco\_95@freenet.de\\ 
				lang-marco-95@web.de& lang-marco-95@gmail.com& lang-marco-95@freenet.de\\
				lang95marco@web.de& lang95marco@gmail.com& lang95marco@freenet.de\\
				lang.95.marco@web.de& lang.95.marco@gmail.com& lang.95.marco@freenet.de\\ 
				lang\_95\_marco@web.de& lang\_95\_marco@gmail.com& lang\_95\_marco@freenet.de\\ 
				lang-95-marco@web.de& lang-95-marco@gmail.com& lang-95-marco@freenet.de\\
				95marcolang@web.de& 95marcolang@gmail.com& 95marcolang@freenet.de\\ 
				95.marco.lang@web.de& 95.marco.lang@gmail.com& 95.marco.lang@freenet.de\\ 
				95\_marco\_lang@web.de& 95\_marco\_lang@gmail.com& 95\_marco\_lang@freenet.de\\
				95-marco-lang@web.de& 95-marco-lang@gmail.com& 95-marco-lang@freenet.de\\
				95langmarco@web.de& 95langmarco@gmail.com& 95langmarco@freenet.de\\
				95.lang.marco@web.de& 95.lang.marco@gmail.com& 95.lang.marco@freenet.de\\ 
				95\_lang\_marco@web.de& 95\_lang\_marco@gmail.com& 95\_lang\_marco@freenet.de\\ 
				95-lang-marco@web.de& 95-lang-marco@gmail.com& 95-lang-marco@freenet.de
		%	\end{tabular}
		
		%\end{table}
	\end{longtable}
	\end{center}
	

\section{Implementierung der E-Mail-Muster}
Ein E-Mail-Muster entspricht einem Lückentext, bei dem die entsprechenden Lücken mit den gewonnen Daten ergänzt werden. Die Texte müssen so erstellt werden, dass sie die Zielperson ansprechen. Aus diesem Grund muss für jede Kombination der gewonnenen Daten ein Muster zur Verfügung stehen. Zusätzlich stellt sich die Frage, wie die E-Mail-Texte möglichst passend kategorisiert werden können.

	\subsection{Kategorien erstellen}
	Die Muster können in zwei große Kategorien unterteilt werden. Es gibt eine private und eine berufliche Kategorie. Der Unterschied zwischen privat und beruflich besteht in der Art und Weise wie ein Text geschrieben wird. Genaugenommen bedeutet das, dass ein privates Muster in einer Umgangssprache und ein berufliches in einer formelleren Sprache erstellt wird. Diese beiden Kategorien haben weitere Unterkategorien, welche verschiedene Kombinationen aus den personenbezogenen Daten verwenden.\\
	Um die Kategorie zu erkennen, werden zu Beginn Abfragen gestartet. Dadurch wird kontrolliert, welche Daten bekannt sind. Im Fall, dass die Institution oder die Tätigkeit der Zielperson bekannt ist, wird ein berufliches Muster gewählt. Andernfalls wird ein privates Muster verwendet.
	
		\subsubsection{Berufliche E-Mail-Muster}
		\label{subsubsec:beruflicheMuster}
		Die beruflichen E-Mail-Muster sind in den folgenden Bildern aufgezeigt. Die Reihenfolge zur Auswahl der Muster im laufenden Programm entspricht der Reihenfolge der Bilder. Dabei werden die kursiv geschriebenen Wörter in den Bildern mit den gewonnenen Daten über die Zielperson ersetzt.\\ 
		Das Bild \ref{img:NameInstitution} beschreibt ein berufliches Muster, welches die Personenattribute Nachname und Institution verwendet. Im Bild \ref{img:NachnameTätigkeit1} ist ein Muster mit den Attributen Nachname und Tätigkeit aufgezeigt. Das Bild \ref{img:NachnameTätigkeit} zeigt das Muster mit denselben Attributen, wobei die Tätigkeit zu Beginn abgefragt wurde. Infolgedessen kann dieses Datenelement als Entscheidungskriterium für die Auswahl eines Musters verwendet werden. In diesem bestimmten Fall wurde das Muster mit dem festgesetzten Wort "'Professor"' gewählt. Im letzten Bild wird das berufliche Muster für die Daten Nachname, Tätigkeit und Institution beschrieben.
			\begin{figure}[h!]
				\label{img:NameInstitution}
				\fbox{\parbox{\linewidth}{\texttt{SUBJECT:\quad \textit{Musterinstitution} - Netzwerkänderungen\\\\
							Hallo \textit{Herr Mustermann},\\
							wir bauen unsere Netzwerkstruktur um. Bitte registrieren Sie sich unter der folgenden Webseite, damit wir Sie in das neue System aufnehmen können.\\https://badlink.com\\\\Mit freundlichen Grüßen\\\\Ihr IT-Team der \textit{Musterinstitution}}}}
				\caption{Ein berufliches E-Mail-Muster mit den Personenattributen Nachname und Institution }
			\end{figure}
		
			\begin{figure}[h!]
				\label{img:NachnameTätigkeit1}
				\fbox{\parbox{\linewidth}{\texttt{SUBJECT:\quad \textit{Mustertätigkeit} gesucht - Im Auftrag des BRD\\\\
							Hallo \textit{Herr Mustermann},\\die Bundesrepublik Deutschland sucht einen kompetenten \textit{Mustertätigkeit}.
							Haben Sie Interesse an einer neuen Herausforderung unter optimalen Arbeitsbedingungen?\\
							Im Anhang finden Sie die offizielle Stellenausschreibung mit den dazugehörigen Voraussetzungen und Gehaltsstufen.\\
							\\\\\
							Ihr Karriere-Team der Bundesrepublik Deutschland}}}
				\caption{Ein berufliches E-Mail-Muster mit den Personenattributen Nachname und Tätigkeit}
			\end{figure}
		
			\begin{figure}[h!]
				\label{img:NachnameTätigkeit}
				\fbox{\parbox{\linewidth}{\texttt{SUBJECT:\quad Feedback zur Ausarbeitung\\\\
							Hallo \textit{Herr} Professor \textit{Mustermann},\\
							wie besprochen befindet sich im Anhang meine vorläufige Ausarbeitung. Könnten Sie diese bitte erneut überprüfen und mir ein Feedback geben?\\\\Mit freundlichen Grüßen\\\\	Max Mustermann}}}
				\caption{Ein berufliches E-Mail-Muster mit dem Personenattribut Nachname und einer festgesetzten Tätigkeit}
			\end{figure}
		
			\begin{figure}[h!]
				\label{img:NachnameTätigkeitInsitution}
				\fbox{\parbox{\linewidth}{\texttt{SUBJECT:\quad \textit{Mustertätigkeit} bei der \textit{Musterinstitution}\\\\
							Hallo \textit{Herr Mustermann},\\
							als \textit{Mustertätigkeit} bei der \textit{Musterinstitution}, stehen Ihnen nun alle Möglichkeiten offen. Sehen Sie nun Ihre neuen Möglichkeiten unter folgendem Link an.\\
							https://badlink.com\\\\Mit freundlichen Grüßen\\\\Ihr Team der \textit{Musterinstitution}}}}
				\caption{Ein berufliches E-Mail-Muster mit den Personenattributen Nachname, Tätigkeit und Institution }
			\end{figure}
			
		\subsubsection{Private E-Mail-Muster}
			Im Nachfolgenden werden die privaten E-Mail-Muster aufgezeigt. Die Ersetzung der kursiven Wörter sowie die Reihenfolge der Muster-Auswahl ist identisch zum vorherigen Kapitel \ref{subsubsec:beruflicheMuster}.\\
			Im ersten Bild wird ein privates E-Mail-Muster beschrieben, welche die gewonnenen Kontaktinformationen verwendet. Genaugenommen ist das der Kontaktname und die Kontaktinformation. Diese Kontaktinformation ist identisch zu einer Information über die Zielperson. Das nächste Bild beschreibt ein Muster, bei dem die Personendaten Vorname und Hobby verwendet werden. Das Bild \ref{img:VornameJahrgang} zeigt die Verwendung des Vornamens und des Geburtsjahrs in einem E-Mail-Muster. Das letzte Muster verwendet die Personenattribute Vorname und Ort. Hierfür wird  der gefundene Ort verwendet.\\
			\FloatBarrier
			
			\begin{figure}[h!]
				\fbox{\parbox{\linewidth}{\texttt{SUBJECT:\quad Fragen bzgl. \textit{Kontaktinformation}\\\\Hi \textit{Max},\\ hier ist \textit{Kontaktname}. Bezüglich \textit{Kontaktinformation} hätte ich noch ein paar Fragen an dich...\\ Könntest du vielleicht in den Anhang schauen und bewerten, was ich dazu so rausgesucht habe?\\Vielen Danke im Voraus!\\\\\textit{Kontaktname}}}}
			\caption{Ein privates E-Mail-Muster mit den Personenattributen Vorname, Kontaktname und Kontaktinformation}
			\end{figure}
			\FloatBarrier
			\begin{figure}[h!]
				\fbox{\parbox{\linewidth}{\texttt{SUBJECT:\quad Verbessere deine Technik im \textit{Musterhobby}\\\\
							Hi \textit{Max},\\ damit du deine Leistung im \textit{Musterhobby} verbessern kannst, musst du unbedingt die Techniken deiner Vorbilder anschauen!\\Im Anhang befindet sich darüber eine kleine Übersicht.\\\\Dein Team der deutschen Förderung}}}
				\caption{Ein privates E-Mail-Muster mit den Personenattributen Vorname und Hobby}
			\end{figure}
		
			\begin{figure}[h!]
				\label{img:VornameJahrgang}
				\fbox{\parbox{\linewidth}{\texttt{SUBJECT:\quad Jahrgang \textit{Geburtsjahr}\\\\
							Hi \textit{Max},\\dieses Jahr findet ein Treffen für alle Personen, die \textit{Geburtsjahr} geboren sind, statt.\\Im Anhang befindet sich eine Liste mit den Leuten die bereits zugesagt haben.\\\\Dein Orga-Team}}}
				\caption{Ein privates E-Mail-Muster mit den Personenattributen Vorname und Jahrgang}
			\end{figure}
		
			\begin{figure}[h!]
				\fbox{\parbox{\linewidth}{\texttt{SUBJECT:\quad Streetfood-Festival in \textit{Musterort}\\\\
							Hi \textit{Max},\\dieses Jahr findet das erste STREETFOOD-FESTIVAL in \textit{Musterort} statt. Im Anhang befindet sich der Plan, auf dem alles weitere erklärt wird.\\
							Wir freuen uns auf dich!\\\\Dein Streefood-Team aus \textit{Musterort}}}}
				\caption{Ein privates E-Mail-Muster mit den Personenattributen Vorname und Ort}
			\end{figure}
		
		\FloatBarrier
\section{Versenden einer Phishing-E-Mail}
Damit eine Phishing-Mail beispielhaft versendet werden kann, wird eine Absender-Adresse benötigt. Sehr große Provider wie Gmail oder Yahoo sind dafür nicht optimal. Das hat den Grund, dass viele Spammer diese Provider in der Vergangenheit verwendet haben. Dadurch werden diese Adressen gerne öfter überprüft. Kleine Provider wie GMX sind dagegen nicht so bekannt. Ein weiterer Vorteil von GMX ist, dass ein Account ohne eine weitere gültige E-Mail-Adresse erzeugt werden kann. Das spricht für GMX, da bei einem gefälschten Account keine Verbindung zu einem verwendeten Profil besteht. Des Weiteren ist dieser kleine Provider großen Plattformen wie Facebook fremd und wird dadurch weniger streng überprüft. Aus den erwähnten Gründen wird eine E-Mail-Adresse bei GMX erstellt.\cite{Bazzell}\\
Mithilfe der erzeugten Adresse und einem Python Skript kann eine Phishing-Mail beispielhaft versendet werden. Dazu wird die Python Bibliothek "'smtplib"' verwendet. Im ersten Schritt werden die erstellten E-Mail-Daten wie Sender-Adresse, Ziel-Adresse, Betreff und Inhalt einer mehrteiligen Nachricht hinzugefügt. Als nächstes wird die Verbindung mit dem GMX-Mailserver aufgebaut. Nach einer erfolgreichen Anmeldung kann eine E-Mail versendet werden. 










	