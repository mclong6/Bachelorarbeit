%Kapitel der Umsetzung

\chapter{Erstellung einer Phishing-Mail}  %Name des Kapitels
\label{cha:ErstellungeinerPhishing-Mail} %Label des Kapitels


\section{Konzept zur Erstellung einer Phishing-Mail}
Die Generierung einer Phishing-Mail läuft voll automatisch ab. Das bedeutet, dass das Programm eigenständig die E-Mail-Adressen generiert und selbst passende E-Mail-Muster auswählt.

	\subsection{Methoden zur Generierung von E-Mail-Adressen}
	Eine Möglichkeit zur Generierung der E-Mail-Adressen kann das Open Source-Tool von Michael Bazzell \cite{EmailAssumptions} sein, welches mit Hilfe eines automatisierten Webbrowsers verwendet werden kann. Bei diesem Tool werden zuerst über ein Formular, Daten für die E-Mail-Generierung eingetragen. Unter anderem sind das Vorname, Nachname und der E-Mail-Provider. Daraufhin werden die vorgeschlagenen E-Mail-Adressen angezeigt,kopiert und in ein Suchfeld eingefügt. Anschließend kann bei Google, Bing, und Facebook nach Einträgen gesucht und falls ein Eintrag gefunden wurde auch angezeigt werden.
	
	Eine Weitere Möglichkeit wäre ein Algorithmus zu entwickeln, der alle möglichen E-Mail-Adressen aus den Kombinationen von Vorname, Nachname, Geburtsjahr, Benutzernamen und den Domains von den bekanntesten E-Mail-Providern generiert. Dazu gehören \textit{GMX}, \textit{WEB.DE}, \textit{Gmail}, \textit{T-Online}, \textit{Freenet} und \textit{1\&1}.\cite{AnbieterMail} \\
	Für den Fall, dass der Arbeitgeber der Zielperson bekannt ist, kann auf der Firmenwebseite nach E-Mail-Adressen gesucht werden. Dadurch ist es möglich die Domain einer Firmen-Mailadresse zu bestimmen und eine Anzahl  möglicher Firmenadressen für die Zielperson zu generieren.\\
	Schon bei der Suche von personenbezogenen Daten wird ebenfalls nach E-Mail-Adressen gesucht. Dadurch kann bereits eine bis jetzt unbekannte Anzahl von Adressen gefunden werden.
	%TODO Erwähnen dass facbook nach emails suchen konnte
	
	\subsection{Methode zur Erstellung von E-Mail-Mustern}
	Für die Erstellung der E-Mail-Muster kann eine eigene Klasse erstellt werden, welche für die Erzeugung des Textes zuständig ist. In dieser Klasse werden Strings gespeichert die einem Lückentext ähneln. Abhängig von den gefundenen Daten wird ein Lückentext ausgewählt, welcher anschließend mit den Daten an den passenden Lücken ergänzt wird. Mit dieser Methode muss jedoch für jede Kombination aus gewonnenen Daten ein Lückentext vorhanden sein.\\
	Die Lückentexte werden so kategorisiert, dass für jede gefundene Information ein passender Lückentext vorhanden ist. Eine denkbare Unterteilung wäre in die Kategorien Privat und Geschäftlich.

\section{Bewertung: E-Mail-Adresse generieren}

Für die E-Mail-Adressgenerierung wird ein eigener Algorithmus entwickelt. Im Gegensatz zu dem Open Souce-Tool \cite{Bazzell} besteht bei diesem Algorithmus eine höhere Wahrscheinlichkeit, dass die richtige E-Mail-Adresse enthalten ist, da das Geburtsjahr, falls es bekannt ist, mit einbezogen wird. 
%TODO Anzahl der Möglichen E-Mail-adressen ausrechenen


\section{Generierung der E-Mail-Adressen}

	\subsection{Funktion des eigenen Algorithmus}
	
	
\section{Validität der generierten Mail-Adressen prüfen}

	\subsection{Methoden zum Prüfen der Validität}
	Die erzeugten Adressen werden anschließend auf Validität geprüft. Hierfür gab es früher eine \textit{VRFY} Anfrage von SMTP. Mit dieser Anfrage konnte eine angegebene E-Mail-Adresse überprüft werden. Allerdings wurde der Dienst von Spammern ausgenutzt und wird dadurch von den meisten SMTP-Servern nicht mehr zu Verfügung gestellt.\cite{balduzzi2010abusing}\\
	Demnach muss die Validität auf einem anderen Weg geprüft werden. Eine Möglichkeit zur Prüfung ist die Verwendung bereitgestellter Webseiten, bei der die zu prüfenden E-Mail-Adresse angegeben werden kann. Eine anschließende Rückmeldung verrät dann, ob die Adresse verwendet wird oder nicht. Eine Webseite dafür wäre "'\textit{https://centralops.net/co/}"'. Als Alternative dazu, ist die Entwicklung eines Skriptes, welches die Validität der Adresse prüft.
	
	Im Fall, dass mehrere Adressen von diesem Adresspool gültig sind, kann nach mit Hilfe dieser Mail-Adressen nach Einträgen im Internet gesucht werden. Wenn es eine Übereinstimmung mit der Zielperson gibt, wird diese E-Mail ausgewählt. Andernfalls wird an jede gültige Adresse eine Phishing-Mail gesendet. 
	
	\subsection{Bewertung: Validität Prüfen}
	Für eine bessere Laufzeit des Programms, wird ein Skript zur Überprüfung der Adressen auf Verfügbarkeit und Gültigkeit, verwendet.
	
\section{E-Mail-Muster erstellen}

	\subsection{Kategorien erstellen}
	Grundsätzlich können die Muster in zwei große Kategorien unterteilt werden. Es gibt einen privaten und geschäftlichen Teil. Der private Teil hat weiter Unterteilungen wie beispielsweise Familie, Hobby und Interessen. Der Text kann hier in einer Alltagssprache erstellt werden. Für ein geschäftliches Muster sollte eine gehobene Sprache verwendet werden und Daten wie der Firmenname muss bekannt sein. 
	
	\subsection{Lückentexte erstellen}
	