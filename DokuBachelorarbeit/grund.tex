%Kapitel des Hauptteils

\chapter {Grundlagen}  %Name des Kapitels
\label{cha:grundlagen} %Label des Kapitels

\section{Personenbezogene Daten}
%TODO möglicherweise unterschied von Information und Daten erklären
Laut der DSGVO sind \textbf{personenbezogene Daten}, alle Informationen, die sich auf eine identifizierbare Person beziehen. Als identifizierbar wird eine natürliche Person angesehen, die mittels einem oder mehreren Merkmalen direkt oder indirekt identifiziert werden kann. Mögliche Kennungen für die Unterscheidung der Merkmale sind der Name, eine Kennnummer, Standortdaten, eine Online-Kennung, et cetera von der Person. Dabei dienen diese Kennungen als Ausdruck der physischen, physiologischen, genetischen, psychischen, wirtschaftlichen, kulturellen oder sozialen Identitäten dieser natürlichen Person. \cite{personenbezogeneDaten}

\section{Social Engineering} %Unterkapitel
\label {sec:Social Engineering} %Label des Unterkapitels
	Die Definition von Social Engineering, kurz \textit{SE}, ist nicht eindeutig, da es sehr verschiedene Ansichten davon gibt. Jedoch ist der Grundgedanke von Social Engineering, eine Zielperson so zu manipulieren, damit sie für den Angreifer bessere Entscheidung trifft. \cite{ArtOfHumanHacking} \\
	Kevin D. Mitnick definiert Social Engineering wie folgt:\\	
	\textit{"'Social Engineering uses influence and persuasion to deceive people by convincing them that the social engineer is someone he is not, or by manipulation. As a result, the social engineer is able to take advantage of people to obtain information with or without the use of technology"'}\cite{ArtOfDeception}
	
	SE wird Menschen von Geburt an beigebracht und begegnet einem beinahe jeden Tag. Schon ein Baby muss wissen wie es die Eltern manipulieren kann, damit es Dinge wie Essen, Zuneigung, oder ähnliches bekommt. Darüber hinaus ist SE in vielen Berufen ein täglicher Bestandteil. %TODO villt Jobbeispiel finden mit Social Engineering
	
	Im Bereich der Informationssicherheit, wird von Social Engineering gesprochen, wenn Angreifer durch die Manipulierung und Täuschung von Menschen vertrauliche Informationen oder Zugänge zu Systemen bekommen. Die bekanntesten Angriffsmethoden sind Phishing, Pretexting, Baiting und Quid Pro Quo. Bei dieser Arbeit wird hauptsächlich auf das Thema E-Mail-Phishing eingegangen.

	Der Aufbau eines SE-Angriffes ist definiert in mehrere Phasen. Das wohl bekannteste Modell für einen Social Engineering-Angriffszyklus ist in dem Buch von Kevin D. Mitnicks \cite{ArtOfDeception} definiert. Dieser Zyklus besteht aus den 4 Phasen \textbf{Research, Developing rapport and trust, Exploiting trust} und \textbf{Utilize information}.\\
	In der \textbf{Research-Phase} geht es um die Informationsbeschaffung. Bei dieser Phase will der Angreifer möglichst viele Informationen über das Ziel herausfinden. Die \textbf{Developing Rapport and Trust-Phase} beschreibt den Kontaktaufbau zum Ziel, da wenn das Opfer dem Angreifer vertraut, hat dieser ein leichteres Spiel in den kommenden Phasen. Das nun erzeugte Vertrauen wird in der \textbf{Exploitung Trust-Phase} ausgenutzt. Hier will der Angreifer die eigentlich Information vom Opfer herausfinden. Dies geschieht einerseits durch bestimmtes Nachfragen oder durch Manipulation.
	\textbf{Utilize Information} ist die letzte Phase. Dort wird die gewonnene Information genutzt um das eigentliche Ziel des Angreifers zu erreichen.\\
	Grundsätzlich werden bei einem Social Engineering Angriff menschliche Wünsche, Ängste und verbreitete Verhaltensmuster verwendet um ein Opfer zu manipulieren.\cite{LeitfadenSE}\\

		\subsection{Phishing}
		Das Wort Phishing wird von dem Wort "'fishing"' abgeleitet, da die Angreifer nach Informationen fischen. Das "'Ph"' kommt von "'sophisticated"' und meint damit, dass die Angreifer ausgeklügelte Techniken verwenden um an Informationen heranzukommen.\cite{PhishingExposed}\\
		Die wohl bekannteste Angriffsmethode von Phishing ist das E-Mail-Phishing. Bei diesem Verfahren, versendet ein Angreifer meist eine gefälschte E-Mail, um ein Opfer zu täuschen und dadurch sein Ziel zu erreichen. Die sogenannten Phishing-Mails enthalten meist eine Aufforderung einen Link zu öffnen und sehen täuschend echt aus.\\
		Ein reales Beispiel könnte sein, dass der Angreifer eine gefälschte E-Mail von Amazon an das Opfer versendet und es dabei auffordert, einen Link in der Mail zu öffnen. Nachdem die Zielperson auf den Link geklickt hat, muss Sie sich anmelden. Hier könnte der Angreifer ein täuschend echtes Anmeldeformular erstellt haben, um die Anmeldedaten der Zielperson zu bekommen. Sobald die Anmeldedaten eingegeben wurden, könnte eine Fehlermeldung erscheinen, die einen Authentifizierungsfehler beinhaltet und das Opfer auffordert sich erneut anzumelden. Jedoch wird während diesem Prozess das originale Anmeldeformular geladen und das Opfer kann sich korrekt bei der entsprechenden Webseite anmelden. \\
		Dieser Verfahren ermöglicht Angreifern die Anmeldedaten von einer Zielperson ohne großen Aufwand zu beschaffen. Allerdings benötigt der Angreifer für diese Methode nicht nur Social Engineering sondern auch technische Fähigkeiten.\cite{PhishingDarkWaters}
		
		\subsection{Spear-Phishing}
		Das Spear-Phishing ist eine erweiterte Methode des herkömmlichen E-Mail-Phishings. Hierbei wird anstatt das Versenden etlicher Phishing-Mails an unbekannte Opfer, eine gezielte Mail an eine ausgewählte Person versendet.\cite{SpearPhishingPaper}\\
		Bei dieser Form von E-Mail-Phishing spielt die Opferauswahl und die Informationsbeschaffung eine sehr große Rolle, da diese Information später für personalisierte E-Mails oder vorgetäuschte Identitäten verwendet werden können. Durch diese Art von Täuschung kann ein Opfer dazu bewegt werden auf einen Link zu klicken und dadurch eine Schadsoftware herunterzuladen.\cite{SpearPhishingPaper} \\
		Der Aufwand für die Informationsbeschaffung wird oft in Kauf genommen, da der Erfolg bei dieser Methode vielversprechender ist als beim herkömmlichen E-Mail-Phishing.\\
		91\% der Advanced Persistent Threat (APT) Angriffe auf Firmen beginnen mit einer Spear-Phishing-E-Mail. Die Schadsoftware wir meisten als Remote Access Trojans (RATs) in einem Zip-Datei überliefert.\cite{SpearPhishing}


\section{Open Source Intelligence}
	\subsection{Definition OSINT}
	Open Source Intelligence kurz OSINT ist definiert in eine Intelligenz, welche aus öffentlich zugänglichen Informationen gewonnen wird. Allerdings kann sich die Bedeutung fallspezifisch ändern. So bedeutet OSINT für die CIA die Informationsgewinnung aus ausländischen Nachrichtensendungen. Doch für die meisten Menschen bedeutet OSINT die Gewinnung eines öffentlichen Inhalts aus dem Internet. \cite{Bazzell}\\
	Unter Open Source wird die öffentlich zugängliche Information, die in gedruckter oder elektronischer Form vorliegt, bezeichnet.\cite{steele1996open} Eine Verbindung mit dem Begriff Open-Source-Software besteht nicht.
	\subsection{Web Crawler}
		Web Crawler sind Computerprogramme, die mit Hilfe der Hypertextstruktur das Internet durchlaufen. Dabei können sie in einen \textbf{internen} und \textbf{externen Web Crawler} unterschieden werden. Der interne Web Crawler durchsucht ausschließliche interne Seiten einer Webseite und der externe Web Crawler durchsucht unbekannte Webseiten im ganzen Netz. \cite{sharma2012study}

		In anderen Worten besteht die Funktionsweise darin, dass in den meisten Fällen ein automatisiertes Programm, Web Crawler, erstellt wird. Dieser lädt Webinhalte herunter und durchsucht den Inhalt nach Hyperlinks. Den gefundenen Links wird gefolgt, um neue Webseiten mit weiteren Links zu laden. So hangelt sich ein Web Crawler von Link zu Link durch das Internet.\cite{WebScraping}
		
	\subsection{Web Scraper}
		In der Theorie bedeutet \textit{web scraping} die Informationsbeschaffung im Internet mit unterschiedlichsten Mitteln. \cite{WebScraping}\\		
		Meist wird dies mit einem automatisierten Programm realisiert, welches Daten von einem Webserver anfragt, entgegen nimmt, analysiert und auswertet. 
		In der Praxis gibt es ein großes Feld von Programmiertechniken und Einsatzmöglichkeiten.
		Mit Hilfe eines Web Scrapers ist es möglich, große Datenmengen zu erfassen und zu verarbeiten.\cite{WebScraping}

		\subsubsection{Natural Language Processing}
			Natural Language Processing kurz \textit{NLP} beschreibt eine Technologie, für die Kommunikation zwischen Mensch und Computer. Mit dem Ziel, dass ein Computer die natürliche Sprache verstehen und verarbeiten kann. Dafür werden verschiedenste Methoden aus der Sprach- und Computerwissenschaft sowie aus der künstliche Intelligenz verwendet. Unter anderem hat eine NLP-Anwendung die Aufgabe von \textbf{Stemming}.\cite{NLP} 
	
			\textbf{Stemming} ist eine Methode der Wortstandardisierung, bei der verwandte Wörter auf ihrer Stammform reduziert werden. Dabei wird bei dem Rechenvorgang auf den Stamm und die Semantik eines Wortes geachtet. Aus diesem Grund fällt der Name Stammformreduktion öfters in Verbindung von Stemming.\cite{eldesouki2009stemming}\\
			%TODO Beispiel von Stammformreduktion
			Die Verwendung von Stemming, kann bei der Schlüsselwortgenerierung von Texten sehr hilfreich sein, da die Anzahl der möglichen Schlüsselwörter reduziert werden können.


%%% Local Variables: 
%%% mode: latex
%%% TeX-master: "Bachelorarbeit"
%%% End: 
