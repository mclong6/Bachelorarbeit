%Kapitel des Hauptteils

\chapter {Grundlagen}  %Name des Kapitels
\label{cha:grundlagen} %Label des Kapitels

\section{Social Engineering} %Unterkapitel
\label {sec:Social Engineering} %Label des Unterkapitels
	Die Definition von Social Engineering (SE) ist nicht eindeutig. Es gibt sehr verschiedene Ansichten von der Definition. Die Idee von Social Engineering ist, eine Ziel so zu manipulieren, damit das Ziel eine für den Angreifer bessere Entscheidung trifft. In dem Buch Social Engineering - The Art of Human Hacking, von Christopher Hadnagy, ist Social Engineering definiert als:
	
	\textit{"'social engineering is the act of manipulating a person to take an action that may or may not be in the "'target’s"' best interest"'}\cite{ArtOfHumanHacking}
	
	Wiederum lautet die Definition in dem Buch von Kevin D. Mitnick:
	
	\textit{"'Social Engineering uses influence and persuasion to deceive people by convincing them that the social engineer is someone he is not, or by manipulation. As a result, the social engineer is able to take advantage of people to obtain information with or without the use of technology"'}\cite{ArtOfDeception}
	
	SE wird Menschen von Geburt an beigebracht und begegnet einem beinahe jeden Tag. Schon ein Baby muss wissen wie es die Eltern manipulieren kann damit es Dinge wie Essen, Zuneigung, o.ä. bekommt. Darüber hinaus ist SE in vielen Berufen ein täglicher Bestandteil. Beispielsweise manipulieren Ärzte viele Patienten mit einer Placebo-Behandlung. Bei dieser Behandlung wird dem Patient ein wirkstoff-freies Medikament verschrieben. Ausschließlich durch die Manipulation des Patienten und den sogenannten Palzebo-Effekt können Erfolge erzielt werden.
	
	Im Bereich der Informationssicherheit, wird von Social Engineering gesprochen, wenn Angreifer durch die Manipulierung und Täuschung von Menschen vertrauliche Informationen oder Zugänge zu Systemen bekommen. Die bekanntesten Angriffsmethoden sind Phishing, Pretexting, Baiting und Quad Pro Quo. Bei dieser Arbeit wird hauptsächlich auf das Thema E-Mail-Phishing eingegangen.

	Der Aufbau eines SE-Angriffes ist definiert in mehrere Phasen. Das wohl bekannteste Modell für einen Social Engineering-Angriffszyklus ist in dem Buch von Kevin D. Mitnicks \cite{ArtOfDeception} definiert. Dieser Zyklus besteht aus den 4 Phasen \textit{Research, Developing rapport and trust, Exploiting trust} und \textit{Utilize information}.\\
	In der \textit{Research-Phase} geht es um die Informationsbeschaffung. Bei dieser Phase will der Angreifer möglichst viel Informationen über das Ziel herausfinden. Die \textit{Developing Rapport and Trust-Phase} beschreibt den Kontaktaufbau zum Ziel, da wenn das Opfer dem Angreifer vertraut, hat dieser ein leichteres Spiel in den kommenden Phasen. Das nun erzeugte Vertrauen wird in der \textit{Exploitung Trust-Phase} ausgenutzt. Hier will der Angreifer die eigentlich Information vom Opfer herausfinden. Dies geschieht einerseits durch bestimmtes Nachfragen oder Manipulation.
	\textit{Utilize Information} ist die letzte Phase. Dort wird die gewonnene Information genutzt um das eigentliche Ziel des Angreifers zu erreichen.\\
	Grundsätzlich werden bei einem Social Engineering Angriff menschliche Wünsche, Ängste und verbreitete Verhaltensmuster verwendet um ein Opfer zu manipulieren.\cite{LeitfadenSE}\\

		\subsection{Phishing}
		Das Wort Phishing wird von dem Wort "'fishing"' abgeleitet, da die Angreifer nach Informationen fischen. Das "'Ph"' kommt von "'sophisticated"' und meint damit, dass die Angreifer ausgeklügelte Techniken verwenden um an Informationen heranzukommen.\cite{PhishingExposed}\\
		Die wohl bekannteste Angriffsmethode von Phishing ist das E-Mail-Phishing. Bei diesem Verfahren, versendet ein Angreifer meist eine gefälschte E-Mail, um ein Opfer zu täuschen und dadurch sein Ziel zu erreichen. Die sogenannten Phishing-Mails enthalten meist eine Aufforderung einen Link zu öffnen und sehen täuschend echt aus. Zum Beispiel könnte der Angreifer ein Layout von Amazon verwenden und das Ziel auffordern, den Link zu öffnen wegen einem Authentifizierungsproblem. Nachdem Sie auf den Link geklickt haben müssen Sie sich anmelden. Hier könnten die Angreifer Ihre Anmeldedaten abgreifen, nachdem sie das Opfer eingeben hat. Sobald die Anmeldedaten eingegeben wurden, könnte eine Fehlermeldung erscheinen, die sagt:"'Hoppla, ein Fehler ist aufgetreten, melden Sie sich bitte neu an!"'. Anschließend wird die originale Seite geladen, das Opfer kann sich korrekt anmelden und der Angreifer hat ohne einen großen Aufwand die Anmeldedaten der Zielperson.\\
		Für diese Methode benötigt der Angreifer nicht nur Social Engineering  sondern auch technische Fähigkeiten.\cite{PhishingDarkWaters}
		
		\subsubsection{Spear-Phishing}
		Das Spear-Phishing ist eine erweiterte Methode des herkömmlichen E-Mail-Phishings. Hierbei wird anstatt das Versenden etlicher Phishing-Mails an unbekannte Opfer, eine gezielte Mail an eine Person versendet.\cite{SpearPhishingPaper}\\
		Bei dieser Form von E-Mail-Phishing spielt die Opferauswahl und die Informationsbeschaffung eine wichtig Rolle, da diese Information später für personalisierte E-Mails oder vorgetäuschte Identitäten verwendet werden können. Durch diese Art von Täuschung kann ein Opfer dazu bewegt werden auf einen Link zu klicken und dadurch eine Schadsoftware herunterzuladen.\cite{SpearPhishingPaper} \\
		Der Aufwand für Informationsbeschaffung wird oft in Kauf genommen, da der Erfolg bei diese Methode vielversprechender ist al beim herkömmlichen E-Mail-Phishing.
		91\% der Advanced Persistent Threat (APT) Angriffe auf Firmen beginnen mit einer Spear-Phishing-E-Mail. Die Schadsoftware wir meisten als Remote Access Trojans (RATs) in einem Zip-Datei überliefert.\cite{SpearPhishing}


\section{Informationsbeschaffung im Internet}
	\subsection{Web Scraping}
		In der Theorie bedeutet \textit{Web Scraping} die Informationsbeschaffung im Internet mit unterschiedlichsten Mitteln. \cite{WebScraping}\\		
		Meist wird dies mit einem automatisierten Programm realisiert, das Daten von einem Webserver anfragt, entgegen nimmt, analysiert und auswertet. 
		In der Praxis gibt es ein großes Feld von Programmiertechniken und Einsatzmöglichkeiten.
		Mit Hilfe von \textit{Web Scraping} ist es möglich große Datenmengen zu erfassen und zu verarbeiten.\cite{WebScraping}

	\subsection{Web Crawler}
	\textit{Web Crawler} sind Computerprogramme, die mit Hilfe der Hypertextstruktur das Internet durchlaufen. Dabei können sie in einen \textit{internen} und \textit{externen Web Crawler} unterschieden werden. Der \textit{interne Web Crawler} durchsucht ausschließliche interne Seiten einer Webseite und der \textit{externe Web Crawler} durchsucht unbekannte Webseiten im ganzen Netz. \cite{sharma2012study}
	
	In anderen Worten besteht die Funktionsweise darin, dass in den meisten Fällen ein automatisiertes Programm \textit{Web Crawler} erstellt wird. Dieser lädt Webinhalte herunter und durchsucht den Inhalt nach Hyperlinks. Den gefundenen Links wird gefolgt, um neue Webseiten mit weiteren Links zu laden. So hangelt sich ein \textit{Web Crawler} von Link zu Link durch das Internet.\cite{WebScraping}


\section{Personenbezogene Daten}
	%TODO möglicherweise unterschied von Information und Daten erklären
	Laut dem DSGVO sind \textit{personenbezogene Daten} "'alle Informationen, die sich auf eine identifizierte oder identifizierbare natürliche Person (im Folgenden „betroffene Person“) beziehen; als identifizierbar wird eine natürliche Person angesehen, die direkt oder indirekt, insbesondere mittels Zuordnung zu einer Kennung wie einem Namen, zu einer Kennnummer, zu Standortdaten, zu einer Online-Kennung oder zu einem oder mehreren besonderen Merkmalen identifiziert werden kann, die Ausdruck der physischen, physiologischen, genetischen, psychischen, wirtschaftlichen, kulturellen oder sozialen Identität dieser natürlichen Person sind;"'\cite{personenbezogeneDaten}
	
\section{Natural Language Processing}
	\textit{Natural Language Processing} kurz NLP und beschreibt einen Technologie, für die Kommunikation zwischen Mensch und Computer. Mit dem Ziel, dass ein Computer die natürliche Sprache verstehen und verarbeiten kann. Dafür werden verschiedenste Methoden aus der Sprach- und Computerwissenschaft sowie aus der künstliche Intelligenz verwendet. Unter anderem hat eine NLP-Anwendung die Aufgabe von \textit{Stemming}.\cite{NLP} 
	
	\textit{Stemming} ist eine Methode der Wortstandardisierung, bei der verwandte Wörter auf ihrer Stammform reduziert werden. Dabei wird bei dem Rechenvorgang auf den Stamm und die Semantik eines Wortes geachtet. Aus diesem Grund fällt der Name Stammformreduktion öfters in Verbindung von Stemming.\cite{eldesouki2009stemming}\\
	Die Verwendung von Stemming, kann bei der Schlüsselwortgenerierung von Texten sehr hilfreich sein, da die Anzahl der möglichen Schlüsselwörter reduziert werden können.
	
\section{Textanalyse}
	\subsection{Stoppwörter}
	\label{sec:Stoppwörter}
	Als Stoppwörter werden Wörter bezeichnet, die sehr oft auftreten und keinen großen Informationsgewinn mit sich bringen. Beispiele dafür sind \textit{und}, \textit{weil}, \textit{der} oder \textit{als}.\cite{Stopwords}
	

%%% Local Variables: 
%%% mode: latex
%%% TeX-master: "Bachelorarbeit"
%%% End: 
