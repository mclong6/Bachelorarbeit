
%% Vorlage fuer Studien- und Bachelorarbeiten an der HS Ravensburg-Weingarten
%% Benoetigt wird KOMA-Skript ab Version 2.8j vom 30.07.2001
%% Vor der Veraenderung irgendwelcher Einstellungen wird dringend empfohlen
%% die Anleitung zum KOMA-Skript (scrguide) zu konsultieren !!
%%
%% Bilder bitte nicht mit Endung einbinden, so ist die Erzeugung von
%% DVI, PS und PDF problemlos m�glich!
%% J. Moehler, 2008-12-10 
%%Kommentar


%% Dokumentendefinition
\documentclass[
   12pt,                % Schriftgroesse 12pt
   a4paper,             % Layout fuer Din A4
   german,              % deutsche Sprache, global   
%  twoside,             % Layout fuer beidseitigen Druck
	 oneside,						  % Layout für einseitigen Druck
   headinclude,         % Kopfzeile wird Seiten-Layouts mit beruecksichtigt
   headsepline,         % horizontale Linie unter Kolumnentitel
   plainheadsepline,    % horizontale Linie auch beim plain-Style
   BCOR12mm,            % Korrektur fuer die Bindung
   DIV18,               % DIV-Wert fuer die Erstellung des Satzspiegels, siehe scrguide
   halfparskip,         % Absatzabstand statt Absatzeinzug
   openany,             % Kapitel können auf geraden und ungeraden Seiten beginnen
   bibtotoc,            % Literaturverz. wird ins Inhaltsverzeichnis eingetragen
   pointlessnumbers,    % Kapitelnummern immer ohne Punkt
   tablecaptionabove,   % korrekte Abstaende bei TabellenUEBERschriften
   fleqn,               % fleqn: Glgen links (statt mittig)
%   draft               % Keine Bilder in der Anzeige, overfull hboxes werden angezeigt !Auskommentieren für Test-Compile!
]{scrbook}[2001/07/30]  % scrbook-Version mind. 2.8j von 2001/07/30

%% Pakete für nützliche Dinge
\usepackage{color} 							 % Schriftfarben verwenden
\usepackage{ngerman}             % neue deutsche Rechtschreibung
%\usepackage[ngerman]{translator} % Übersetzung 
%\usepackage[ansinew]{inputenc}   % Input-Encodung: ansinew fuer Windows
\usepackage[utf8]{inputenc}   % Input-Encodung: latin1 fuer Unix
\usepackage[T1]{fontenc}         % T1-kodierte Schriften, korrekte Trennmuster fuer Worte mit Umlauten
\usepackage{ae}                  % Für PDF-Erstellung
\usepackage[hang]{caption2}      % mehrzeilige Captions ausrichten

\usepackage[centertags]{amsmath} % AMS-Mathematik, centertags zentriert Nummer bei split
\usepackage{latexsym}            % verschiedene Symbole
\usepackage{textcomp}            % verschiedene Symbole
 \usepackage{microtype}          % bessere Optik     
\usepackage{graphicx}            % zum Einbinden von Grafiken
\usepackage{float}               % u.a. genaue Plazierung von Gleitobjekten mit H
% \usepackage{pstricks}          % PostScript Macros
% \usepackage{lscape}            % Seite im Querformat bei Erhalt der Kopfzeile
% \usepackage{verbatim}          % Quellcode einbinden (\verbatiminput)
% \usepackage{multicol}          % Mehrspaltiger Text 
\usepackage{placeins}

%%% Literatur und sonstige Referenzen
\usepackage{cite}              % Sortierte und zusammengefasste Zitatnummern 
\usepackage{url}							 % URL in Literatur wird unterst�tzt
\usepackage{varioref}          % Verbesserte Referenzen
\usepackage{hyperref}          % Verlinkte Verzeichnisse
% \usepackage[numbers, sort]{natbib} % DIN Literaturverzeichnis; nicht zusammen mit cite verwenden!

%% Index
\usepackage{makeidx}						 % Index verwenden
\makeindex											 % index erstellen

%% Zeilenabstand
\usepackage{setspace}            % Zeilenabstand einstellbar
\onehalfspacing                  % eineinhalbzeilig einstellen

%% Kopf- und Fusszeilen
\usepackage{scrpage2}                             % Kopf und Fusszeilen-Layout 
\renewcommand{\headfont}{\normalfont\sffamily}    % Kolumnentitel serifenlos
\renewcommand{\pnumfont}{\normalfont\sffamily}    % Seitennummern serifenlos
\pagestyle{scrheadings}
\ihead[]{\headmark}               % Kolumnentitel immer oben innen
\chead[]{}                        % oben Mitte
\ohead[\pagemark]{\pagemark}      % Seitennummern immer oben aussen
\ofoot[]{}                        % Fusszeile aussen
\cfoot[]{}                        % Fusszeile Mitte
\ifoot[]{}                        % Fusszeile innen

%% Fussnotenzähler
\usepackage{chngcntr}              % Paket um Counter zu steuern
\counterwithout{footnote}{chapter} % Fussnoten nicht pro Chapter, sondern Global
\usepackage{threeparttable}        % Tabelle mit Fussnoten

%% Namen von Verzeichnissen definieren
\renewcommand{\bibname}{Literatur}               % Literaturverzeichnis wird zu Literatur
\renewcommand{\figurename}{Bild}                 % Abbildung wird zu Bild
\renewcommand{\listfigurename}{Bildverzeichnis}
\renewcommand{\indexname}{Stichwortverzeichnis}

%% Schrift mit Serifen auch fuer Ueberschriften benutzen
%\renewcommand*{\sectfont}{\bfseries}
%\renewcommand*{\descfont}{\bfseries}

\typearea[current]{current}        % Neuberechnung des Satzspiegels mit alten Werten nach Änderung von Zeilenabstand,etc

% -- Glossar --
\usepackage[toc, acronym]{glossaries} % Glossareinträge, muss nach hyperref (insofern dies verwendet wird) geladen werden
% (aufgrund der Seitenzahlverlinkung im Glossarverzeichnis), (benötigt die Packete
% "xkeyval" und "supertabular", welche dann automatisch eingebunden werden)
\newglossary[slg]{symbolslist}{syi}{syg}{Symbolverzeichnis} %Ein eigenes Symbolverzeichnis erstellen
\renewcommand*{\glspostdescription}{} %Den Punkt am Ende jeder Beschreibung deaktivieren
\makeglossaries % erstellt ein Glossar (Verzeichnis für Begriffserklärungen,
% z.B.: Abkürzungen)
% Glossareinträge (MUSS für JEDEN Glossareintrag überarbeitet werden):
%% Symbole:
%%\newglossaryentry{symb:Name}{name=Symbolname, description={Beschreibung}, sort=alphabetisches Wort für die Einreihung, type=symbolslist}
\newglossaryentry{symb:Pi}{
name=$\pi$,
description={Die Kreiszahl.},
sort=symbolpi, type=symbolslist
}
%%Abkürzungen:
%%\newacronym{Referenz}{Abkürzung}{Beschreibung}
\newacronym{BSP}{BSP}{Beispiel}


%%Eine Abkürzung mit Glossareintrag:
%%\newacronym{Referenz}{Abkürzung}{Beschreibung\protect\glsadd{glos:Referenz}}
\newacronym{AD}{AD}{Active Directory\protect\glsadd{glos:AD}}

%%Glossareintrag:
%%\newglossaryentry{glos:Referenz}{name=Name, description={Beschreibung}}
\newglossaryentry{glos:AD}{
name=Active Directory,
description={Active Directory ist in einem Windows Server 2000, Windows
Server 2003, oder Windows Server 2008-Netzwerk der Verzeichnisdienst, 
der die zentrale Organisation und Verwaltung aller Netzwerkressourcen erlaubt. Es
ermöglicht den Benutzern über eine einzige zentrale Anmeldung den
Zugriff auf alle Ressourcen und den Administratoren die zentral
organisierte Verwaltung, transparent von der Netzwerktopologie und
den eingesetzten Netzwerkprotokollen. Das dafür benötigte
Betriebssystem ist entweder Windows Server 2000, 
Windows Server 2003, oder Windows Server 2008, welches auf dem zentralen
Domänencontroller installiert wird. Dieser hält alle Daten des
Active Directory vor, wie z.B. Benutzernamen und
Kennwörter.}
}

\newglossaryentry{glos:Glossareintrag}{name=Glossareintrag, description={Erweiterte Informationen zum
einem Wort oder einer Abkürzung, ähnlich einem Eintrag im Duden.}}




%%% Local Variables: 
%%% mode: latex
%%% TeX-master: "Bachelorarbeit"
%%% End: 

%%% PDF-Erzeugung: pdflatex statt latex aufrufen!
%% BEI WINDOWS UND TEXNICCENTER AUSKOMMENTIERT LASSEN !!! 
%\pdfoutput=1                  % PDF-Ausgabe
%\usepackage[pdftex, a4paper,  % muss letztes Package sein!
%     pdftitle={Titel der Arbeit},%
%     pdfauthor={Name Autor},%
%     pdfsubject={Studien- bzw. Diplomarbeit},%
%     pdfkeywords={Stichwort zur Arbeit},%
%    ]{hyperref} % 



%\graphicspath{{figs/}{bilder/}}    % Falls texinput nicht gesetzt -> Bildverzeichnisse

%\includeonly{}


%%%%%%%%%%%%%%%%%%%%%%%%%%%%%%%%%%%%%%%%%%%%%%%%%%%%%%%%%%%%%%%
\begin{document}

\pagenumbering{Roman}           % Nummerierung Römisch Start bei I

%% Deckblatt fuer Studien- und Diplomarbeiten am der
%% Hochschule Weingarten

\thispagestyle{empty}
%~
{
\normalsize\fontfamily{phv} \fontsize{12pt}{10}\selectfont 
\vspace{-1cm}
\begin{minipage}[b]{9.4cm}
{\fontsize{13pt}{13} \selectfont%
Hochschule\\[1ex]
Ravensburg-Weingarten}\\[1ex]
\end{minipage}
}
\begin{minipage}[b]{10cm}
\includegraphics*[height=2.7cm]{bilder/HSLogoWGd}
\end{minipage}


\vspace{10mm}
 
\hrule 
\vspace{1cm}
{
\fontseries{b} \fontsize{20pt}{20}  \selectfont%
\begin{center}
\textcolor{red}{Titel der Arbeit} % Titel der Arbeit
\end{center}
}

\begin{center}
\large \textbf{Dokumentation zu Seminar} % Hier Praxisarbeit, Studienarbeit, Bachelorarbeit, Dokumentation zu Seminar, etc. eintragen
\end{center}

\begin{center}
\textbf{Multimedia Technik SS09} % Hier den Zusatz wie Fach oder Semester eintragen
\end{center}

\vspace{5mm}

\begin{center}
im Studiengang \textcolor{red}{Angewandte Informatik} % Hier den Studiengang eintragen
\end{center}

\begin{center}
an der Hochschule Ravensburg - Weingarten 
\end{center}
\begin{center}

\end{center}
\vspace{5mm}
\begin{center}
von
\end{center}




\begin{center}
{\fontsize{12pt}{12} \selectfont%
\begin{tabular}{ll}
Autor Name & \textcolor{red}{Matr.-Nr.: xxxxx}\\[0.5ex] % Hier den Autor und die Matrikelnummer statt xxxxx eintragen
%Autor 2 & \textcolor{red}{Matr.-Nr.: xxxxx}\\[0.5ex] % Für weitere Autoren Zeilen auskommentieren bzw. kopieren und ausfüllen
%Autor 3 & \textcolor{red}{Matr.-Nr.: xxxxx}\\[0.5ex]
Abgabedatum :& \today   % Das Abgabedatum wird gleichgesetzt mit dem Datum der letzten Compilierung. Statt \today kann auch Datum von Hand geschrieben werden
\end{tabular}
}
\end{center}
                               

\vspace{1cm}

\vspace{1cm}
\hrule


%%% Local Variables: 
%%% mode: latex
%%% TeX-master: "Bachelorarbeit"
%%% End: 

                % Deckblatt Nummerierung unterdrückt (In Deckblatt festgelegt)
%%\cleardoubleemptypage         % Die Eidesstattliche Erklaerung auf einer rechten Seite beginnen
%% Eidestattliche Erklärung %%
%% Die Erklärung sollte nach dem Deckblatt fest abgeheftet werden
\addchap*{Erklärung} %* nicht entfernen, sonst erhält Erklärung eine Nummer und erscheint im Inhaltsverzeichnis

\thispagestyle{empty} %Keine Seitenzahl, keine Kopf- und Fusszeile

Hiermit erkläre ich, dass ich die vorliegende Arbeit mit dem Titel   % ein Autor
% Hiermit erkl"aren wir, dass wir die vorliegende Arbeit mit dem Titel \newline % mehrere Autoren
\begin{center}
\textbf{Entwicklung einer Anwendung zur automatisierten Beschaffung von personenbezogenen Daten im Internet und deren Integration in Phishing-Mails}
\end{center}
selbstständig angefertigt, nicht anderweitig zu Prüfungszwecken vorgelegt, keine anderen als die angegebenen Hilfsmittel benutzt und wörtliche sowie sinngemäße Zitate als solche gekennzeichnet habe.\newline  % ein Autor

\begin{flushleft}
Weingarten, 29. April 2019 % Ort eintragen, /today kann durch Datum 2009-10-21 oder 21.10.2009 ersetzt werden
\end{flushleft}

%%% Unterschriftenblock für einen Autor
\begin{tabular}{l}   
Autor Name        \\% Hier Autor eintragen
 \\
------------------------------------ \\
\end{tabular}

%%% Unterschriftenblock für mehrere Autoren
%\begin{tabular}{lll}
%Autor 1       &Autor 2      &Autor 3 \\% Hier eintragen
% & & \\
%------------------------------------ & ------------------------------------ & ------------------------------------ \\
%\end{tabular}

%Hier unterschreiben


%%% Local Variables: 
%%% mode: latex
%%% TeX-master: "Bachelorarbeit"
%%% End: 
                 % Eidesstattliche Erklaerung Nummerierung unterdrückt
%%\cleardoubleemptypage         % Das Inhaltsverzeichnis auf einer rechten Seite beginnen

\pagenumbering{Roman}           % Nummerierung Römisch start bei I 

\begin{spacing}{1.0}            % Verzeichnisse werden mit einzeiligem Abstand gesetzt
 \tableofcontents               % Inhaltsverzeichnis
\end{spacing}
%%% Vorbemerkungen %%%  Nummerierung unterdrückt durch *
\addchap{Kurzfassung}
\label{cha:kurzfassung} 
Es wird gezeigt, wie eine automatisierte Suche nach personenbezogenen Daten im Internet aussehen kann und wie diese Daten für einen Phishing-Mail-Angriff verwendet werden können. 

Wird erweitert!

%%% Local Variables: 
%%% mode: latex
%%% TeX-master: "Bachelorarbeit"
%%% End: 


             % Kurzfassung der Arbeit
\addchap{Abstract}
\label{cha:abtract} 







%%% Local Variables: 
%%% mode: latex
%%% TeX-master: "Bachelorarbeit"
%%% End: 
             % Abstract der Arbeit (englische Kurzfassung der Arbeit)
\addchap{Danksagung}
\label{cha:danksagung}



%%% Local Variables: 
%%% mode: latex
%%% TeX-master: "Bachelorarbeit"
%%% End: 								  % Danksagung (optional)
\addchap{Vorwort}
\label{cha:vorwort}



%%% Local Variables: 
%%% mode: latex
%%% TeX-master: "Bachelorarbeit"
%%% End: 							    % Vorwort (optional)

%%% Hauptteil %%%       Nummerierung beginnend bei 1
%%\cleardoubleplainpage         % Das erste Kapitel des Hauptteils auf einer rechten Seite beginnen
\mainmatter                     % den Hauptteil beginnen
\chapter{Einleitung}
\label{cha:einleitung}

%%% Local Variables: 
%%% mode: latex
%%% TeX-master: "Bachelorarbeit"
%%% End: 

\section{Motivation}
\label {sec:Motivation}
In der heutigen Zeit wird das Thema Informationssicherheit immer wichtiger. Systeme werden immer komplexer und Firewalls immer besser.
Doch laut dem Bundeskriminalamt hat sich die Zahl der Cyberkriminalität mit einem klaren Trend nach oben entwickelt. \cite{Cyberkriminalitaet}\\
Eine häufig verwendetet Technik von Cyberkriminalität ist das E-Mail-Phishing. Hier wird der Mensch als Schwachstelle des Systems genutzt. In den neusten Fällen von Phishing-Attacken zeigt die Verbraucherzentrale Nordrhein-Westfalen, dass diese meist direkt an eine Person adressiert sind. Beispielsweise wird man in den gefälschten DSGVO-E-Mails, im Namen der Sparkasse, persönlich mit Namen angesprochen. \cite{VerbraucherzentraleNW} \\
Im Rahmen dieser Abschlussarbeit wird gezeigt, mit welchem Aufwand solche Angriffe verbunden sind und wie die Suche nach privaten Informationen im Internet aussieht.

\section{Problem}
Leser Problem komplett erklären, weiterführende Motivation

Persönliche Informationen werden im Internet immer leichter zugänglich gemacht. !!!ZITAT!!!\\
Es gibt viele Webseiten die persönliche Information von Menschen bereitstellt. Eine davon ist auch www.fupa.net. Hier können persönliche Informationen ohne Anmeldung ausgelesen werden. Diese Art von Webseite ist eine perfekte Informationsquelle für Angreifer.\\
Im Bereich von Social Engineering Angriffen wird diese Information oft genutzt um ein Opfer zu manipulieren.
Das hier beschriebene Problem zeigt dass der Zugang für persönliche Information durch das Internet für viele Menschen einfacher gemacht wird. Es soll gezeigt werden wie einfach es ist, personenbezogene Daten aus dem Internet herauszulesen, analysieren und für einen Phishing-Angriffe zu verwenden.
!!!!ZITATE HINZUFÜGEN!!! statista z.B.

\section{Eigene Leistung}
\label {sec:Leistung} 
In dieser Arbeit wird ein Phishing-Mailgenerator erstellt. Dieser liest automatisiert Informationen von der Webseite www.fupa.net heraus und erstellt potentiellen Opferprofile. Zusätzlich wird mit dieser Information und einem Web-Crawler das Internet nach weiteren Informationen durchstöbert. Mit dem Vornamen, Nachnamen und dem Geburtsjahr werden die E-Mail-Adressen generiert. Die gefundenen Informationen werde automatisch in eine personalisierte Phishing-E-Mail eingebaut. Für einen höheren Erfolg werden E-Mail-Muster erstellt.

\section{Aufbau der Arbeit}
\label {sec:Aufbau} 
Meine Arbeit gliedert sich in zwei Teile. Einem theoretischen und einem praktischen Teil. Der Theorie-Teil beginnt im zweiten Kapitel und beschreibt die Grundbegriffe im Bereich Social Engineering, Webtools, E-Mails und Programmiersprachen. Im nächsten Kapitel befindet sich die Anforderungsanalyse. Hier werden die Anforderungen an die Arbeit festgelegt. Darauf folgen die Lösungsvorschläge im Kapitel vier und die ausgewählte Lösung anhand den Anforderungen im Kapitel 5. Im Anschluss wird bei der Umsetzung auf den Praktischen Teil eingegangen.Am Ende befindet sich das Fazit, der Ausblick und der Anhang.






               % Einleitung
%Kapitel des Hauptteils

\chapter {Grundbegriffe}  %Name des Kapitels
\label{cha:grundlagen} %Label des Kapitels

\section{Social Engineering} %Unterkapitel
\label {sec:Unterkapitel} %Label des Unterkapitels
\subsubsection{Definition}
Die Definition von Social Engineering (SE) ist nicht eindeutig. Es gibt sehr verschiedene Ansichten von der Definition. Die Idee von Social Engineering ist, eine Ziel so zu manipulieren, damit das Ziel eine für den Angreifer bessere Entscheidung trifft. In dem Buch Social Engineering - The Art of Human Hacking, von Christopher Hadnagy, ist Social Engineering definiert als "'social engineering is the act of manipulating a person to take an action that may or may not be in the "'target’s"' best interest"'\cite{ArtOfHumanHacking}. Die Definition in dem Buch von Kevin D. Mitnick lautet:"'Social Engineering uses influence and persuasion to deceive people by convincing them that the social engineer is someone he is not, or by manipulation. As a result, the social engineer is able to take advantage of people to obtain information with or without the use of technology"'\cite{ArtOfDeception}.\\
!!!!!!!!!!Umschreiben!!!!!!!!!!\\Wie bereits erwähnt, nutzt ein Social Engineer menschliche Wünsche, Ängste
und verbreitete Verhaltensmuster aus, um seine Opfer zu manipulieren.\cite{LeitfadenSE}\\

\subsubsection{SE im Leben}
SE wird einem von Geburt an beigebracht und begegnet einem beinahe jeden Tag. Schon ein Baby muss wissen wie es die Eltern manipulieren kann damit man Dinge wie Essen, Zuneigung, o.ä. bekommt. Darüber hinaus ist SE in vielen Berufen ein täglicher Bestandteil. Beispielsweise manipulieren Ärzte viele Patienten mit einer Placebo-Behandlung. Bei dieser Behandlung wird dem Patient ein wirkstoff-freies Medikament verschrieben. Nur durch die Manipulation des Patienten und den sogenannten Palzebo-Effekt können Erfolge erzielt werden.\\

\subsubsection{SE in der Informationssicherheit}
Im Bereich der Informationssicherheit spricht man von Social Engineering wenn man durch Manipulierung bzw. das Hacken von Menschen Passwörter, Zugänge zu Systemen oder vertrauliche Information bekommt. Die bekanntesten Angriffsmethoden sind Phishing, Pretexting, Baiting und Quad Pro Quo. Bei dieser Arbeit wird aber hauptsächlich auf das Thema Phishing eingegangen.

\subsection{Social Engineering Angriffe}

Der Aufbau eines Social-Engineering-Angriffes ist definiert in mehrere Phasen. Das wohl bekannteste Modell für einen Social Engineering-Angriffszyklus ist in dem Buch von Kevin D. Mitnicks - The art of deception: controlling the human element of security \cite{ArtOfDeception} definiert. Dieser Zyklus besteht aus den 4 Phasen Research, Developing rapport and trust, Exploiting trust und Utilize information.
In der Research-Phase geht es um die Informationsbeschaffung, bei der der Angreifer möglichst viel Informationen über das Ziel herausfindet. Die Developing rapport and trust Phase beschreibt den aufbau für einen guten Kontakt, da der Angreifer ein leichteres Spiel hat wenn das Ziel dem Angreifer vertraut. Das nun erzeugte Vertrauen wird in der Exploitung trust Phase ausgenutzt. Hier will der Angreifer die eigentlich Information vom Opfer herausfinden. Dies geschieht einerseits durch bestimmtes nachfragen oder Manipulation. Utilize information ist die letzte Phase. Dort wird die gewonnene Information genutzt um das eigentliche Ziel des Angreifers zu erreichen.\\

!!!!!!!!BILD EINFÜGEN!!!!\\

Leider sind die Phasen in dem Buch von Mintnick \cite{ArtOfDeception} nicht sehr detailliert beschrieben. Aus diesem Grund haben die Autoren von der Publikation "'Social Engineering Attack Framework"' \cite{AttackFramework} ein Framework erstellt, was eine Erweiterung von Mitnick's Angriffszykluses darstellt.\\

!!!!!!!!BILD EINFÜGEN!!!!\\

\subsubsection{Phishing}
Das Wort Phishing wird von dem Wort "'fishing"' abgeleitet, da die Angreifer nach Informationen fischen. Das "'Ph"' kommt von "'sophisticated"' und meint damit, dass die Angreifer ausgeklügelte Techniken verwenden um an Informationen heranzukommen.\cite{PhishingExposed}\\
Phishing ist ein Angriffsmethode, bei dem ein Angreifer glaubwürdige E-Mails versendet, um von einem Opfer Informationen zu erhalten. Die sogenannten E-Mails enthalten meist eine Aufforderung einen Link zu öffnen und sehen täuschend echt aus. Zum Beispiel könnten der Angreifer ein Layout von Amazon verwenden und Sie auffordern, den Link zu öffnen, wegen einem Authentifizierungsproblem. Nachdem Sie auf den Link geklickt haben müssen Sie sich anmelden. Hier könnten die Angreifer Ihre Anmeldedaten abgreifen, nachdem Sie sie eingeben haben. Sobald Sie die Anmeldedaten haben könnten Sie mit der Meldung :"'Hoppla, ein Fehler ist aufgetreten, melden Sie sich bitte neu an!"' auf die originale Seite weitergeleitet werden. Durch diesen Vorgang hätten die Angreifer ihre Anmeldedaten bekommen.\\
Für diese Methode benötigt der Angreifer nicht nur Social Engineering Fähigkeiten sonder auch technische.\cite{PhishingDarkWaters}

\subsubsection{Spear-Phishing}
Spear-Phishing ist im Prinzp die gleiche Angriffsmethode wie Phishing. Nur dass hier anstatt einer anonymen E-Mail eine persönliche Mail gesendet wird. Beispielsweise wird man hier mit einem Namen angesprochen oder man bekommt Mails mit Inhalten die einen interessieren. Aus diesem Grund benötigt man hier Zeit für die Informationsbeschaffung. Dennoch ist der Erfolg hier sehr vielversprechender als beim normalen Phishing. Desweitern ist Spear-Phishing oft mit E-Mail-Spoofing verbunden.
91\% der APT Angriffe auf Firmen beginnen mit einer Spear-Phishing-E-Mail. Advanced Persistent Threat (APT). Die Schadsoftware wir meisten als Remote Access Trojans (RATs) in einem Zip-Datei überliefert.

\subsubsection{Pretexting}
Als Polizist verkleiden, etwas vortäuschen

\subsubsection{Baiting}
ähnelt Phishing, wie Trojanisches Pferd aber mit physischen Gegenständen. Gier und Neugier werden ausgenutzt. Beispielsweise USB-Stick liegen lassen mit Malware und warten bis ihn jemand findet und öffnet.

\subsubsection{Quad Pro Quo}
Etwas geben dafür etwas bekommen. Werben für einen Service. Kostenlose Musik für Anmeldedaten.

\subsubsection{Web-Crawler}
Suchmaschine
\subsubsection{Web-Scraper}

\FloatBarrier
\begin{figure}
 \begin{center}
  \includegraphics*{bilder/HSLogoWGd}
  \caption{Logo der HS -- oder nicht?}
  \label{fig:logo}
 \end{center}
\end{figure}

Und es gibt auch ein Beispiel für eine Tabelle\index{Tabelle}.

\begin{table}
 \begin{center}
 \caption{Verwendete Matrizen}
 \label{matrizen}
  \begin{tabular}{|l|l|l|}
   \hline
   Matrix & Dimension & Symbol \\
   \hline
   Systemmatrix & $n \times n$ & ${\bf A}$  \\
   \hline
   Ausgangsmatrix & $m \times n$ & ${\bf C}$  \\
   \hline
  \end{tabular}
 \end{center}
\end{table}
\FloatBarrier
Man beachte: Bilder haben Bild{\bf unter}schriften, 
Tabellen haben Tabellen{\bf "uber}schriften.

Für jedes Kapitel sollte ein neues \TeX  File erstellt und eingebunden werden. \newline

Ein Symbol wie \gls{symb:Pi} Kann mathematisch korrekt dargestellt werden. Auch \gls{glos:Glossareintrag} zu Abkürzungen wie \gls{AD} können in \LaTeX behandelt werden.
Zum Demonstrieren wird hier noch eine Webseite von Microsoft zitiert\cite{SE}, und noch eine Stelle\cite{SE}


 
%%% Local Variables: 
%%% mode: latex
%%% TeX-master: "Bachelorarbeit"
%%% End: 
                % Ein Kapitel des Hauptteils
%Kapitel des Hauptteils

\chapter{}  %Name des Kapitels
\label{cha:} %Label des Kapitels
\section{} %Unterkapitel
\label{sec:} %Label des Unterkapitels
\subsection{} %Unterunterkapitel
\label{sse:}
\subsubsection{} %Unterkapitel 3. Ordnung
\label{sss:}
%%% Local Variables: 
%%% mode: latex
%%% TeX-master: "Bachelorarbeit"
%%% End: 
		% Ein weiteres Kapitel des Hauptteils
%Kapitel des Hauptteils

\chapter{}  %Name des Kapitels
\label{cha:} %Label des Kapitels
\section{} %Unterkapitel
\label{sec:} %Label des Unterkapitels
\subsection{} %Unterunterkapitel
\label{sse:}
\subsubsection{} %Unterkapitel 3. Ordnung
\label{sss:}
%%% Local Variables: 
%%% mode: latex
%%% TeX-master: "Bachelorarbeit"
%%% End: 
		% Die unterschiedlichen Kapitel
%Kapitel des Hauptteils

\chapter{}  %Name des Kapitels
\label{cha:} %Label des Kapitels
\section{} %Unterkapitel
\label{sec:} %Label des Unterkapitels
\subsection{} %Unterunterkapitel
\label{sse:}
\subsubsection{} %Unterkapitel 3. Ordnung
\label{sss:}
%%% Local Variables: 
%%% mode: latex
%%% TeX-master: "Bachelorarbeit"
%%% End: 
		% des Hauptteils
%Kapitel des Hauptteils

\chapter{}  %Name des Kapitels
\label{cha:} %Label des Kapitels
\section{} %Unterkapitel
\label{sec:} %Label des Unterkapitels
\subsection{} %Unterunterkapitel
\label{sse:}
\subsubsection{} %Unterkapitel 3. Ordnung
\label{sss:}
%%% Local Variables: 
%%% mode: latex
%%% TeX-master: "Bachelorarbeit"
%%% End: 
		% sollten hier in der
%Kapitel des Hauptteils

\chapter{}  %Name des Kapitels
\label{cha:} %Label des Kapitels
\section{} %Unterkapitel
\label{sec:} %Label des Unterkapitels
\subsection{} %Unterunterkapitel
\label{sse:}
\subsubsection{} %Unterkapitel 3. Ordnung
\label{sss:}
%%% Local Variables: 
%%% mode: latex
%%% TeX-master: "Bachelorarbeit"
%%% End: 
		% Masterdatei einen sinnvollen
%Kapitel des Hauptteils

\chapter{}  %Name des Kapitels
\label{cha:} %Label des Kapitels
\section{} %Unterkapitel
\label{sec:} %Label des Unterkapitels
\subsection{} %Unterunterkapitel
\label{sse:}
\subsubsection{} %Unterkapitel 3. Ordnung
\label{sss:}
%%% Local Variables: 
%%% mode: latex
%%% TeX-master: "Bachelorarbeit"
%%% End: 
		% Kommentar erhalten
%Kapitel des Hauptteils

\chapter{}  %Name des Kapitels
\label{cha:} %Label des Kapitels
\section{} %Unterkapitel
\label{sec:} %Label des Unterkapitels
\subsection{} %Unterunterkapitel
\label{sse:}
\subsubsection{} %Unterkapitel 3. Ordnung
\label{sss:}
%%% Local Variables: 
%%% mode: latex
%%% TeX-master: "Bachelorarbeit"
%%% End: 
		% z.B. den Titel des Kapitels
%Kapitel des Hauptteils

\chapter{}  %Name des Kapitels
\label{cha:} %Label des Kapitels
\section{} %Unterkapitel
\label{sec:} %Label des Unterkapitels
\subsection{} %Unterunterkapitel
\label{sse:}
\subsubsection{} %Unterkapitel 3. Ordnung
\label{sss:}
%%% Local Variables: 
%%% mode: latex
%%% TeX-master: "Bachelorarbeit"
%%% End: 
		% um für Korrekturen oder Umstellungen
%Kapitel des Hauptteils

\chapter{}  %Name des Kapitels
\label{cha:} %Label des Kapitels
\section{} %Unterkapitel
\label{sec:} %Label des Unterkapitels
\subsection{} %Unterunterkapitel
\label{sse:}
\subsubsection{} %Unterkapitel 3. Ordnung
\label{sss:}
%%% Local Variables: 
%%% mode: latex
%%% TeX-master: "Bachelorarbeit"
%%% End: 
		% leichter gefunden zu werden
\chapter{Fazit und Ausblick}
\label{chap:SchlussUndAusblick}
\section{Fazit}
Das Ergebnis der Testfälle ist überwiegend positiv. Dennoch sind nur zwei der vier Testergebnisse vollständig korrekt. Demnach ist die Antwort auf die Forschungsfrage, ob es möglich ist, ausschließlich korrekte Opferprofile zu erstellen, nein. Dennoch ist die Mehrzahl der Personenattribute bei den meisten fällen richtig. Der Aufwand zu Erstellung einer Phishing-E-Mail, ist durch die Automatisierung verschwindend gering. Lediglich die variierende Laufzeit der Anwendung muss beachtet werden. Diese ist abhängig von der gefundenen Information. Auf die Frage, wie glaubwürdig automatisierte Phishing-E-Mails mit integrierten personenbezogenen Daten sind, gibt es keine korrekte Antwort. Es ist möglich glaubwürdige Phishing-Mails mit allen Kriterien zu erstellen. Dennoch hängt die Glaubwürdigkeit davon ab, wie misstrauisch ein Opfer ist.

Die definierten Ziele in Kapitel \ref{sec:Zielsetzung} sind erfüllt. Die erstellte Suchfunktion bietet die Möglichkeit bekannte Daten über die Zielperson einzugeben. Diese Daten ermöglichen die Identifizierung der Person und das auslesen von bedeutender Information. Allerdings ist eine vollständig korrekte Identifizierung der Zielperson nicht möglich. E-Mail-Adressen werden aus den Webseiten herausgelesen. Falls keine übereinstimmende Adresse gefunden wird, generiert ein Algorithmus ein Pool an möglichen Adressen. Um zu Beweisen, dass die Phishing-Mail mit der entwickelten Anwendung versendet werden kann, wurde ein Zieladresse festgelegt. Der Inhalt einer Mail, wird abhängig von den gewonnen Informationen ausgewählt und mit den entsprechenden Daten ergänzt.

\section{Ausblick}
Es stellt sich die Frage, ob die Laufzeit der Anwendung, durch die Optimierung der Methode zur Erkennung von wichtigen Informationen, verbessert werden kann. Dafür wäre es denkbar, den Vergleich der Schlüsselwörter mit den Elementen der Wortsammlungen zu optimieren. Dazu werden die Datenbanken sortiert und die Schlüsselwörter mit einem angewendeten Suchalgorithmus verglichen. Des Weiteren könnte ein neuronales Netz trainiert werden. Als Trainingsdaten können die Wortsammlungen mit den entsprechenden Kategorien dienen. Das dabei entstehende Netz, würde beispielsweise eigenständig das Schlüsselwort "'Fußball"' aus dem Text herauslesen und in die Kategorie Hobby einordnen. 
%TODO Worstammlungen können ergänzt werden

Um die Personenidentifikation zu erweitern, können Bilderkennungen verwendet werden. Dadurch dienen gleiche Profilbilder auf unterschiedlichen Social-Media-Plattformen als weitere Identifikationsschlüssel. Für diese Methode eignet sich eine Bildererkennungssoftware oder die Google-Bildersuche. Eine weitere Möglichkeit die Identifizierung der gesuchten Person zu optimieren, kann das Beachten von Zeiträumen sein. Dabei wird erkannt, ob der Zeitrahmen des Artikels oder das Erstellungsdatum einer Webseite mit dem Alter der Person grundsätzlich übereinstimmt. Hierfür können Jahreszahlen und mögliche Metadaten der Webseite beziehungsweise der Domain ausgelesen werden.

Damit die Wahrscheinlichkeit erhöht wird, dass sich die korrekte E-Mail-Adresse in dem erzeugten Adresspool befindet, können weitere Adresse generiert werden. Als Ideengeber könnte hierfür das OSINT-Tool \cite{EmailAssumptions} dienen. Darüber hinaus können dem Adresspool mögliche Firmenadressen hinzugefügt werden. Dazu müsste allerdings die Institution der Zielperson bekannt sein. Der erzeugt Adresspool beinhaltet viele mögliche E-Mail-Adressen der gesuchten Person. Jedoch ist nicht jede Adresse korrekt. Aus diesem Grund können die generierten Adressen validiert werden. Möglichkeiten dafür sind bereitgestellte Webseiten oder ein eigenes Skript.

Aktuell wird die Phishing-E-Mail mit der Adresse des gefälschten GMX-Accounts versendet. Dadurch steht diese Adresse als Absender in der entsprechenden Mail. Um die Glaubwürdigkeit der Phishing-Mail zu steigern, kann die Absenderadresse verschleiert werden. Dies ist möglich, indem der E-Mail-Header verändert wird. Im Fall, dass Informationen über Kontakte der Zielperson gefunden werden, können diese Daten zur Generierung einer gefälschten Absenderadresse verwendet werden.

%%% Local Variables: 
%%% mode: latex
%%% TeX-master: "Bachelorarbeit"
%%% End: 
               % Schluss

%%% Anhang %%%          Nummerierung beginnend bei A
\appendix                       % Anhang
 \chapter{Ein Kapitel des Anhangs}
\label{cha:anhang}




%%% Local Variables: 
%%% mode: latex
%%% TeX-master: "Bachelorarbeit"
%%% End: 
               % Erstes Kapitel des Anhangs

%%% Verzeichnisse %%%
\begin{spacing}{1.0}            % Verzeichnisse werden mit einzeiligem Abstand gesetzt

% \listoffigures                   % Abbildungsverzeichnis (optional)
% \listoftables                    % Tabellenverzeichnis (optional)

%Glossar ausgeben
\printglossary[style=altlist,title=Glossar]

%Abk�rzungen ausgeben
\renewcommand{\acronymname}{Abkürzungsverzeichnis}
\printglossary[type=\acronymtype,style=long]

%Symbole ausgeben
\printglossary[type=symbolslist,style=long]

\bibliographystyle{geralpha}       % Look&Feel vom Literaturverzeichnis
\bibliography{bib}                 % Literaturverzeichnis
\addcontentsline {toc}{chapter}{Stichwortverzeichnis} % Stichwortverzeichnis soll im Inhaltsverzeichnis auftauchen (optional)
\printindex % Stichwortverzeichnis endgueltig anzeigen

\end{spacing}

\end{document}

%%% WICHTIG:
%% Um den Glossareintrag (Abkürzungsverzeichnis richtig darstellen zu können, muss makeindex mit dem Parameter "-s Bachelorarbeit.ist -t Bachelorarbeit.alg -o
%% Bachelorarbeit.acr Bachelorarbeit.acn" aufgerufen werden --> Einstellung im TeXnicCenter unter Ausgabe -> Ausgabeprofil definieren -> LaTeX=>PDF -> Nachbearbeitung
%% Unter Postprozessoren neuen Eintrag anlegen, z.B. Makeindex1. Unter Anwendung makeindex.exe auswählen. (c:\programme\miktex2.7\miktex\bin\makeindex.exe)
%% unter Argumente die obige Parameterzeile eintragen. Bei anderer TeX-Distri makeindex suchen. Unter Linux ein shell-script erstellen, das makeindex mit 
%% den Parametern aufruft.
%% makeindex erneut starten mit folgender Parameterzeile:  " -s Bachelorarbeit.ist -t Bachelorarbeit.glg -o Bachelorarbeit.gls Bachelorarbeit.glo"
%% Analog zu oben im TeXnicCenter einen weiteren Eintrag Makeindex2 erstellen. In Linux einen weiteren makeindex-Aufruf im Script hinzufügen.
%% makeindex erneut starten mit folgender Parameterzeile:  " -s Bachelorarbeit.ist -t Bachelorarbeit.slg -o Bachelorarbeit.syi Bachelorarbeit.syg"
%% Analog zu oben im TeXnicCenter einen weiteren Eintrag Makeindex3 erstellen. In Linux einen weiteren makeindex-Aufruf im Script hinzufügen.

%% Bachelorarbeit muss durch den Namen der Hauptdatei ausgetauscht werden. Hauptdatei unter Windows mindestens 5 Mal compilieren, dann betrachten

%% Local Variables:
%% mode: latex
%% TeX-master:
%% End:
