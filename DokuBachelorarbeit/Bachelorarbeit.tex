
%% Vorlage fuer Studien- und Bachelorarbeiten an der HS Ravensburg-Weingarten
%% Benoetigt wird KOMA-Skript ab Version 2.8j vom 30.07.2001
%% Vor der Veraenderung irgendwelcher Einstellungen wird dringend empfohlen
%% die Anleitung zum KOMA-Skript (scrguide) zu konsultieren !!
%%
%% Bilder bitte nicht mit Endung einbinden, so ist die Erzeugung von
%% DVI, PS und PDF problemlos m�glich!
%% J. Moehler, 2008-12-10 
%%Kommentar


%% Dokumentendefinition
\documentclass[
   12pt,                % Schriftgroesse 12pt
   a4paper,             % Layout fuer Din A4
   german,              % deutsche Sprache, global   
%  twoside,             % Layout fuer beidseitigen Druck
	 oneside,						  % Layout für einseitigen Druck
   headinclude,         % Kopfzeile wird Seiten-Layouts mit beruecksichtigt
   headsepline,         % horizontale Linie unter Kolumnentitel
   plainheadsepline,    % horizontale Linie auch beim plain-Style
   BCOR12mm,            % Korrektur fuer die Bindung
   DIV18,               % DIV-Wert fuer die Erstellung des Satzspiegels, siehe scrguide
   halfparskip,         % Absatzabstand statt Absatzeinzug
   openany,             % Kapitel können auf geraden und ungeraden Seiten beginnen
   bibtotoc,            % Literaturverz. wird ins Inhaltsverzeichnis eingetragen
   pointlessnumbers,    % Kapitelnummern immer ohne Punkt
   tablecaptionabove,   % korrekte Abstaende bei TabellenUEBERschriften
   fleqn,               % fleqn: Glgen links (statt mittig)
%   draft               % Keine Bilder in der Anzeige, overfull hboxes werden angezeigt !Auskommentieren für Test-Compile!
]{scrbook}[2001/07/30]  % scrbook-Version mind. 2.8j von 2001/07/30

%% Pakete für nützliche Dinge
\usepackage{color} 							 % Schriftfarben verwenden
\usepackage{ngerman}             % neue deutsche Rechtschreibung
%\usepackage[ngerman]{translator} % Übersetzung 
%\usepackage[ansinew]{inputenc}   % Input-Encodung: ansinew fuer Windows
\usepackage[utf8]{inputenc}   % Input-Encodung: latin1 fuer Unix
\usepackage[T1]{fontenc}         % T1-kodierte Schriften, korrekte Trennmuster fuer Worte mit Umlauten
\usepackage{ae}                  % Für PDF-Erstellung
\usepackage[hang]{caption2}      % mehrzeilige Captions ausrichten

\usepackage[centertags]{amsmath} % AMS-Mathematik, centertags zentriert Nummer bei split
\usepackage{latexsym}            % verschiedene Symbole
\usepackage{textcomp}            % verschiedene Symbole
 \usepackage{microtype}          % bessere Optik     
\usepackage{graphicx}            % zum Einbinden von Grafiken
\usepackage{float}               % u.a. genaue Plazierung von Gleitobjekten mit H
% \usepackage{pstricks}          % PostScript Macros
% \usepackage{lscape}            % Seite im Querformat bei Erhalt der Kopfzeile
% \usepackage{verbatim}          % Quellcode einbinden (\verbatiminput)
% \usepackage{multicol}          % Mehrspaltiger Text 
\usepackage{placeins}

%%% Literatur und sonstige Referenzen
\usepackage{cite}              % Sortierte und zusammengefasste Zitatnummern 
\usepackage{url}							 % URL in Literatur wird unterst�tzt
\usepackage{varioref}          % Verbesserte Referenzen
\usepackage{hyperref}          % Verlinkte Verzeichnisse
% \usepackage[numbers, sort]{natbib} % DIN Literaturverzeichnis; nicht zusammen mit cite verwenden!

%% Index
\usepackage{makeidx}						 % Index verwenden
\makeindex											 % index erstellen

%% Zeilenabstand
\usepackage{setspace}            % Zeilenabstand einstellbar
\onehalfspacing                  % eineinhalbzeilig einstellen

%% Kopf- und Fusszeilen
\usepackage{scrpage2}                             % Kopf und Fusszeilen-Layout 
\renewcommand{\headfont}{\normalfont\sffamily}    % Kolumnentitel serifenlos
\renewcommand{\pnumfont}{\normalfont\sffamily}    % Seitennummern serifenlos
\pagestyle{scrheadings}
\ihead[]{\headmark}               % Kolumnentitel immer oben innen
\chead[]{}                        % oben Mitte
\ohead[\pagemark]{\pagemark}      % Seitennummern immer oben aussen
\ofoot[]{}                        % Fusszeile aussen
\cfoot[]{}                        % Fusszeile Mitte
\ifoot[]{}                        % Fusszeile innen

%% Fussnotenzähler
\usepackage{chngcntr}              % Paket um Counter zu steuern
\counterwithout{footnote}{chapter} % Fussnoten nicht pro Chapter, sondern Global
\usepackage{threeparttable}        % Tabelle mit Fussnoten

%% Namen von Verzeichnissen definieren
\renewcommand{\bibname}{Literatur}               % Literaturverzeichnis wird zu Literatur
\renewcommand{\figurename}{Bild}                 % Abbildung wird zu Bild
\renewcommand{\listfigurename}{Bildverzeichnis}
\renewcommand{\indexname}{Stichwortverzeichnis}

%% Schrift mit Serifen auch fuer Ueberschriften benutzen
%\renewcommand*{\sectfont}{\bfseries}
%\renewcommand*{\descfont}{\bfseries}

\typearea[current]{current}        % Neuberechnung des Satzspiegels mit alten Werten nach Änderung von Zeilenabstand,etc

% -- Glossar --
\usepackage[toc, acronym]{glossaries} % Glossareinträge, muss nach hyperref (insofern dies verwendet wird) geladen werden
% (aufgrund der Seitenzahlverlinkung im Glossarverzeichnis), (benötigt die Packete
% "xkeyval" und "supertabular", welche dann automatisch eingebunden werden)
\newglossary[slg]{symbolslist}{syi}{syg}{Symbolverzeichnis} %Ein eigenes Symbolverzeichnis erstellen
\renewcommand*{\glspostdescription}{} %Den Punkt am Ende jeder Beschreibung deaktivieren
\makeglossaries % erstellt ein Glossar (Verzeichnis für Begriffserklärungen,
% z.B.: Abkürzungen)
% Glossareinträge (MUSS für JEDEN Glossareintrag überarbeitet werden):
%% Symbole:
%%\newglossaryentry{symb:Name}{name=Symbolname, description={Beschreibung}, sort=alphabetisches Wort für die Einreihung, type=symbolslist}
\newglossaryentry{symb:Pi}{
name=$\pi$,
description={Die Kreiszahl.},
sort=symbolpi, type=symbolslist
}
%%Abkürzungen:
%%\newacronym{Referenz}{Abkürzung}{Beschreibung}
\newacronym{BSP}{BSP}{Beispiel}


%%Eine Abkürzung mit Glossareintrag:
%%\newacronym{Referenz}{Abkürzung}{Beschreibung\protect\glsadd{glos:Referenz}}
\newacronym{AD}{AD}{Active Directory\protect\glsadd{glos:AD}}

%%Glossareintrag:
%%\newglossaryentry{glos:Referenz}{name=Name, description={Beschreibung}}
\newglossaryentry{glos:AD}{
name=Active Directory,
description={Active Directory ist in einem Windows Server 2000, Windows
Server 2003, oder Windows Server 2008-Netzwerk der Verzeichnisdienst, 
der die zentrale Organisation und Verwaltung aller Netzwerkressourcen erlaubt. Es
ermöglicht den Benutzern über eine einzige zentrale Anmeldung den
Zugriff auf alle Ressourcen und den Administratoren die zentral
organisierte Verwaltung, transparent von der Netzwerktopologie und
den eingesetzten Netzwerkprotokollen. Das dafür benötigte
Betriebssystem ist entweder Windows Server 2000, 
Windows Server 2003, oder Windows Server 2008, welches auf dem zentralen
Domänencontroller installiert wird. Dieser hält alle Daten des
Active Directory vor, wie z.B. Benutzernamen und
Kennwörter.}
}

\newglossaryentry{glos:Glossareintrag}{name=Glossareintrag, description={Erweiterte Informationen zum
einem Wort oder einer Abkürzung, ähnlich einem Eintrag im Duden.}}




%%% Local Variables: 
%%% mode: latex
%%% TeX-master: "Bachelorarbeit"
%%% End: 

%%% PDF-Erzeugung: pdflatex statt latex aufrufen!
%% BEI WINDOWS UND TEXNICCENTER AUSKOMMENTIERT LASSEN !!! 
%\pdfoutput=1                  % PDF-Ausgabe
%\usepackage[pdftex, a4paper,  % muss letztes Package sein!
%     pdftitle={Titel der Arbeit},%
%     pdfauthor={Name Autor},%
%     pdfsubject={Studien- bzw. Diplomarbeit},%
%     pdfkeywords={Stichwort zur Arbeit},%
%    ]{hyperref} % 



%\graphicspath{{figs/}{bilder/}}    % Falls texinput nicht gesetzt -> Bildverzeichnisse

%\includeonly{}


%%%%%%%%%%%%%%%%%%%%%%%%%%%%%%%%%%%%%%%%%%%%%%%%%%%%%%%%%%%%%%%
\begin{document}

\pagenumbering{Roman}           % Nummerierung Römisch Start bei I

%% Deckblatt fuer Studien- und Diplomarbeiten am der
%% Hochschule Weingarten

\thispagestyle{empty}
%~
{
\normalsize\fontfamily{phv} \fontsize{12pt}{10}\selectfont 
\vspace{-1cm}
\begin{minipage}[b]{9.4cm}
{\fontsize{13pt}{13} \selectfont%
Hochschule\\[1ex]
Ravensburg-Weingarten}\\[1ex]
\end{minipage}
}
\begin{minipage}[b]{10cm}
\includegraphics*[height=2.7cm]{bilder/HSLogoWGd}
\end{minipage}


\vspace{10mm}
 
\hrule 
\vspace{1cm}
{
\fontseries{b} \fontsize{20pt}{20}  \selectfont%
\begin{center}
\textcolor{red}{Titel der Arbeit} % Titel der Arbeit
\end{center}
}

\begin{center}
\large \textbf{Dokumentation zu Seminar} % Hier Praxisarbeit, Studienarbeit, Bachelorarbeit, Dokumentation zu Seminar, etc. eintragen
\end{center}

\begin{center}
\textbf{Multimedia Technik SS09} % Hier den Zusatz wie Fach oder Semester eintragen
\end{center}

\vspace{5mm}

\begin{center}
im Studiengang \textcolor{red}{Angewandte Informatik} % Hier den Studiengang eintragen
\end{center}

\begin{center}
an der Hochschule Ravensburg - Weingarten 
\end{center}
\begin{center}

\end{center}
\vspace{5mm}
\begin{center}
von
\end{center}




\begin{center}
{\fontsize{12pt}{12} \selectfont%
\begin{tabular}{ll}
Autor Name & \textcolor{red}{Matr.-Nr.: xxxxx}\\[0.5ex] % Hier den Autor und die Matrikelnummer statt xxxxx eintragen
%Autor 2 & \textcolor{red}{Matr.-Nr.: xxxxx}\\[0.5ex] % Für weitere Autoren Zeilen auskommentieren bzw. kopieren und ausfüllen
%Autor 3 & \textcolor{red}{Matr.-Nr.: xxxxx}\\[0.5ex]
Abgabedatum :& \today   % Das Abgabedatum wird gleichgesetzt mit dem Datum der letzten Compilierung. Statt \today kann auch Datum von Hand geschrieben werden
\end{tabular}
}
\end{center}
                               

\vspace{1cm}

\vspace{1cm}
\hrule


%%% Local Variables: 
%%% mode: latex
%%% TeX-master: "Bachelorarbeit"
%%% End: 

                % Deckblatt Nummerierung unterdrückt (In Deckblatt festgelegt)
%%\cleardoubleemptypage         % Die Eidesstattliche Erklaerung auf einer rechten Seite beginnen
%% Eidestattliche Erklärung %%
%% Die Erklärung sollte nach dem Deckblatt fest abgeheftet werden
\addchap*{Erklärung} %* nicht entfernen, sonst erhält Erklärung eine Nummer und erscheint im Inhaltsverzeichnis

\thispagestyle{empty} %Keine Seitenzahl, keine Kopf- und Fusszeile

Hiermit erkläre ich, dass ich die vorliegende Arbeit mit dem Titel   % ein Autor
% Hiermit erkl"aren wir, dass wir die vorliegende Arbeit mit dem Titel \newline % mehrere Autoren
\begin{center}
\textbf{Entwicklung einer Anwendung zur automatisierten Beschaffung von personenbezogenen Daten im Internet und deren Integration in Phishing-Mails}
\end{center}
selbstständig angefertigt, nicht anderweitig zu Prüfungszwecken vorgelegt, keine anderen als die angegebenen Hilfsmittel benutzt und wörtliche sowie sinngemäße Zitate als solche gekennzeichnet habe.\newline  % ein Autor

\begin{flushleft}
Weingarten, 29. April 2019 % Ort eintragen, /today kann durch Datum 2009-10-21 oder 21.10.2009 ersetzt werden
\end{flushleft}

%%% Unterschriftenblock für einen Autor
\begin{tabular}{l}   
Autor Name        \\% Hier Autor eintragen
 \\
------------------------------------ \\
\end{tabular}

%%% Unterschriftenblock für mehrere Autoren
%\begin{tabular}{lll}
%Autor 1       &Autor 2      &Autor 3 \\% Hier eintragen
% & & \\
%------------------------------------ & ------------------------------------ & ------------------------------------ \\
%\end{tabular}

%Hier unterschreiben


%%% Local Variables: 
%%% mode: latex
%%% TeX-master: "Bachelorarbeit"
%%% End: 
                 % Eidesstattliche Erklaerung Nummerierung unterdrückt
%%\cleardoubleemptypage         % Das Inhaltsverzeichnis auf einer rechten Seite beginnen

\pagenumbering{Roman}           % Nummerierung Römisch start bei I 

\begin{spacing}{1.0}            % Verzeichnisse werden mit einzeiligem Abstand gesetzt
 \tableofcontents               % Inhaltsverzeichnis
\end{spacing}
%%% Vorbemerkungen %%%  Nummerierung unterdrückt durch *
\addchap{Kurzfassung}
\label{cha:kurzfassung} 
Es wird gezeigt, wie eine automatisierte Suche nach personenbezogenen Daten im Internet aussehen kann und wie diese Daten für einen Phishing-Mail-Angriff verwendet werden können. 

Wird erweitert!

%%% Local Variables: 
%%% mode: latex
%%% TeX-master: "Bachelorarbeit"
%%% End: 


             % Kurzfassung der Arbeit
\addchap{Abstract}
\label{cha:abtract} 







%%% Local Variables: 
%%% mode: latex
%%% TeX-master: "Bachelorarbeit"
%%% End: 
             % Abstract der Arbeit (englische Kurzfassung der Arbeit)
\addchap{Danksagung}
\label{cha:danksagung}



%%% Local Variables: 
%%% mode: latex
%%% TeX-master: "Bachelorarbeit"
%%% End: 								  % Danksagung (optional)
\addchap{Vorwort}
\label{cha:vorwort}



%%% Local Variables: 
%%% mode: latex
%%% TeX-master: "Bachelorarbeit"
%%% End: 							    % Vorwort (optional)

%%% Hauptteil %%%       Nummerierung beginnend bei 1
%%\cleardoubleplainpage         % Das erste Kapitel des Hauptteils auf einer rechten Seite beginnen
\mainmatter                     % den Hauptteil beginnen
\chapter{Einleitung}
\label{cha:einleitung}

%%% Local Variables: 
%%% mode: latex
%%% TeX-master: "Bachelorarbeit"
%%% End: 

\section{Motivation}
\label {sec:Motivation}
In der heutigen Zeit wird das Thema Informationssicherheit immer wichtiger. Systeme werden immer komplexer und Firewalls immer besser.
Doch laut dem Bundeskriminalamt hat sich die Zahl der Cyberkriminalität mit einem klaren Trend nach oben entwickelt. \cite{Cyberkriminalitaet}\\
Eine häufig verwendetet Technik von Cyberkriminalität ist das E-Mail-Phishing. Hier wird der Mensch als Schwachstelle des Systems genutzt. In den neusten Fällen von Phishing-Attacken zeigt die Verbraucherzentrale Nordrhein-Westfalen, dass diese meist direkt an eine Person adressiert sind. Beispielsweise wird man in den gefälschten DSGVO-E-Mails, im Namen der Sparkasse, persönlich mit Namen angesprochen. \cite{VerbraucherzentraleNW} \\
Im Rahmen dieser Abschlussarbeit wird gezeigt, mit welchem Aufwand solche Angriffe verbunden sind und wie die Suche nach privaten Informationen im Internet aussieht.

\section{Problem}
Leser Problem komplett erklären, weiterführende Motivation

Persönliche Informationen werden im Internet immer leichter zugänglich gemacht. !!!ZITAT!!!\\
Es gibt viele Webseiten die persönliche Information von Menschen bereitstellt. Eine davon ist auch www.fupa.net. Hier können persönliche Informationen ohne Anmeldung ausgelesen werden. Diese Art von Webseite ist eine perfekte Informationsquelle für Angreifer.\\
Im Bereich von Social Engineering Angriffen wird diese Information oft genutzt um ein Opfer zu manipulieren.
Das hier beschriebene Problem zeigt dass der Zugang für persönliche Information durch das Internet für viele Menschen einfacher gemacht wird. Es soll gezeigt werden wie einfach es ist, personenbezogene Daten aus dem Internet herauszulesen, analysieren und für einen Phishing-Angriffe zu verwenden.
!!!!ZITATE HINZUFÜGEN!!! statista z.B.

\section{Eigene Leistung}
\label {sec:Leistung} 
In dieser Arbeit wird ein Phishing-Mailgenerator erstellt. Dieser liest automatisiert Informationen von der Webseite www.fupa.net heraus und erstellt potentiellen Opferprofile. Zusätzlich wird mit dieser Information und einem Web-Crawler das Internet nach weiteren Informationen durchstöbert. Mit dem Vornamen, Nachnamen und dem Geburtsjahr werden die E-Mail-Adressen generiert. Die gefundenen Informationen werde automatisch in eine personalisierte Phishing-E-Mail eingebaut. Für einen höheren Erfolg werden E-Mail-Muster erstellt.

\section{Aufbau der Arbeit}
\label {sec:Aufbau} 
Meine Arbeit gliedert sich in zwei Teile. Einem theoretischen und einem praktischen Teil. Der Theorie-Teil beginnt im zweiten Kapitel und beschreibt die Grundbegriffe im Bereich Social Engineering, Webtools, E-Mails und Programmiersprachen. Im nächsten Kapitel befindet sich die Anforderungsanalyse. Hier werden die Anforderungen an die Arbeit festgelegt. Darauf folgen die Lösungsvorschläge im Kapitel vier und die ausgewählte Lösung anhand den Anforderungen im Kapitel 5. Im Anschluss wird bei der Umsetzung auf den Praktischen Teil eingegangen.Am Ende befindet sich das Fazit, der Ausblick und der Anhang.






               % Einleitung
%Kapitel des Hauptteils

\chapter {Grundlagen}  %Name des Kapitels
\label{cha:grundlagen} %Label des Kapitels

\section{Personenbezogene Daten}
%TODO möglicherweise unterschied von Information und Daten erklären
Laut dem DSGVO sind \textit{personenbezogene Daten} 

\textit{"'alle Informationen, die sich auf eine identifizierte oder identifizierbare natürliche Person (im Folgenden „betroffene Person“) beziehen; als identifizierbar wird eine natürliche Person angesehen, die direkt oder indirekt, insbesondere mittels Zuordnung zu einer Kennung wie einem Namen, zu einer Kennnummer, zu Standortdaten, zu einer Online-Kennung oder zu einem oder mehreren besonderen Merkmalen identifiziert werden kann, die Ausdruck der physischen, physiologischen, genetischen, psychischen, wirtschaftlichen, kulturellen oder sozialen Identität dieser natürlichen Person sind;"'}\cite{personenbezogeneDaten}

\section{Social Engineering} %Unterkapitel
\label {sec:Social Engineering} %Label des Unterkapitels
	Die Definition von Social Engineering (SE) ist nicht eindeutig. Es gibt sehr verschiedene Ansichten von der Definition. Die Idee von Social Engineering ist, eine Ziel so zu manipulieren, damit das Ziel eine für den Angreifer bessere Entscheidung trifft. In dem Buch Social Engineering - The Art of Human Hacking, von Christopher Hadnagy, ist Social Engineering definiert als:
	
	\textit{"'social engineering is the act of manipulating a person to take an action that may or may not be in the "'target’s"' best interest"'}\cite{ArtOfHumanHacking}
	
	Wiederum lautet die Definition in dem Buch von Kevin D. Mitnick:
	
	\textit{"'Social Engineering uses influence and persuasion to deceive people by convincing them that the social engineer is someone he is not, or by manipulation. As a result, the social engineer is able to take advantage of people to obtain information with or without the use of technology"'}\cite{ArtOfDeception}
	
	SE wird Menschen von Geburt an beigebracht und begegnet einem beinahe jeden Tag. Schon ein Baby muss wissen wie es die Eltern manipulieren kann damit es Dinge wie Essen, Zuneigung, o.ä. bekommt. Darüber hinaus ist SE in vielen Berufen ein täglicher Bestandteil. Beispielsweise manipulieren Ärzte viele Patienten mit einer Placebo-Behandlung. Bei dieser Behandlung wird dem Patient ein wirkstoff-freies Medikament verschrieben. Ausschließlich durch die Manipulation des Patienten und den sogenannten Palzebo-Effekt können Erfolge erzielt werden.
	
	Im Bereich der Informationssicherheit, wird von Social Engineering gesprochen, wenn Angreifer durch die Manipulierung und Täuschung von Menschen vertrauliche Informationen oder Zugänge zu Systemen bekommen. Die bekanntesten Angriffsmethoden sind Phishing, Pretexting, Baiting und Quad Pro Quo. Bei dieser Arbeit wird hauptsächlich auf das Thema E-Mail-Phishing eingegangen.

	Der Aufbau eines SE-Angriffes ist definiert in mehrere Phasen. Das wohl bekannteste Modell für einen Social Engineering-Angriffszyklus ist in dem Buch von Kevin D. Mitnicks \cite{ArtOfDeception} definiert. Dieser Zyklus besteht aus den 4 Phasen \textit{Research, Developing rapport and trust, Exploiting trust} und \textit{Utilize information}.\\
	In der \textit{Research-Phase} geht es um die Informationsbeschaffung. Bei dieser Phase will der Angreifer möglichst viel Informationen über das Ziel herausfinden. Die \textit{Developing Rapport and Trust-Phase} beschreibt den Kontaktaufbau zum Ziel, da wenn das Opfer dem Angreifer vertraut, hat dieser ein leichteres Spiel in den kommenden Phasen. Das nun erzeugte Vertrauen wird in der \textit{Exploitung Trust-Phase} ausgenutzt. Hier will der Angreifer die eigentlich Information vom Opfer herausfinden. Dies geschieht einerseits durch bestimmtes Nachfragen oder Manipulation.
	\textit{Utilize Information} ist die letzte Phase. Dort wird die gewonnene Information genutzt um das eigentliche Ziel des Angreifers zu erreichen.\\
	Grundsätzlich werden bei einem Social Engineering Angriff menschliche Wünsche, Ängste und verbreitete Verhaltensmuster verwendet um ein Opfer zu manipulieren.\cite{LeitfadenSE}\\

		\subsection{Phishing}
		Das Wort Phishing wird von dem Wort "'fishing"' abgeleitet, da die Angreifer nach Informationen fischen. Das "'Ph"' kommt von "'sophisticated"' und meint damit, dass die Angreifer ausgeklügelte Techniken verwenden um an Informationen heranzukommen.\cite{PhishingExposed}\\
		Die wohl bekannteste Angriffsmethode von Phishing ist das E-Mail-Phishing. Bei diesem Verfahren, versendet ein Angreifer meist eine gefälschte E-Mail, um ein Opfer zu täuschen und dadurch sein Ziel zu erreichen. Die sogenannten Phishing-Mails enthalten meist eine Aufforderung einen Link zu öffnen und sehen täuschend echt aus. Zum Beispiel könnte der Angreifer ein Layout von Amazon verwenden und das Ziel auffordern, den Link zu öffnen wegen einem Authentifizierungsproblem. Nachdem Sie auf den Link geklickt haben müssen Sie sich anmelden. Hier könnten die Angreifer Ihre Anmeldedaten abgreifen, nachdem sie das Opfer eingeben hat. Sobald die Anmeldedaten eingegeben wurden, könnte eine Fehlermeldung erscheinen, die sagt: "'Hoppla, ein Fehler ist aufgetreten, melden Sie sich bitte neu an!"'. Anschließend wird die originale Seite geladen, das Opfer kann sich korrekt anmelden und der Angreifer hat ohne einen großen Aufwand die Anmeldedaten der Zielperson.\\
		Für diese Methode benötigt der Angreifer nicht nur Social Engineering  sondern auch technische Fähigkeiten.\cite{PhishingDarkWaters}
		
		\subsubsection{Spear-Phishing}
		Das Spear-Phishing ist eine erweiterte Methode des herkömmlichen E-Mail-Phishings. Hierbei wird anstatt das Versenden etlicher Phishing-Mails an unbekannte Opfer, eine gezielte Mail an eine ausgewählte Person versendet.\cite{SpearPhishingPaper}\\
		Bei dieser Form von E-Mail-Phishing spielt die Opferauswahl und die Informationsbeschaffung eine wichtig Rolle, da diese Information später für personalisierte E-Mails oder vorgetäuschte Identitäten verwendet werden können. Durch diese Art von Täuschung kann ein Opfer dazu bewegt werden auf einen Link zu klicken und dadurch eine Schadsoftware herunterzuladen.\cite{SpearPhishingPaper} \\
		Der Aufwand für die Informationsbeschaffung wird oft in Kauf genommen, da der Erfolg bei dieser Methode vielversprechender ist als beim herkömmlichen E-Mail-Phishing.\\
		91\% der Advanced Persistent Threat (APT) Angriffe auf Firmen beginnen mit einer Spear-Phishing-E-Mail. Die Schadsoftware wir meisten als Remote Access Trojans (RATs) in einem Zip-Datei überliefert.\cite{SpearPhishing}


\section{Informationsbeschaffung im Internet}
%TODO Entweder Web Crawling und Web Scraping oder Web Crawler und Web Scraper
	\subsection{Web Crawler}
		Web Crawler sind Computerprogramme, die mit Hilfe der Hypertextstruktur das Internet durchlaufen. Dabei können sie in einen \textbf{internen} und \textbf{externen Web Crawler} unterschieden werden. Der interne Web Crawler durchsucht ausschließliche interne Seiten einer Webseite und der externe Web Crawler durchsucht unbekannte Webseiten im ganzen Netz. \cite{sharma2012study}

		In anderen Worten besteht die Funktionsweise darin, dass in den meisten Fällen ein automatisiertes Programm Web Crawler erstellt wird. Dieser lädt Webinhalte herunter und durchsucht den Inhalt nach Hyperlinks. Den gefundenen Links wird gefolgt, um neue Webseiten mit weiteren Links zu laden. So hangelt sich ein Web Crawler von Link zu Link durch das Internet.\cite{WebScraping}
		
	\subsection{Web Scraping}
		In der Theorie bedeutet Web Scraping die Informationsbeschaffung im Internet mit unterschiedlichsten Mitteln. \cite{WebScraping}\\		
		Meist wird dies mit einem automatisierten Programm realisiert, welches Daten von einem Webserver anfragt, entgegen nimmt, analysiert und auswertet. 
		In der Praxis gibt es ein großes Feld von Programmiertechniken und Einsatzmöglichkeiten.
		Mit Hilfe von Web Scraping ist es möglich große Datenmengen zu erfassen und zu verarbeiten.\cite{WebScraping}

		\subsubsection{Natural Language Processing}
			\textit{Natural Language Processing} kurz NLP und beschreibt einen Technologie, für die Kommunikation zwischen Mensch und Computer. Mit dem Ziel, dass ein Computer die natürliche Sprache verstehen und verarbeiten kann. Dafür werden verschiedenste Methoden aus der Sprach- und Computerwissenschaft sowie aus der künstliche Intelligenz verwendet. Unter anderem hat eine NLP-Anwendung die Aufgabe von \textit{Stemming}.\cite{NLP} 
	
			\textit{Stemming} ist eine Methode der Wortstandardisierung, bei der verwandte Wörter auf ihrer Stammform reduziert werden. Dabei wird bei dem Rechenvorgang auf den Stamm und die Semantik eines Wortes geachtet. Aus diesem Grund fällt der Name Stammformreduktion öfters in Verbindung von Stemming.\cite{eldesouki2009stemming}\\
			Die Verwendung von Stemming, kann bei der Schlüsselwortgenerierung von Texten sehr hilfreich sein, da die Anzahl der möglichen Schlüsselwörter reduziert werden können.


%%% Local Variables: 
%%% mode: latex
%%% TeX-master: "Bachelorarbeit"
%%% End: 
                % Ein Kapitel des Hauptteils
%Kapitel des Hauptteils

\chapter{Anforderungsanalyse und Priorisierung}  %Name des Kapitels
\label{cha:Anforderungsanalyse und Prioriesierung} %Label des Kapitels
Die im Kapitel \ref{sec:Zielsetzung} definierten Ziele sollen mit den folgenden Anforderungen gewährleistet werden.

%TODO Weitere Vorteile von Python herausfinden 
\section{Anforderung an die Informationsbeschaffung}
Die Anforderung an die Informationsbeschaffung von personenbezogenen Daten lässt sich in zwei Teile gliedern. Der erste Teil beinhaltet die Informationsbeschaffung von ausgewählten Personen und der zweite Teil die Informationsbeschaffung von einer großen Menge unbekannter Personen.
	
	\subsection{Informationsbeschaffung von einer ausgewählten Person}
	Bei dieser Informationsbeschaffung soll eine Suchfunktion entwickelt werden, welche Daten zu einer angegeben Person im Internet sucht. Hierbei sollen so viele Daten wie möglich gefunden und gespeichert werden.\\
	Das zu entwickelnde Programm soll für die Suche bekannte Daten wie Vorname, Nachname, Geburtsjahr, Ort und Benutzernamen von Social Media Plattformen einlesen können. Die Eingabe kann mit Hilfe einer Konsole oder einer grafische Oberfläche realisiert werden.\\
	Die Herausforderung besteht darin, zu erkennen, wann und ob es sich um die Information der gesuchten Person handelt. Sowie die Analyse und das Herauslesen dieser Daten.
	
	\subsection{Informationsbeschaffung von unbekannten Personen}
	Es soll eine Prototyp-Suchfunktion entwickelt werden, die eine komplette Website nach personenbezogenen Daten durchsucht. Dabei sollen möglichst viele Informationen von möglichst vielen Personen herausgefunden werden. Jedoch sind diese Personen dem Programm-Anwender unbekannt. Die Informationen wird von einer festgesetzten Webseite herausgelesen, welche eine großen Anzahl von Mitgliedern haben muss.\\
	Dabei soll der zu entwickelnde Web Scraper möglichst performant arbeiten und kann hartkodiert werden.
		
\section{Anforderung an die Datenverwaltung/-speicherung}
Ausgelesene Daten sollen vor dem speichern formatiert und klassifiziert werden, damit die Daten später korrekt in die Phishing-Mails eingesetzt werden können. Die Schwierigkeit besteht darin, zu erkennen, um welche Art von Information es sich handelt. Zusätzlich sollen die Daten in einer gut übersichtlichen Struktur gespeichert werden und müssen beliebig erweiterbar sein.
	
\section{Anforderung an die Generierung der E-Mail-Adressen}
Da nicht zu jeder Suche eine E-Mail-Adresse im Internet gefunden werden kann, muss die E-Mail-Adresse aus den vorhandenen Informationen generiert werden. Es soll eine größere Anzahl von möglichen E-Mail-Adressen erzeugt werden. Durch den Pool an erzeugten E-Mail-Adressen soll die Wahrscheinlichkeit erhöht werden, dass die richtige E-Mail-Adresse dabei ist. Des Weiteren sollen die Adresse auf Verfügbarkeit und Gültigkeit geprüft werden.
	
\section{Anforderung an die E-Mail-Muster}
Bei der Erstellung der E-Mail-Muster handelt es sich ausschließlich um das Erstellen potentieller Inhalte einer E-Mail, welche mit den gewonnenen Informationen über eine Person erweitert werden kann. Die Muster sollen erstellt werden und so klassifiziert sein, dass für jedes gefundene Opferprofil ein passendes Muster vorhanden ist. Des Weiteren soll der E-Mail-Text mit den eingesetzten Informationen Sinn ergeben und eine korrekte Grammatik beinhalten. Weiterführend können SE-Fähigkeiten genutzt werden um die Zielperson tatsächlich zu manipulieren und zu täuschen. Hierfür können beispielsweise Gefühle wie Freude und Angst ausgenützt oder gefälschte E-Mails von bekannten Firmen in Betracht gezogen werden.
	
\section{Anforderung an die Erstellung der Phishing-Mail}
Die Phishing-Mails sollen automatisiert erstellt werden. Die Auswahl des richtigen E-Mail-Musters zu der gewonnenen Opferinformation soll ebenfalls automatisiert ablaufen.

\section{Weitere Anforderungen}
Unter anderem soll die Arbeit Antworten auf die folgenden Fragen finden. Mit welchem Aufwand ist eine Phishing-Mail-Angriff verbunden? Ist es möglich ein Personenprofil zu erstellen, bei dem ausschließlich korrekte Informationen vorhanden sind?
%TODO in Zielsetzung
\FloatBarrier

\section{Priorisierung} %Unterkapitel
\label{sec:} %Label des Unterkapitels
Die Tabelle \ref{tab:prio} zeigt die Priorisierung der Anforderungen. Dabei liegt der eindeutige Fokus auf der Informationsbeschaffung von personenbezogenen Daten und der Erstellung von E-Mail-Mustern.
%TODO Begründungen für Prioritäten

\begin{table}
	\caption{Priorisierung der Anforderungen}
	\label{tab:prio}
	\begin{center} 
		\begin{tabular}{|l|l|}
			\hline
			Anforderung & Priorisierung (A-C) \\
			\hline
			Informationsbeschaffung von ausgewählten Personen & $ A $ \\
			\hline
			Informationsbeschaffung von vielen ubekannten Personen & $ A $ \\
			\hline
			E-Mail-Muster erstellen & $ A $		\\
			\hline
			Phsishing-Mail erzeugen & $ B $ 	\\
			\hline
			Datenverwaltung/-speicherung & $ B $   \\
			\hline
		\end{tabular}
	\end{center}
\end{table}
		% Ein weiteres Kapitel des Hauptteils
%Kapitel des Hauptteils

\chapter{Lösungsideen}  %Name des Kapitels
\label{cha:Lösungsideen} %Label des Kapitels
Für die Umsetzung der im Kapitel \ref{sec:Zielsetzung} definierten Ziele, werden folgende Lösungsideen vorgeschlagen.

\section{Programmiersprache}
Python
%TODO Warum Python? Antworten finden

\section{Informationsbeschaffung}
Für die Eingabe von Suchdaten, besteht für beide Informationsbeschaffungen die Möglichkeit eine Grafische-Bedienoberfläche oder eine Konsolen-Eingabe zu verwenden.
	\subsection{Informationsbeschaffung von bestimmten/ausgewählten Personen}
		
		\subsubsection{Suche nach Informationen}
		\label{sec:Suche nach Information}
		{\bf Idee 1} \textit{Die Art der Suche wird anhand den eingegebenen Daten angepasst.}\\\\
		Abhängig von der Anzahl und Art der Daten die vom Programm-Anwender eingeben wurden, wird die Art beziehungsweise die Reihenfolge der Suche variiert. Die nachfolgenden Fälle sollen diesen Ansatz verdeutlichen.\\\\
		\textit{Fall 1: Vorname, Nachname, Ort wird eingeben:}\\
		In diesem Fall wird mit Hilfe der Suchmaschine von Google nach Information gesucht. Die von Google vorgeschlagenen Seiten werden analysiert, interpretiert und gespeichert. Dadurch können weitere Informationen gewonnen werden. Wenn Benutzernamen von anderen Webseiten wie Instagram, Facebook oder ähnliches vorgeschlagen wird, kann somit die Suche speziell auf der entsprechenden Seite erweitert werden.\\\\
		\textit{Fall 2: Benutzername einer Webseite(Facebook,Instagram,usw.) wird eingeben:}\\
		Hier kann zuallererst auf der entsprechenden Webseite nach Informationen zu dem angegebenen Benutzername gesucht werden. Möglicherweise werden dadurch zusätzliche Daten herausgefunden, die bei der weiteren Suche von Vorteil wären.\\
		Nachdem die Webseite nach dem Nutzernamen durchsucht und ausgewertet wurde, kann nun mit herkömmlichen Suchmaschinen die Suche erweitert werden.\\\\
		{\bf Idee 2} \textit{Es wird unabhängig von den eingegebenen Daten direkt mit einer Suchmaschine nach Informationen gesucht.}\\\\
		Man verwendet ausschließlich die herkömmlichen Suchmaschinen und geht anhand den vorgeschlagenen Links auf die Suche nach Informationen.\\\\
		{\bf Idee 3} \textit{Es wird nur auf ausgewählten Webseiten nach Informationen gesucht}\\\\
		Verschiedenste Webseiten durchsuchen. Ideen dafür sind Facebook, FuPa, Instagram, Xing, LinkedIn, Google und Twitter.\\
		
		\subsubsection{Wann handelt es sich um die gleiche Person?}
		Bei jeder einzelnen Suchvariante, besteht die Herausforderung darin, zu erkennen, wann es sich um die gesuchte Person handelt. Für diese Problematik werden folgende Lösungsideen vorgeschlagen.\\
		
		{\bf Lösungsidee 1} \textit{Die Art der Suche wird anhand den eingegebenen Daten angepasst.}\\
		Diese Lösung entspricht dem Ansatz 1 von Kapitel \ref{sec:Suche nach Information}. Die Suche kann dadurch verfeinert werden und die Anzahl der fehlerhaften Vorschläge wird geringer. Dadurch wird die Wahrscheinlichkeit höher, dass es sich um die richtige Person handelt.\\\\
		{\bf Lösungsidee 2} \textit{Bei keiner perfekten Übereinstimmung wird die Suche erweitert}\\
		Hier kann das Such erweitert werden, indem auf Verbindungen der Zielperson eingegangen wird. Das heißt bekannte Facebook-Freunden, FuPa-Teammitglieder, oder Xing-Arbeitskollegen gesucht können ebenfalls durchsucht werden.\\
		
		{\bf Lösungsidee 3} \textit{Profilbilder können verglichen werden}\\ Entweder mit Bilderkennungssoftware oder Googl-Bildersuche\\
		
		{\bf Lösungsidee 4} \textit{Die Personen-Suche mit Hilfe von korrekten Suchbefehlen verfeinern.}\\
		In dem Buch "'Open Source Intelligence Techniques"' \cite{Bazzell}, werden Suchbefehle aufgezeigt, mit denen die Suche mit bekannter Suchmaschinen verbessert und verfeinert werden kann. Auch bei dieser Lösungsidee wird die Wahrscheinlichkeit erhöht, dass es sich um die gesuchte Person handelt. %TODO Buchzitat in bib einfügen
		
	\subsection{Informationsbeschaffung von einer großen Menge unbestimmter Personen}
	Webseiten mit großen Menge von Daten, ausgenommen von den bekannten Social Media Seiten, sind das Fußballportal FuPa, Xing und LinkedIn.
	
\section{Analyse und Speicherung der Information}
	\subsection{Textanalyse}
	Die Textanalyse wird nur bei der Suche einer bestimmten Person benötigt. Die  zweite Suchfunktion wird hartkodiert und benötigt dadurch keine Textanalyse, da der Aufbau der Webseite im voraus bekannt ist. Das bedeutet, dass das Programm genau weiß wo welche Information auf der Webseite steht. Beispielsweise befindet das Geburtsjahr bei der Seite von dem Fußballportal FuPa immer an der gleichen Position einer Tabelle.\\
	Dies ist bei der ersten Suchfunktion allerdings nicht möglich. Es können verschiedenste Arten von Webseiten vorgeschlagen werden. Aus diesem Grund muss das Programm "'clever"' sein und die wichtigen Daten aus der Seite herausfiltern können. Dazu gibt es folgende Lösungsideen.\\
	
	{\bf Idee 1} \textit{Textanalyse mit Hilfe von Python NTLK.}\\
	Mit dem\textit{Natural Language Toolkit} ist es möglich, den vorhandenen Webseitentext zu analysieren. Zu beginn können sogenannte "'stopwords"' aus dem vorgegebenen Text herausgefiltert werden. Stopwords sind Wörter die sehr oft auftreten und keinen großen Informationsgewinn mit sich bringen. Beispiele dafür sind ist, ein, einer, usw. Dadurch verringert sich die Anzahl der gesamten Wörter im Text. Anschließend können Funktionen wie das Zählen des Vorkommens einzelner Wörter angewendet werden, um einen Überblick von dem Text zu bekommen. Des Weiteren kann der Text in Fragmente zerlegt werden um weitere Informationen über den Inhalt zu erlangen. Abschließend können die analysierten Wörter bzw. Fragmente nach Schlüsselbegriffen durchsucht werden.\\
	
	{\bf Idee 2} \textit{Textanalyse indem nach Schlüsselwörtern gesucht wird.}\\
	Es kann ein Algorithmus entwickelt werden, der nach Schlüsselwörtern in einer Webseite sucht. Hierfür wäre es denkbar, Datenbanken bzw Wortsammlungen zu erstellen, die die zu suchenden Schlüsselwörter beinhalten. Mit diesen Datenbanken kann nun die Webseite nach den Schlüsselwörter durchsucht,analysiert und interpretiert werden. Die Datenbanken werden mit Hilfe von bekannter Listen im Internet befüllt. Beispiele hierfür wären eine aktuelle Liste aller Hochschulen in Deutschland, Berufsbezeichnungen, Studiengänge, Hobbys, Städte, etc...\\
	
	{\bf Idee 3} \textit{Textanalyse mit Hilfe Machine Learning}\\
	Möglicherweise irgendeine Python bib.
	
	Scikit\\
	
	{\bf Idee 4} \textit{Mit Hilfe von NLTK Rake den Text interpretieren}\\
	Rake hat die Aufgabe, einen Text mit vielen Wörten auf eine geringe Anzahl von Schlüsselwörter zu reduzieren und dadurch kann möglicherweise der Inhalt des Textes erkannt werden ohne ihn komplett gelesen zu haben.
	
	\subsection{Speicherung der Information}
	Für die Suche einzelner Person, kann ein erweiterbares Personen-Objekt erstellt werden.\\
	Für die Informationsbeschaffung von vielen unbekannten Personen, könnte eine SQL-Datenbank erstellt werden. Ein weiterer Idee wäre, eine Datei anzulegen, bei der alle Personen gut strukturiert gespeichert werden können. Möglichkeiten dafür wären die Dateiformate CSV und TXT.%TODO ist TXT Dateiformat wirklich eine gute Möglichkeit

\section{Generierung der E-Mail-Adressen}
Es kann das opensource tool von intelligencetechniques mit Hilfe eines automatisierten Webbrowsers verwendet werden. Algorithmus entwickeln, der alle möglichen Mail-Adressen aus den Daten Vorname, Nachname, Geburtsjahr und den bekanntesten Mail-Providern erzeugt.

\section{Erstellung der E-Mail-Muster}
Die Muster können in zwei große Kategorien unterteilt werden. Es gibt einen privaten und geschäftlichen Teil. Der private Teil hat weiter Unterteilungen wie Familie, Hobby/Interessen.
\section{Erzeugung der Phishing-Mail}
		% Die unterschiedlichen Kapitel
%Kapitel des Hauptteils

\chapter{Lösungsvorschläge}  %Name des Kapitels
\label{cha:} %Label des Kapitels
\section{Informationsbeschaffung} %Unterkapitel

\begin{itemize}
	\item Es soll durch Web Scraping die Internetseite www.fupa.net durchsucht werden. Persönliche Spielerinformationen werden ausgelesen und gespeichert.
	\begin{itemize}
		\item Vorname, Nachname, Spitzname, Geburtsjahr, Verein (must)
		\item Bild (could)
	\end{itemize}
	\item Es soll ein Web Crawler erstellt werden der mit der vorhandenen Information nach weiteren Informationen im Internet sucht.
\end{itemize}
\section{Informationsbeschaffung}
Das Thema Informationsbeschaffung von personenbezogenen Daten lässt sich in zwei Teile gliedern. Erstens in die Informationsbeschaffung von bestimmten bzw. ausgewählten Personen und zweitens die Informationsbeschaffung von vielen unbestimmten Personen.
\subsection{Informationsbeschaffung von bestimmten/ausgewählten Personen}
\subsection{Informationsbeschaffung von unbestimmten Personen}
\section{Datenverwaltung/-speicherung}
\section{Phishing-Mail Erzeugung}		% des Hauptteils
%Kapitel des Hauptteils

\chapter{Umsetzung}  %Name des Kapitels
\label{cha:} %Label des Kapitels

		


%%% Local Variables: 
%%% mode: latex
%%% TeX-master: "Bachelorarbeit"
%%% End: 
		% sollten hier in der
%Kapitel des Hauptteils

\chapter{}  %Name des Kapitels
\label{cha:} %Label des Kapitels
\section{} %Unterkapitel
\label{sec:} %Label des Unterkapitels
\subsection{} %Unterunterkapitel
\label{sse:}
\subsubsection{} %Unterkapitel 3. Ordnung
\label{sss:}
%%% Local Variables: 
%%% mode: latex
%%% TeX-master: "Bachelorarbeit"
%%% End: 
		% Masterdatei einen sinnvollen
%Kapitel des Hauptteils

\chapter{}  %Name des Kapitels
\label{cha:} %Label des Kapitels
\section{} %Unterkapitel
\label{sec:} %Label des Unterkapitels
\subsection{} %Unterunterkapitel
\label{sse:}
\subsubsection{} %Unterkapitel 3. Ordnung
\label{sss:}
%%% Local Variables: 
%%% mode: latex
%%% TeX-master: "Bachelorarbeit"
%%% End: 
		% Kommentar erhalten
%Kapitel des Hauptteils

\chapter{}  %Name des Kapitels
\label{cha:} %Label des Kapitels
\section{} %Unterkapitel
\label{sec:} %Label des Unterkapitels
\subsection{} %Unterunterkapitel
\label{sse:}
\subsubsection{} %Unterkapitel 3. Ordnung
\label{sss:}
%%% Local Variables: 
%%% mode: latex
%%% TeX-master: "Bachelorarbeit"
%%% End: 
		% z.B. den Titel des Kapitels
%Kapitel des Hauptteils

\chapter{}  %Name des Kapitels
\label{cha:} %Label des Kapitels
\section{} %Unterkapitel
\label{sec:} %Label des Unterkapitels
\subsection{} %Unterunterkapitel
\label{sse:}
\subsubsection{} %Unterkapitel 3. Ordnung
\label{sss:}
%%% Local Variables: 
%%% mode: latex
%%% TeX-master: "Bachelorarbeit"
%%% End: 
		% um für Korrekturen oder Umstellungen
%Kapitel des Hauptteils

\chapter{}  %Name des Kapitels
\label{cha:} %Label des Kapitels
\section{} %Unterkapitel
\label{sec:} %Label des Unterkapitels
\subsection{} %Unterunterkapitel
\label{sse:}
\subsubsection{} %Unterkapitel 3. Ordnung
\label{sss:}
%%% Local Variables: 
%%% mode: latex
%%% TeX-master: "Bachelorarbeit"
%%% End: 
		% leichter gefunden zu werden
\chapter{Fazit und Ausblick}
\label{chap:SchlussUndAusblick}
\section{Fazit}
Das Ergebnis der Testfälle ist überwiegend positiv. Dennoch sind nur zwei der vier Testergebnisse vollständig korrekt. Demnach ist die Antwort auf die Forschungsfrage, ob es möglich ist, ausschließlich korrekte Opferprofile zu erstellen, nein. Dennoch ist die Mehrzahl der Personenattribute bei den meisten fällen richtig. Der Aufwand zu Erstellung einer Phishing-E-Mail, ist durch die Automatisierung verschwindend gering. Lediglich die variierende Laufzeit der Anwendung muss beachtet werden. Diese ist abhängig von der gefundenen Information. Auf die Frage, wie glaubwürdig automatisierte Phishing-E-Mails mit integrierten personenbezogenen Daten sind, gibt es keine korrekte Antwort. Es ist möglich glaubwürdige Phishing-Mails mit allen Kriterien zu erstellen. Dennoch hängt die Glaubwürdigkeit davon ab, wie misstrauisch ein Opfer ist.

Die definierten Ziele in Kapitel \ref{sec:Zielsetzung} sind erfüllt. Die erstellte Suchfunktion bietet die Möglichkeit bekannte Daten über die Zielperson einzugeben. Diese Daten ermöglichen die Identifizierung der Person und das auslesen von bedeutender Information. Allerdings ist eine vollständig korrekte Identifizierung der Zielperson nicht möglich. E-Mail-Adressen werden aus den Webseiten herausgelesen. Falls keine übereinstimmende Adresse gefunden wird, generiert ein Algorithmus ein Pool an möglichen Adressen. Um zu Beweisen, dass die Phishing-Mail mit der entwickelten Anwendung versendet werden kann, wurde ein Zieladresse festgelegt. Der Inhalt einer Mail, wird abhängig von den gewonnen Informationen ausgewählt und mit den entsprechenden Daten ergänzt.

\section{Ausblick}
Es stellt sich die Frage, ob die Laufzeit der Anwendung, durch die Optimierung der Methode zur Erkennung von wichtigen Informationen, verbessert werden kann. Dafür wäre es denkbar, den Vergleich der Schlüsselwörter mit den Elementen der Wortsammlungen zu optimieren. Dazu werden die Datenbanken sortiert und die Schlüsselwörter mit einem angewendeten Suchalgorithmus verglichen. Des Weiteren könnte ein neuronales Netz trainiert werden. Als Trainingsdaten können die Wortsammlungen mit den entsprechenden Kategorien dienen. Das dabei entstehende Netz, würde beispielsweise eigenständig das Schlüsselwort "'Fußball"' aus dem Text herauslesen und in die Kategorie Hobby einordnen. 
%TODO Worstammlungen können ergänzt werden

Um die Personenidentifikation zu erweitern, können Bilderkennungen verwendet werden. Dadurch dienen gleiche Profilbilder auf unterschiedlichen Social-Media-Plattformen als weitere Identifikationsschlüssel. Für diese Methode eignet sich eine Bildererkennungssoftware oder die Google-Bildersuche. Eine weitere Möglichkeit die Identifizierung der gesuchten Person zu optimieren, kann das Beachten von Zeiträumen sein. Dabei wird erkannt, ob der Zeitrahmen des Artikels oder das Erstellungsdatum einer Webseite mit dem Alter der Person grundsätzlich übereinstimmt. Hierfür können Jahreszahlen und mögliche Metadaten der Webseite beziehungsweise der Domain ausgelesen werden.

Damit die Wahrscheinlichkeit erhöht wird, dass sich die korrekte E-Mail-Adresse in dem erzeugten Adresspool befindet, können weitere Adresse generiert werden. Als Ideengeber könnte hierfür das OSINT-Tool \cite{EmailAssumptions} dienen. Darüber hinaus können dem Adresspool mögliche Firmenadressen hinzugefügt werden. Dazu müsste allerdings die Institution der Zielperson bekannt sein. Der erzeugt Adresspool beinhaltet viele mögliche E-Mail-Adressen der gesuchten Person. Jedoch ist nicht jede Adresse korrekt. Aus diesem Grund können die generierten Adressen validiert werden. Möglichkeiten dafür sind bereitgestellte Webseiten oder ein eigenes Skript.

Aktuell wird die Phishing-E-Mail mit der Adresse des gefälschten GMX-Accounts versendet. Dadurch steht diese Adresse als Absender in der entsprechenden Mail. Um die Glaubwürdigkeit der Phishing-Mail zu steigern, kann die Absenderadresse verschleiert werden. Dies ist möglich, indem der E-Mail-Header verändert wird. Im Fall, dass Informationen über Kontakte der Zielperson gefunden werden, können diese Daten zur Generierung einer gefälschten Absenderadresse verwendet werden.

%%% Local Variables: 
%%% mode: latex
%%% TeX-master: "Bachelorarbeit"
%%% End: 
               % Schluss

%%% Anhang %%%          Nummerierung beginnend bei A
\appendix                       % Anhang
 \chapter{Ein Kapitel des Anhangs}
\label{cha:anhang}




%%% Local Variables: 
%%% mode: latex
%%% TeX-master: "Bachelorarbeit"
%%% End: 
               % Erstes Kapitel des Anhangs

%%% Verzeichnisse %%%
\begin{spacing}{1.0}            % Verzeichnisse werden mit einzeiligem Abstand gesetzt

% \listoffigures                   % Abbildungsverzeichnis (optional)
% \listoftables                    % Tabellenverzeichnis (optional)

%Glossar ausgeben
\printglossary[style=altlist,title=Glossar]

%Abk�rzungen ausgeben
\renewcommand{\acronymname}{Abkürzungsverzeichnis}
\printglossary[type=\acronymtype,style=long]

%Symbole ausgeben
\printglossary[type=symbolslist,style=long]

\bibliographystyle{geralpha}       % Look&Feel vom Literaturverzeichnis
\bibliography{bib}                 % Literaturverzeichnis
\addcontentsline {toc}{chapter}{Stichwortverzeichnis} % Stichwortverzeichnis soll im Inhaltsverzeichnis auftauchen (optional)
\printindex % Stichwortverzeichnis endgueltig anzeigen

\end{spacing}

\end{document}

%%% WICHTIG:
%% Um den Glossareintrag (Abkürzungsverzeichnis richtig darstellen zu können, muss makeindex mit dem Parameter "-s Bachelorarbeit.ist -t Bachelorarbeit.alg -o
%% Bachelorarbeit.acr Bachelorarbeit.acn" aufgerufen werden --> Einstellung im TeXnicCenter unter Ausgabe -> Ausgabeprofil definieren -> LaTeX=>PDF -> Nachbearbeitung
%% Unter Postprozessoren neuen Eintrag anlegen, z.B. Makeindex1. Unter Anwendung makeindex.exe auswählen. (c:\programme\miktex2.7\miktex\bin\makeindex.exe)
%% unter Argumente die obige Parameterzeile eintragen. Bei anderer TeX-Distri makeindex suchen. Unter Linux ein shell-script erstellen, das makeindex mit 
%% den Parametern aufruft.
%% makeindex erneut starten mit folgender Parameterzeile:  " -s Bachelorarbeit.ist -t Bachelorarbeit.glg -o Bachelorarbeit.gls Bachelorarbeit.glo"
%% Analog zu oben im TeXnicCenter einen weiteren Eintrag Makeindex2 erstellen. In Linux einen weiteren makeindex-Aufruf im Script hinzufügen.
%% makeindex erneut starten mit folgender Parameterzeile:  " -s Bachelorarbeit.ist -t Bachelorarbeit.slg -o Bachelorarbeit.syi Bachelorarbeit.syg"
%% Analog zu oben im TeXnicCenter einen weiteren Eintrag Makeindex3 erstellen. In Linux einen weiteren makeindex-Aufruf im Script hinzufügen.

%% Bachelorarbeit muss durch den Namen der Hauptdatei ausgetauscht werden. Hauptdatei unter Windows mindestens 5 Mal compilieren, dann betrachten

%% Local Variables:
%% mode: latex
%% TeX-master:
%% End:
