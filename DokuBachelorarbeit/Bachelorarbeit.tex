
%% Vorlage fuer Studien- und Bachelorarbeiten an der HS Ravensburg-Weingarten
%% Benoetigt wird KOMA-Skript ab Version 2.8j vom 30.07.2001
%% Vor der Veraenderung irgendwelcher Einstellungen wird dringend empfohlen
%% die Anleitung zum KOMA-Skript (scrguide) zu konsultieren !!
%%
%% Bilder bitte nicht mit Endung einbinden, so ist die Erzeugung von
%% DVI, PS und PDF problemlos m�glich!
%% J. Moehler, 2008-12-10 
%%Kommentar


%% Dokumentendefinition
\documentclass[
   12pt,                % Schriftgroesse 12pt
   a4paper,             % Layout fuer Din A4
   german,              % deutsche Sprache, global   
%  twoside,             % Layout fuer beidseitigen Druck
	 oneside,						  % Layout für einseitigen Druck
   headinclude,         % Kopfzeile wird Seiten-Layouts mit beruecksichtigt
   headsepline,         % horizontale Linie unter Kolumnentitel
   plainheadsepline,    % horizontale Linie auch beim plain-Style
   BCOR12mm,            % Korrektur fuer die Bindung
   DIV18,               % DIV-Wert fuer die Erstellung des Satzspiegels, siehe scrguide
   halfparskip,         % Absatzabstand statt Absatzeinzug
   openany,             % Kapitel können auf geraden und ungeraden Seiten beginnen
   bibtotoc,            % Literaturverz. wird ins Inhaltsverzeichnis eingetragen
   pointlessnumbers,    % Kapitelnummern immer ohne Punkt
   tablecaptionabove,   % korrekte Abstaende bei TabellenUEBERschriften
   fleqn,               % fleqn: Glgen links (statt mittig)
%   draft               % Keine Bilder in der Anzeige, overfull hboxes werden angezeigt !Auskommentieren für Test-Compile!
]{scrbook}[2001/07/30]  % scrbook-Version mind. 2.8j von 2001/07/30

%% Pakete für nützliche Dinge
\usepackage{color} 							 % Schriftfarben verwenden
\usepackage{ngerman}             % neue deutsche Rechtschreibung
%\usepackage[ngerman]{translator} % Übersetzung 
%\usepackage[ansinew]{inputenc}   % Input-Encodung: ansinew fuer Windows
\usepackage[utf8]{inputenc}   % Input-Encodung: latin1 fuer Unix
\usepackage[T1]{fontenc}         % T1-kodierte Schriften, korrekte Trennmuster fuer Worte mit Umlauten
\usepackage{ae}                  % Für PDF-Erstellung
\usepackage[hang]{caption2}      % mehrzeilige Captions ausrichten
\usepackage{rotate}
\usepackage[centertags]{amsmath} % AMS-Mathematik, centertags zentriert Nummer bei split
\usepackage{latexsym}            % verschiedene Symbole
\usepackage{textcomp}            % verschiedene Symbole
 \usepackage{microtype}          % bessere Optik     
\usepackage{graphicx}            % zum Einbinden von Grafiken
\usepackage{float}               % u.a. genaue Plazierung von Gleitobjekten mit H
% \usepackage{pstricks}          % PostScript Macros
% \usepackage{lscape}            % Seite im Querformat bei Erhalt der Kopfzeile
% \usepackage{verbatim}          % Quellcode einbinden (\verbatiminput)
% \usepackage{multicol}          % Mehrspaltiger Text 
\usepackage{placeins}

%%% Literatur und sonstige Referenzen
\usepackage{cite}              % Sortierte und zusammengefasste Zitatnummern 
\usepackage{url}							 % URL in Literatur wird unterst�tzt
\usepackage{varioref}          % Verbesserte Referenzen
\usepackage{hyperref}          % Verlinkte Verzeichnisse
% \usepackage[numbers, sort]{natbib} % DIN Literaturverzeichnis; nicht zusammen mit cite verwenden!

%% Index
\usepackage{makeidx}						 % Index verwenden
\makeindex											 % index erstellen

%% Zeilenabstand
\usepackage{setspace}            % Zeilenabstand einstellbar
\onehalfspacing                  % eineinhalbzeilig einstellen

%% Kopf- und Fusszeilen
\usepackage{scrpage2}                             % Kopf und Fusszeilen-Layout 
\renewcommand{\headfont}{\normalfont\sffamily}    % Kolumnentitel serifenlos
\renewcommand{\pnumfont}{\normalfont\sffamily}    % Seitennummern serifenlos
\pagestyle{scrheadings}
\ihead[]{\headmark}               % Kolumnentitel immer oben innen
\chead[]{}                        % oben Mitte
\ohead[\pagemark]{\pagemark}      % Seitennummern immer oben aussen
\ofoot[]{}                        % Fusszeile aussen
\cfoot[]{}                        % Fusszeile Mitte
\ifoot[]{}                        % Fusszeile innen

%% Fussnotenzähler
\usepackage{chngcntr}              % Paket um Counter zu steuern
\counterwithout{footnote}{chapter} % Fussnoten nicht pro Chapter, sondern Global
\usepackage{threeparttable}        % Tabelle mit Fussnoten

%% Namen von Verzeichnissen definieren
\renewcommand{\bibname}{Literatur}               % Literaturverzeichnis wird zu Literatur
\renewcommand{\figurename}{Bild}                 % Abbildung wird zu Bild
\renewcommand{\listfigurename}{Bildverzeichnis}
\renewcommand{\indexname}{Stichwortverzeichnis}

%% Schrift mit Serifen auch fuer Ueberschriften benutzen
%\renewcommand*{\sectfont}{\bfseries}
%\renewcommand*{\descfont}{\bfseries}

\typearea[current]{current}        % Neuberechnung des Satzspiegels mit alten Werten nach Änderung von Zeilenabstand,etc

% -- Glossar --
\usepackage[toc, acronym]{glossaries} % Glossareinträge, muss nach hyperref (insofern dies verwendet wird) geladen werden
% (aufgrund der Seitenzahlverlinkung im Glossarverzeichnis), (benötigt die Packete
% "xkeyval" und "supertabular", welche dann automatisch eingebunden werden)
\newglossary[slg]{symbolslist}{syi}{syg}{Symbolverzeichnis} %Ein eigenes Symbolverzeichnis erstellen
\renewcommand*{\glspostdescription}{} %Den Punkt am Ende jeder Beschreibung deaktivieren
\makeglossaries % erstellt ein Glossar (Verzeichnis für Begriffserklärungen,
% z.B.: Abkürzungen)
% Glossareinträge (MUSS für JEDEN Glossareintrag überarbeitet werden):
%% Symbole:
%%\newglossaryentry{symb:Name}{name=Symbolname, description={Beschreibung}, sort=alphabetisches Wort für die Einreihung, type=symbolslist}
%\newglossaryentry{symb:Pi}{
%name=$\pi$,
%description={Die Kreiszahl.},
%sort=symbolpi, type=symbolslist
%}
%%Abkürzungen:
%%\newacronym{Referenz}{Abkürzung}{Beschreibung}
\newacronym{BSP}{BSP}{Beispiel}


%%Eine Abkürzung mit Glossareintrag:
%%\newacronym{Referenz}{Abkürzung}{Beschreibung\protect\glsadd{glos:Referenz}}
%\newacronym{AD}{AD}{Active Directory\protect\glsadd{glos:AD}}

%%Glossareintrag:
%%\newglossaryentry{glos:Referenz}{name=Name, description={Beschreibung}}
%\newglossaryentry{glos:AD}{
%name=Active Directory,
%description={Active Directory ist in einem Windows Server 2000, Windows
%Server 2003, oder Windows Server 2008-Netzwerk der Verzeichnisdienst, 
%der die zentrale Organisation und Verwaltung aller Netzwerkressourcen erlaubt. Es
%ermöglicht den Benutzern über eine einzige zentrale Anmeldung den
%Zugriff auf alle Ressourcen und den Administratoren die zentral
%organisierte Verwaltung, transparent von der Netzwerktopologie und
%den eingesetzten Netzwerkprotokollen. Das dafür benötigte
%Betriebssystem ist entweder Windows Server 2000, 
%Windows Server 2003, oder Windows Server 2008, welches auf dem zentralen
%Domänencontroller installiert wird. Dieser hält alle Daten des
%Active Directory vor, wie z.B. Benutzernamen und
%Kennwörter.}
%}

%\newglossaryentry{glos:Glossareintrag}{name=Glossareintrag, description={Erweiterte Informationen zum
%einem Wort oder einer Abkürzung, ähnlich einem Eintrag im Duden.}}




%%% Local Variables: 
%%% mode: latex
%%% TeX-master: "Bachelorarbeit"
%%% End: 
\usepackage[left = 2.54cm, right = 2.54cm]{geometry}
%%% PDF-Erzeugung: pdflatex statt latex aufrufen!
%% BEI WINDOWS UND TEXNICCENTER AUSKOMMENTIERT LASSEN !!! 
%\pdfoutput=1                  % PDF-Ausgabe
%\usepackage[pdftex, a4paper,  % muss letztes Package sein!
%     pdftitle={Titel der Arbeit},%
%     pdfauthor={Name Autor},%
%     pdfsubject={Studien- bzw. Diplomarbeit},%
%     pdfkeywords={Stichwort zur Arbeit},%
%    ]{hyperref} % 



%\graphicspath{{figs/}{bilder/}}    % Falls texinput nicht gesetzt -> Bildverzeichnisse

%\includeonly{}


%%%%%%%%%%%%%%%%%%%%%%%%%%%%%%%%%%%%%%%%%%%%%%%%%%%%%%%%%%%%%%%
\begin{document}

\pagenumbering{Roman}           % Nummerierung Römisch Start bei I

%% Deckblatt fuer Studien- und Diplomarbeiten am der
%% Hochschule Weingarten

\thispagestyle{empty}
%~
{
\normalsize\fontfamily{phv} \fontsize{12pt}{10}\selectfont 
\vspace{-1cm}
\begin{minipage}[b]{9.4cm}
{\fontsize{13pt}{13} \selectfont%
Hochschule\\[1ex]
Ravensburg-Weingarten}\\[1ex]
\end{minipage}
}
\begin{minipage}[b]{10cm}
\includegraphics*[height=2.7cm]{bilder/HSLogoWGd}
\end{minipage}


\vspace{10mm}
 
\hrule 
\vspace{1cm}
{
\fontseries{b} \fontsize{20pt}{20}  \selectfont%
\begin{center}
Entwicklung einer Anwendung zur automatisierten Beschaffung von personenbezogenen Daten im Internet und deren Integration in Phishing-Mails % Titel der Arbeit
\end{center}
}

\begin{center}
\large \textbf{Bachelorarbeit} % Hier Praxisarbeit, Studienarbeit, Bachelorarbeit, Dokumentation zu Seminar, etc. eintragen
\end{center}

\begin{center}
\textbf{Wintersemester 2018/2019} % Hier den Zusatz wie Fach oder Semester eintragen
\end{center}

\vspace{5mm}

\begin{center}
im Studiengang Angewandte Informatik % Hier den Studiengang eintragen
\end{center}

\begin{center}
an der Hochschule Ravensburg - Weingarten 
\end{center}
\begin{center}

\end{center}
\vspace{5mm}
\begin{center}
von
\end{center}




\begin{center}
{\fontsize{12pt}{12} \selectfont%
\begin{tabular}{ll}
Marco Lang & Matr.-Nr.: 27416\\[0.5ex] % Hier den Autor und die Matrikelnummer statt xxxxx eintragen
%Autor 2 & \textcolor{red}{Matr.-Nr.: xxxxx}\\[0.5ex] % Für weitere Autoren Zeilen auskommentieren bzw. kopieren und ausfüllen
%Autor 3 & \textcolor{red}{Matr.-Nr.: xxxxx}\\[0.5ex]
Abgabedatum :& \today   % Das Abgabedatum wird gleichgesetzt mit dem Datum der letzten Compilierung. Statt \today kann auch Datum von Hand geschrieben werden
\end{tabular}
}
\end{center}
                               

\vspace{1cm}

\vspace{1cm}
\hrule


%%% Local Variables: 
%%% mode: latex
%%% TeX-master: "Bachelorarbeit"
%%% End: 

                % Deckblatt Nummerierung unterdrückt (In Deckblatt festgelegt)
%%\cleardoubleemptypage         % Die Eidesstattliche Erklaerung auf einer rechten Seite beginnen
%% Eidestattliche Erklärung %%
%% Die Erklärung sollte nach dem Deckblatt fest abgeheftet werden
\addchap*{Erklärung} %* nicht entfernen, sonst erhält Erklärung eine Nummer und erscheint im Inhaltsverzeichnis

\thispagestyle{empty} %Keine Seitenzahl, keine Kopf- und Fusszeile

Hiermit erkläre ich, dass ich die vorliegende Arbeit mit dem Titel   % ein Autor
% Hiermit erkl"aren wir, dass wir die vorliegende Arbeit mit dem Titel \newline % mehrere Autoren
\begin{center}
\textbf{Entwicklung einer Anwendung zur automatisierten Beschaffung von personenbezogenen Daten im Internet und deren Integration in Phishing-Mails}
\end{center}
selbstständig angefertigt, nicht anderweitig zu Prüfungszwecken vorgelegt, keine anderen als die angegebenen Hilfsmittel benutzt und wörtliche sowie sinngemäße Zitate als solche gekennzeichnet habe.\newline  % ein Autor

\begin{flushleft}
Weingarten, 29. April 2019 % Ort eintragen, /today kann durch Datum 2009-10-21 oder 21.10.2009 ersetzt werden
\end{flushleft}

%%% Unterschriftenblock für einen Autor
\begin{tabular}{l}   
Autor Name        \\% Hier Autor eintragen
 \\
------------------------------------ \\
\end{tabular}

%%% Unterschriftenblock für mehrere Autoren
%\begin{tabular}{lll}
%Autor 1       &Autor 2      &Autor 3 \\% Hier eintragen
% & & \\
%------------------------------------ & ------------------------------------ & ------------------------------------ \\
%\end{tabular}

%Hier unterschreiben


%%% Local Variables: 
%%% mode: latex
%%% TeX-master: "Bachelorarbeit"
%%% End: 
                 % Eidesstattliche Erklaerung Nummerierung unterdrückt
%%\cleardoubleemptypage         % Das Inhaltsverzeichnis auf einer rechten Seite beginnen

\pagenumbering{Roman}           % Nummerierung Römisch start bei I 

\begin{spacing}{1.0}            % Verzeichnisse werden mit einzeiligem Abstand gesetzt
 \tableofcontents               % Inhaltsverzeichnis
\end{spacing}
%%% Vorbemerkungen %%%  Nummerierung unterdrückt durch *
\addchap{Kurzfassung}
\label{cha:kurzfassung} 
Es wird gezeigt, wie eine automatisierte Suche nach personenbezogenen Daten im Internet aussehen kann und wie diese Daten für einen Phishing-Mail-Angriff verwendet werden können. 

Wird erweitert!

%%% Local Variables: 
%%% mode: latex
%%% TeX-master: "Bachelorarbeit"
%%% End: 


             % Kurzfassung der Arbeit
%\addchap{Abstract}
\label{cha:abtract} 







%%% Local Variables: 
%%% mode: latex
%%% TeX-master: "Bachelorarbeit"
%%% End: 
             % Abstract der Arbeit (englische Kurzfassung der Arbeit)
\addchap{Danksagung}
\label{cha:danksagung}
An dieser Stelle möchte ich mich bei all denjenigen bedanken, die mich bei meiner Bachelorarbeit und während meines Studiums unterstützt haben.

Mein Dank gilt meinen Eltern Wolfgang und Birgit Lang ohne die mein Studium in dieser Weise nicht möglich gewesen wäre.

Ebenfalls möchte ich meinem Vater, meinem Bruder und meiner Freundin für das Korrekturlesen dieser Bachelorarbeit danken.

Ein besonderer Dank geht an meinen Prüfer Prof. Dr. rer. nat. Eggendorfer für die zahlreichen Anregungen und Tipps, sowie für das Korrektur lesen.



%%% Local Variables: 
%%% mode: latex

%%% TeX-master: "Bachelorarbeit"
%%% End: 								  % Danksagung (optional)
%\addchap{Vorwort}
\label{cha:vorwort}



%%% Local Variables: 
%%% mode: latex
%%% TeX-master: "Bachelorarbeit"
%%% End: 							    % Vorwort (optional)

%%% Hauptteil %%%       Nummerierung beginnend bei 1
%%\cleardoubleplainpage         % Das erste Kapitel des Hauptteils auf einer rechten Seite beginnen
\mainmatter                     % den Hauptteil beginnen
\chapter{Einleitung}
\label{cha:einleitung}

%%% Local Variables: 
%%% mode: latex
%%% TeX-master: "Bachelorarbeit"
%%% End: 

\section{Motivation}
\label {sec:Motivation}
In der heutigen Zeit wird das Thema Informationssicherheit immer wichtiger. Systeme werden immer komplexer und Firewalls immer besser.
Doch laut dem Bundeskriminalamt hat sich die Zahl der Cyberkriminalität mit einem klaren Trend nach oben entwickelt. \cite{Cyberkriminalitaet}\\
Eine häufig verwendetet Technik von Cyberkriminalität ist das E-Mail-Phishing. Hier wird der Mensch als Schwachstelle des Systems genutzt. In den neusten Fällen von Phishing-Attacken zeigt die Verbraucherzentrale Nordrhein-Westfalen, dass diese meist direkt an eine Person adressiert sind. Beispielsweise wird man in den gefälschten DSGVO-E-Mails, im Namen der Sparkasse, persönlich mit Namen angesprochen. \cite{VerbraucherzentraleNW} \\
Im Rahmen dieser Abschlussarbeit wird gezeigt, mit welchem Aufwand solche Angriffe verbunden sind und wie die Suche nach privaten Informationen im Internet aussieht.
%TODO Mehr auf die Informationsbeschaffung eingehen

\section{Zielsetzung}
\label {sec:Zielsetzung}
 
\section{Eigene Leistung}
\label {sec:Eigene Leistung} 
In dieser Arbeit wird ein Tool erstellt, welches personenbezogene Daten automatisiert aus dem Internet heraussucht und diese in potentielle Opferprofile ablegt. Die gewonnenen Informationen werden durch einen Phishing-Mailgenerator automatisiert in eine personalisierte Phishing-E-Mail eingebaut. Für einen höheren Erfolg werden E-Mail-Muster erstellt.

\section{Aufbau der Arbeit}
\label {sec:Aufbau der Arbeit} 
Die Arbeit gliedert sich in einem theoretischen und praktischen Teil auf. Der Theorie-Teil beginnt im zweiten Kapitel und beschreibt die Grundbegriffe im Bereich Social Engineering, Webtools, E-Mails und Programmiersprachen. Im nächsten Kapitel befindet sich die Anforderungsanalyse. Hier werden die Anforderungen an die Arbeit festgelegt. Darauf folgen die Lösungsvorschläge im Kapitel vier und die ausgewählte Lösung anhand den Anforderungen im Kapitel 5. Anschließend wird bei der Umsetzung auf den Praktischen Teil eingegangen.Am Ende befindet sich das Fazit, der Ausblick und der Anhang.






               % Einleitung
%Kapitel des Hauptteils

\chapter {Grundlagen}  %Name des Kapitels
\label{cha:grundlagen} %Label des Kapitels

\section{Personenbezogene Daten}
%TODO möglicherweise unterschied von Information und Daten erklären
Laut der DSGVO sind \textbf{personenbezogene Daten}, alle Informationen, die sich auf eine identifizierbare Person beziehen. Als identifizierbar wird eine natürliche Person angesehen, die mittels einem oder mehreren Merkmalen direkt oder indirekt identifiziert werden kann. Mögliche Kennungen für die Unterscheidung der Merkmale sind der Name, eine Kennnummer, Standortdaten, eine Online-Kennung, et cetera von der Person. Dabei dienen diese Kennungen als Ausdruck der physischen, physiologischen, genetischen, psychischen, wirtschaftlichen, kulturellen oder sozialen Identitäten dieser natürlichen Person. \cite{personenbezogeneDaten}

\section{Social Engineering} %Unterkapitel
\label {sec:Social Engineering} %Label des Unterkapitels
	Die Definition von Social Engineering, kurz \textit{SE}, ist nicht eindeutig, da es sehr verschiedene Ansichten davon gibt. Jedoch ist der Grundgedanke von Social Engineering, eine Zielperson so zu manipulieren, damit sie für den Angreifer bessere Entscheidung trifft. \cite{ArtOfHumanHacking} \\
	Kevin D. Mitnick definiert Social Engineering wie folgt:\\	
	\textit{"'Social Engineering uses influence and persuasion to deceive people by convincing them that the social engineer is someone he is not, or by manipulation. As a result, the social engineer is able to take advantage of people to obtain information with or without the use of technology"'}\cite{ArtOfDeception}
	
	SE wird Menschen von Geburt an beigebracht und begegnet einem beinahe jeden Tag. Schon ein Baby muss wissen wie es die Eltern manipulieren kann, damit es Dinge wie Essen, Zuneigung, oder ähnliches bekommt. Darüber hinaus ist SE in vielen Berufen ein täglicher Bestandteil. %TODO villt Jobbeispiel finden mit Social Engineering
	
	Im Bereich der Informationssicherheit, wird von Social Engineering gesprochen, wenn Angreifer durch die Manipulierung und Täuschung von Menschen vertrauliche Informationen oder Zugänge zu Systemen bekommen. Die bekanntesten Angriffsmethoden sind Phishing, Pretexting, Baiting und Quid Pro Quo. Bei dieser Arbeit wird hauptsächlich auf das Thema E-Mail-Phishing eingegangen.

	Der Aufbau eines SE-Angriffes ist definiert in mehrere Phasen. Das wohl bekannteste Modell für einen Social Engineering-Angriffszyklus ist in dem Buch von Kevin D. Mitnicks \cite{ArtOfDeception} definiert. Dieser Zyklus besteht aus den 4 Phasen \textbf{Research, Developing rapport and trust, Exploiting trust} und \textbf{Utilize information}.\\
	In der \textbf{Research-Phase} geht es um die Informationsbeschaffung. Bei dieser Phase will der Angreifer möglichst viele Informationen über das Ziel herausfinden. Die \textbf{Developing Rapport and Trust-Phase} beschreibt den Kontaktaufbau zum Ziel, da wenn das Opfer dem Angreifer vertraut, hat dieser ein leichteres Spiel in den kommenden Phasen. Das nun erzeugte Vertrauen wird in der \textbf{Exploitung Trust-Phase} ausgenutzt. Hier will der Angreifer die eigentlich Information vom Opfer herausfinden. Dies geschieht einerseits durch bestimmtes Nachfragen oder durch Manipulation.
	\textbf{Utilize Information} ist die letzte Phase. Dort wird die gewonnene Information genutzt um das eigentliche Ziel des Angreifers zu erreichen.\\
	Grundsätzlich werden bei einem Social Engineering Angriff menschliche Wünsche, Ängste und verbreitete Verhaltensmuster verwendet um ein Opfer zu manipulieren.\cite{LeitfadenSE}\\

		\subsection{Phishing}
		Das Wort Phishing wird von dem Wort "'fishing"' abgeleitet, da die Angreifer nach Informationen fischen. Das "'Ph"' kommt von "'sophisticated"' und meint damit, dass die Angreifer ausgeklügelte Techniken verwenden um an Informationen heranzukommen.\cite{PhishingExposed}\\
		Die wohl bekannteste Angriffsmethode von Phishing ist das E-Mail-Phishing. Bei diesem Verfahren, versendet ein Angreifer meist eine gefälschte E-Mail, um ein Opfer zu täuschen und dadurch sein Ziel zu erreichen. Die sogenannten Phishing-Mails enthalten meist eine Aufforderung einen Link zu öffnen und sehen täuschend echt aus.\\
		Ein reales Beispiel könnte sein, dass der Angreifer eine gefälschte E-Mail von Amazon an das Opfer versendet und es dabei auffordert, einen Link in der Mail zu öffnen. Nachdem die Zielperson auf den Link geklickt hat, muss Sie sich anmelden. Hier könnte der Angreifer ein täuschend echtes Anmeldeformular erstellt haben, um die Anmeldedaten der Zielperson zu bekommen. Sobald die Anmeldedaten eingegeben wurden, könnte eine Fehlermeldung erscheinen, die einen Authentifizierungsfehler beinhaltet und das Opfer auffordert sich erneut anzumelden. Jedoch wird während diesem Prozess das originale Anmeldeformular geladen und das Opfer kann sich korrekt bei der entsprechenden Webseite anmelden. \\
		Dieser Verfahren ermöglicht Angreifern die Anmeldedaten von einer Zielperson ohne großen Aufwand zu beschaffen. Allerdings benötigt der Angreifer für diese Methode nicht nur Social Engineering sondern auch technische Fähigkeiten.\cite{PhishingDarkWaters}
		
		\subsection{Spear-Phishing}
		Das Spear-Phishing ist eine erweiterte Methode des herkömmlichen E-Mail-Phishings. Hierbei wird anstatt das Versenden etlicher Phishing-Mails an unbekannte Opfer, eine gezielte Mail an eine ausgewählte Person versendet.\cite{SpearPhishingPaper}\\
		Bei dieser Form von E-Mail-Phishing spielt die Opferauswahl und die Informationsbeschaffung eine sehr große Rolle, da diese Information später für personalisierte E-Mails oder vorgetäuschte Identitäten verwendet werden können. Durch diese Art von Täuschung kann ein Opfer dazu bewegt werden auf einen Link zu klicken und dadurch eine Schadsoftware herunterzuladen.\cite{SpearPhishingPaper} \\
		Der Aufwand für die Informationsbeschaffung wird oft in Kauf genommen, da der Erfolg bei dieser Methode vielversprechender ist als beim herkömmlichen E-Mail-Phishing.\\
		91\% der Advanced Persistent Threat (APT) Angriffe auf Firmen beginnen mit einer Spear-Phishing-E-Mail. Die Schadsoftware wir meisten als Remote Access Trojans (RATs) in einem Zip-Datei überliefert.\cite{SpearPhishing}


\section{Open Source Intelligence}
	\subsection{Definition OSINT}
	Open Source Intelligence kurz OSINT ist definiert in eine Intelligenz, welche aus öffentlich zugänglichen Informationen gewonnen wird. Allerdings kann sich die Bedeutung fallspezifisch ändern. So bedeutet OSINT für die CIA die Informationsgewinnung aus ausländischen Nachrichtensendungen. Doch für die meisten Menschen bedeutet OSINT die Gewinnung eines öffentlichen Inhalts aus dem Internet. \cite{Bazzell}\\
	Unter Open Source wird die öffentlich zugängliche Information, die in gedruckter oder elektronischer Form vorliegt, bezeichnet.\cite{steele1996open} Eine Verbindung mit dem Begriff Open-Source-Software besteht nicht.
	\subsection{Web Crawler}
		Web Crawler sind Computerprogramme, die mit Hilfe der Hypertextstruktur das Internet durchlaufen. Dabei können sie in einen \textbf{internen} und \textbf{externen Web Crawler} unterschieden werden. Der interne Web Crawler durchsucht ausschließliche interne Seiten einer Webseite und der externe Web Crawler durchsucht unbekannte Webseiten im ganzen Netz. \cite{sharma2012study}

		In anderen Worten besteht die Funktionsweise darin, dass in den meisten Fällen ein automatisiertes Programm, Web Crawler, erstellt wird. Dieser lädt Webinhalte herunter und durchsucht den Inhalt nach Hyperlinks. Den gefundenen Links wird gefolgt, um neue Webseiten mit weiteren Links zu laden. So hangelt sich ein Web Crawler von Link zu Link durch das Internet.\cite{WebScraping}
		
	\subsection{Web Scraper}
		In der Theorie bedeutet \textit{web scraping} die Informationsbeschaffung im Internet mit unterschiedlichsten Mitteln. \cite{WebScraping}\\		
		Meist wird dies mit einem automatisierten Programm realisiert, welches Daten von einem Webserver anfragt, entgegen nimmt, analysiert und auswertet. 
		In der Praxis gibt es ein großes Feld von Programmiertechniken und Einsatzmöglichkeiten.
		Mit Hilfe eines Web Scrapers ist es möglich, große Datenmengen zu erfassen und zu verarbeiten.\cite{WebScraping}

		\subsubsection{Natural Language Processing}
			Natural Language Processing kurz \textit{NLP} beschreibt eine Technologie, für die Kommunikation zwischen Mensch und Computer. Mit dem Ziel, dass ein Computer die natürliche Sprache verstehen und verarbeiten kann. Dafür werden verschiedenste Methoden aus der Sprach- und Computerwissenschaft sowie aus der künstliche Intelligenz verwendet. Unter anderem hat eine NLP-Anwendung die Aufgabe von \textbf{Stemming}.\cite{NLP} 
	
			\textbf{Stemming} ist eine Methode der Wortstandardisierung, bei der verwandte Wörter auf ihrer Stammform reduziert werden. Dabei wird bei dem Rechenvorgang auf den Stamm und die Semantik eines Wortes geachtet. Aus diesem Grund fällt der Name Stammformreduktion öfters in Verbindung von Stemming.\cite{eldesouki2009stemming}\\
			%TODO Beispiel von Stammformreduktion
			Die Verwendung von Stemming, kann bei der Schlüsselwortgenerierung von Texten sehr hilfreich sein, da die Anzahl der möglichen Schlüsselwörter reduziert werden können.


%%% Local Variables: 
%%% mode: latex
%%% TeX-master: "Bachelorarbeit"
%%% End: 
                % Ein Kapitel des Hauptteils

%Kapitel des Hauptteils

\chapter{Problemspezifikation}  %Name des Kapitels
\label{cha:Problemspezifikation} %Label des Kapitels
Persönliche Daten sind im Internet oft frei zugänglich. Das heißt, dass unterschiedlichste Webseiten persönliche Information von Menschen öffentlich bereitstellen. Die bekanntesten Webseiten sind wahrscheinlich die Social Media Seiten wie Twitter, Facebook und Instagram. Allerdings wird auch auf anderen Webseiten personenbezogene Daten in großen Mengen bereitgestellt. Ein Beispiel dafür ist das Fußballportal \textit{"'www.fupa.net"'}. Diese Art von Webseiten sind perfekte Informationsquellen für Phisher, da im Bereich von Social Engineering, diese Informationen oft genutzt werden um ein Opfer zu täuschen oder zu manipulieren.\\
Dass hier beschriebene Problem zeigt, dass der Zugang für persönliche Information durch das Internet für die Öffentlichkeit einfacher gemacht wird. Es soll mit einem kritisch Blick darauf gezeigt werden, mit welchem Aufwand, personenbezogene Daten aus dem Internet herausgelesen, analysiert und für einen Phishing-Mail-Angriff verwendet werden kann.
%TODO Werden Persönliche Informationen werden im Internet immer leichter zugänglich gemacht?????.Möglicherweise Geschichte zur Verdeutlichung.		% Ein weiteres Kapitel des Hauptteils
%Ethische Betrachtung
\chapter{Ethische und rechtliche Betrachtung}  %Name des Kapitels
\label{cha:EthischeUndRechtlicheBetrachtung} %Label des Kapitels
%http://analysis.seclab.tuwien.ac.at/papers/raid2010.pdf
Das Sammeln von personenbezogenen Daten auf sozialen Netzwerken ist ethisch gesehen ein sehr sensibler Bereich. Dennoch werden in dieser Arbeit ausschließlich die Daten verwendet, die öffentlich frei Zugänglich sind. Das heißt, unter den Informationen befinden sich keine Passwörter oder Informationen die nicht an die Öffentlichkeit gehören.\\
Mit diesem realen Experiment, soll die Privatsphäre der Benutzer geschützt werden, indem aufgezeigt wird, wozu öffentlich zugänglichen Daten verwendet werden können. Genau aus diesem Grund ist es wichtig, dass das Experiment in der realen Welt durchgeführt wird.\\
Des Weitern kann gesagt werden, dass der hier verwendete Crawler nicht stark genug ist, um die Leistung eines sozialen Netzwerkes zu beeinflussen.
%TODO MD5?? Datensicherheitsbeauftrager kontaktieren wie es mit Namen anonymisieren aussieht.
%Kapitel des Hauptteils

\chapter{Anforderungsanalyse und Priorisierung}  %Name des Kapitels
\label{cha:Anforderungsanalyse und Prioriesierung} %Label des Kapitels
\section{Anforderungsanalyse} %Unterkapitel
\label{sec:Anforderunsanalyse} %Label des Unterkapitels
Die im Kapitel \ref{sec:Zielsetzung} definierten Ziele sollen mit den folgenden Anforderungen gewährleistet werden.

	\subsection{Anforderung an die Informationsbeschaffung}
	Die Anforderungen an die Informationsbeschaffung von personenbezogenen Daten lässt sich in zwei Teile gliedern. Erstens in die Informationsbeschaffung von bestimmten bzw. ausgewählten Personen und zweitens die Informationsbeschaffung von vielen unbestimmten Personen.
		\subsubsection{Informationsbeschaffung von bestimmten/ausgewählten Personen}
		Bei dieser Informationsbeschaffung soll ein Tool entwickelt werden, welches Informationen zu einer angegeben Person sucht. Dies soll mit Hilfe eines Web-Crawlers und mit einem Web-Scraper umgesetzt werden. Das zu entwickelnde Tool soll bekannte Information/Daten (Name, Geburtsjahr, Ort, Usernames von Social Media Webseiten) über eine Konsolen-Abfrage einlesen können.
	
		\subsubsection{Informationsbeschaffung von unbestimmten Personen}
		Es soll ein Prototyp-Tool entwickelt werden, der möglichst viele Informationen von möglichst vielen Personen herausfindet. Jedoch sind diese Personen dem Tool-Anwender unbekannt. Die Informationen werden aus Webseiten mit einer großen Anzahl von Mitgliedern herausgelesen. Bei dem Prototyp soll es möglich sein, Webseiten auszuwählen, die ausgelesen werden sollen.
	\subsection{Anforderung an die Datenverwaltung/-speicherung}
	Die Spieler- bzw. Opferinformationen sollen in einer gut übersichtlichen Struktur gespeichert werden. Informationen müssen erweiterbar sein.
	
	\subsection{Anforderung an die Phishing-Mail Erzeugung}
	Es sollen E-Mail-Muster erstellt werden. Diese Muster sollen kategorisiert werden, damit für alle Opferinformationen ein passendes Muster vorhanden ist.
	Die Phishing-E-Mails sollen automatisiert erstellt werden und die gewonnene Opferinformation verwenden.
	
	\subsection{Ergänzende Anforderungen}
	Unter anderem soll die Arbeit Antworten auf folgende Fragen finden:\\
	Wie können Webseiten am effizientesten ausgelesen werden?\\
	Welche zusätzlichen Webseiten liefern die meisten Informationen zu potentiellen Opfern?\\
	Wie soll nach Informationen gesucht werden?\\
	Gibt es bereits einen Algorithmus der mit Hilfe von Vorname, Nachname und Geburtsjahr eine E-Mail-Adresse generieren?\\
	Wie können die Phishing-E-Mails möglichst auf einzelne Personen zutreffend erstellt werden? Ist es sinnvoll E-Mail-Muster zu erstellen?\\

\FloatBarrier
\section{Priorisierung} %Unterkapitel
\label{sec:} %Label des Unterkapitels
Die Tabelle \ref{tab:prio} zeigt die Priorisierung der Anforderungen.

\begin{table}
	
	\caption{Priorisierung der Anforderungen}
	\label{tab:prio}
	\begin{center} 
		\begin{tabular}{|l|l|}
			\hline
			Anforderung & Priorisierung (A-C) \\
			\hline
			Informationsbeschaffung von ausgewählten Personen & $ A $ \\
			\hline
			Informationsbeschaffung von vielen ubekannten Personen & $ A $ \\
			\hline
			E-Mail-Muster erstellen & $ A $    \\
			\hline
			Phsishing-Mail erzeugen & $ B $   \\
			\hline
		\end{tabular}
	\end{center}
\end{table}
\FloatBarrier
Man beachte: Bilder haben Bild{\bf unter}schriften, 
Tabellen haben Tabellen{\bf "uber}schriften.		% Die unterschiedlichen Kapitel
%Kapitel des Hauptteils

\chapter{Lösungsideen}  %Name des Kapitels
\label{cha:Lösungsideen} %Label des Kapitels
%TODO Struktur wie??
In diesem Kapitel werden die Lösungsideen für die Umsetzung der im Kapitel \ref{sec:Zielsetzung} definierten Ziele beschreiben.

\section{OSINT einer ausgewählten Person}
	\subsection{Verwendung von OSINT-Tools}
	Die Personensuche wird durch die Verwendung kostenloser OSINT-Tools durchgeführt.\\ 
	Eine entsprechende Webseite die mehrere OSINT-Methoden bereit stellt, ist unter dem URL \textit{"'https://inteltechniques.com/index.html"'} erreichbar. Sie stellt Methoden zur Suche nach E-Mail-Adressen, Benutzernamen, Social-Media-Profilen, und noch viele mehr zu Verfügung. Allerdings werden nicht nur selbstentwickelt OSINT-Methoden von Michael Bazzell bereitgestellt, sondern auch andere Webseiten mit weiteren OSINT-Tools vorgeschlagen.
	
	\subsection{Algorithmus für OSINT entwickeln}
	Es wird ein Algorithmus für OSINT entwickelt, der aus einem Web Crawler und Web Scraper besteht. Mit diesem ist es möglich eigenständig nach Information zu suchen. Hierfür wird eine Suchmaschine, wie die von Google, verwendet.\\
	Die Suchergebnisse können mit Hilfe des Web Crawlers verfolgt werden. Anschließen wird der Webseitentext, durch den Web Scraper, ausgelesen. Im letzten Schritt, wird der Text analysiert und interpretiert.\\
	All diese Prozesse laufen unabhängig von den vorgeschlagenen Webseiten voll automatisiert ab.


\section{Webseiten für OSINT mehrerer unbekannter Personen}
Für das OSINT mehrere unbekannter Personen stehen die Webseiten von FuPa, Xing und LinkedIn zu Auswahl.
	\subsection{XING}
	XING ist ein soziales Netzwerk für Berufstätige mit über 15 Millionen Mitgliedern. Hier vernetzen sich Kontakte aus allen Branchen um Jobs, Mitarbeiter, Aufträge oder ähnliches zu suchen und zu finden.\cite{WasIstXING}\\
	XING bietet allerdings viele Möglichkeiten zum Schutz der Privatsphäre. So kann ein Nutzer einstellen, ob er von einer Suchmaschine gefunden werden oder nur für Xing-Mitglieder sichtbar sein will. %TODO in die Bewertung
	\begin{figure}[H]
		\centering
		\includegraphics[ scale=0.2]{bilder/XING_profil.png}
		\caption{Profil von der Webseite XING}
		\label{img:XING}
	\end{figure}

	\subsection{LinkedIn}
	LinkedIn ist das weltweit größte soziale Netzwerk für Berufstätige mit hunderten von Millionen Mitgliedern. Es vernetzt berufliche Kontakte der ganzen Welt und stellt ebenfalls Möglichkeiten zum Schutz der Privatsphäre zu Verfügung. \cite{WasIstLinkedIn}
	\begin{figure}[H]
		\centering
		\includegraphics[ scale=0.2]{bilder/LinkedIn_profil.png}
		\caption{Prifl von der Webseite LinkedIn}
		\label{img:Linkedin}
	\end{figure}

		\subsection{Fupa}
		Die Webseite Fupa stellt ein regionales Fußballportal dar, welches zur Berichterstattung des Amateurfußballs vorhanden ist. Allerdings werden nicht nur Berichte sondern auch aussagekräftige Spielerprofile zur Verfügung gestellt.\cite{WasIstFUPA} Außer dem kann FuPa eine Mitgliederzahl von über 200.000 verzeichnet.\cite{FuPaMitglieder}\\
		Das Bild \ref{img:FuPa} zeigt ein Spielerprofil, wie es auf dieser Webseite angezeigt wird. Allerdings kann sich die Vollständigkeit eines Profils variieren.
		\begin{figure}[H]
			\centering
			\includegraphics[ scale=0.2]{bilder/fupa_screenshot.png}
			\caption{Spielerprofil von der Webseite FuPa}
			\label{img:FuPa}
		\end{figure}
		%TODO wie viele Mitglieder hat fupa?
	
	\section{Konzept für die Erstellung einer Phishing-Mail}
	Die Generierung einer realen Phishing-Mail benötigt eine korrekte E-Mail-Adresse der Zielperson und einen sinnvollen Inhalt, der die gewonnenen Informationen verwendet.
	\subsection{E-Mail-Adresse Generierung}
		\subsubsection{Algorithmus entwickeln zum generieren}
		Es kann ein Algorithmus entwickelt werden, der mögliche E-Mail-Adressen aus den gewonnen Daten generiert. Dies ist durch die Kombination aus Vorname, Nachname, Geburtsjahr und den bekanntesten E-Mail-Providern realisierbar. Dabei entsteht ein Adresspool, von dem jede einzelne E-Mail-Adresse auf Validität geprüft werden muss.\\
		\subsubsection{Automatisierbare OSINT-Tools verwenden}
		Für die Generierung der E-Mail-Adressen kann ein kostenloses OSINT-Tools von Michael Bazzel verwendet werden. Diese Tool ermöglicht es, die gewonnenen Informationen über eine Formular einzugeben und anschließend mögliche E-Mail-Adressen zu generieren. Auch hier entsteht ein Adresspool, bei dem die E-Mail-Adressen auf Validität geprüft werden müssen. Allerdings bringt das Tool eine weitere Funktion mit sich. Es wird automatisch nach Einträgen, der generierten E-Mail-Adressen, im Internet gesucht und angezeigt. \cite{EmailAssumptions}
		%TODO OSINT TOOL URL einfügen
	\subsection{E-Mail Inhalt}
		\subsubsection{E-Mail-Muster erstellen}
		Die zu erstellenden E-Mail-Muster entsprechen hier kategorisierten Lückentexten. Abhängig von den gefundenen Daten, wird ein Lückentext ausgewählt und anschließend mit den Daten an den passenden Stellen ergänzt.\\
		Die Lückentexte werden so kategorisiert, dass für jede gefundene Information ein passender Lückentext vorhanden ist. Eine denkbare Unterteilung wären die Kategorien Privat und Geschäftlich.
		\subsubsection{Text aus Fragmenten erzeugen}
		Bei dieser Methode besteht der E-Mail-Text aus zusammengesetzten Fragmenten. Dafür wird zu jeder gefundenen Information ein Fragment erstellt und anschließend werden alle Fragmente zu einem Text zusammengefügt.
		
		
		

%Kapitel des Hauptteils

\chapter{Lösungsideen}  %Name des Kapitels
\label{cha:Lösungsideen} %Label des Kapitels
%TODO Struktur wie??
In diesem Kapitel werden die Lösungsideen für die Umsetzung der im Kapitel \ref{sec:Zielsetzung} definierten Ziele beschreiben.

\section{Konzept zur Informationsbeschaffung einer ausgewählten Person}	
	\subsection{Methoden zur Suche nach einer Person im Internet}
	\label{sec:Suche nach Information}
	Für die Suche einer Person im Internet, wird abhängig von den eingegebenen Daten des Programm-Anwenders, die Art der Suche angepasst. Genau genommen heißt das, dass die eingegebenen Daten vor der Suche analysiert werden und dementsprechend die Suche danach angepasst wird. \\
	Die Art der Personensuche lässt sich in zwei Methoden gliedern.
		\subsubsection{Personensuche mit Hilfe von Suchmaschinen}
		\label{subsubsec:PersonensucheMitHilfevonSuchmaschine}
		Hier wird mit Hilfe einer Suchmaschine nach Informationen gesucht. Allerdings muss nicht für jede Suche eine Suchmaschine verwendet werden. Die nachfolgenden Fälle sollen diesen Ansatz verdeutlichen.
		
		Im Fall, dass der Vorname, Nachname und Wohnort der gesuchten Person eingegeben wird, kann mit Hilfe von herkömmlichen Suchmaschinen wie die von Google und Bing nach Information gesucht werden. Die von den Suchmaschinen vorgeschlagenen Seiten werden anschließend analysiert, ausgelesen und gespeichert. Dadurch können weitere Informationen gewonnen werden. Falls Benutzernamen von anderen Webseiten wie Instagram, Facebook oder ähnliches vorgeschlagen werden, kann somit die Suche mit diesen Daten speziell auf den entsprechenden Seiten erweitert werden.
		
		Ein weiterer Fall beschreibt das Szenario, wenn ein Benutzername der gesuchten Person in das Programm eingegeben wird. Hierbei handelt es sich um einen Benutzernamen von Social-Media-Webseiten wie Facebook, Instagram, et cetera. \\
		Zuallererst, wird hier die entsprechende Webseite nach Informationen zu dem angegebenen Benutzername durchsucht. Dadurch können zusätzliche Daten herausgefunden werden, die bei der weiteren Suche von Vorteil sind.\\
		Sobald die Webseite nach dem Nutzernamen durchsucht und ausgewertet wurde, kann mit herkömmlichen Suchmaschinen die Suche erweitert werden.
		
		\subsubsection{Personensuche ohne Suchmaschinen}
		\label{subsubsec:PersonensucheohneSuchmaschine}
		Unabhängig von den eingegebenen Daten, wird eine festgesetzte Anzahl von Webseiten durchsucht. Diese Art der Personensuche arbeitet allerdings ohne die Verwendung einer Suchmaschine. Vorschläge für die ausgewählten Webseiten sind Facebook, FuPa, Instagram, Xing, LinkedIn und Twitter.

	
	\subsection{Methoden zum Erkennen einer Person}
	\label{sec:WannhandeltessichumdiegesuchtePerson}
	Bei jeder einzelnen Suchvariante, besteht die Herausforderung darin, zu erkennen, wann es sich um die gesuchte Person handelt. Durch die große Anzahl an verfügbaren Informationen im Internet, besteht eine hohe Wahrscheinlichkeit, dass Personen mit sehr ähnlichen Profilen gefunden werden. Um diesem Problem entgegen zu wirken, kann die Art der Suche anhand den eingegebene Daten angepasst werden. Dies entspricht dem Ansatz im Kapitel \ref{sec:Suche nach Information}. Die Suche kann dadurch verfeinert werden und die Anzahl der fehlerhaften Vorschläge wird geringer. Dadurch wird die Wahrscheinlichkeit höher, dass es sich um die richtige Person handelt.\\
	Darüber hinaus kann die Personensuche mit einer Suchmaschine durch verbesserte Suchbefehle ebenfalls verfeinert werden. In dem Buch "'Open Source Intelligence Techniques"' \cite{Bazzell}, werden Suchbefehle für bekannte Suchmaschinen aufgezeigt, mit denen die Suche verbessert werden kann. Dies bedeutet, bei einer Personensuche ist es mit den richtigen Suchbefehlen möglich, die Anzahl der Vorschläge um einen großen Teil zu verringern. Ein Beispiel in dem Buch von Michael Bazzell zeigt, wie es funktioniert, die Suchergebnisse von 8770 Vorschlägen auf lediglich neun Vorschläge zu reduzieren.\cite{Bazzell} 
	Auch bei dieser Lösungsidee wird die Wahrscheinlichkeit erhöht, dass es sich um die gesuchte Person handelt.
	
	Im Fall dass nach diese Maßnahmen dennoch verschiedene Profile angezeigt werden, können die folgenden Erweiterungen in die Suche mit einfließen.
		\subsubsection{Erweiterte Kriterien}
		\label{sec:ErweiterteKriteriern}
		Hierbei handelt es sich um weitere Kriterien, welche die Suche noch mehr eingrenzen sollen. Bekannte Informationen über die Zielperson dienen dazu, die vorgeschlagenen Seiten einer Suchmaschine weiter zu filtern. Ausführlich bedeutet dies, dass das Programm in erster Linie nur die Webseite als Informationsquelle verwendet, die alle Suchbegriffe beinhaltet. Darüber hinaus kann das genaue oder grobe Alter der Zielperson mit in die Suche einfließen. Dadurch kann erkannt werden ob der Zeitrahmen des Artikels oder das Erstellungsdatum einer Webseite mit dem Alter der Person grundsätzlich übereinstimmt.
		%TODO Erklären wie man Zeitpunkt des Eintrages oder Alter der Webseite erkennen kann.
		\subsubsection{Kontakte der Suchperson werden in Betracht gezogen}	
		Hier kann die Suche erweitert werden, indem auf soziale und berufliche Verbindungen der Zielperson eingegangen wird. Das heißt, dass bekannte Kontakte der gesuchten Person ebenfalls durchsucht und ausgewertet werden. In diesem Fall könnten Facebook-Freunden, FuPa-Teammitglieder, Instagram-Follower oder LinkedIn/Xing-Kontakte als Kontaktquelle dienen.\\
		Durch dieses Verfahren können weitere Informationen gewonnen werden, die zur Unterscheidung von Profilen nützlich sein könnten.
		\subsubsection{Identifikationsschlüssel verwenden}
		Bekannte Information zur Person können als Identifikationsschlüssel verwendetet werden. Allerdings müssen dies einzigartige Daten sein. Als einzigartige Daten zählen beispielsweise die E-Mail-Adresse oder eine Verbindung von mehreren Daten, da der vollständige Name nicht einzigartig ist. Das heißt, häufig verwendete Namen können oft in Verbindung mit unterschiedlichen Personen im Internet vorkommen und sind dadurch nicht als Identifikationsschlüssel verwendbar. Des Weiteren, kann eine Zielperson auf einer Webseite einen erfundenen Benutzernamen und auf der nächsten Seite den vollständigen Namen verwenden.\\
		
		Im Fall das auch mit diesen Maßnahmen nicht die gesuchte Person identifiziert werden kann, können mehrere Personenprofile erstellt und angezeigt werde. Der Programm-Anwender kann anschließend aus den vorgeschlagenen Profilen eines auswählen.  
		
	\subsection{Methoden zum Erkennen von wichtigen Informationen auf einer Webseite}
	\label{subsec:ErkennenVonInformation}
	Für die Suche nach einer ausgewählten Person können verschiedenste Arten von Webseiten gefunden werden. Aus diesem Grund muss das Programm eine gewisse "'Intelligenz"' mit sich bringen um die wichtigsten Daten aus einer Seite herauszufiltern. Dabei ist es nicht möglich eine Hartkodierung zu verwenden, um festgelegte Bereiche einer Webseite auszulesen, da jede Webseite eine individuelle Struktur hat.\\
	Die Grundidee zur Lösung diese Problems ist die Analyse des vorliegenden Webseiten-Textes. Eine Methode zur Textanalyse ist die automatisierte Schlüsselwort-Gewinnung. Hierbei wird die HTML-Seite zu einem verwendbaren Text formatiert, wobei die meisten Sonderzeichen herausgefiltert werden. Sonderzeichen wie "'."' und "'@"' werden dabei nicht herausgefiltert, da sie für die E-Mail-Erkennung wichtig sind. Anschließend werden Schlüsselwörter aus dem formatierten Webseitentext generiert. Möglichkeiten zur automatisierten Schlüsselwortgenerierung sind die Verfahren RAKE \ref{sec:RAKE} und die Automatic Keyword Extraction mit NLP \ref{sec:Automatic Keyword Extraction}, welche im Laufe dieser Arbeit detailliert beschrieben werden.\\
	Nachdem die Schlüsselwörter generiert und in Listen gespeichert wurden, werden Wortsammlung erstellt. Diese Wortsammlungen sind Listen, welche aussagekräftige Schlüsselwörter enthalten und nach Themen kategorisiert werden. Beispiele für den Inhalt der Listen sind alle Hochschulen und Universitäten in Deutschland, Berufsbezeichnungen und Tätigkeiten, Studiengänge, Hobbybezeichnungen und alle Städte und Gemeinden in Deutschland.\\
	Mit diesen Wortsammlungen kann nun die Liste mit den bereits generierten Schlüsselwörtern aus dem Webseitentext verglichen werden. Bei einer Übereinstimmung eines Schlüsselwortes wird das Wort mit der entsprechenden Kategorie vorgemerkt und später in die verwendete Speicherstruktur eingetragen. \\
	Die Wortsammlungen werden mit Hilfe von bekannten Listen im Internet eigenständig befüllt. Als Informationsquelle dafür, dient jegliche Art von Webseite, die nützliche Information enthält.	
		\subsubsection{RAKE}
		\label{sec:RAKE}
		RAKE steht für \textit{Rapid Automatic Keyword Extraction} und stellt eine sehr effiziente Methode zur Schlüsselwortgenerierung dar. Die Funktion von RAKE basiert darin, dass Schlüsselwörter mehrere Wörter mit inhaltlicher Relevanz enthalten, allerdings selten Stoppwörter und Sonderzeichen.\cite{rose2010automatic}\\
		Als Stoppwörter werden Wörter bezeichnet, die sehr oft auftreten und keinen großen Informationsgewinn mit sich bringen. Beispiele dafür sind \textit{und}, \textit{weil}, \textit{der} oder \textit{als}.\cite{Stopwords}\\
		
		\begin{figure}[h!]
			\fbox{\parbox{\linewidth}{In einer jungen Wissenschaft wie der Informatik mit ihrer Vielschichtigkeit und ihrer unüberschaubaren Anwendungsvielfalt ist man oftmals noch bestrebt, eine Charakterisierung des Wesens dieser Wissenschaft und Gemeinsamkeiten und Abgrenzungen zu anderen Wissenschaften zu finden. Etablierte Wissenschaften haben es da leichter, sei es, dass sie es aufgegeben haben, sich zu definieren, oder sei es, dass ihre Struktur und ihre Inhalte allgemein bekannt sind.}}
			\caption{Beispieltext}
			\label{fig:text}
		\end{figure}
	%TODO QUELLE für Beispieltext einfügen.
		
		Zu Beginn wird der zu analysierende Text, hier der Beispieltext in  Bild \ref{fig:text}, durch einen Worttrenner in ein Array, bestehen aus möglichen Schlüsselwörtern, aufgeteilt. Das erzeugte Array wird anschließend in Sequenzen von zusammenhängenden Wörtern unterteilt. Dabei erhalten die Wörter in einer Sequenz die gleiche Position und Reihenfolge wie im Ursprungstext und dienen gemeinsam als Kandidatenschlüsselwort.\cite{rose2010automatic}\\		
		Nachdem die möglichen Schlüsselwörter identifiziert sind, wird für jeden einzelnen Kandidaten ein Score ausgerechnet. Dieser besteht aus dem Quotient des Grades $deg(w)$ und der Häufigkeit des Vorkommens eines Wortes innerhalb der Kandidaten $freq(w)$. Daraus ergibt sich die Formel:
		\begin{center}
			$deg(w)/freq(w) $
		\end{center}	
		
		Dabei beschreibt der Grad eines Wortes, dass gemeinsame Auftreten mit sich selbst und anderen Schlüsselwörtern. In der Tabelle \ref{tab:Co-occurance} ist der Grad für jedes Wort ablesbar, indem die Einträge in der entsprechenden Reihe summiert werden. Beispielsweise Beträgt der Grad des Wortes \textit{"'Wissenschaft"'} den Wert \textit{3}. Dies ergibt sich aus der Rechnung:
		\begin{center}
			$2 + 1 = 3$
		\end{center}
		Das Wort \textit{"'Wissenschaft"'} kommt hier selbst zweimal in dem Kandidaten-Array vor und davon einmal in Verbindung mit dem Worten "'jungen"'.\\
		Die Häufigkeit des Vorkommens eines Wortes lässt sich ebenfalls in der Tabelle \ref{tab:Co-occurance} ablesen. Allerdings muss hier in der Reihe und Spalte des jeweiligen Wortes nachgeschaut werden. Für das Wort \textit{"'Wissenschaft"'} beträgt die Häufigkeit des Vorkommens den Wert \textit{3}.\\
		Zusammenfassend kann gesagt werden, dass \textit{deg(w)} die Kandidaten bevorzugt, welche oft und in langen Schlüsselwörtern, die mehrere Wörter enthalten, vorkommen. Dies bedeutet, dass beispielsweise \textit{deg(etabliert)} eine höhere Bewertung als \textit{deg(informatik)} bekommt, obwohl beide Wörter gleich oft im Text vorkommen. Dagegen wird bei \textit{freq(w)}, ausschließlich die Häufigkeit des Vorkommens bewertet. Bei der Formel \textit{deg(w)/freq(w)} werden die Wörter bevorzugt, welche überwiegend in langen Kandidatenwörtern vorkommen. Diese Formel bietet dadurch einen guten Mittelweg zur Schlüsselwortgewinnung. Ein Beispiel dafür sind die Wörter \textit{"'Wissenschaften} und \textit{"'allgemein"'}. Hier ist der Quotient von \textit{deg(allgemein)/freq(allgemein)} höher als von \textit{deg(Wissenschaften)/freq(Wissenschaften)}, obwohl die Häufigkeit des Wortes \textit{"'Wissenschaften"'} höher und der Grad gleich hoch ist. \cite{rose2010automatic}
		%TODO Möglicherweise kann Tabelle mit deg(w), freq(w) und deg(w)/feq(w)
		
		Durch das genannte Verfahren und der Formel \textit{deg(w)/freq(w)} für die Bewertung, ergeben sich die im Bild \ref{fig:SchlüsselwörterMitScore} befindenden Kandidaten mit den dazugehörigem Endbewertungen. \cite{rose2010automatic}
		
		\begin{center}
			\begin{table}[h!]
				\scriptsize
				\begin{tabular}{*{24}{l|}}				
					\rotatebox[origin=c]{90}{} 
					&\rotatebox[origin=c]{90}{wissenschaften} &\rotatebox[origin=c]{90}{wissenschaft} &\rotatebox[origin=c]{90}{sei} &\rotatebox[origin=c]{90}{etablierte} &\rotatebox[origin=c]{90}{informatik} &\rotatebox[origin=c]{90}{aufgegeben} &\rotatebox[origin=c]{90}{gemeinsamkeiten} &\rotatebox[origin=c]{90}{oftmals} &\rotatebox[origin=c]{90}{charakterisierung} &\rotatebox[origin=c]{90}{jungen} &\rotatebox[origin=c]{90}{inhalte} &\rotatebox[origin=c]{90}{allgemein} &\rotatebox[origin=c]{90}{bekannt} &\rotatebox[origin=c]{90}{struktur} &\rotatebox[origin=c]{90}{wesens} &\rotatebox[origin=c]{90}{bestrebt} &\rotatebox[origin=c]{90}{unüberschaubaren} &\rotatebox[origin=c]{90}{anwendungsvielfalt} &\rotatebox[origin=c]{90}{definieren} &\rotatebox[origin=c]{90}{abgrenzungen}
					&\rotatebox[origin=c]{90}{leichter}
					&\rotatebox[origin=c]{90}{finden}
					&\rotatebox[origin=c]{90}{vielschichtigkeit}\\
					\hline
					wissenschaften & 2 & & & 1 & & & & & & & & & & & & & & & & & & &\\
					\hline
					wissenschaft & & 2 & & & & & & & & 1 & & & & & & & & & & & & & \\
					\hline
					sei & & & 1 & & & & & & & & & & & & & & & & & & & &	\\
					\hline
					etablierte & 1 & & &1 & & & & & & & & & & & & & & & & & & &	\\
					\hline
					informatik & & & & &1 & & & & & & & & & & & & & & & & & & \\
					\hline
					aufgegeben & & & & & &1 & & & & & & & & & & & & & & & & &	\\
					\hline
					gemeinsamkeiten & & & & & & & 1& & & & & & & & & & & & & & & &	\\
					\hline
					oftmals & & & & & & & & 1& & & & & & & & & & & & & & &\\
					\hline
					charakterisierung & & & & & & & & & 1& & & & & & & & & & & & & & \\
					\hline
					jungen & & 1 & & & & & & & & 1 & & & & & & & & & & & & &	\\
					\hline
					inhalte & & & & & & & & & & & 1 & 1 & 1 & & & & & & & & & &	\\
					\hline
					allgemein & & & & & & & & & & & 1 & 1 & 1 & & & & & & & & & & \\
					\hline
					bekannt & & & & & & & & & & & 1 & 1 & 1 & & & & & & & & & &	\\
					\hline
					struktur & & & & & & & & & & & & & &1 & & & & & & & & &	\\
					\hline
					wesens & & & & & & & & & & & & & & &1 & & & & & & & &\\
					\hline
					bestrebt & & & & & & & & & & & & & & & & 1& & & & & & & \\
					\hline
					unüberschaubaren & & & & & & & & & & & & & & & & & 1 & 1 & & & & &	\\
					\hline
					anwendungsvielfalt & & & & & & & & & & & & & & & & & 1 & 1 & & & & &	\\
					\hline
					definieren & & & & & & & & & & & & & & & & & & & 1 & & & & \\
					\hline
					abgrenzungen & & & & & & & & & & & & & & & & & & & & 1 & & &	\\
					\hline
					leichter & & & & & & & & & & & & & & & & & & & & & 1 & &	\\
					\hline
					finden & & & & & & & & & & & & & & & & & & & & & & 1 & \\
					\hline
					vielschichtigkeit & & & & & & & & & & & & & & & & & & & & & & & 1	\\
					\hline
				\end{tabular}
				\label{tab:Co-occurance}
				\caption{Co-occurance}
			\end{table}
		\end{center}
	
		\begin{figure}[h!]
			\fbox{\parbox{\linewidth}{ inhalte allgemein bekannt (9.0), unüberschaubaren anwendungsvielfalt (4.0), jungen wissenschaft(3.5), etablierte wissenschaften (3.5), wissenschaften (1.5), wissenschaft (1.5), wesens (1.0), vielschichtigkeit (1.0), struktur (1.0), sei (1.0), oftmals (1.0), leichter (1.0), informatik (1.0), gemeinsamkeiten (1.0), finden (1.0), definieren (1.0), dass (1.0), charakterisierung (1.0), bestrebt (1.0), aufgegeben (1.0), abgrenzungen (1.0)}}
			\caption{Schlüsselwörter mit zugehörigem Score}
			\label{fig:SchlüsselwörterMitScore}
		\end{figure}
	\FloatBarrier
	
		\subsubsection{Automatic Keyword Extraction mit NLP}
		\label{sec:Automatic Keyword Extraction}
		Bei dieser Methode wird der vorliegende Text in die einzelnen Wörter unterteilt. Dabei wird eine Liste mit potentiellen Schlüsselwörtern erstellt, in der \textit{Stoppwörter} und Sonderzeichen herausgefiltert werden. Bei den Schlüsselwörtern handelt es sich nicht ausschließlich um ein Wort sondern auch um Wortsequenzen.\\
		Mit Hilfe von Stemming kann nun die Anzahl der Wörter in der Liste weiter reduziert werden, wodurch eine bessere Schlüsselwortgenerierung möglich ist. \\
		Die Liste mit den möglichen Schlüsselwörtern, kann nach der Häufigkeit des Vorkommens eines Wortes im Text sortiert werden. Das hat den Vorteil, dass die Schlüsselwörter, welche am Häufigsten im Text vorkommen, in den darauf folgenden Schritten zuerst verwendet werden und dadurch eine Laufzeitverbesserung der Anwendung entsteht.\\
		Ergänzende Regeln wie, eine Mindestanzahl von Buchstaben in einem Wort, können die Schlüsselwörter weiter begrenzen. 

	
\section{Konzept zur Informationsbeschaffung von einer großen Menge unbekannter Personen}
Für die \textit{real-world} Simulation eines Phishing-Mail-Angriffs eine Webseiten mit einer großen Menge von personenbezogenen Daten benötigt. 	Hierfür wird manuell nach einer Webseite gesucht, die eine große Menge an personenbezogenen Daten enthält. Diese wird anschließen als Informationsquelle festgelegt. Möglichkeiten, ausgenommen von den bekannten Social Media Seiten, sind die Webseiten FuPa, Xing und LinkedIn.\\
	\subsection{Methode zur Suche nach Information}
	In diesem Konzept gibt es keine automatisierte Suche nach Informationen, jedoch eine automatisierte Suche nach internen Links. Diese interne Suche kann mit einem Web Crawler realisiert werden. In Vorbereitung darauf wird der Aufbau der Seite analysiert.\\
	
	\subsection{Methode zum Auslesen der Information}
	Zum Auslesen einer großen Menge an Daten wird ein Web Scraper erstellt. Dieser könnte für die ausgewählte Webseite hartkodiert werden. Eine Alternative dazu, wäre die Analyse des Webseitentextes, was dem Ansatz \ref{subsec:ErkennenVonInformation} von der Suchfunktion einer ausgewählten Person entsprechen würde.

%TODO BILDER von Webseite einfügen für groben Überblick oder in Umsetztung
\section{Konzept zur Erstellung einer Phishing-Mail}
Die Generierung einer Phishing-Mail läuft voll automatisch ab. Das bedeutet, dass das Programm eigenständig die E-Mail-Adressen generiert und selbst passende E-Mail-Muster auswählt.
	\subsection{Methoden zur Generierung von E-Mail-Adressen}
	Eine Möglichkeit zur Generierung der E-Mail-Adressen kann das Open Source-Tool von Michael Bazzell \cite{EmailAssumptions} sein, welches mit Hilfe eines automatisierten Webbrowsers verwendet werden kann. Bei diesem Tool werden zuerst über ein Formular, Daten für die E-Mail-Generierung eingetragen. Unter anderem sind das Vorname, Nachname und der E-Mail-Provider. Daraufhin werden die vorgeschlagenen E-Mail-Adressen angezeigt,kopiert und in ein Suchfeld eingefügt. Anschließend kann bei Google, Bing, und Facebook nach Einträgen gesucht und falls ein Eintrag gefunden wurde auch angezeigt werden.

	Eine Weitere Möglichkeit wäre ein Algorithmus zu entwickeln, der alle möglichen E-Mail-Adressen aus den Kombinationen von Vorname, Nachname, Geburtsjahr, Benutzernamen und den Domains von den bekanntesten E-Mail-Providern generiert. Dazu gehören \textit{GMX}, \textit{WEB.DE}, \textit{Gmail}, \textit{T-Online}, \textit{Freenet} und \textit{1\&1}.\cite{AnbieterMail} \\
	Für den Fall, dass der Arbeitgeber der Zielperson bekannt ist, kann auf der Firmenwebseite nach E-Mail-Adressen gesucht werden. Dadurch ist es möglich die Domain einer Firmen-Mailadresse zu bestimmen und eine Anzahl  möglicher Firmenadressen für die Zielperson zu generieren.\\
	Schon bei der Suche von personenbezogenen Daten wird ebenfalls nach E-Mail-Adressen gesucht. Dadurch kann bereits eine bis jetzt unbekannte Anzahl von Adressen gefunden werden.

%TODO Erwähnen dass facbook nach emails suchen konnte

	\subsection{Methode zur Erstellung von E-Mail-Mustern}
	Für die Erstellung der E-Mail-Muster kann eine eigene Klasse erstellt werden, welche für die Erzeugung des Textes zuständig ist. In dieser Klasse werden Strings gespeichert die einem Lückentext ähneln. Abhängig von den gefundenen Daten wird ein Lückentext ausgewählt, welcher anschließend mit den Daten an den passenden Lücken ergänzt wird. Mit dieser Methode muss jedoch für jede Kombination aus gewonnenen Daten ein Lückentext vorhanden sein.\\
	Die Lückentexte werden so kategorisiert, dass für jede gefundene Information ein passender Lückentext vorhanden ist. Eine denkbare Unterteilung wäre in die Kategorien Privat und Geschäftlich.

		% des Hauptteils

%Kapitel des Hauptteils

\chapter{Auswahl der Lösung anhand den Anforderungen}  %Name des Kapitels
\label{cha:Auswahl der Lösung anhand Anforderungen} %Label des Kapitels
\section{Informationsbeschaffung} %Unterkapitel
	\subsection{Informationsbeschaffung von bestimmten/ausgewählten Personen}
	
	\subsection{Informationsbeschaffung von einer großen Menge unbestimmter Personen}


\section{Datenanalyse/-speicherung}


\section{Generierung der E-Mail-Adressen}
%TODO Anzahl der Möglichen E-Mail-adressen ausrechenen

\section{Erstellung der E-Mail-Muster}

\section{Erzeugung der Phishing-Mail}		% sollten hier in der
%Kapitel der Umsetzung

\chapter{OSINT einer ausgewählten Person}  %Name des Kapitels
\label{cha:Informationsbeschaffung einer ausgewählten Person} %Label des Kapitels

\section{Programmiersprache}
Damit das Programm anhand den Lösungsideen umgesetzt werden kann, ist der erste Schritt die Auswahl der Programmiersprache.\\
\textbf{Anforderung an das Programm bzw. an die Programmiersprache}\\
	Es soll eine möglichst übersichtliche und performante Skriptsprache verwendet werden, mit der eine automatisierte Informationsbeschaffung gut möglich ist. Eine Eingabe über die Konsole oder über eine graphische Benutzeroberfläche soll ebenfalls möglich sein. Aus diesem muss die Programmiersprache keine GUI-Programmierung mit sich bringen.\\
\textbf{Lösungsideen für Programmiersprache}\\
	Für die Auswahl der Programmiersprache gibt es viele Auswahlmöglichkeiten. Allerdings bringt die Programmiersprache Pyhton, alle Nötigen Eigenschaften mit sich.\\
	%TODO Warum Python? Antworten finden Welche Eingeschaften
	Für die Eingabe von Suchdaten, besteht für beide Informationsbeschaffungen die Möglichkeit eine Grafische-Bedienoberfläche oder Konsolen-Eingabe zu verwenden.\\
\textbf{Bewertung Programmiersprache}\\
	Mit der Programmiersprache Python lässt sich das Programm entsprechend den Anforderungen entwickeln und es kann sowohl eine Konsolenanwendung als auch eine Oberflächenanwendung programmiert werden. Es bringt alle Module mit sich um das Projekt mit dem vorgegebenen Zielen umzusetzen. Außerdem eignet sich Python sehr gut für die Bearbeitung von linguistischen Daten. \cite{bird2009natural}
	%TODO So geht das nicht: Schnappen Sie sich einen Schwung Sprachen, stellen Sie die vor, jeweils, neutral. Und dann können sie sagen: Die nicht, die nicht... Uuups, es bleibt nur Python übrig.
	
\section{Methoden zur Suche nach einer Person im Internet}
\label{sec:Suche nach Information}
Für die Suche einer Person im Internet, wird abhängig von den eingegebenen Daten, des Programm-Anwenders, die Art der Suche angepasst. Genau genommen heißt das, dass die eingegebenen Daten vor der Suche analysiert werden und dementsprechend die Suche danach angepasst wird. \\
Die Art der Personensuche lässt sich in zwei mögliche Methoden gliedern.

	\subsection{Personensuche mit Hilfe von Suchmaschinen}
	\label{subsubsec:PersonensucheMitHilfevonSuchmaschine}
	Hier wird mit Hilfe einer Suchmaschine nach Informationen gesucht. Mögliche Suchmaschinen sind die von Google und Bing. Allerdings muss nicht für jede Suche eine Suchmaschine verwendet werden. Die nachfolgenden Fälle sollen diesen Ansatz verdeutlichen.
	
	Im Fall, dass der Vorname, Nachname und Wohnort der gesuchten Person eingegeben wird, kann mit Hilfe von herkömmlichen Suchmaschinen wie die von Google und Bing nach Information gesucht werden. Die von den Suchmaschinen vorgeschlagenen Seiten werden anschließend analysiert, ausgelesen und gespeichert. Dadurch können weitere Informationen gewonnen werden. Falls Benutzernamen von anderen Webseiten wie Instagram, Facebook oder ähnliches vorgeschlagen werden, kann somit die Suche mit diesen Daten speziell auf den entsprechenden Seiten erweitert werden.
	
	Ein weiterer Fall beschreibt das Szenario, wenn ein Benutzername der gesuchten Person in das Programm eingegeben wird. Hierbei handelt es sich um einen Benutzernamen von Social-Media-Webseiten wie Facebook, Instagram, et cetera. \\
	Zuallererst, wird hier die entsprechende Webseite nach Informationen zu dem angegebenen Benutzername durchsucht. Dadurch können zusätzliche Daten herausgefunden werden, die bei der weiteren Suche von Vorteil sind.\\
	Sobald die Webseite nach dem Nutzernamen durchsucht und ausgewertet wurde, kann mit herkömmlichen Suchmaschinen die Suche erweitert werden.
	
	\subsection{Personensuche ohne Suchmaschinen}
	\label{subsubsec:PersonensucheohneSuchmaschine}
	Unabhängig von den eingegebenen Daten, wird eine festgesetzte Anzahl von Webseiten durchsucht. Diese Art der Personensuche arbeitet allerdings ohne die Verwendung einer Suchmaschine. Vorschläge für die ausgewählten Webseiten sind Facebook, FuPa, Instagram, Xing, LinkedIn und Twitter.
	
\section{Bewertung: Art der Personensuche}
Um möglichst viele Informationen über eine Person im Internet zu finden, bietet die Personensuche mit der Verwendung einer Suchmaschine die beste Lösung. Es wird anstatt ausschließlich festgelegten Seiten das ganze Internet durchsucht. Dadurch können wesentlich mehr individuelle Einträge gefunden werden. Des Weiteren wird keine Logik zur Suche nach Einträgen im Internet benötigt, da lediglich den vorgeschlagenen Suchergebnissen gefolgt werden kann.\\
Allerdings muss beachtet werden, dass Benutzer bei verschiedensten Social-Media-Seiten auswählen können, ob das Benutzerprofil von einer Suchmaschine gefunden werden kann oder nicht. Aus diesem Grund, werden bei dieser Suche die Ergebnisse kontrolliert ob sich die geforderten Seiten darin befinden. Wenn das nicht der Fall ist, wird separat auf den festgelegten Seiten nach Information gesucht. Zu diesen Webseiten zählen beispielsweise \textit{XING} und \textit{LinkedIn}.\\
%TODO Google oder Bing?!
	
\section{Umsetzung: Personensuche mit Hilfe der Google-Suchmaschine im Internet}
Für die Personensuche im Internet wird die Google-Suchmaschine verwendet. Gesucht wird nach den eingegebenen Daten. Dafür werden die Daten über eine Konsole eingelesen.

	\subsection{Eingabe der bekannten Daten}
	Es besteht die Möglichkeit den \textbf{Vorname, Nachname, Wohnort, Arbeitgeber, Instagram Benutzername, Facebook Benutzername, Twitter Benutzername}, und das genaue beziehungsweise geschätzte \textbf{Geburtsjahr} der gesuchten Person über eine Konsole einzugeben. Falls der genaue Jahrgang der Zielperson nicht bekannt ist, kann ein geschätztes Geburtsjahr eingetragen werden. Dies kann später bei der Identifizierung der gesuchten Person verwendet werden.\\
	Zu Beginn werden alle Personen-Variablen mit einem leeren String initialisiert. Das bedeutet dass all die Variablen, zu denen keine Information eingegeben wurde, einen leeren String enthalten.
	%TODO Eingabe kann gespeichert werden wenn z.B. Wohnort noch nicht in Wortsamnmlung gespeichert wurde.
	
		\subsubsection{Verarbeitung der Daten}
		Zu Beginn der Anwendung werden Abfragen gemacht, um zu erkennen in welchen Variablen sich Information befindet. Anschließend werden mit diesen Variable Kombinationen für die spätere URL-Generierung erstellt. Mögliche Kombinationen für erfolgreiche Suchergebnisse sind:\\
		Vorname, Nachname;\\
		Vorname, Nachname, Wohnort;\\
		Vorname, Nachname, Geburtsjahr;\\
		Vorname, Nachname, Arbeitgeber;\\
		Vorname, Nachname, Benutzername einer Social-Media-Seite;\\
		Vorname, Nachname, Wohnort, Geburtsjahr;\\
		Die Kombination aus vielen oder allen Daten ist ebenfalls eine mögliche Option, allerdings wird dadurch oft kein Ergebnis gefunden, da nicht zur jeder Information ein Eintrag im Internet ist.\\
		Sobald die Kombinationen aus den Daten bekannt sind, werden die Such-URLs für die Google-Suchmaschine generiert.
			
		\subsection{Aufbau Google Such-URL}
		\label{subsec:AufbauGoogleURL}
		Für den Aufbau eines Google-URLs gilt, dass der URL-Teil "'https://www.google.com/search?"', bis auf die Protokolle HTTP und HTTPS, gleich bleibt. Des Weitern repräsentieren "'\%20"' ein Leerzeichen und "'\%22"' ein Anführungszeichen. \\
		Hier befindet sich ein Beispiel link:\\
		\textbf{https://www.google.com/search?q=marco+lang}
		%TODO Aufbau URL erklären und Quelle finden.
		
			\subsubsection{Such-URL optimieren}
			Um die Suchergebnisse zu verbessern, können die Suchbegriffe in Anführungszeichen gesetzt werden. Das bedeutet, dass ausschließlich nach diesen Begriffen gesucht wird und nicht nach einer Abwandlung. Ein Beispiel hierfür ist die Suche nach "'Mike Bazzell"'. Wenn diese Suche ohne Anführungszeichen durchgeführt wird, werden Webseiten vorgeschlagen die den Namen Mike Bazzell anstatt Micheal Bazzell beinhalten. Diese erweiterte Suche kann dazu führen, dass unzählige Webseiten vorgeschlagen werden, die nicht unbedingt was mit dem Thema der Suchbegriffe zu tun hat. Um dem vorzubeugen können Anführungzeichen verwendet werden, welche die Anzahl der Suchergebnisse um einen sehr großen Teil verringern wird. \cite{Bazzell}\\
			Für die Suche nach \textbf{Marco Lang} werden ungefähr \textbf{96.400.000} Ergebnisse mit Hilfe der Google-Suchmaschine gefunden. Wird die Suche mit den Anführungszeichen verfeinert, werden für \textbf{"'Marco"' "'Lang"'} etwa \textbf{55.600.000} Ergebnisse gefunden. Allerdings werden hier Webseiten vorgeschlagen, welche die Wörter "'Marco"' und "'Lang"' beinhalten, jedoch müssen diese nicht direkt nebeneinander und auch nicht in der Reihenfolge vorkommen. Es wäre Möglich, dass bei dieser Suche Webseite mit Referenzen auf die Namen "'Marco Mustermann"' und "'Max Lang"' beinhaltet. Aus diesem Grund kann nach \textbf{"'Marco Lang"'} gegoogelt werden. Dadurch wird die Anzahl der Suchergebnisse auf \textbf{45.500} Ergebnisse reduziert, da nun ausschließlich die Webseiten vorgeschlagen werden, die den kompletten String "'Marco Lang"' beinhalten. Für eine weitere Optimierung der Ergebnisse, der Wohnort hinzugefügt, wie in dem Beispiel \textbf{"'Marco Lang"' Tettnang}. Dadurch werden die Suchvorschläge auf lediglich \textbf{113} Ergebnisse reduziert. Der URL zu dieser optimierten Suche lautet: 
			
			\textit{https://www.google.com/search?q=\%22Marco+Lang\%22+Tettnang}
			
			Nicht nur die Reduzierung der Suchergebnisse, sondern auch das herausfiltern von unerwünschten Webseiten hat einen positiven Effekt auf diese Arbeit. Die vorgeschlagenen Seiten müssen nämlich in den folgenden Schritten analysiert werden. Das bedeutet, dass jede unerwünschte Seite die allein durch die Suche herausgefiltert werden kann, einen großen Laufzeitvorteil mit sich bringt. 
			
		\subsection{Erstellen des eigenen Such-URLs}
		In diesem Absatz wird beschrieben wie Google-URLs zur Suche, mit dem Wissen aus Kapitel \ref{subsec:AufbauGoogleURL}, erstellt werden.\\
		Für jede genannte Kombination aus den eingegebenen Daten werden Link-Muster erzeugt, die einem Lückentext entsprechen. Sobald die entsprechenden Muster ausgewählt wurden, werden die Lücken mit den Daten befüllt. Dadurch wird eine variierende Anzahl von Suchlinks erstellt.\\
		Wenn allerdings der Benutzername einer Social-Media-Seite bekannt ist, wird ein anderer Aufbau des Suchlinks verwendet, da speziell nach Einträgen auf der entsprechenden Webseite gesucht wird.
		
			\subsubsection{URL für beliebige Webseiten}
			https://www.google.com/search?q=\%22Marco+Lang\%22+Tettnang
			\subsubsection{URL für Social-Media-Seiten}
			https://www.google.com/search?q=site\%3Ainstagram.com+\%22Lamarcong\%22
			\\Probleme mit Facebook
			%TODO Social Media URL, bei der nur seite durchsucht wird
	
		
			
		
		\subsection{Mit welcher Bibliothek werden Serveranfragen umgesetzt?}
		Damit eine Person im Internet gesucht werden kann, muss das Programm dazu in der Lage sein, Anfragen an einen Server zu schicken. \\
		Um Anfragen an einen Server zu versenden, kann die Python "'request"' Bibliothek verwendet werden. Eine Alternative dazu wäre ein automatisierten Web-Browser, welcher mit Hilfe der Selenium Webdriver Bibliothek erstellt werden kann.\\
		Für einfach Anfragen an einen Server eignet sich die request Bibliothek von Python sehr gut. Des Weiteren hat die Bibliothek einen großen Laufzeit-Vorteil gegenüber dem automatisierten Webbrowser. Allerdings lässt sich mit der request Bibliothek keine Javascript-Seite anfordern.\\
		Da Webseiten wie Facebook und Xing Javascript verwenden und diese Seiten elementar für diese Arbeit sind, wird ein automatisierter Webbrowser für die Suche nach einer Person verwendet.
	
		\subsection{Web Crawler erstellen}
		Nachdem der automatisierte Browser und die Personensuche implementiert wurde, wird ein Web Crawler benötigt um den, von den Suchmaschinen, vorgeschlagenen Seiten, zu folgen. Dazu muss die Google-Seite mit den Vorschlägen analysiert werden, damit erkannt werden kann wo sich die vorgeschlagenen Links auf der Seite befinden. Diesen Links kann anschließend gefolgt werden.
			\subsubsection{Googel-Suchseite analysieren}
			Wie werden Links herausgesucht?
			Wie werden korrekte links erkannt?
			%TODO Weiter schreiben
			
\section{Methoden zum Erkennen einer Person}
\label{sec:WannhandeltessichumdiegesuchtePerson}
Bei jeder einzelnen Suche, besteht die Herausforderung darin, zu erkennen, wann es sich um die gesuchte Person handelt. Durch die große Anzahl an verfügbaren Informationen im Internet, besteht eine hohe Wahrscheinlichkeit, dass Personen mit sehr ähnlichen Profilen gefunden werden.\\
Im Fall das auch mit den folgenden Methoden nicht die gesuchte Person identifiziert werden kann, können mehrere Personenprofile erstellt und angezeigt werde. Der Programm-Anwender kann anschließend aus den vorgeschlagenen Profilen eines auswählen.

	\subsection{Art der Suche anpassen}
	Um diesem Problem entgegen zu wirken, kann die Art der Suche anhand den eingegebene Daten angepasst werden. Dies entspricht dem Ansatz im Kapitel \ref{sec:Suche nach Information}. Die Suche kann dadurch verfeinert werden und die Anzahl der fehlerhaften Vorschläge wäre geringer. Dies hätte zur Folge, dass die Wahrscheinlichkeit steigt, dass es sich um die richtige Person handelt.\\
	
	\subsection{Verbesserte Suchbefehle verwenden}
	Darüber hinaus kann die Personensuche durch verbesserte Suchbefehle verfeinert werden. In dem Buch \cite{Bazzell}, werden Suchbefehle für bekannte Suchmaschinen aufgezeigt, mit denen die Suche verbessert werden kann. Dies bedeutet, bei einer Personensuche ist es mit den richtigen Suchbefehlen möglich, die Anzahl der Vorschläge um einen großen Teil zu verringern. Ein Beispiel in dem Buch von Michael Bazzell zeigt, wie es funktioniert, die Suchergebnisse von 8770 Vorschlägen auf lediglich neun Vorschläge zu reduzieren.\cite{Bazzell}\\ 
	Auch bei dieser Lösungsidee wird die Wahrscheinlichkeit erhöht, dass es sich um die gesuchte Person handelt.
	
	\subsection{Zeitraum beachten}
	Eine weitere Methode für das Erkennen von Personen kann das Beachten von Zeiträumen sein. Dabei fließt das Alter der Zielperson mit in die Suche ein. Das bedeutet, dass nach dem Alter der Webseite gesucht wird, indem Jahreszahlen aus dem Webseitentext ausgelesen werden Dadurch wird erkannt, ob der Zeitrahmen des Artikels oder das Erstellungsdatum einer Webseite mit dem Alter der Person grundsätzlich übereinstimmt.
%TODO Erklären wie man Zeitpunkt des Eintrages oder Alter der Webseite erkennen kann.

	\subsection{Kontakte der Suchperson werden in Betracht gezogen}	
	Hier kann die Suche erweitert werden, indem auf soziale und berufliche Verbindungen der Zielperson eingegangen wird. Das heißt, dass bekannte Kontakte der gesuchten Person ebenfalls durchsucht und ausgewertet werden. In diesem Fall könnten Facebook-Freunden, FuPa-Teammitglieder, Instagram-Follower oder LinkedIn/Xing-Kontakte als Kontaktquelle dienen.\\
	Durch dieses Verfahren können weitere Informationen gewonnen werden, die zur Unterscheidung von Profilen nützlich sein könnten.
	
	\subsection{Identifikationsschlüssel verwenden}
	Bekannte Information zur Person können als Identifikationsschlüssel verwendetet werden. Allerdings müssen dies einzigartige Daten sein. Als einzigartige Daten zählen beispielsweise die E-Mail-Adresse oder eine Verbindung von mehreren Daten, da der vollständige Name nicht einzigartig ist. Das heißt, häufig verwendete Namen können oft in Verbindung mit unterschiedlichen Personen im Internet vorkommen und sind dadurch nicht als Identifikationsschlüssel verwendbar. Des Weiteren, kann eine Zielperson auf einer Webseite einen erfundenen Benutzernamen und auf der nächsten Seite den vollständigen Namen verwenden.\\
	

	
\section{Bewertung: Die gesuchten Person erkennen}
Grundsätzlich gilt, dass alle Methoden zur Erkennung einer Person eine Verbesserungen der Ergebnisse mit sich bringen. Allerdings gibt es Unterschied in der Wirksamkeit und in der Laufzeit des Programms.

Die Erweiterung der Kriterien \ref{sec:ErweiterteKriteriern} bringt keine große Laufzeitänderung mit sich und stellt eine sehr gute Eigenschaft zur Optimierung der Informationsfindung dar, da die Zeit ebenfalls mit einbezogen wird.
\section{Umsetzung: Die gesuchte Person erkennen}
	\subsection{Zeitrahmen wird mit Beachtet}
		\subsubsection{Wie kann Alter der Webseite herausgefunden werden}
	\subsection{Kontakte in Betracht ziehen}
		\subsubsection{Auf welcher Seite können mögliche Kontakte gefunden werden}
		\subsubsection{Wie werden Kontakte ausgelesen?}
	\subsubsection{Identifikationsschlüssel erstellen}
		\subsubsection{Was dient als Identifikationsschlüssel}
		
		
\section{Methoden zum Erkennen von wichtigen Informationen auf einer Webseite}
\label{subsec:ErkennenVonInformation}
Für die Suche nach einer ausgewählten Person können verschiedenste Arten von Webseiten gefunden werden. Aus diesem Grund muss das Programm eine gewisse "'Intelligenz"' mit sich bringen um die wichtigsten Daten aus einer Seite herauszufiltern. Dabei ist es nicht möglich eine Hartkodierung zu verwenden, um festgelegte Bereiche einer Webseite auszulesen, da jede Webseite eine individuelle Struktur hat.\\
Die Grundidee zur Lösung diese Problems ist die Analyse des vorliegenden Webseiten-Textes. Eine Methode zur Textanalyse ist die automatisierte Schlüsselwort-Gewinnung. Hierbei wird die HTML-Seite zu einem verwendbaren Text formatiert, wobei die meisten Sonderzeichen herausgefiltert werden. Sonderzeichen wie "'."' und "'@"' werden dabei nicht herausgefiltert, da sie für die E-Mail-Erkennung wichtig sind. Anschließend werden Schlüsselwörter aus dem formatierten Webseitentext generiert. Möglichkeiten zur automatisierten Schlüsselwortgenerierung sind die Verfahren RAKE \ref{sec:RAKE} und die Automatic Keyword Extraction mit NLP \ref{sec:Automatic Keyword Extraction}, welche im Laufe dieser Arbeit detailliert beschrieben werden.\\
Nachdem die Schlüsselwörter generiert und in Listen gespeichert wurden, werden Wortsammlung erstellt. Diese Wortsammlungen sind Listen, welche aussagekräftige Schlüsselwörter enthalten und nach Themen kategorisiert werden. Beispiele für den Inhalt der Listen sind alle Hochschulen und Universitäten in Deutschland, Berufsbezeichnungen und Tätigkeiten, Studiengänge, Hobbybezeichnungen und alle Städte und Gemeinden in Deutschland.\\
Mit diesen Wortsammlungen kann nun die Liste mit den bereits generierten Schlüsselwörtern aus dem Webseitentext verglichen werden. Bei einer Übereinstimmung eines Schlüsselwortes wird das Wort mit der entsprechenden Kategorie vorgemerkt und später in die verwendete Speicherstruktur eingetragen. \\
Die Wortsammlungen werden mit Hilfe von bekannten Listen im Internet eigenständig befüllt. Als Informationsquelle dafür, dient jegliche Art von Webseite, die nützliche Information enthält.

	\subsection{RAKE}
	\label{sec:RAKE}
	RAKE steht für \textit{Rapid Automatic Keyword Extraction} und stellt eine sehr effiziente Methode zur Schlüsselwortgenerierung dar. Die Funktion von RAKE basiert darin, dass Schlüsselwörter mehrere Wörter mit inhaltlicher Relevanz enthalten, allerdings selten Stoppwörter und Sonderzeichen.\cite{rose2010automatic}\\
	Als Stoppwörter werden Wörter bezeichnet, die sehr oft auftreten und keinen großen Informationsgewinn mit sich bringen. Beispiele dafür sind \textit{und}, \textit{weil}, \textit{der} oder \textit{als}.\cite{Stopwords}\\
	
	\begin{figure}[h!]
		\fbox{\parbox{\linewidth}{In einer jungen Wissenschaft wie der Informatik mit ihrer Vielschichtigkeit und ihrer unüberschaubaren Anwendungsvielfalt ist man oftmals noch bestrebt, eine Charakterisierung des Wesens dieser Wissenschaft und Gemeinsamkeiten und Abgrenzungen zu anderen Wissenschaften zu finden. Etablierte Wissenschaften haben es da leichter, sei es, dass sie es aufgegeben haben, sich zu definieren, oder sei es, dass ihre Struktur und ihre Inhalte allgemein bekannt sind.}}
		\caption{Beispieltext}
		\label{fig:text}
	\end{figure}
	%TODO QUELLE für Beispieltext einfügen.
	
	Zu Beginn wird der zu analysierende Text, hier der Beispieltext in  Bild \ref{fig:text}, durch einen Worttrenner in ein Array, bestehen aus möglichen Schlüsselwörtern, aufgeteilt. Das erzeugte Array wird anschließend in Sequenzen von zusammenhängenden Wörtern unterteilt. Dabei erhalten die Wörter in einer Sequenz die gleiche Position und Reihenfolge wie im Ursprungstext und dienen gemeinsam als Kandidatenschlüsselwort.\cite{rose2010automatic}\\		
	Nachdem die möglichen Schlüsselwörter identifiziert sind, wird für jeden einzelnen Kandidaten ein Score ausgerechnet. Dieser besteht aus dem Quotient des Grades $deg(w)$ und der Häufigkeit des Vorkommens eines Wortes innerhalb der Kandidaten $freq(w)$. Daraus ergibt sich die Formel:
	\begin{center}
		$deg(w)/freq(w) $
	\end{center}	
	
	Dabei beschreibt der Grad eines Wortes, dass gemeinsame Auftreten mit sich selbst und anderen Schlüsselwörtern. In der Tabelle \ref{tab:Co-occurance} ist der Grad für jedes Wort ablesbar, indem die Einträge in der entsprechenden Reihe summiert werden. Beispielsweise Beträgt der Grad des Wortes \textit{"'Wissenschaft"'} den Wert \textit{3}. Dies ergibt sich aus der Rechnung:
	\begin{center}
		$2 + 1 = 3$
	\end{center}
	Das Wort \textit{"'Wissenschaft"'} kommt hier selbst zweimal in dem Kandidaten-Array vor und davon einmal in Verbindung mit dem Worten "'jungen"'.\\
	Die Häufigkeit des Vorkommens eines Wortes lässt sich ebenfalls in der Tabelle \ref{tab:Co-occurance} ablesen. Allerdings muss hier in der Reihe und Spalte des jeweiligen Wortes nachgeschaut werden. Für das Wort \textit{"'Wissenschaft"'} beträgt die Häufigkeit des Vorkommens den Wert \textit{3}.\\
	Zusammenfassend kann gesagt werden, dass \textit{deg(w)} die Kandidaten bevorzugt, welche oft und in langen Schlüsselwörtern, die mehrere Wörter enthalten, vorkommen. Dies bedeutet, dass beispielsweise \textit{deg(etabliert)} eine höhere Bewertung als \textit{deg(informatik)} bekommt, obwohl beide Wörter gleich oft im Text vorkommen. Dagegen wird bei \textit{freq(w)}, ausschließlich die Häufigkeit des Vorkommens bewertet. Bei der Formel \textit{deg(w)/freq(w)} werden die Wörter bevorzugt, welche überwiegend in langen Kandidatenwörtern vorkommen. Diese Formel bietet dadurch einen guten Mittelweg zur Schlüsselwortgewinnung. Ein Beispiel dafür sind die Wörter \textit{"'Wissenschaften} und \textit{"'allgemein"'}. Hier ist der Quotient von \textit{deg(allgemein)/freq(allgemein)} höher als von \textit{deg(Wissenschaften)/freq(Wissenschaften)}, obwohl die Häufigkeit des Wortes \textit{"'Wissenschaften"'} höher und der Grad gleich hoch ist. \cite{rose2010automatic}
	%TODO Möglicherweise kann Tabelle mit deg(w), freq(w) und deg(w)/feq(w)
	
	Durch das genannte Verfahren und der Formel \textit{deg(w)/freq(w)} für die Bewertung, ergeben sich die im Bild \ref{fig:SchlüsselwörterMitScore} befindenden Kandidaten mit den dazugehörigem Endbewertungen. \cite{rose2010automatic}
	
	\begin{center}
		\begin{table}[h!]
			\scriptsize
			\begin{tabular}{*{24}{l|}}				
				\rotatebox[origin=c]{90}{} 
				&\rotatebox[origin=c]{90}{wissenschaften} &\rotatebox[origin=c]{90}{wissenschaft} &\rotatebox[origin=c]{90}{sei} &\rotatebox[origin=c]{90}{etablierte} &\rotatebox[origin=c]{90}{informatik} &\rotatebox[origin=c]{90}{aufgegeben} &\rotatebox[origin=c]{90}{gemeinsamkeiten} &\rotatebox[origin=c]{90}{oftmals} &\rotatebox[origin=c]{90}{charakterisierung} &\rotatebox[origin=c]{90}{jungen} &\rotatebox[origin=c]{90}{inhalte} &\rotatebox[origin=c]{90}{allgemein} &\rotatebox[origin=c]{90}{bekannt} &\rotatebox[origin=c]{90}{struktur} &\rotatebox[origin=c]{90}{wesens} &\rotatebox[origin=c]{90}{bestrebt} &\rotatebox[origin=c]{90}{unüberschaubaren} &\rotatebox[origin=c]{90}{anwendungsvielfalt} &\rotatebox[origin=c]{90}{definieren} &\rotatebox[origin=c]{90}{abgrenzungen}
				&\rotatebox[origin=c]{90}{leichter}
				&\rotatebox[origin=c]{90}{finden}
				&\rotatebox[origin=c]{90}{vielschichtigkeit}\\
				\hline
				wissenschaften & 2 & & & 1 & & & & & & & & & & & & & & & & & & &\\
				\hline
				wissenschaft & & 2 & & & & & & & & 1 & & & & & & & & & & & & & \\
				\hline
				sei & & & 1 & & & & & & & & & & & & & & & & & & & &	\\
				\hline
				etablierte & 1 & & &1 & & & & & & & & & & & & & & & & & & &	\\
				\hline
				informatik & & & & &1 & & & & & & & & & & & & & & & & & & \\
				\hline
				aufgegeben & & & & & &1 & & & & & & & & & & & & & & & & &	\\
				\hline
				gemeinsamkeiten & & & & & & & 1& & & & & & & & & & & & & & & &	\\
				\hline
				oftmals & & & & & & & & 1& & & & & & & & & & & & & & &\\
				\hline
				charakterisierung & & & & & & & & & 1& & & & & & & & & & & & & & \\
				\hline
				jungen & & 1 & & & & & & & & 1 & & & & & & & & & & & & &	\\
				\hline
				inhalte & & & & & & & & & & & 1 & 1 & 1 & & & & & & & & & &	\\
				\hline
				allgemein & & & & & & & & & & & 1 & 1 & 1 & & & & & & & & & & \\
				\hline
				bekannt & & & & & & & & & & & 1 & 1 & 1 & & & & & & & & & &	\\
				\hline
				struktur & & & & & & & & & & & & & &1 & & & & & & & & &	\\
				\hline
				wesens & & & & & & & & & & & & & & &1 & & & & & & & &\\
				\hline
				bestrebt & & & & & & & & & & & & & & & & 1& & & & & & & \\
				\hline
				unüberschaubaren & & & & & & & & & & & & & & & & & 1 & 1 & & & & &	\\
				\hline
				anwendungsvielfalt & & & & & & & & & & & & & & & & & 1 & 1 & & & & &	\\
				\hline
				definieren & & & & & & & & & & & & & & & & & & & 1 & & & & \\
				\hline
				abgrenzungen & & & & & & & & & & & & & & & & & & & & 1 & & &	\\
				\hline
				leichter & & & & & & & & & & & & & & & & & & & & & 1 & &	\\
				\hline
				finden & & & & & & & & & & & & & & & & & & & & & & 1 & \\
				\hline
				vielschichtigkeit & & & & & & & & & & & & & & & & & & & & & & & 1	\\
				\hline
			\end{tabular}
			\label{tab:Co-occurance}
			\caption{Co-occurance}
		\end{table}
	\end{center}
	
	\begin{figure}[h!]
		\fbox{\parbox{\linewidth}{ inhalte allgemein bekannt (9.0), unüberschaubaren anwendungsvielfalt (4.0), jungen wissenschaft(3.5), etablierte wissenschaften (3.5), wissenschaften (1.5), wissenschaft (1.5), wesens (1.0), vielschichtigkeit (1.0), struktur (1.0), sei (1.0), oftmals (1.0), leichter (1.0), informatik (1.0), gemeinsamkeiten (1.0), finden (1.0), definieren (1.0), dass (1.0), charakterisierung (1.0), bestrebt (1.0), aufgegeben (1.0), abgrenzungen (1.0)}}
		\caption{Schlüsselwörter mit zugehörigem Score}
		\label{fig:SchlüsselwörterMitScore}
	\end{figure}
	\FloatBarrier
	
	\subsection{Automatic Keyword Extraction mit NLP}
	\label{sec:Automatic Keyword Extraction}
	Bei dieser Methode wird der vorliegende Text in die einzelnen Wörter unterteilt. Dabei wird eine Liste mit potentiellen Schlüsselwörtern erstellt, in der \textit{Stoppwörter} und Sonderzeichen herausgefiltert werden. Bei den Schlüsselwörtern handelt es sich nicht ausschließlich um ein Wort sondern auch um Wortsequenzen.\\
	Mit Hilfe von Stemming kann nun die Anzahl der Wörter in der Liste weiter reduziert werden, wodurch eine bessere Schlüsselwortgenerierung möglich ist. \\
	Die Liste mit den möglichen Schlüsselwörtern, kann nach der Häufigkeit des Vorkommens eines Wortes im Text sortiert werden. Das hat den Vorteil, dass die Schlüsselwörter, welche am Häufigsten im Text vorkommen, in den darauf folgenden Schritten zuerst verwendet werden und dadurch eine Laufzeitverbesserung der Anwendung entsteht.\\
	Ergänzende Regeln wie, eine Mindestanzahl von Buchstaben in einem Wort, können die Schlüsselwörter weiter begrenzen. 
	
\section{Bewertung: Herausfiltern von wichtigen Informationen auf einer Webseite}
	
\section{Umsetzung: Herausfiltern von wichtigen Informationen auf einer Webseite}
	\subsection{Automatic Keyword Extraction}%TODO Muss noch festgelegt werden
		\subsubsection{Schlüsselwortgenerierung mit Python NTLK}
		Mit dem \textit{Natural Language Toolkit} ist es möglich, den vorhandenen Webseitentext zu analysieren. Zu Beginn können sogenannte "'stopwords"' aus dem vorgegebenen Text herausgefiltert werden. Stopwords sind Wörter die sehr oft auftreten und keinen großen Informationsgewinn mit sich bringen. Beispiele dafür sind ist, ein, einer, usw. Dadurch verringert sich die Anzahl der gesamten Wörter im Text um einen sehr großen Teil. Anschließend können Funktionen wie das Zählen des Vorkommens einzelner Wörter angewendet werden, um einen Überblick von dem Text zu bekommen. Des Weiteren kann der Text in Fragmente zerlegt werden um weitere Informationen über den Inhalt zu erlangen. Abschließend kann eine Liste der analysierten Wörter bzw. Fragmente erstellt werden.\\
		Für die Erkennung wichtiger Schlüsselwörter
		Es wäre denkbar, Datenbanken bzw. Wortsammlungen zu erstellen, welche die zu suchenden Schlüsselwörter beinhalten. Mit diesen Datenbanken kann nun die Liste mit den bereits verarbeiteten Wörter verglichen werden. Die Datenbanken können mit Hilfe von bekannter Listen im Internet befüllt werden. Beispiele hierfür sind eine aktuelle Liste aller Hochschulen in Deutschland, Berufsbezeichnungen, Studiengänge, Hobbys, Städte und Gemeinden, etc..
	\subsection{Wortsammlungen erstellen}
		\subsubsection{Wie werden Wortsammlungen befüllt?}
		\subsubsection{Wie werden sie am effektivsten verglichen?}

\section{Speicherung der gewonnenen Daten}
	Die gewonnenen Daten können in einem beliebig erweiterbaren Personen-Objekt gespeichert werden. Darüber hinaus lässt sich das Objekt mit bekannten Kontakten der zu suchenden Person erweitern.\\
	Eine andere Möglichkeit wäre die Daten in eine Datei auszulagern. Hierfür wäre eine Datei mit dem Format \textit{CSV} oder \textit{TXT} möglich.		% Masterdatei einen sinnvollen
		% Kommentar erhalten
%Kapitel der Umsetzung

\chapter{OSINT einer großen Anzahl von Person}  %Name des Kapitels
\label{cha:Informationsbeschaffung einer grossen Anzahl von Person} %Label des Kapitels
Für die \textit{real-world} Simulation eines Phishing-Mail-Angriffs eine Webseiten mit einer großen Menge von personenbezogenen Daten benötigt. 	Hierfür wird manuell nach einer Webseite gesucht, die eine große Menge an personenbezogenen Daten enthält. Diese wird anschließen als Informationsquelle festgelegt. Möglichkeiten, ausgenommen von den bekannten Social Media Seiten, sind die Webseiten FuPa, Xing und LinkedIn.\\


\section{Methoden für OSINT}
	\subsection{Methode für die Suche nach Information}
	In diesem Konzept gibt es keine automatisierte Suche nach Informationen, jedoch eine automatisierte Suche nach internen Links. Diese interne Suche kann mit einem Web Crawler realisiert werden. In Vorbereitung darauf wird der Aufbau der Seite analysiert.\\

	\subsection{Methode zum Auslesen der Information}
	Zum Auslesen einer großen Menge an Daten wird ein Web Scraper erstellt. Dieser könnte für die ausgewählte Webseite hartkodiert werden. Eine Alternative dazu, wäre die Analyse des Webseitentextes, was dem Ansatz \ref{subsec:ErkennenVonInformation} von der Suchfunktion einer ausgewählten Person entsprechen würde.
%TODO BILDER von Webseite einfügen für groben Überblick oder in Umsetztung


\section{Bewertung: OSINT große Anzahl}
Die Suchfunktion für eine große Anzahl von Personen kann \textit{hartkodiert} werden und benötigt dadurch keine Textanalyse, da der Aufbau der Webseite im voraus bekannt ist. Das bedeutet, dass das Programm genau weiß wo welche Information auf einer Webseite steht. Auf der Seite "'\textit{www.fupa.net}"' befindet sich beispielsweise der Name einer Person immer an der gleichen Position einer Tabelle. Das bringt den Vorteil mit sich, dass der Text nicht analysiert werden muss und das Programm genau weiß, was mit diesen Daten gemacht werden muss. Zusätzlich entsteht eine sehr performante Methode zur Auslesung von personenbezogenen Daten.


\section{Aufbau einer Webseite analysieren}

\section{Erstellung eines internen Web Crawlers} %Unterunterkapitel
	\label{sse:}
	Damit die Webseite \textit{www.fupa.net} komplett nach Spielerdaten durchsucht werden kann, wird ein interner Web Crawler benötigt. Dieser wird sich anhand den internen Links, über die ganze Seite hinweg, durchhangeln.\\
	Für die Erstellung eines hartkodierten Web Crawlers muss zuerst einmal der komplette Aufbau einer Webseite bekannt sein. Dies lässt sich einfach mit Hilfe der Entwicklertools in einem Browser durchführen. %TODO Möglicherweise Bild von dem Aufbau der Seite
	
	\subsection{Funktionsweise des Web Crawlers}
	Links mit Spielerinformationen speichern.
	Die Funktionsweise des Web Crawlers besteht darin, dass das Programm auf der Startseite von Fupa.net beginnt nach links zu suchen und diesen folgt.
	\subsection{Probleme bei der Erstellung} %Unterkapitel 3. Ordnung
	\begin{enumerate}
		\item Python hat einen verkürzten und erkennbaren Standard http-Header. Dieser wird von vielen Administratoren geblockt und mit der Fehlermeldung 451 erkennbar gemacht. 451 for legal reason
		\item Honeypots gewollt oder ungewollt, hier Kalender darstellung mit links zu neuen Jahren die eine sehr hohe bis überhaupt keine Begrenzung haben.
		\item Rekursion erreicht schnell die Maximale tiefe von 1500.
		\item Zu langsamer Algorithmus
	\end{enumerate}
	
	
	\subsection{Lösungen}
	\begin{enumerate}
		\item http-Header selber konfigurieren
		\item Links mit möglichen Honeypots nicht beachten
		\item Stack Klasse schreiben damit keine Rekursion benötigt wird
		\item Algorithmus anpassen auf fupa-Webseite
	\end{enumerate}

\section{Auslesen der Webseite durch Hartkodierung}

\section{Datenverwaltung und Speicherung}
%TODO Speicherung kommt in Umsetzung, da es nicht zur Kernfrage gehört.
Für die Speicherung der gewonnen Daten kann eine SQL-Datenbank erstellt werden.
Als Alternative kann eine Datei angelegt werden, bei der alle Daten zu allen Personen gut strukturiert gespeichert werden können. Eine Möglichkeit dafür wäre das Dateiformat \textit{CSV} oder \textit{TXT}.		% z.B. den Titel des Kapitels
%Kapitel der Umsetzung

\chapter{Erstellung einer Phishing-Mail}  %Name des Kapitels
\label{cha:ErstellungeinerPhishing-Mail} %Label des Kapitels
In diesem Kapitel wird die Umsetzung zur Erstellung einer Phishing-Mail beschrieben.

\section{Implementierung der Methode zur Generierung der E-Mail-Adressen}	
Für den den zu entwickelnden Algorithmus wird eine eigene Klasse erstellt. Diese Klasse ist ausschließlich für die Generierung der E-Mail-Adressen zuständig.
	\subsection{Funktion des eigenen Algorithmus}
	Der lokale Teil einer E-Mail-Adresse befindet sich vor dem At-Zeichen. Dieser kann aus verschiedensten Daten bestehen. Allerdings wird in den meisten Fällen der bürgerlichen Namen verwendet. \cite{NameAlsEMail} Aus diesem Grund verwendet der Algorithmus die Personenattribute Vorname, Nachname und das Geburtsjahr.\\
	Im ersten Schritt wird kontrolliert, welche Daten bekannt sind. Im Idealfall sind das alle drei Attribute. Im zweiten Schritt wird festgelegt aus welchen Daten der lokale Teil bestehen kann. Im Folgenden sind möglichen Kombinationen aufgezeigt.
	
	\textit{Vorname;}\\
	\textit{Nachname;}\\
	\textit{Vorname, Nachname;}\\
	\textit{Vorname, Nachname, vollständiges Geburtsjahr;}\\
	\textit{Vorname, Nachname, Kurzform von Geburtsjahr;}
	
	Ein lokaler Teil kann somit aus mehreren Daten bestehen. Es kann vorkommen, dass anstatt "'Max Mustermann"' "'Mustermann Max"' als lokaler Namen verwendet wird. Aus diesem Grund wir für jeden lokalen Teil, der aus mehreren Daten besteht, eine Permutation ohne Wiederholung angewendet. Dadurch werden alle möglichen Kombinationen aus den Daten gewonnen, da bei der Zusammensetzung der Daten zusätzlich auf die Reihenfolge geachtet wird. Außerdem werden bei der Zusammensetzung der Daten die bekannten Trennzeichen "'."',"'\_"' und "'-"' hinzugefügt. Jedoch gibt es ebenfalls jede Kombination ohne Trennzeichen. Die lokale Namen werden anschließend in einer Liste gespeichert.\\
	Für den Domainteil werden die bekannte Mailprovider in Deutschland verwendet.  Dazu gehören die Provider GMX, WEB.DE, Gmail, T-Online, Freenet und 1\&1.\cite{AnbieterMail}. Das bedeutet, es wird für jeden lokalen Namen eine E-Mail-Adresse mit den jeweiligen Mailprovidern und der Landeskennung "'de"' erzeugt. Die folgende Tabelle  zeigt die erzeugten E-Mail-Adressen des Algorithmus für die Daten "'Marco"', "'Lang"' und "'1995"'. Es sind allerdings nur die Mailadressen für die Provider WEB.DE, Gmail und Freenet aufgelistet.
	
	\begin{center}
		%\begin{table}[h!]
		\scriptsize
		\begin{longtable}{c|c|c}
			\label{EMailAdressen}

			%\centering
			%\scriptsize
			%\begin{tabular}{c|c|c}
				marco@web.de& marco@gmail.com& marco@freenet.de\\ 
				lang@web.de& lang@gmail.com& lang@freenet.de\\
				marcolang@web.de& marcolang@gmail.com& marcolang@freenet.de\\
				marco.lang@web.de& marco.lang@gmail.com& marco.lang@freenet.de\\ 
				marco\_lang@web.de& marco\_lang@gmail.com& marco\_lang@freenet.de\\ 
				marco-lang@web.de& marco-lang@gmail.com& marco-lang@freenet.de\\
				langmarco@web.de& langmarco@gmail.com& langmarco@freenet.de\\
				lang.marco@web.de& lang.marco@gmail.com& lang.marco@freenet.de\\
				lang\_marco@web.de& lang\_marco@gmail.com& lang\_marco@freenet.de\\
				lang-marco@web.de& lang-marco@gmail.com& lang-marco@freenet.de\\
				marcolang1995@web.de& marcolang1995@gmail.com& marcolang1995@freenet.de\\
				marco.lang.1995@web.de& marco.lang.1995@gmail.com& marco.lang.1995@freenet.de\\
				marco\_lang\_1995@web.de& marco\_lang\_1995@gmail.com& marco\_lang\_1995@freenet.de\\
				marco-lang-1995@web.de& marco-lang-1995@gmail.com& marco-lang-1995@freenet.de\\ 
				marco1995lang@web.de& marco1995lang@gmail.com& marco1995lang@freenet.de\\
				marco.1995.lang@web.de& marco.1995.lang@gmail.com& marco.1995.lang@freenet.de\\ 
				marco\_1995\_lang@web.de& marco\_1995\_lang@gmail.com& marco\_1995\_lang@freenet.de\\
				marco-1995-lang@web.de& marco-1995-lang@gmail.com& marco-1995-lang@freenet.de\\
				langmarco1995@web.de& langmarco1995@gmail.com& langmarco1995@freenet.de\\
				lang.marco.1995@web.de& lang.marco.1995@gmail.com& lang.marco.1995@freenet.de\\ 
				lang\_marco\_1995@web.de& lang\_marco\_1995@gmail.com& lang\_marco\_1995@freenet.de\\
				lang-marco-1995@web.de& lang-marco-1995@gmail.com& lang-marco-1995@freenet.de\\
				lang1995marco@web.de& lang1995marco@gmail.com& lang1995marco@freenet.de\\
				lang.1995.marco@web.de& lang.1995.marco@gmail.com& lang.1995.marco@freenet.de\\ 
				lang\_1995\_marco@web.de& lang\_1995\_marco@gmail.com& lang\_1995\_marco@freenet.de\\ 
				lang-1995-marco@web.de& lang-1995-marco@gmail.com& lang-1995-marco@freenet.de\\ 
				1995marcolang@web.de& 1995marcolang@gmail.com& 1995marcolang@freenet.de\\
				1995.marco.lang@web.de& 1995.marco.lang@gmail.com& 1995.marco.lang@freenet.de\\ 
				1995\_marco\_lang@web.de& 1995\_marco\_lang@gmail.com& 1995\_marco\_lang@freenet.de\\ 
				1995-marco-lang@web.de& 1995-marco-lang@gmail.com& 1995-marco-lang@freenet.de\\
				1995langmarco@web.de& 1995langmarco@gmail.com& 1995langmarco@freenet.de\\
				1995.lang.marco@web.de& 1995.lang.marco@gmail.com& 1995.lang.marco@freenet.de\\ 
				1995\_lang\_marco@web.de& 1995\_lang\_marco@gmail.com& 1995\_lang\_marco@freenet.de\\
				1995-lang-marco@web.de& 1995-lang-marco@gmail.com& 1995-lang-marco@freenet.de\\
				marcolang95@web.de& marcolang95@gmail.com& marcolang95@freenet.de\\
				marco.lang.95@web.de& marco.lang.95@gmail.com& marco.lang.95@freenet.de\\ 
				marco\_lang\_95@web.de& marco\_lang\_95@gmail.com& marco\_lang\_95@freenet.de\\ 
				marco-lang-95@web.de& marco-lang-95@gmail.com& marco-lang-95@freenet.de\\ 
				marco95lang@web.de& marco95lang@gmail.com& marco95lang@freenet.de\\ 
				marco.95.lang@web.de& marco.95.lang@gmail.com& marco.95.lang@freenet.de\\ 
				marco\_95\_lang@web.de& marco\_95\_lang@gmail.com& marco\_95\_lang@freenet.de\\
				marco-95-lang@web.de& marco-95-lang@gmail.com& marco-95-lang@freenet.de\\
				langmarco95@web.de& langmarco95@gmail.com& langmarco95@freenet.de\\ 
				lang.marco.95@web.de& lang.marco.95@gmail.com& lang.marco.95@freenet.de\\ 
				lang\_marco\_95@web.de& lang\_marco\_95@gmail.com& lang\_marco\_95@freenet.de\\ 
				lang-marco-95@web.de& lang-marco-95@gmail.com& lang-marco-95@freenet.de\\
				lang95marco@web.de& lang95marco@gmail.com& lang95marco@freenet.de\\
				lang.95.marco@web.de& lang.95.marco@gmail.com& lang.95.marco@freenet.de\\ 
				lang\_95\_marco@web.de& lang\_95\_marco@gmail.com& lang\_95\_marco@freenet.de\\ 
				lang-95-marco@web.de& lang-95-marco@gmail.com& lang-95-marco@freenet.de\\
				95marcolang@web.de& 95marcolang@gmail.com& 95marcolang@freenet.de\\ 
				95.marco.lang@web.de& 95.marco.lang@gmail.com& 95.marco.lang@freenet.de\\ 
				95\_marco\_lang@web.de& 95\_marco\_lang@gmail.com& 95\_marco\_lang@freenet.de\\
				95-marco-lang@web.de& 95-marco-lang@gmail.com& 95-marco-lang@freenet.de\\
				95langmarco@web.de& 95langmarco@gmail.com& 95langmarco@freenet.de\\
				95.lang.marco@web.de& 95.lang.marco@gmail.com& 95.lang.marco@freenet.de\\ 
				95\_lang\_marco@web.de& 95\_lang\_marco@gmail.com& 95\_lang\_marco@freenet.de\\ 
				95-lang-marco@web.de& 95-lang-marco@gmail.com& 95-lang-marco@freenet.de
		%	\end{tabular}
		
		%\end{table}
	\end{longtable}
	\end{center}

\section{Implementierung der Methode zur Erstellung von E-Mail-Mustern/Inhalt}
	\subsection{}

	
\section{Validität der generierten Mail-Adressen prüfen}

	\subsection{Methoden zum Prüfen der Validität}
	Die erzeugten Adressen werden anschließend auf Validität geprüft. Hierfür gab es früher eine \textit{VRFY} Anfrage von SMTP. Mit dieser Anfrage konnte eine angegebene E-Mail-Adresse überprüft werden. Allerdings wurde der Dienst von Spammern ausgenutzt und wird dadurch von den meisten SMTP-Servern nicht mehr zu Verfügung gestellt.\cite{balduzzi2010abusing}\\
	Demnach muss die Validität auf einem anderen Weg geprüft werden. Eine Möglichkeit zur Prüfung ist die Verwendung bereitgestellter Webseiten, bei der die zu prüfenden E-Mail-Adresse angegeben werden kann. Eine anschließende Rückmeldung verrät dann, ob die Adresse verwendet wird oder nicht. Eine Webseite dafür wäre "'\textit{https://centralops.net/co/}"'. Als Alternative dazu, ist die Entwicklung eines Skriptes, welches die Validität der Adresse prüft.
	
	Im Fall, dass mehrere Adressen von diesem Adresspool gültig sind, kann nach mit Hilfe dieser Mail-Adressen nach Einträgen im Internet gesucht werden. Wenn es eine Übereinstimmung mit der Zielperson gibt, wird diese E-Mail ausgewählt. Andernfalls wird an jede gültige Adresse eine Phishing-Mail gesendet. 
	
	\subsection{Bewertung: Validität Prüfen}
	Für eine bessere Laufzeit des Programms, wird ein Skript zur Überprüfung der Adressen auf Verfügbarkeit und Gültigkeit, verwendet.
	
\section{E-Mail-Muster erstellen}

	\subsection{Kategorien erstellen}
	Grundsätzlich können die Muster in zwei große Kategorien unterteilt werden. Es gibt einen privaten und geschäftlichen Teil. Der private Teil hat weiter Unterteilungen wie beispielsweise Familie, Hobby und Interessen. Der Text kann hier in einer Alltagssprache erstellt werden. Für ein geschäftliches Muster sollte eine gehobene Sprache verwendet werden und Daten wie der Firmenname muss bekannt sein. 
	
	\subsection{Lückentexte erstellen}

\section{Absender-Adresse}
Spoofing, Kontakte von Fupa oder Instagram nutzen.
%TODO in ausblick

			% um für Korrekturen oder Umstellungen

%Kapitel des Evaluation

\chapter{Evaluation der Implementation}  %Name des Kapitels
\label{cha:Evaluation der Implementation} %Label des Kapitels
\section{Validierung des Gesamtkonzeptes}
Das Gesamtkonzept dieser Anwendung funktioniert gut. Eine große Herausforderung ist die Identifizierung einer Person. Dazu wurden die Methoden zur Kontaktanalyse und zur Generierung von Identifikationsschlüsseln erstellt. Darüber hinaus wurden die Sucher-URL optimiert, damit die Suchergebnisse reduziert und verbessert werden. Dennoch besteht die Möglichkeit zur Verwechslung der Person, bei zu identischen Profilen.\\
Zum Herausfiltern von wichtigen Informationen, wurden Schlüsselwörter aus dem Text erzeugt und mit den Elementen aus den Wortsammlungen verglichen. Dabei bestehen alle Elemente, außer die der Wortsammlung Institution, aus einem Wort. Das bedeutet es können nur Schlüsselwörter bestehend aus einem Wort gefunden werden. Dagegen kann eine Institution mehrere Wörter enthalten. Allerdings kann in diesem Fall die Häufigkeit des Vorkommens einer Institution auf einer Webseite nicht erkannt werden. Somit besteht bei beiden Fällen die Gefahr von Fehlinterpretation oder Missachtung einer Information.\\
Für die Bestimmung, welche Daten verwendet werden, wurde ein Algorithmus entwickelt. Dieser berechnet unter Beachtung von bestimmten Kriterien eine Wertung. Das Element mit der höchsten Wertung wird ausgewählt. Diese Berechnung kann im Fall, dass nur ein Element einer Kategorie auf einer Webseite gefunden wurde, zu Problemen führen. In dieser Ausnahme, wird das eine Element sehr hoch gewichtet. Dadurch kann es zu einer fehlerhaften Auswahl kommen. Jedoch ist die Berechnung der Wertung unter Berücksichtigung der Kriterien nötig, um einen dauerhaften Auswahlfehler zu umgehen.\\
Die Phishing-E-Mails werden mit Mustern erzeugt. Dadurch enthält die E-Mail einen sinnvollen Inhalt mit einer korrekten Grammatik. In einzelnen Fällen kann es zu sonderbaren Formulierungen kommen.

\section{Beschreibung und Motivation der Testfälle}
Bei den Testfällen wurden verschiedene Personen mit unterschiedlichen Daten gesucht. Für jede Zielperson wurde eine Einverständniserklärung eingeholt. Somit besteht jeder einzelne Testfall aus einer Suche nach einer realen Personen.
	\subsection{Testfall 1}
	\label{subsec:Testfall1}
	Im ersten Testfall wurde nach der Person mit dem Namen "'Marco Lang"' und dem Wohnort Tettnang gesucht. Dabei ist zusätzlich der Instagram-Benutzername dieser Zielperson bekannt.
	\subsubsection{Ergebnisse}
		\begin{figure}[h!]
			\fbox{\parbox{\linewidth}{\texttt{Vorname: marco\\
						Nachname: lang\\
						Wohnort: tettnang\\
						Geburtsjahr: 1995\\
						Ort: tettnang\\
						Tätigkeit: bäcker\\
						Hobby: fussball\\
						Institution: \\
						E-Mails:  []\\
						Kontaktinformation:  ['sophie', 'fitness']\\\\
						Phishing-Mail:\\
						Betreff: Fragen bzgl. Fitness\\
						Hi Marco,\\
						hier ist Sophie. Bezüglich Fitness hätte ich noch ein paar fragen an dich...\\
						Könntest du zufällig in den Anhang schauen und bewerten was ich da so rausgesucht habe?\\\\
						Grüße,\\					
						Sophie}}}
			\caption{Programmausgabe zu dem Testfall 1}
		\end{figure}
		\FloatBarrier
	\subsection{Testfall 2}
	\label{subsec:Testfall2}
	Für diesen Testfall wird nach der Person "'Anika Zeilmann"' gesucht. Der Wohnort ist bei dieser Suche keine Stadt, sondern die Gemeinde Heidesheim aus dem Bundesland Rheinland-Pfalz. Es werden keine zusätzlichen Personendaten angegeben.
		\subsubsection{Ergebnisse}
			\begin{figure}[h!]
			\fbox{\parbox{\linewidth}{\texttt{Vorname: anika\\
						Nachname: zeilmann\\
						Wohnort: heidesheim\\
						Geburtsjahr:\\
						Ort: heidesheim\\
						Tätigkeit: student\\
						Hobby: basketball\\
						Institution: hochschule mainz\\
						E-Mails: []\\
						Kontaktinformation: []\\\\
						Phishing-Mail:\\
						Betreff: Rückmeldung - Hochschule Mainz\\
						Hallo Frau Zeilmann,\\
						leider ist und ein Fehler unterlaufen. Aus diesem Grund müssen sie sich erneut zurückmelden.
						Um den Vorgang zu beschleunigen, klicken Sie bitte auf den folgenden Link.\\
						https://badlink.com\\\\
						Mit freundlichen Grüßen\\\\
						Ihr Team der Hochschule Mainz}}}
			\caption{Programmausgabe zu der Suche "'Anika Zeilmann"' \& "'Heidesheim"'}
		\end{figure}
		\FloatBarrier
	\subsection{Testfall 3}
	\label{subsec:Testfall3}
	Für diesen Testfall ist die Zielperson "'Wolfgang Lang"'. Hierbei wird zweimal nach der gleichen Person gesucht. Allerdings mit zwei unterschiedlichen Orten. Der erste Ort ist Meckenbeuren und entspricht dem Arbeitsort. Der zweiten Fall ist der Wohnort Tettnang.
		\subsubsection{Ergebnisse}
			\begin{figure}[h!]
				\fbox{\parbox{\linewidth}{\texttt{Vorname: wolfgang\\
				Nachname: lang\\
				Wohnort: meckenbeuren\\
				Geburtsjahr:\\
				Ort: meckenbeuren\\
				Tätigkeit: prokurist\\
				Hobby: politik\\
				Institution: p+w metallbau gmbh \& co. kg\\
				E-Mails: [lang@pw-metallbau.de]\\
				Kontaktinformation: []\\\\
				Phishing-Mail:\\
				Betreff: Prokurist bei der P+W Metallbau Gmbh \& Co. Kg\\
				Hallo Herr Lang,\\
				als Prokurist bei der Institution P+W Metallbau Gmbh \& Co. Kg, stehen Ihnen nun alle Möglichkeiten offen. Sehen Sie sich die neuen Möglichkeiten unter folgendem Link an.\\
				https://badlink.com\\\\
				Mit freundlichen Grüßen\\\\
				Ihr Karriere-Team der Institution P+W Metallbau Gmbh \& Co. Kg}}}
				\caption{Programmausgabe zum Tesfall 1 - Meckenbeuren}
			\end{figure}
			\FloatBarrier
			\begin{figure}[h!]
				\fbox{\parbox{\linewidth}{\texttt{Vorname: wolfgang\\
							Nachname: lang\\
							Wohnort: tettnang\\
							Geburtsjahr:\\
							Ort: meckenbeuren\\
							Tätigkeit: bäcker\\
							Hobby: reisen\\
							Institution: europäische fachhochschule\\
							E-Mails: []\\
							Kontaktinformation: []\\\\
							Phishing-Mail:\\
							Betreff: Bäcker bei der Institution Europäische Fachhochschule\\
							Hallo Herr Lang,\\
							als Bäcker bei der Institution Europäische Fachhochschule, stehen Ihnen nun alle Möglichkeiten offen. Sehen Sie sich die neuen Möglichkeiten unter folgendem Link an.\\
							https://badlink.com\\\\
							Mit freundlichen Grüßen\\\\
							Ihr Karriere-Team der Institution Europäische Fachhochschule}}}
				\caption{Programmausgabe zum Tesfall 1 - Tettnang}
			\end{figure}
			\FloatBarrier
			
\section{Übersicht und Bewertung der erzielten Ergebnisse}
	\subsection{Bewertung Testfall 1}
	Im Testfall \ref{subsec:Testfall1} wurde mit Hilfe eines vollständigen Namens, dem Wohnort und dem einmaligen Instagram-Benutzername gesucht. Dabei wurde ein überwiegend richtiges Personenprofil erstellt. Das einzige falsche Attribut ist die Tätigkeit. Hierbei handelt es sich um einen Fehler. Die Tätigkeit "'Bäcker"' wird auf der Webseite der Schwäbischen Zeitung als Werbung angezeigt. Dazu kommt, dass dieser Beruf die einzige Tätigkeit auf der entsprechenden Webseite ist. Aus diesem Grund wird das Element "'Bäcker"' hoch gewertet und infolgedessen ausgewählt.\\
	Das Geburtsjahr, der Ort, das Hobby und die Kontaktinformation ist gefunden worden und spricht mit dem tatsächlichen Personenprofil überein. Das Hobby "'Fitness"' wird auf dem Instagram-Profil der Ziel- sowie der Kontaktperson gefunden. Allerdings hat der Kontakt keinen vollständigen Namen angegeben. Dadurch konnte nur der Vorname "'Sophie"' gefunden werden.\\
	Die Anrede für die Phishing-Mail wurde korrekt ausgewählt. Die gewonnene Kontaktinformation ist verwendet worden, um das Opfer zu täuschen. Des Weiteren ergibt die E-Mail Sinn und erweist eine korrekte Grammatik. 
	\subsection{Bewertung von Testfall 2}
	Der Testfall \ref{subsec:Testfall2} zeigt ein perfektes Ergebnis. Alle Personenattribute sind korrekt und stimmen mit der gesuchten Person überein. Die gefundenen Daten sind Ort, Tätigkeit, Hobby und Institution. Dabei wird die gefundene Tätigkeit und Institution für die Generierung der Phishing-Mail verwendet. Die Auswahl des Geschlechts ist ebenfalls richtig.
	\subsection{Bewertung von Testfall 3}
	Bei dem Testfall \ref{subsec:Testfall3} wurden zwei Personensuchen für das selbe Opfer mit unterschiedlichen Orten durchgeführt. Dabei ist zu sehen, dass das gefundene Profil mit dem Ort "'Meckenbeuren"', bis auf den angegebenen Wohnort, vollständig mit der gesuchten Person übereinstimmt. Wogegen der tatsächliche Wohnort zu einem komplett fehlerhaften Profil führt. Das hat den Grund, dass diese Person in Meckenbeuren arbeitet und alle Einträge im Internet des Opfers berufsbezogen sind.\\
	Bei diesem Ergebnis ist zu sehen, wie ausschlaggebend der angegebene Wohnort für die Suche ist.
		% leichter gefunden zu werden
\chapter{Fazit und Ausblick}
\label{chap:SchlussUndAusblick}
\section{Fazit}
Das Ergebnis der Testfälle ist überwiegend positiv. Dennoch ist nur eins der vier Testergebnisse vollständig korrekt. Demnach ist die Antwort auf die Forschungsfrage, ob es möglich ist, ausschließlich korrekte Opferprofile zu erstellen, nein. Dennoch ist die Mehrzahl der Personenattribute bei den meisten Fällen richtig. Der Aufwand zu Erstellung einer Phishing-E-Mail ist durch die Automatisierung verschwindend gering. Lediglich die variierende Laufzeit der Anwendung muss beachtet werden. Diese ist abhängig von der gefundenen Information. Somit können glaubwürdige Phishing-Mails mit allen Kriterien erstellt werden. Dennoch hängt die Glaubwürdigkeit und der Erfolg solch einer Mail von dem Charakter eines Opfers ab.

Die definierten Ziele in Kapitel \ref{sec:Zielsetzung} sind erfüllt. Die erstellte Suchfunktion bietet die Möglichkeit bekannte Daten über die Zielperson einzugeben. Diese Daten dienen zur Identifizierung der gesuchten Person und ermöglichen das Auslesen von bedeutender Information. Allerdings ist eine vollständig korrekte Identifizierung der Zielperson nicht möglich. E-Mail-Adressen werden aus den Webseiten herausgelesen. Falls keine übereinstimmende Adresse gefunden wird, generiert ein Algorithmus einen Pool an möglichen Adressen. Um zu Beweisen, dass die Phishing-Mail mit der entwickelten Anwendung versendet werden kann, wurde ein Zieladresse festgelegt. Der Inhalt einer Mail wird abhängig von den gewonnen Informationen ausgewählt und mit den entsprechenden Daten ergänzt.

\section{Ausblick}
Um die Personenidentifikation zu erweitern, könnten Bilderkennungen verwendet werden. Dadurch dienen gleiche Profilbilder auf unterschiedlichen Social-Media-Plattformen als weitere Identifikationsschlüssel. Für diese Methode eignet sich eine Bildererkennungssoftware oder die Google-Bildersuche. Eine weitere Optimierung der Personenidentifizierung kann das Beachten von Zeiträumen sein. Dabei wird erkannt, ob der Inhalt oder das Erstellungsdatum einer Webseite mit dem Alter der Person grundsätzlich übereinstimmt. Hierfür können Jahreszahlen und mögliche Metadaten der Webseite beziehungsweise der Domain ausgelesen werden.

Es stellt sich die Frage, ob die Laufzeit der Anwendung durch die Optimierung der Methode zur Erkennung von wichtigen Informationen verbessert werden kann. Dafür wäre es denkbar, den Vergleich der Schlüsselwörter mit den Elementen der Wortsammlungen zu optimieren. Dazu werden die Datenbanken sortiert und die Schlüsselwörter mit einem angewendeten Suchalgorithmus verglichen. Des Weiteren könnte ein neuronales Netz trainiert werden. Als Trainingsdaten können die Wortsammlungen mit den entsprechenden Kategorien dienen. Das dabei entstehende Netz würde beispielsweise eigenständig das Schlüsselwort "'Fußball"' aus dem Text herauslesen und in die Kategorie Hobby einordnen. \\
Des Weiteren können die gefundene Elemente durch Stemming auf den Wortstamm zurückgeführt werden. Wodurch eine verbesserte Wertung der vorkommenden Elemente entstehen könnte.

Damit die Wahrscheinlichkeit erhöht wird, dass sich die korrekte E-Mail-Adresse in dem erzeugten Adresspool befindet, können weitere Adresse generiert werden. Als Ideengeber könnte hierfür das OSINT-Tool \cite{EmailAssumptions} dienen. Darüber hinaus können dem Adresspool mögliche Firmenadressen hinzugefügt werden. Dazu müsste allerdings die Institution der Zielperson bekannt sein. Der erzeugt Adresspool beinhaltet viele mögliche E-Mail-Adressen der gesuchten Person. Jedoch ist nicht jede dieser Adresse gültig. Aus diesem Grund können die generierten Adressen validiert werden. Möglichkeiten dafür sind bereitgestellte Webseiten oder ein eigenes Skript.

Aktuell wird die Phishing-E-Mail mit der Adresse des gefälschten GMX-Accounts versendet. Dadurch steht diese Adresse als Absender in der entsprechenden Mail. Um die Glaubwürdigkeit der Phishing-Mail zu steigern, kann die Absenderadresse verschleiert werden. Dies ist möglich, indem der E-Mail-Header verändert wird. Im Fall, dass Informationen über Kontakte der Zielperson gefunden werden, können diese Daten zur Generierung einer gefälschten Absenderadresse verwendet werden.

%%% Local Variables: 
%%% mode: latex
%%% TeX-master: "Bachelorarbeit"
%%% End: 
               % Schluss

%%% Anhang %%%          Nummerierung beginnend bei A
\appendix                       % Anhang
 \chapter{Ein Kapitel des Anhangs}
\label{cha:anhang}

Quellcode der Anwendung und Wortsammlungen werden hinzugefügt!


%%% Local Variables: 
%%% mode: latex
%%% TeX-master: "Bachelorarbeit"
%%% End: 
               % Erstes Kapitel des Anhangs

%%% Verzeichnisse %%%
\begin{spacing}{1.0}            % Verzeichnisse werden mit einzeiligem Abstand gesetzt

% \listoffigures                   % Abbildungsverzeichnis (optional)
% \listoftables                    % Tabellenverzeichnis (optional)

%Glossar ausgeben
\printglossary[style=altlist,title=Glossar]

%Abk�rzungen ausgeben
\renewcommand{\acronymname}{Abkürzungsverzeichnis}
\printglossary[type=\acronymtype,style=long]

%Symbole ausgeben
\printglossary[type=symbolslist,style=long]

\bibliographystyle{geralpha}       % Look&Feel vom Literaturverzeichnis
\bibliography{bib}                 % Literaturverzeichnis
\addcontentsline {toc}{chapter}{Stichwortverzeichnis} % Stichwortverzeichnis soll im Inhaltsverzeichnis auftauchen (optional)
\printindex % Stichwortverzeichnis endgueltig anzeigen

\end{spacing}

\end{document}

%%% WICHTIG:
%% Um den Glossareintrag (Abkürzungsverzeichnis richtig darstellen zu können, muss makeindex mit dem Parameter "-s Bachelorarbeit.ist -t Bachelorarbeit.alg -o
%% Bachelorarbeit.acr Bachelorarbeit.acn" aufgerufen werden --> Einstellung im TeXnicCenter unter Ausgabe -> Ausgabeprofil definieren -> LaTeX=>PDF -> Nachbearbeitung
%% Unter Postprozessoren neuen Eintrag anlegen, z.B. Makeindex1. Unter Anwendung makeindex.exe auswählen. (c:\programme\miktex2.7\miktex\bin\makeindex.exe)
%% unter Argumente die obige Parameterzeile eintragen. Bei anderer TeX-Distri makeindex suchen. Unter Linux ein shell-script erstellen, das makeindex mit 
%% den Parametern aufruft.
%% makeindex erneut starten mit folgender Parameterzeile:  " -s Bachelorarbeit.ist -t Bachelorarbeit.glg -o Bachelorarbeit.gls Bachelorarbeit.glo"
%% Analog zu oben im TeXnicCenter einen weiteren Eintrag Makeindex2 erstellen. In Linux einen weiteren makeindex-Aufruf im Script hinzufügen.
%% makeindex erneut starten mit folgender Parameterzeile:  " -s Bachelorarbeit.ist -t Bachelorarbeit.slg -o Bachelorarbeit.syi Bachelorarbeit.syg"
%% Analog zu oben im TeXnicCenter einen weiteren Eintrag Makeindex3 erstellen. In Linux einen weiteren makeindex-Aufruf im Script hinzufügen.

%% Bachelorarbeit muss durch den Namen der Hauptdatei ausgetauscht werden. Hauptdatei unter Windows mindestens 5 Mal compilieren, dann betrachten

%% Local Variables:
%% mode: latex
%% TeX-master:
%% End:
