\chapter{Einleitung}
\label{cha:einleitung}

%%% Local Variables: 
%%% mode: latex
%%% TeX-master: "Bachelorarbeit"
%%% End: 

\section{Motivation}
\label {sec:Motivation}
In der heutigen Zeit wird das Thema Informationssicherheit immer wichtiger. Systeme werden immer komplexer und Firewalls immer besser.
Doch laut dem Bundeskriminalamt hat sich die Zahl der Cyberkriminalität mit einem klaren Trend nach oben entwickelt. \cite{Cyberkriminalitaet}\\
Eine häufig verwendetet Technik von Cyberkriminalität ist das E-Mail-Phishing. Hier wird der Mensch als Schwachstelle des Systems genutzt. In den neusten Fällen von Phishing-Attacken zeigt die Verbraucherzentrale Nordrhein-Westfalen, dass diese meist direkt an eine Person adressiert sind. Beispielsweise wird man in den gefälschten DSGVO-E-Mails, im Namen der Sparkasse, persönlich mit Namen angesprochen. \cite{VerbraucherzentraleNW} Hier stellt sich die Frage, wie kann es sein dass der persönliche Name in der E-Mail steht?\\
Im Rahmen dieser Abschlussarbeit wird gezeigt, wie die Suche nach personenbezogenen Daten Internet aussieht und mit welchem Aufwand solche Angriffe verbunden sind.

\section{Zielsetzung}
\label {sec:Zielsetzung}
 Ziel ist ein Tool zu entwickeln, welches automatisiert nach personenbezogenen Daten im Internet sucht und daraus eine Phishing-Mail erzeugt. Dabei soll das Tool zwei verschiedene Suchfunktionen haben. \\\\
 {\bf Ziel 1} \textit{Informationen zu einer bestimmten Person im Internet suchen.}\\\\
 Diese Suchfunktion beinhaltet die Suche nach Informationen einer bestimmten Person. Dadurch können bereits bekannte Daten über die Person angegeben werden und somit die Suche verfeinert und verbessert werden.\\\\
 {\bf Ziel 2} \textit{Webseiten, die eine große Menge von personenbezogener Daten enthalten, auslesen und analysieren.}\\\\
 Bei dieser Suchfunktion kann nur die Webseite angegeben werden, welche ausgelesen und analysiert werden soll. Durch diese Funktion ist es möglich einen weitläufigen Phishing-Mail-Angriff zu simulieren.\\\\
 {\bf Ziel 3} \textit{Die gewonnenen Informationen Klassifizieren}\\\\
 Ausgelesene Daten sollen vor dem speichern klassifiziert werden, damit die Daten später korrekt eingesetzt werden können.\\\\
 {\bf Ziel 4} \textit{E-Mail-Adressen aus den gewonnen Daten generieren.}\\\\
 Durch die Zusammensetzung von Vorname, Name und Geburtsjahr können E-Mail-Adressen generiert werden.\\\\
 {\bf Ziel 5} \textit{Anhand der oben genannten Klassifizierung werden Phishing-Mail-Muster erstellt}\\\\
 Diese Muster können ebenfalls Klassifiziert werden. Das bedeutet, dass je nach gefunden Information ein passendes Muster vorhanden sein muss.\\\\
 {\bf Ziel 6} \textit{Phishing-Mail erzeugen.}\\\\
 Mit der vorhandenen Information, der E-Mail-Adresse und einem passende Muster, wird eine Phishing erzeugt und versendet.
 
 %TODO wie Formatieren Enteder Ziele oder Fließtext?!


 	
\section{Eigene Leistung}
\label {sec:Eigene Leistung} 
In dieser Arbeit wird ein Tool erstellt, welches personenbezogene Daten automatisiert aus dem Internet heraussucht und diese in potentielle Opferprofile ablegt. Die gewonnenen Informationen werden durch einen Phishing-Mailgenerator automatisiert in eine personalisierte Phishing-E-Mail eingebaut. Für einen höheren Erfolg werden E-Mail-Muster erstellt.

\section{Aufbau der Arbeit}
\label {sec:Aufbau der Arbeit} 
Die Arbeit gliedert sich in einem theoretischen und praktischen Teil auf. Der Theorie-Teil beginnt im zweiten Kapitel und beschreibt die Grundbegriffe im Bereich Social Engineering, Webtools, E-Mails und Programmiersprachen. Im nächsten Kapitel befindet sich die Anforderungsanalyse. Hier werden die Anforderungen an die Arbeit festgelegt. Darauf folgen die Lösungsvorschläge im Kapitel vier und die ausgewählte Lösung anhand den Anforderungen im Kapitel 5. Anschließend wird bei der Umsetzung auf den Praktischen Teil eingegangen.Am Ende befindet sich das Fazit, der Ausblick und der Anhang.






