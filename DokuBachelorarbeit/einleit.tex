\chapter{Einleitung}
\label{cha:einleitung}

%%% Local Variables: 
%%% mode: latex
%%% TeX-master: "Bachelorarbeit"
%%% End: 

\section{Motivation}
\label {sec:Motivation}
In der heutigen Zeit wird das Thema Informationssicherheit immer wichtiger. Systeme werden immer komplexer und Firewalls immer besser.
Doch laut dem Bundeskriminalamt hat sich die Zahl der Cyberkriminalität mit einem klaren Trend nach oben entwickelt. \cite{Cyberkriminalitaet}\\
Eine häufig verwendetet Technik von Cyberkriminalität ist das E-Mail-Phishing. Hier wird der Mensch als Schwachstelle des Systems genutzt. In den neusten Fällen von Phishing-Attacken zeigt die Verbraucherzentrale Nordrhein-Westfalen, dass diese meist direkt an eine Person adressiert sind. Beispielsweise wird man in den gefälschten DSGVO-E-Mails, im Namen der Sparkasse, persönlich mit Namen angesprochen. \cite{VerbraucherzentraleNW} \\
Im Rahmen dieser Abschlussarbeit wird gezeigt, mit welchem Aufwand solche Angriffe verbunden sind und wie die Suche nach privaten Informationen im Internet aussieht.

\section{Problem}
Leser Problem komplett erklären, weiterführende Motivation


\section{Eigene Leistung}
\label {sec:Leistung} 
Was werd ich neues Erfinden/Schaffen?!!!!

In dieser Arbeit wird ein Phishing-Mailgenerator erstellt. Dieser liest automatisiert Informationen von der Webseite www.fupa.net heraus und erstellt potentiellen Opferprofile. Zusätzlich wird mit dieser Information und einem Web-Crawler das Internet nach weiteren Informationen durchstöbert. Mit dem Vornamen, Nachnamen und dem Geburtsjahr werden die E-Mail-Adressen generiert. Die gefundenen Informationen werde automatisch in eine personalisierte Phishing-E-Mail eingebaut. Für einen höheren Erfolg werden E-Mail-Muster erstellt.

\section{Aufbau der Arbeit}
\label {sec:Aufbau} 
Beispielsweise in Kapitel 3 finden sie das und in Kapitel 4 das.

Meine Arbeit gliedert sich in zwei Teile. Einem theoretischen und einem praktischen Teil. In der Theorie wird auf das Thema Social Engineering eingegangen. Speziell auf das Thema E-Mail Phishing. In dem praktische Teil wird der Phishing-Mailgenerator erzeugt und beschrieben. Der hier enthaltene Suchalgorithmus und die verbundene Verwaltung der Information, sowie die E-Mail-Generierung wird der Forschungsaspekt sein.





