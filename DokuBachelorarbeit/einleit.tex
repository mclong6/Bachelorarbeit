\chapter{Einleitung}
\label{cha:einleitung}

%%% Local Variables: 
%%% mode: latex
%%% TeX-master: "Bachelorarbeit"
%%% End: 

\section{Motivation}
\label {sec:Motivation}
In der heutigen Zeit wird das Thema Informationssicherheit immer wichtiger. Systeme werden immer komplexer und Firewalls immer besser.
Doch laut dem Bundeskriminalamt hat sich die Zahl der Cyberkriminalität mit einem klaren Trend nach oben entwickelt. \cite{Cyberkriminalitaet}\\
Eine häufig verwendetet Technik von Cyberkriminalität ist das E-Mail-Phishing. Hier wird der Mensch als Schwachstelle des Systems genutzt. In den neusten Fällen von Phishing-Attacken zeigt die Verbraucherzentrale Nordrhein-Westfalen, dass diese meist direkt an eine Person adressiert sind. Beispielsweise wird man in den gefälschten DSGVO-E-Mails, im Namen der Sparkasse, persönlich mit Namen angesprochen. \cite{VerbraucherzentraleNW} \\
Im Rahmen dieser Abschlussarbeit wird gezeigt, mit welchem Aufwand solche Angriffe verbunden sind und wie die Suche nach privaten Informationen im Internet aussieht.
%TODO Mehr auf die Informationsbeschaffung eingehen

\section{Zielsetzung}
\label {sec:Zielsetzung}
 
\section{Eigene Leistung}
\label {sec:Eigene Leistung} 
In dieser Arbeit wird ein Tool erstellt, welches personenbezogene Daten automatisiert aus dem Internet heraussucht und diese in potentielle Opferprofile ablegt. Die gewonnenen Informationen werden durch einen Phishing-Mailgenerator automatisiert in eine personalisierte Phishing-E-Mail eingebaut. Für einen höheren Erfolg werden E-Mail-Muster erstellt.

\section{Aufbau der Arbeit}
\label {sec:Aufbau der Arbeit} 
Die Arbeit gliedert sich in einem theoretischen und praktischen Teil auf. Der Theorie-Teil beginnt im zweiten Kapitel und beschreibt die Grundbegriffe im Bereich Social Engineering, Webtools, E-Mails und Programmiersprachen. Im nächsten Kapitel befindet sich die Anforderungsanalyse. Hier werden die Anforderungen an die Arbeit festgelegt. Darauf folgen die Lösungsvorschläge im Kapitel vier und die ausgewählte Lösung anhand den Anforderungen im Kapitel 5. Anschließend wird bei der Umsetzung auf den Praktischen Teil eingegangen.Am Ende befindet sich das Fazit, der Ausblick und der Anhang.






