\chapter{Einleitung}
\label{cha:einleitung}

%%% Local Variables: 
%%% mode: latex
%%% TeX-master: "Bachelorarbeit"
%%% End: 
\section{Motivation}
70\% der Internetnutzer sehen sich laut einer Umfrage durch das Risiko einer missbräuchlichen Verwendung ihrer Daten nach einem Hack nicht gefährdet \cite{AngstDatendiebstahl}\\
Das Ergebnis dieser Umfrage spricht für die Behauptung, dass viele Personen Informationen über die eigenen Person im Internet preis geben, da keine Ängste vorhanden sind. Doch diese Informationspreisgabe kann in den falschen Händen schwerwiegende Folgen haben. So kann beispielsweise bei einem Phishing-Mail-Angriff diese Art von Information genutzt werden, um ein potentielles Opfer zu täuschen oder zu manipulieren. 
Ein Beispiel dafür, sind die gefälschten DSGVO-E-Mails, bei denen der Angreifer das Opfer durch scheinbar echte Mails der Sparkasse täuscht. Dabei wird die Zielpersonen persönlich mit ihrem Namen angesprochen, wodurch die Mail an Glaubwürdigkeit gewinnt. \cite{VerbraucherzentraleNW}
\\
Solch ein Angriff benötigt allerdings im Voraus eine ausführliche Recherche über das Opfer. Als Informationsquelle für die Recherche dienen beliebig viele Medien. Doch in der heutigen Zeit ist das Internet die meistgenutzte Informationsquelle für Menschen und birgt dadurch Gefahren für jeden einzelnen Internetnutzer, der personenbezogene Daten im Internet teilt. \cite{Inforamtionsquellen} Diese Gefahr wird unter anderem durch die Entwicklung von kostenlosen OSINT-Tools erhöht. Diese Tools sammeln Informationen über Opfer von öffentlichen und frei zugänglichen Medien. Dadurch wird die Recherche im Internet nach persönlichen Informationen deutlich einfacher. Das hat zu Folge, dass jeder Internetnutzer ohne großen Aufwand OSINT im Internet betreiben kann.
%TODO automatisierungsantiel rein bringen
\section{Zielsetzung und Forschungsfragen}
\label {sec:Zielsetzung}
Ziel ist es eine OSINT-Anwendung zu entwickeln, welche automatisiert nach personenbezogenen Daten im Internet sucht. Die gewonnen Daten werden anschließend in eine Phishing-Mail integriert. Dabei soll der Fokus auf der automatisierten Informationsbeschaffung liegen.\\ 
Unter anderem sollen Antworten auf die folgenden Fragen gefunden werden:\\ Mit welchem Aufwand ist ein automatisierte Spear-Phishing-Mail-Angriff verbunden? Ist es möglich ein Personenprofil zu erstellen, bei dem ausschließlich korrekte Informationen vorhanden sind? Wie glaubwürdig sind automatisierte Phishing-E-Mails mit integrierten personenbezogenen Daten?
 \\\\
 {\bf Ziel 1} \textit{Informationen zu einer ausgewählten Person im Internet suchen.}\\\\
 Die zu erstellende Suchfunktion beinhaltet die Suche nach Informationen einer bestimmten Person. Dadurch können bereits bekannte Daten über die Person angegeben und somit die Suche verfeinert beziehungsweise verbessert werden. Die Herausforderung besteht darin, zu erkennen, wann es sich um eine Information der gesuchten Person handelt.\\\\
 {\bf Ziel 2} \textit{E-Mail-Adressen finden oder aus den gewonnen Daten generieren.}\\\\
 Wenn eine E-Mail-Adresse zu einer gesuchten Person nicht gefunden werden kann, soll diese mit Hilfe der gewonnen Daten generiert werden. Durch die Zusammensetzung von Vorname, Name und Geburtsjahr können die möglichen E-Mail-Adressen einer Zielperson erzeugt werden. Des Weiteren kann die Institution der gesuchten Person, falls diese bekannt ist, mit in den Generierungsprozess einfließen.\\\\
 {\bf Ziel 3} \textit{Phishing-Mail erzeugen.}\\\\
 Mit den gewonnen Informationen soll eine Phishing-E-Mail erzeugt werden. Dabei wird der Inhalt dieser Mail abhängig von den gewonnen Informationen erstellt. Das Ziel ist in diesem Fall, dass eine glaubhafte und sinnvolle Spear-Phishing-Mail generiert und versendet werden kann.
 

 	
\section{Eigene Leistung}
\label {sec:Eigene Leistung} 
In dieser Arbeit wird eine Anwendung erstellt, welche personenbezogene Daten zu einer gesuchten Person automatisiert aus dem Internet heraussucht. Die gewonnen Daten werden in einem potentiellen Opferprofil gespeichert und anschließend in eine personalisierte Phishing-E-Mail integriert. Für einen höheren Erfolg der Phishing-Mails werden Methoden für die Generierung des Mailtextes herausgearbeitet und realisiert.\\
Damit ein kompletter Ablauf eines Phishing-Mail-Angriffs simuliert werden kann, wird zu jeder Personensuche eine passende E-Mail-Adresse benötigt. Allerdings kann nicht bei jeder Suche eine korrekte E-Mail gefunden werden. Aus diesem Grund wird zusätzlich ein Algorithmus entwickelt, der im Fall, dass keine E-Mail-Adresse zu der Zielperson gefunden wurde, ein Pool aus möglichen Mail-Adressen mit Hilfe den gefundenen Informationen generiert.

\section{Aufbau der Arbeit}
\label {sec:Aufbau der Arbeit} 
Die Arbeit gliedert sich in einen theoretischen und praktischen Teil auf: Die Theorie beginnt im zweiten Kapitel und beschreibt die Grundlagen im Bereich von personenbezogenen Daten, Social Engineering und der Informationsbeschaffung im Internet. In Kapitel \ref{cha:Problemspezifikation} wird das Problem des Umgangs mit personenbezogenen Daten aufgezeigt, auf welches in dieser Arbeit eingegangen wird. Darauf folgt die ethische und rechtliche Betrachtung in Kapitel \ref{cha:EthischeUndRechtlicheBetrachtung}. Als nächstes werden die Anforderungen festgelegt und analysiert \ref{cha:Anforderungsanalyse} dargestellt, in der Anforderungen und Prioritäten der Arbeit festgelegt werden. Darauf folgen die Lösungsvorschläge im Kapitel \ref{cha:Lösungsideen} und die Auswahl der Lösung anhand den Anforderungen aus Kapitel \ref{cha:BewertungLösungsideenAnhandAnforderung}. Anschließend wird bei der Umsetzung auf den Praktischen Teil eingegangen. Dieser unterteilt sich in die Themen OSINT einer ausgewählten Person \ref{cha:Informationsbeschaffung einer ausgewählten Person} und die Generierung einer Phishing-Mail \ref{cha:ErstellungeinerPhishing-Mail}. Am Ende dieser Arbeit befindet sich die Evaluation der Implementation in Kapitel \ref{cha:Evaluation der Implementation} sowie die Schlussbemerkung und der Ausblick in Kapitel \ref{chap:SchlussUndAusblick}.
%TODO FALLS Anhang hier hinzufügen.






