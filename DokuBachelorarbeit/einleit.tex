\chapter{Einleitung}
\label{cha:einleitung}

%%% Local Variables: 
%%% mode: latex
%%% TeX-master: "Bachelorarbeit"
%%% End: 
\section{Motivation}
70\% der Internetnutzer sehen sich laut einer Umfrage durch das Risiko einer missbräuchlichen Verwendung ihrer Daten nach einem Hack nicht gefährdet \cite{AngstDatendiebstahl}\\
Das Ergebnis dieser Umfrage spricht für die Behauptung, dass viele Personen Informationen über die eigenen Person im Internet preis geben, da keine Ängste vorhanden sind. Doch diese Informationspreisgabe kann in den falschen Händen schwerwiegende Folgen haben. So kann beispielsweise bei einem Phishing-Mail-Angriff diese Art von Information genutzt werden, um ein potentielles Opfer zu täuschen oder zu manipulieren. 
Ein Beispiel dafür, sind die gefälschten DSGVO-E-Mails, bei denen der Angreifer das Opfer durch scheinbar echte Mails der Sparkasse täuscht. Dabei wird die Zielpersonen persönlich mit ihrem Namen angesprochen, wodurch die Mail an Glaubwürdigkeit gewinnt. \cite{VerbraucherzentraleNW}
\\
Solch ein Angriff benötigt allerdings im Voraus eine ausführliche Recherche über das Opfer. Als Informationsquelle für die Recherche dienen beliebig viele Medien. Doch in der heutigen Zeit ist das Internet die meistgenutzte Informationsquelle für Menschen und birgt dadurch Gefahren für jeden einzelnen Internetnutzer, der personenbezogene Daten im Internet teilt. \cite{Inforamtionsquellen} Diese Gefahr wird unter anderem durch die Entwicklung von kostenlosen OSINT-Tools erhöht. Diese Tools sammeln Informationen über Opfer von öffentlichen und frei zugänglichen Medien. Dadurch wird die Recherche im Internet nach persönlichen Informationen deutlich einfacher. Das hat zu Folge, dass jeder Internetnutzer ohne großen Aufwand OSINT im Internet betreiben kann.
%TODO automatisierungsantiel rein bringen
\section{Zielsetzung und Forschungsfragen}
\label {sec:Zielsetzung}
Ziel ist es eine OSINT-Anwendung zu entwickeln, welche automatisiert nach personenbezogenen Daten im Internet sucht. Die gewonnen Daten werden anschließend in eine Phishing-Mail integriert. Dabei soll der Fokus auf der automatisierten Informationsbeschaffung liegen. Für die Beschaffung der Daten gibt es zwei Arten von Suchfunktionen, siehe Ziel 1 und Ziel 2.\\ 
Unter anderem sollen Antworten auf die folgenden Fragen gefunden werden. Mit welchem Aufwand ist ein automatisierte Spear-Phishing-Mail-Angriff verbunden? Ist es möglich ein Personenprofil zu erstellen, bei dem ausschließlich korrekte Informationen vorhanden sind?
 \\\\
 {\bf Ziel 1} \textit{Informationen zu einer ausgewählten Person im Internet suchen.}\\\\
 Die erste Suchfunktion beinhaltet die Suche nach Informationen einer bestimmten Person. Dadurch können bereits bekannte Daten über die Person angegeben und somit die Suche verfeinert beziehungsweise verbessert werden. Hierbei ist es wichtig zu erkenne wann es sich um eine Information der gesuchten Person handelt.\\\\
 {\bf Ziel 2} \textit{Nach Informationen einer großen Anzahl von unbekannten Personen suchen, indem eine festgesetzte Webseite vollständig durchsucht wird.}\\\\
 Bei dieser Suchfunktion soll eine bestimmte Webseiten vorgegeben werden, welche durchsucht, analysiert und ausgelesen wird. Dadurch ist es möglich einen weitläufigen "'\textit{real-world}"' Phishing-Mail-Angriff zu simulieren.\\\\
 {\bf Ziel 3} \textit{E-Mail-Adressen aus den gewonnen Daten generieren.}\\\\
 Durch die Zusammensetzung von Vorname, Name und Geburtsjahr werden die E-Mail-Adressen generiert. Außerdem kann der Arbeitgeber, falls er bekannt ist, mit in den Generierungsprozess einfließen.\\\\
 {\bf Ziel 4} \textit{Phishing-Mail-Muster erstellt}\\\\
 Abhängig von den gefundenen Informationen, soll mit Hilfe dieser Muster, eine Phishing-Mail mit glaubhaftem und sinnvollem Inhalt erstellt werden.\\\\
 {\bf Ziel 5} \textit{Phishing-Mail erzeugen.}\\\\
 Mit der vorhandenen Information, der E-Mail-Adresse und einem passende Muster, soll eine Phishing-Mail erzeugt und versendet werden können.
 
 %TODO wie Formatieren Enteder Ziele oder Fließtext?!


 	
\section{Eigene Leistung}
\label {sec:Eigene Leistung} 
In dieser Arbeit wird ein Programm erstellt, welches personenbezogene Daten automatisiert aus dem Internet heraussucht und diese in potentielle Opferprofile ablegt. Die gewonnenen Informationen werden  automatisiert in eine personalisierte Phishing-E-Mail eingebaut. Für einen höheren Erfolg werden E-Mail-Muster konzeptioniert und realisiert.\\
Damit ein kompletter Ablauf eines Phishing-Mail-Angriffs simuliert werden kann, wird zu jeder Personensuche eine passende E-Mail-Adresse benötigt. Allerdings kann nicht bei jeder Suche eine korrekte E-Mail gefunden werden. Aus diesem Grund wird zusätzlich ein Algorithmus entwickelt, der im Fall, dass keine E-Mail-Adresse zu der Zielperson gefunden wurde, eine Adresse aus den gefundenen Informationen generiert.

\section{Aufbau der Arbeit}
\label {sec:Aufbau der Arbeit} 
Die Arbeit gliedert sich in einen theoretischen und praktischen Teil auf. Die Theorie beginnt im zweiten Kapitel und beschreibt die Grundlagen \ref{cha:grundlagen} im Bereich von personenbezogenen Daten, Social Engineering und der Informationsbeschaffung im Internet. In Kapitel \ref{cha:Problemspezifikation} wird das Problem aufgezeigt, auf welches in dieser Arbeit eingegangen wird. Darauf folgt die ethische und rechtliche Betrachtung in Kapitel \ref{cha:EthischeUndRechtlicheBetrachtung}. Die Anforderungsanalyse \ref{cha:Anforderungsanalyse und Prioriesierung} beschreibt das nächste Kapitel, in welchem die Anforderungen und Prioritäten der Arbeit festgelegt werden. Darauf folgen die Lösungsvorschläge im Kapitel \ref{cha:Lösungsideen} und die Auswahl der Lösung anhand den Anforderungen im Kapitel \ref{cha:AuswahlderLösunganhandAnforderungen}. Anschließend wird bei der Umsetzung auf den Praktischen Teil eingegangen. Dieser unterteilt sich in die Themen Informationsbeschaffung einer ausgewählten Person \ref{cha:Informationsbeschaffung einer ausgewählten Person}, Informationsbeschaffung einer großen Menge an unbekannten Personen \ref{cha:Informationsbeschaffung einer grossen Anzahl von Person} und die Erstellung einer Phishing-Mail \ref{cha:ErstellungeinerPhishing-Mail}. Am Ende dieser Arbeit befindet sich die Evaluation der Implementation in Kapitel \ref{cha:Evaluation der Implementation} und die Schlussbemerkung und der Ausblick in Kapitel \ref{chap:SchlussUndAusblick}.
%TODO FALLS Anhang hier hinzufügen.






