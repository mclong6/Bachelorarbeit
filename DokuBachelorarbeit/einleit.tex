\chapter{Einleitung}
\label{cha:einleitung}




%%% Local Variables: 
%%% mode: latex
%%% TeX-master: "Bachelorarbeit"
%%% End: 

In der heutigen Zeit wird das Thema Informationssicherheit immer wichtiger. Systeme werden immer komplexer und Firewalls immer besser. Doch was ist mit uns Menschen? \\
Social Engineering (SE) wird oft mit etwas bösem bzw. schlechtem verbunden. Ist es aber grundsätzlich nicht! \\
In der Informationssicherheit spricht man oft über den Mensch als Schwachstelle des Systems. Da dieses Thema allgegenwärtig ist und sowohl Privatpersonen als auch Weltkonzerne betrifft habe ich mich entschieden meine Arbeit in diesem Bereich zu schreiben.


\section{Motivation}
\label {sec:Motivation}
Das Thema Social Engineering ist derzeit sehr aktuell. Es begegnet einem quasi jederzeit im Alltag. Man bekommt Anrufe, welche nur das Ziel haben ein Passwort herauszufinden.Man bekommt Nachrichten auf das Handy, die nur private Informationen oder Überweisungen als Ziel haben. Man bekommt E-Mails, bei denen man persönlich angewiesen wird auf einen Link o.ä. zu drücken. Meiner Meinung nach ist es sehr verblüffend wie mein Name oder ähnliche privates in solch einer E-Mail stehen kann. Aus diesem Grund habe ich mich gefragt mit welchem Aufwand und ob es möglich ist einen automatisierten Phishing-Mailgenerator zu erzeugen, der personalisierte Informationen aus dem Internet verwendet.

\section{Zielsetzung}
\label {sec:Zielsetzung} 
Das Ziel meiner Arbeit ist es einen Phishing-Mailgenerator zu entwickeln. Dieser soll automatisiert Informationen zu Personen oder Mailadressen aus dem Internet finden und die gewonnenen Informationen in einer Phishing-Mail verwenden. Es sollen E-Mail-Muster erstellt werden, die abhängig von der gewonnenen Information verwendet werden können. Beispielsweise wird das Muster Hobby für eine fußballinteressiert Person verwendet.

\section{Eigene Leistung}
\label {sec:Leistung} 
Die Aufgabe wird es sein einen Algorithmus zu entwickeln. Der Algorithmus soll das Internet nach Informationen durchstöbern können und sowohl erkennen was wichtige Information sein könnte, als auch diese Information auslesen bzw. verwenden und speichern. Desweiteren müssen E-Mail-Muster erstellt werden, die möglichst passend auf übergreifende Themen treffen, wie an dem Beispiel Fußball und Hobby kurz erläutert wurde.

\section{Aufbau der Arbeit}
\label {sec:Aufbau} 
Meine Arbeit gliedert sich in zwei Teile. Einem theoretischen und einem praktischen Teil. In der Theorie wird auf das Thema Social Engineering eingegangen. Speziell auf das Thema E-Mail Phishing. In dem praktische Teil wird der Phishing-Mailgenerator erzeugt und beschrieben. Der hier enthaltene Suchalgorithmus und die verbundene Verwaltung der Information, sowie die E-Mail-Generierung wird der Forschungsaspekt sein.


