\chapter{Einleitung}
\label{cha:einleitung}

%%% Local Variables: 
%%% mode: latex
%%% TeX-master: "Bachelorarbeit"
%%% End: 

\section{Motivation}
\label {sec:Motivation}
In der heutigen Zeit wird das Thema Informationssicherheit immer wichtiger. Systeme werden immer komplexer und Firewalls immer besser.
Doch laut dem Bundeskriminalamt hat sich die Zahl der Cyberkriminalität mit einem klaren Trend nach oben entwickelt. \cite{Cyberkriminalitaet}\\
Eine häufig verwendetet Technik von Cyberkriminalität ist das E-Mail-Phishing. Hier wird der Mensch als Schwachstelle des Systems genutzt. In den neusten Fällen von Phishing-Attacken zeigt die Verbraucherzentrale Nordrhein-Westfalen, dass diese meist direkt an eine Person adressiert sind. Beispielsweise wird man in den gefälschten DSGVO-E-Mails, im Namen der Sparkasse, persönlich mit Namen angesprochen. \cite{VerbraucherzentraleNW} \\
Im Rahmen dieser Abschlussarbeit wird gezeigt, mit welchem Aufwand solche Angriffe verbunden sind und wie die Suche nach privaten Informationen im Internet aussieht.

\section{Problem}
Leser Problem komplett erklären, weiterführende Motivation

Persönliche Informationen werden im Internet immer leichter zugänglich gemacht. !!!ZITAT!!!\\
Es gibt viele Webseiten die persönliche Information von Menschen bereitstellt. Eine davon ist auch www.fupa.net. Hier können persönliche Informationen ohne Anmeldung ausgelesen werden. Diese Art von Webseite ist eine perfekte Informationsquelle für Angreifer.\\
Im Bereich von Social Engineering Angriffen wird diese Information oft genutzt um ein Opfer zu manipulieren.
Das hier beschriebene Problem zeigt dass der Zugang für persönliche Information durch das Internet für viele Menschen einfacher gemacht wird. Es soll gezeigt werden wie einfach es ist, personenbezogene Daten aus dem Internet herauszulesen, analysieren und für einen Phishing-Angriffe zu verwenden.
!!!!ZITATE HINZUFÜGEN!!! statista z.B.

\section{Eigene Leistung}
\label {sec:Leistung} 
In dieser Arbeit wird ein Phishing-Mailgenerator erstellt. Dieser liest automatisiert Informationen von der Webseite www.fupa.net heraus und erstellt potentiellen Opferprofile. Zusätzlich wird mit dieser Information und einem Web-Crawler das Internet nach weiteren Informationen durchstöbert. Mit dem Vornamen, Nachnamen und dem Geburtsjahr werden die E-Mail-Adressen generiert. Die gefundenen Informationen werde automatisch in eine personalisierte Phishing-E-Mail eingebaut. Für einen höheren Erfolg werden E-Mail-Muster erstellt.

\section{Aufbau der Arbeit}
\label {sec:Aufbau} 
Meine Arbeit gliedert sich in zwei Teile. Einem theoretischen und einem praktischen Teil. Der Theorie-Teil beginnt im zweiten Kapitel und beschreibt die Grundbegriffe im Bereich Social Engineering, Webtools, E-Mails und Programmiersprachen. Im nächsten Kapitel befindet sich die Anforderungsanalyse. Hier werden die Anforderungen an die Arbeit festgelegt. Darauf folgen die Lösungsvorschläge im Kapitel vier und die ausgewählte Lösung anhand den Anforderungen im Kapitel 5. Im Anschluss wird bei der Umsetzung auf den Praktischen Teil eingegangen.Am Ende befindet sich das Fazit, der Ausblick und der Anhang.






