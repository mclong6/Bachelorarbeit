
%Kapitel des Hauptteils

\chapter{Bewertung der Lösungsideen anhand der Anforderung}  %Name des Kapitels
\label{cha:BewertungLösungsideenAnhandAnforderung} %Label des Kapitels

\section{Bewertung der OSINT-Methoden für eine ausgewählte Person}
Hierfür gibt es zwei verschiedene Methoden um OSINT zu betreiben. Die erste Lösungsidee beschreibt die Verwendung von einem öffentlich frei zugänglichen OSINT-Tool. Diese Tool bietet sehr viele Möglichkeiten um eine Person beziehungsweise Daten über eine Person zu finden. Allerdings ist es auf dieser Webseite nicht möglich ein zu suchendes Profil anzugeben, um eine Person zu finden. Die Suchen sind aufgeteilt in verschiedenste Daten wie Name, E-Mail, et cetera. Aus diesem Grund wird bei einer Suche ausschließlich nach dem Namen oder einer E-Mail gesucht. Dadurch ist das Suchergebnis am Ende kein vollständiges Personen-Profil, sondern lediglich Verweise auf weiterer Webseiten mit möglichen Einträgen. Dazu ist die Eingabemöglichkeiten der im Voraus bekannten, Daten begrenzt, da die Formulare nicht individuell erweiterbar sind.\\
Im Gegensatz zu diesem Tool, nutzt der eigenen Algorithmus alle im Vorfeld bekannten Daten für eine Suche. Des Weiteren kann die Laufzeit verbessert werden und bekannten Suchtechniken dieses Tools, mit Hilfe des Buches \cite{Bazzell} verwendet werden. Durch die eigene Anwendung wird die Suche beliebig erweiterbar programmiert. Dadurch kann jede Information zur Personensuche verwendet werden.

\section{Bewertung der Methoden zur Erstellung einer Phishing-Mail}
	\subsubsection{Generierung der E-Mail-Adressen}
	Bei der Verwendung eines bereits fertigen OSINT-Tools wird keine große Arbeit mehr benötigt, es ist ein komplettes System was funktioniert. Lediglich die Automatisierung muss entwickelt werden. Allerdings kann nicht jede individuelle Information für die Generierung genutzt werden. Dies ist für die zu entwickelnde Anwendung ein großer Nachteil. Die Wahrscheinlichkeit, dass sich die richtige E-Mail-Adresse darunter befindet, wird dadurch kleiner.\\
	Im Gegensatz dazu, kann ein eigener Algorithmus jegliche Information mit in die Generierung einer E-Mail-Adresse einfließen lassen. Ein Beispiel hierfür wäre das Geburtsjahr einer Zielperson. Das OSINT-Tool \cite{EmailAssumptions} verwendet das nicht. Allerdings können die möglichen Adressen, welche von dem OSINT-Tool generiert wurden, als Anregung und Ideengeber für den eigenen Algorithmus dienen.\\
	Für die erfolgreiche Simulation eines Phishing-Mail-Angriffes, wird die richtige E-Mail-Adresse benötigt. Aus diesem Grund wird der eigene Algorithmus verwendet, damit die Wahrscheinlichkeit erhöht wird, dass sich die korrekte Maildresse in dem Pool befindet.
	%TODO Anzahl der Möglichen E-Mail-adressen ausrechenen
	
	\subsubsection{E-Mail-Text}
	Der Inhalt einer E-Mail ist sehr wichtig für die Glaubwürdigkeit einer Phishing-Mail. Aus diesem Grund ist es von Bedeutung, dass der E-Mail-Text Sinn ergibt und mit einer guten Grammatik geschrieben wurde. Bei der dynamischen Texterzeugung mit der Fragment-basierten Methode \ref{subsubsec:EMailTextFragment}, kann die Grammatik und der Zusammenhang des Textes zu einer Problematik führen. Durch die Verkettung von verschiedensten Fragmenten kann ein Text erzeugt werden, welcher kein sinnvoller Zusammenhang hat. Allerdings muss hierbei nicht für jede Kombination aus gewonnen Daten ein vollständiges Fragment-Muster erstellt werden. Wogegen bei der Verwendung von fertigen Lückentexten, ein Muster für jede Kombination aus gewonnen Daten vorhanden sein muss. Dennoch ist die Glaubwürdigkeit durch einen sinnvollen E-Mail-Text höher. Aus diesem Grund werden die vollständigen E-Mail-Muster umgesetzt.


	
