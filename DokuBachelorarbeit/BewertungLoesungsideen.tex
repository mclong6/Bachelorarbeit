
%Kapitel des Hauptteils

\chapter{Bewertung der Lösungsideen anhand der Anforderung}  %Name des Kapitels
\label{cha:BewertungLösungsideenAnhandAnforderung} %Label des Kapitels

\section{Bewertung der OSINT-Methoden für eine ausgewählte Person}
Es gibt zwei verschiedene Methoden um OSINT zu betreiben. Die erste Lösungsidee beschreibt die Verwendung von einem öffentlich frei zugänglichen OSINT-Tool. Dieses Tool bietet zahlreiche Möglichkeiten um eine Person beziehungsweise Daten über eine Person zu finden. Allerdings ist es auf dieser Webseite nicht möglich, ein Profil zur gesuchten Person anzugeben. Es kann nur eine begrenzte Anzahl an Daten über ein Formular eingegeben werden. Des Weiteren wird bei einer Suche ausschließlich nach dem Namen oder einer E-Mail gesucht. Das Suchergebnis ist dadurch kein vollständiges Personenprofil, es werden lediglich Verweise auf weiterer Webseiten mit möglichen Einträgen angezeigt.\\
Im Gegensatz zu diesem Tool nutzt der eigenen Algorithmus alle im Vorfeld bekannten Daten für eine Suche. Es wird nicht auf Webseiten mit möglichen Informationen verwiesen, sondern Profile der Zielpersonen erstellt. Durch die Verwendung eines eigenen Algorithmus kann die Suche an die Anforderungen beliebig angepasst werden. Zusätzlich besteht die Möglichkeit zur Optimierung der Suche durch die Verwendung der Techniken aus \cite{Bazzell}.
\section{Bewertung der Methoden zur Erstellung einer Phishing-Mail}
	\subsubsection{Generierung der E-Mail-Adressen}
	Ein bereitgestelltes OSINT-Tool ist ein komplettes und funktionsfähiges System. Dadurch wird kein zusätzlicher Aufwand für die Entwicklung eines Algorithmus benötigt. Lediglich die Automatisierung des Tools muss erstellt werden. Allerdings kann nicht jede Information über die Person zur Generierung der E-Mail-Adresse genutzt werden. Dies ist ein großer Nachteil. Die Wahrscheinlichkeit, dass sich die richtige Adresse unter den erzeugten befindet, sinkt dadurch.\\
	Bei einem eigenen Algorithmus fließen dagegen alle Information mit in die Generierung einer E-Mail-Adresse ein. Beispielsweise auch das Geburtsjahr einer Zielperson, was bei dem OSINT-Tool \cite{EmailAssumptions} nicht verwendet wird. Jedoch können die vom OSINT-Tool generierten Adressen, als Anregung und Ideengeber für den eigenen Algorithmus dienen.\\
	Für die erfolgreiche Simulation eines Phishing-Mail-Angriffes wird die korrekte E-Mail-Adresse benötigt. Aus diesem Grund wird der eigene Algorithmus verwendet. Dadurch wird die Wahrscheinlichkeit erhöht, dass sich die korrekte Maildresse in dem Pool befindet.
	%TODO Anzahl der Möglichen E-Mail-adressen ausrechenen
	
	\subsubsection{E-Mail-Text}
	Der Inhalt einer E-Mail ist sehr wichtig für die Glaubwürdigkeit einer Phishing-Mail. Es ist die einzige Möglichkeit, bei der ein Angreifer mit dem Opfer kommunizieren kann. Aus diesem Grund ist es von Bedeutung, dass der E-Mail-Text Sinn ergibt und eine korrekte Grammatik enthält. Bei der dynamischen Texterzeugung mit der Fragment-basierten Methode \ref{subsubsec:EMailTextFragment} kann die Grammatik und der Zusammenhang des Textes zu einer Problematik führen. Durch die Verkettung von verschiedensten Fragmenten könnte ein Text erzeugt werden, welcher kein sinnvoller Zusammenhang hat. Allerdings muss hierbei nicht für jede Kombination aus gewonnen Daten ein vollständiges Fragment-Muster erstellt werden. Wogegen bei der Verwendung von fertigen Lückentexten ein Muster für jede Kombination aus gewonnen Daten vorhanden sein muss. Dennoch ist die Glaubwürdigkeit durch einen sinnvollen E-Mail-Text höher. Aus diesem Grund werden die vollständigen E-Mail-Muster umgesetzt.


	
