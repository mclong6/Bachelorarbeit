
%Kapitel des Hauptteils

\chapter{Bewertung der Lösungsideen anhand der Anforderung}  %Name des Kapitels
\label{cha:BewertungLösungsideenAnhandAnforderung} %Label des Kapitels

\section{OSINT einer ausgewählten Person}
Hierfür gibt es zwei verschiedene Methoden um OSINT zu betreiben. Die erste Lösungsidee beschreibt die Verwendung von einem öffentlich frei zugänglichen OSINT-Tool. Diese Tool bietet sehr viele Möglichkeiten um eine Person beziehungsweise Daten über eine Person zu finden. Allerdings ist es auf dieser Webseite nicht möglich ein zu suchendes Profil anzugeben, um eine Person zu finden. Die Suchen sind aufgeteilt in verschiedenste Daten wie Name, E-Mail, et cetera. Aus diesem Grund wird bei einer Suche ausschließlich nach dem Namen oder einer E-Mail gesucht. Dadurch ist das Suchergebnis am Ende kein vollständiges Personen-Profil, sondern lediglich Verweise auf weiterer Webseiten mit möglichen Einträgen. Dazu ist die Eingabemöglichkeiten der im Voraus bekannten, Daten begrenzt, da die Formulare nicht individuell erweiterbar sind.
Im Gegensatz zu diesem Tool, kann bei einem eigenen Algorithmus, welcher alle im Vorfeld bekannten Daten für eine Suche nutzt. Des Weiteren kann die Laufzeit verbessert werden und bekannten Suchtechniken dieses Tools, mit Hilfe des Buches \cite{Bazzell} verwendet werden. Durch die eigene Anwendung wird die Suche beliebig erweiterbar programmiert. Dadurch kann jede Information zur Personensuche verwendet werden.

\section{OSINT einer großen Anzahl unbekannter Personen}
Bei den sozialen Netzwerken XING und LinkedIn lässt sich die Privatsphäre eines Benutzers in den Kontoeinstellungen konfigurieren. Dies hat zu folge, dass viele Profile nicht von Suchmaschinen sondern ausschließlich von angemeldeten Mitgliedern gefunden werden kann. Somit muss für das vollständige Auslesen dieser Webseite, ein Benutzerkonto angelegt werden. Das anlegen eines Kontos ist ein negativer Aspekt, da zusätzliche Arbeit für ein anonymes Konto entsteht.\\ Beide Netzwerke stellen eine Mitgliedersuche zur Verfügung. Des Weiteren lassen sich von bekannten Firmen Mitgliederliste anzeigen, wodurch Personenprofile mit zugehörigem Arbeitgeber angezeigt werden. Dies kann später für die Generierung einer E-Mail-Adresse sehr hilfreich sein.\\
Auf der Webseite Xing ist es möglich, ein Mitgliederverzeichnis anzuzeigen, ohne ein Benutzerkonto zu erstellen. Jedoch sind dort nur die Mitglieder aufgelistet, welche keinen besonderen Privatsphären-Schutz eingestellt haben. Trotzdem wäre dieses Verzeichnis eine gute Informationsquelle, da es eine große Liste mit vielen Verweisen zu persönlichen Profilen ist. Dennoch werden die Profile nicht vollständig angezeigt, da eine Anmeldung von Nöten ist. Dies könnte allerdings im folgenden Schritt getan werden.
FuPa hingegen benötigt keine Anmeldung. Es besteht keine Möglichkeit, den Schutz der Privatsphäre von Spielerprofilen zu verstärken. Die einfach Struktur der Webseite ist ein weiterer Vorteil. Dennoch gibt es kein Mitgliederverzeichnis, welches jedes Spielerprofil auf einer Seite auflistet. Des Weiteren ist die Gewinnung des Geburtsjahres für beinahe jeden Spieler ein erheblicher Vorteil. Für die E-Mail-Generierung in den nächsten Schritten ist das Geburtsjahr sehr wichtig. Möglicherweise kein Javascript, was beim auslesen für Laufzeitverbesserungen führen kann???
%TODO Wo JavaScript... Hat Fupa Mitgliederverzeichnis?

\section{Erstellung einer Phishing-Mail}
	\subsubsection{Generierung der E-Mail-Adressen}
	Bei der Verwendung eines bereits fertigen OSINT-Tools wird keine große Arbeit mehr benötigt, es ist ein komplettes System was funktioniert. Lediglich die Automatisierung muss entwickelt werden. Jedoch, bringt es einen großen Nachteil mit sich, und zwar das es nicht jede Information für die Generierung verwendet. Im Gegensatz dazu, kann ein eigener Algorithmus jegliche Information mit in die Generierung einer E-Mail-Adresse einfließen lassen. Ein Beispiel hierfür wäre das Geburtsjahr einer Zielperson. Das OSINT-Tool von Michael Bazzell verwendet das nicht. Allerdings können die möglichen Adressen, welche von dem OSINT-Tool generiert wurden, als Anregung und Ideengeber für den Algorithmus dienen.
	\subsubsection{E-Mail-Inhalt}
	Muss noch festgleget werden!
	%TODO Versuchen was besser Funktionieren könnte!!