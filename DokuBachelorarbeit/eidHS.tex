%% Eidestattliche Erklärung %%
%% Die Erklärung sollte nach dem Deckblatt fest abgeheftet werden
\addchap*{Erklärung} %* nicht entfernen, sonst erhält Erklärung eine Nummer und erscheint im Inhaltsverzeichnis

\thispagestyle{empty} %Keine Seitenzahl, keine Kopf- und Fusszeile

Hiermit erkl"are ich, dass ich die vorliegende Arbeit mit dem Titel \newline    % ein Autor
% Hiermit erkl"aren wir, dass wir die vorliegende Arbeit mit dem Titel \newline % mehrere Autoren
\begin{center}
{\LARGE{Titel der Arbeit}}
\end{center}
selbst"andig angefertigt, nicht anderweitig zu Prüfungszwecken vorgelegt, keine anderen als die angegebenen Hilfsmittel benutzt und wörtliche sowie sinngemäße Zitate als solche gekennzeichnet habe.\newline  % ein Autor
%selbst"andig angefertigt, nicht anderweitig zu Prüfungszwecken vorgelegt, keine anderen als die angegebenen Hilfsmittel benutzt und wörtliche sowie sinngemäße Zitate als solche gekennzeichnet haben.\newline  % mehrere Autoren

\begin{flushleft}
Weingarten, \today % Ort eintragen, /today kann durch Datum 2009-10-21 oder 21.10.2009 ersetzt werden
\end{flushleft}

%%% Unterschriftenblock für einen Autor
\begin{tabular}{l}   
Autor Name        \\% Hier Autor eintragen
 \\
------------------------------------ \\
\end{tabular}

%%% Unterschriftenblock für mehrere Autoren
%\begin{tabular}{lll}
%Autor 1       &Autor 2      &Autor 3 \\% Hier eintragen
% & & \\
%------------------------------------ & ------------------------------------ & ------------------------------------ \\
%\end{tabular}

%Hier unterschreiben


%%% Local Variables: 
%%% mode: latex
%%% TeX-master: "Bachelorarbeit"
%%% End: 
