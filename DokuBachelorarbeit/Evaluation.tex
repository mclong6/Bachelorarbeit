
%Kapitel des Evaluation

\chapter{Evaluation der Implementation}  %Name des Kapitels
\label{cha:Evaluation der Implementation} %Label des Kapitels
\section{Validierung des Gesamtkonzeptes}
Das Gesamtkonzept dieser Anwendung wurde entsprechend den Anforderungen umgesetzt. Eine große Herausforderung ist die Identifizierung einer Person. Dazu wurde die Methode zur Generierung von Identifikationsschlüsseln erstellt. Darüber hinaus wurden die Sucher-URLs optimiert, damit die Suchergebnisse reduziert und verbessert werden. Dennoch besteht die Möglichkeit der Verwechslung einer Person, wenn sich die gefundene Profile sehr ähneln.\\
Zum Herausfiltern von wichtigen Informationen wurden Schlüsselwörter aus dem Text erzeugt und mit den Elementen aus den Wortsammlungen verglichen. Dabei bestehen alle Elemente, außer die der Wortsammlung Institution, aus ein bis zwei Wörtern. Das bedeutet, es können nur Schlüsselwörter bestehend aus einem Wort oder zwei Wörtern gefunden werden. Dagegen kann eine Institution mehrere Wörter enthalten. Allerdings kann in diesem Fall die Häufigkeit des Vorkommens einer Institution auf einer Webseite nicht erkannt werden. Somit besteht bei beiden Fällen die Gefahr von Fehlinterpretation oder Missachtung einer Information.\\
Für die Bestimmung, welche Daten verwendet werden, wurde ein Algorithmus entwickelt. Dieser berechnet unter Beachtung von bestimmten Kriterien einen Score. Anschließend wird das Element mit der höchsten Score ausgewählt. Diese Berechnung kann im Fall, dass nur ein Element einer Kategorie auf einer Webseite gefunden wurde, zu Problemen führen. Bei dieser Ausnahme wird das eine Element sehr hoch gewichtet. Dadurch kann es zu einer fehlerhaften Auswahl kommen. Jedoch ist die Berechnung der Wertung unter Berücksichtigung der Kriterien nötig, um einen dauerhaften Auswahlfehler zu vermeiden.\\
Die Phishing-E-Mails werden mit Mustern erzeugt. Dadurch enthält die E-Mail einen sinnvollen Inhalt mit einer korrekten Grammatik. In einzelnen Fällen kann es jedoch zu sonderbaren Formulierungen kommen.

\section{Beschreibung und Motivation der Testfälle}
Bei den Testfällen wurden verschiedene Personen mit unterschiedlichen Daten gesucht. Dazu wurde für jede Person eine Einverständniserklärung eingeholt. Somit besteht jeder einzelne Testfall aus einer Suche nach einer realen Personen. Dennoch werden Pseudonyme zum Schutz der persönlichen Daten verwendet.
	\subsection{Testfall 1}
	\label{subsec:Testfall1}
	Im ersten Testfall wurde nach der Person mit dem Namen "'Marco Lang"' und dem Wohnort Tettnang gesucht. Dabei ist zusätzlich der Instagram-Benutzername dieser Zielperson bekannt.
	\newpage
	\subsubsection{Ergebnisse}
		\begin{figure}[h!]
			\fbox{\parbox{\linewidth}{\texttt{Vorname: marco\\
						Nachname: lang\\
						Wohnort/Standort: tettnang\\
						Geburtsjahr: 1995\\
						Ort: tettnang\\
						Tätigkeit: maurer\\
						Hobby: fitness\\
						Institution: None\\
						E-Mails:  []\\
						Kontaktinformation:  ['sophie', 'fitness']\\\\
						Phishing-Mail:\\
						Betreff: Fragen bzgl. Fitness\\
						Hi Marco,\\
						hier ist Sophie. Bezüglich Fitness hätte ich noch ein paar fragen an dich...\\
						Könntest du zufällig in den Anhang schauen und bewerten was ich da so rausgesucht habe?\\\\
						Grüße,\\					
						Sophie}}}
			\caption{Programmausgabe zum Testfall 1 - "'Marco Lang"', "'Tettnang"' und "'Instagram-Benutzername"'}
		\end{figure}
		\FloatBarrier
	\subsection{Testfall 2}
	\label{subsec:Testfall2}
	Für diesen Testfall wird nach der Person "'Anita Schmidt"' gesucht. Der Wohnort ist bei dieser Suche keine Stadt, sondern die Gemeinde Heidesheim aus dem Bundesland Rheinland-Pfalz. Es werden keine zusätzlichen Personendaten angegeben.
	\newpage
		\subsubsection{Ergebnisse}
			\begin{figure}[h!]
			\fbox{\parbox{\linewidth}{\texttt{Vorname: anita\\
						Nachname: schmidt\\
						Wohnort: heidesheim\\
						Geburtsjahr: None\\
						Ort: mainz\\
						Tätigkeit: student\\
						Hobby: motorrad\\
						Institution: hochschule mainz\\
						E-Mails: []\\
						Kontaktinformation: []\\\\
						Phishing-Mail:\\
						Betreff: Rückmeldung - Hochschule Mainz\\
						Hallo Frau Schmidt,\\
						leider ist und ein Fehler unterlaufen. Aus diesem Grund müssen sie sich erneut zurückmelden.
						Um den Vorgang zu beschleunigen, klicken Sie bitte auf den folgenden Link.\\
						https://badlink.com\\\\
						Mit freundlichen Grüßen\\\\
						Ihr Studentenservice der Hochschule Mainz}}}
			\caption{Programmausgabe zum Testfall 2 - "'Anita Schmidt"' und "'Heidesheim"'}
		\end{figure}
		\FloatBarrier
	\subsection{Testfall 3}
	\label{subsec:Testfall3}
	Für diesen Testfall ist die Zielperson "'Klaus Maier"'. Hierbei wird zweimal nach der gleichen Person gesucht. Allerdings mit zwei unterschiedlichen Orten. Der erste Ort ist Meckenbeuren und entspricht dem Arbeitsort. Im zweiten Fall wird der Wohnort Tettnang verwendet.
	\newpage
		\subsubsection{Ergebnisse}
			\begin{figure}[h!]
				\fbox{\parbox{\linewidth}{\texttt{Vorname: klaus\\
				Nachname: maier\\
				Wohnort: meckenbeuren\\
				Geburtsjahr: None\\
				Ort: meckenbeuren\\
				Tätigkeit: industriemechaniker\\
				Hobby: politik\\
				Institution: p+w metallbau gmbh \& co. kg\\
				E-Mails: [maier@pw-metallbau.de]\\
				Kontaktinformation: []\\\\
				Phishing-Mail:\\
				Betreff: Industriemechaniker bei der P+W Metallbau GmbH \& Co. KG\\
				Hallo Herr Maier,\\
				als Industriemechaniker bei der Institution P+W Metallbau GmbH \& Co. KG, stehen Ihnen nun alle Möglichkeiten offen. Sehen Sie sich die neuen Möglichkeiten unter folgendem Link an.\\
				https://badlink.com\\\\
				Mit freundlichen Grüßen\\\\
				Ihr Karriere-Team der Institution P+W Metallbau GmbH \& Co. KG}}}
				\caption{Programmausgabe zum Testfall 3 - "'Klaus Maier"' und "'Meckenbeuren"'}
			\end{figure}
			\FloatBarrier
			\begin{figure}[h!]
				\fbox{\parbox{\linewidth}{\texttt{Vorname: klaus\\
							Nachname: maier\\
							Wohnort: tettnang\\
							Geburtsjahr: None\\
							Ort: meckenbeuren\\
							Tätigkeit: bäcker\\
							Hobby: reisen\\
							Institution: europäische fachhochschule\\
							E-Mails: []\\
							Kontaktinformation: []\\\\
							Phishing-Mail:\\
							Betreff: Bäcker bei der Institution Europäische Fachhochschule\\
							Hallo Herr Maier,\\
							als Bäcker bei der Institution Europäische Fachhochschule, stehen Ihnen nun alle Möglichkeiten offen. Sehen Sie sich die neuen Möglichkeiten unter folgendem Link an.\\
							https://badlink.com\\\\
							Mit freundlichen Grüßen\\\\
							Ihr Karriere-Team der Institution Europäische Fachhochschule}}}
				\caption{Programmausgabe zum Testfall 3 - "'Klaus Maier"' und "'Tettnang"'}
			\end{figure}
			\FloatBarrier
			
\section{Übersicht und Bewertung der erzielten Ergebnisse}
	\subsection{Bewertung von Testfall 1}
	Im Testfall \ref{subsec:Testfall1} wurde mit Hilfe eines vollständigen Namens, dem Wohnort und dem einmaligen Instagram-Benutzername gesucht. Dabei wurde ein nahezu richtiges Personenprofil erstellt.\\
	Das Geburtsjahr, der Ort, das Hobby und die Kontaktinformation ist gefunden worden und spricht mit dem tatsächlichen Personenprofil überein. Das Hobby "'Fitness"' wird auf dem Instagram-Profil der Ziel- sowie der Kontaktperson gefunden. Allerdings hat der Kontakt keinen vollständigen Namen angegeben. Dadurch konnte nur der Vorname "'Sophie"' gefunden werden. Die Tätigkeit ist allerdings nicht korrekt. Hierbei wurde auf einer Webseite ein Nachname falsch interpretiert und somit als Tätigkeit festgelegt. Das hat den Grund, dass keine weitere Tätigkeit gefunden wurde und somit "'Maurer"' höchste Gewichtung hat. \\ %TODO letzer Satz!!
	Die Anrede für die Phishing-Mail wurde korrekt ausgewählt. Die gewonnene Kontaktinformation ist verwendet worden, um das Opfer zu täuschen. Des Weiteren ergibt die E-Mail Sinn und weist eine korrekte Grammatik auf. 
	\subsection{Bewertung von Testfall 2}
	Der Testfall \ref{subsec:Testfall2} zeigt ebenfalls ein annähernd perfektes Ergebnis. Alle Personenattribute bis auf das Hobby sind korrekt und stimmen mit der gesuchten Person überein. Die gefundenen Daten sind Ort, Tätigkeit, Hobby und Institution. Dabei wird die gefundene Tätigkeit und Institution für die Generierung der Phishing-Mail verwendet. Das Hobby "'Motorrad"' ist nicht richtig. Allerdings wird dieses Hobby auf zwei unterschiedlichen Webseiten gefunden. Dadurch entsteht eine Wertung von 0.34. Das korrekt Hobby "'Basketball"' wurde ebenfalls gefunden, aber mit dem Score 0.33. Somit wurde laut der Berechnung das richtige Hobby ausgewählt, dennoch ist es in diesem Fall falsch.
	\subsection{Bewertung von Testfall 3}
	Bei dem Testfall \ref{subsec:Testfall3} wurden zwei Personensuchen für das selbe Opfer mit unterschiedlichen Orten durchgeführt. Dabei ist festzustellen, dass das gefundene Profil mit dem Ort "'Meckenbeuren"', bis auf den angegebenen Wohnort, vollständig mit der gesuchten Person übereinstimmt. Wogegen der tatsächliche Wohnort zu einem komplett fehlerhaften Profil führt. Grund hierfür ist, dass diese Person in Meckenbeuren arbeitet und alle gefundenen Einträge des Opfers berufsbezogen sind.\\
	Bei diesem Ergebnis ist zu erkennen, wie ausschlaggebend der angegebene Wohnort für die Suche ist.
