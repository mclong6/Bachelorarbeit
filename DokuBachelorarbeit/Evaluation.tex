
%Kapitel des Evaluation

\chapter{Evaluation der Implementation}  %Name des Kapitels
\label{cha:Evaluation der Implementation} %Label des Kapitels




\begin{figure}[h!]
	\fbox{\parbox{\linewidth}{\texttt{Vorname: marco\\
				Nachname: lang\\
				Wohnort: tettnang\\
				Geburtsjahr: 1995\\
				Ort: tettnang\\
				Tätigkeit: bäcker\\
				Hobby: fussball\\
				Institution: \\
				E-Mails:  []\\
				Kontaktinformation:  ['sophie', 'fitness']\\\\
				Phishing-Mail:\\
				Betreff: Bäcker gesucht - Im Auftrag der BRD\\
				Hallo Herr Lang,\\
				die Bundesrepublik Deutschland sucht einen kompetenten Bäcker. Haben Sie Interesse an einer neuen Herausforderung unter optimalen Arbeitsbedingungen? Im Anhang finden Sie die offizielle Stellenausschreibung mit den dazugehörigen Voraussetzungen und Gehaltsstufen.\\\\
				Ihr Karriere-Team der Bundesrepublik Deutschland}}}
	\caption{Programmausgabe zu der Suche "'Marco Lang"' \& "'Tettnang"' \& Instagram-Benutzername}
\end{figure}

\begin{figure}[h!]
	\fbox{\parbox{\linewidth}{\texttt{Vorname: anika\\
				Nachname: zeilmann\\
				Wohnort: heidesheim\\
				Geburtsjahr:\\
				Ort: heidesheim\\
				Tätigkeit: student\\
				Hobby: basketball\\
				Institution: hochschule mainz\\
				E-Mails: []\\
				Kontaktinformation: []\\\\
				Phishing-Mail:\\
				Betreff: Rückmeldung - Hochschule Mainz\\
				Hallo Frau Zeilmann,\\
				leider ist und ein Fehler unterlaufen. Aus diesem Grund müssen sie sich erneut zurückmelden.
				Um den Vorgang zu beschleunigen, klicken Sie bitte auf den folgenden Link.\\
				https://badlink.com\\\\
				Mit freundlichen Grüßen\\\\
				Ihr Team der Hochschule Mainz}}}
	\caption{Programmausgabe zu der Suche "'Anika Zeilmann"' \& "'Heidesheim"'}
\end{figure}
