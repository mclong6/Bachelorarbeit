\addchap{Kurzfassung}
\label{cha:kurzfassung} 
In dieser Abschlussarbeit wird eine Open-Source-Intelligence-Anwendung (OSINT-Anwendung) entwickelt, welche automatisiert personenbezogene Daten zu einer angegebenen Person heraussucht und die gefundenen Daten selbständig in eine Phishing-E-Mail integriert. Es wird mit einem kritischen Blick aufgezeigt, wie veröffentlichte personenbezogen Daten für einen Social-Engineering-Angriff missbraucht werden können. Dabei stellen sich die folgenden Fragen: Ist es möglich ein Opferprofil aus den gewonnenen Daten zu erstellen, welches ausschließlich korrekte Informationen enthält? Mit welchem Aufwand ist ein automatisierter Spear-Phishing-Mail-Angriff verbunden?\\
Hierfür wurden Methoden entwickelt um eine Person weitestgehend zu Identifizieren, wichtige Informationen herauszulesen, Phishing-Mail-Muster zu erzeugen und E-Mail-Adressen aus den gewonnen Daten zu generieren. Die Ergebnisse dazu zeigen, dass mit den erstellten Methoden kein vollständig korrektes Opferprofi erstellt werden kann. Dennoch ist die Generierung einer korrekten Phishing-Mail unter Verwendung der gefunden Daten möglich.\\
Somit ist die Gefahr einer missbräuchlichen Verwendung von personenbezogen Daten durch eine automatisierte OSINT-Anwendung real, bei der ein minimalen Aufwand benötigt wird.

%%% Local Variables: 
%%% mode: latex
%%% TeX-master: "Bachelorarbeit"
%%% End: 


