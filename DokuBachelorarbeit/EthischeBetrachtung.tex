%Ethische Betrachtung
\chapter{Ethische und rechtliche Betrachtung}  %Name des Kapitels
\label{cha:EthischeUndRechtlicheBetrachtung} %Label des Kapitels
%http://analysis.seclab.tuwien.ac.at/papers/raid2010.pdf
Das Sammeln von personenbezogenen Daten auf sozialen Netzwerken ist ethisch und rechtlich gesehen ein sehr sensibles Thema. Jedoch werden in dieser Arbeit ausschließlich die Daten verwendet, die öffentlich frei zugänglich sind. Das heißt, unter den Informationen befinden sich keine Passwörter oder Informationen die nicht an die Öffentlichkeit gehören. Des Weiteren ist der hier verwendete Crawler nicht stark genug, um die Leistung eines Servers von einem sozialen Netzwerk zu beeinflussen. Darüber hinaus werden die gefundenen personenbezogenen Daten nicht gespeichert.\\
Mit diesem realen Experiment, soll die Privatsphäre der Benutzer geschützt werden, indem aufgezeigt wird, wozu veröffentlichte Daten über eine Person im negativen Sinn verwendet werden können. Genau aus diesem Grund ist es wichtig, dass das Experiment in der realen Welt durchgeführt wird.\\