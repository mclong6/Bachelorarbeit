%Kapitel des Hauptteils

\chapter{Umsetzung}  %Name des Kapitels
\label{cha:} %Label des Kapitels
\section{Informationsbeschaffung von der Website www.fupa.net} %Unterkapitel
\label{sec:} %Label des Unterkapitels


\subsection{Erstellung eines Web Crawlers} %Unterunterkapitel
\label{sse:}
\subsubsection{Anforderung}
Der Web Crawler soll die komplette Webseite www.fupa.net durchgehen und Links mit Spielerinformationen speichern.
Die Funktionsweise des Web Crawlers besteht darin, dass das Programm auf der Startseite von Fupa.net beginnt nach links zu suchen und diesen folgt.
\subsubsection{Probleme} %Unterkapitel 3. Ordnung
\begin{enumerate}
	\item Python hat einen verkürzten und erkennbaren Standard http-Header. Dieser wird von vielen Administratoren geblockt und mit der Fehlermeldung 451 erkennbar gemacht. 451 for legal reason
	\item Honeypots gewollt oder ungewollt, hier Kalender darstellung mit links zu neuen Jahren die eine sehr hohe bis überhaupt keine Begrenzung haben.
	\item Rekursion erreicht schnell die Maximale tiefe von 1500.
	\item Zu langsamer Algorithmus
\end{enumerate}


\subsubsection{Lösungen}
\begin{enumerate}
	\item http-Header selber konfigurieren
	\item Links mit möglichen Honeypots nicht beachten
	\item Stack Klasse schreiben damit keine Rekursion benötigt wird
	\item Algorithmus anpassen auf fupa-Webseite
\end{enumerate}
\label{sss:}
%%% Local Variables: 
%%% mode: latex
%%% TeX-master: "Bachelorarbeit"
%%% End: 
