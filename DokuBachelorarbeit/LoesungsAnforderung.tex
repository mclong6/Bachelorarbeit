
%Kapitel des Hauptteils

\chapter{Bewertung der Lösungsideen anhand der Anforderung}  %Name des Kapitels
\label{cha:Auswahl der Lösung anhand Anforderungen} %Label des Kapitels
Mit der Programmiersprache Python lässt sich das Programm entsprechend den Anforderungen entwickeln und es kann sowohl eine Konsolenanwendung als auch eine Oberflächenanwendung programmiert werden. Es bringt alle Module mit sich um das Projekt mit dem vorgegebenen Zielen umzusetzen.

Um möglichst viele Informationen über eine Person im Internet zu finden, bietet die Personensuche, welche sich abhängig von den eingegebenen Daten variieren kann, die Lösung mit den meisten Vorteilen. Unter anderem kann die Arbeit des web crawlings ausgelagert werden, da nur noch die Suchergebnisse analysiert werden müssen. Allerdings muss beachtet werden, dass Benutzern bei verschiedensten Social-Media-Seiten auswählen können, ob das Benutzerprofil von einer Suchmaschine gefunden werden kann oder nicht. Aus diesem Grund, werden bei dieser Suchart die Ergebnisse kontrolliert ob sich die geforderten Seiten darin befinden. Wenn das nicht der Fall ist, wird separat auf diesen Seiten nach Information gesucht. Zu den geforderten Seiten zählen beispielsweise \textit{XING} und \textit{LinkedIn}.\\
Für die Bewertung der Lösungsideen zur Frage, wann es sich um die gesuchte Person handelt in Kapitel \ref{sec:WannhandeltessichumdiegesuchtePerson}, gilt, dass alle Ideen eine Verbesserungen des Ergebnisses mit sich bringen. Allerdings gibt es Unterschied in der Wirksamkeit und in der Laufzeit des Programms. Die Erweiterung der Kriterien \ref{sec:ErweiterteKriteriern} bringt keine große Laufzeitänderung mit sich und stellt eine sehr gute Eigenschaft zur Optimierung der Informationsfindung dar, da die Zeit ebenfalls mit einbezogen wird.
%TODO Weiter bei Kontakte der Suchperson in Betracht ziehen

Wenn die Kontakte der Suchperson in Betracht gezogen werden, kann erkannt werden wann es sich um die gesuchte Person handelt. Darüber hin
Für die optimal Informationsbeschaffung einer ausgewählten Person eignet sich die Methode der Automatic Keyword Extraction um die Information wird bei der Informationsbeschaffung einer ausgewählten Person der Ansa
!!XING kann man angeben ob man durch google gefunden wird!!!

Die Suchfunktion für eine große Anzahl von Personen kann \textit{hartkodiert} werden und benötigt dadurch keine Textanalyse, da der Aufbau der Webseite im voraus bekannt ist. Das bedeutet, dass das Programm genau weiß wo welche Information auf einer Webseite steht. Auf der Seite "'\textit{www.fupa.net}"' befindet sich beispielsweise der Name einer Person immer an der gleichen Position einer Tabelle. Das bringt den Vorteil mit sich, dass der Text nicht analysiert werden muss und das Programm genau weiß, was mit diesen Daten gemacht werden muss. Zusätzlich entsteht eine sehr performante Methode zur Auslesung von personenbezogenen Daten.

Für die E-Mail-Adressgenerierung wird ein eigener Algorithmus entwickelt. Im Gegensatz zu dem Open Souce-Tool \cite{Bazzell} besteht bei diesem Algorithmus eine höhere Wahrscheinlichkeit, dass die richtige E-Mail-Adresse enthalten ist, da das Geburtsjahr, falls es bekannt ist, mit einbezogen wird. Für eine bessere Laufzeit des Programms, wird ein Skript zur Überprüfung der Adressen auf Verfügbarkeit und Gültigkeit, verwendet.

\section{Generierung der E-Mail-Adressen}
%TODO Anzahl der Möglichen E-Mail-adressen ausrechenen

\section{Erstellung der E-Mail-Muster}

\section{Erzeugung der Phishing-Mail}