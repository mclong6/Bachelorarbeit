
%Kapitel des Hauptteils

\chapter{Bewertung der Lösungsideen anhand der Anforderung}  %Name des Kapitels
\label{cha:Auswahl der Lösung anhand Anforderungen} %Label des Kapitels
Mit der Programmiersprache Python lässt sich das Programm entsprechend den Anforderungen entwickeln und es kann sowohl eine Konsolenanwendung als auch eine Oberflächenanwendung programmiert werden. Es bringt alle Module mit sich um das Projekt mit dem vorgegebenen Zielen umzusetzen.

Damit möglichst viele zutreffende Informationen über eine Person im Internet gefunden werden können, wird die Art der Personensuche anhand den eingegebenen Daten angepasst. Falls dennoch keine Information zu der Person gefunden wird kann auf ausgewählten Seiten nach der Person gesucht werden. Des Weiteren muss beachtet werden, dass auf verschiedenen Social-Media-Seiten angegeben werden kann, ob man von Suchmaschinen gefunden und angezeigt wird. Aus diesem Grund werden bei jeder Suche die Ergebnisse kontrolliert ob sich die vorgegebenen Seiten darin befinden. Wenn das nicht der Fall ist, wird separat auf diesen Seiten nach Information gesucht. Zu diesen Seiten gehören \textit{XING} und \textit{LinkedIn}\\%TODO können mehr hinzugefügt werden
Wenn die Kontakte der Suchperson in Betracht gezogen werden, kann erkannt werden wann es sich um die gesuchte Person handelt. Darüber hin
Für die optimal Informationsbeschaffung einer ausgewählten Person eignet sich die Methode der Automatic Keyword Extraction um die Information wird bei der Informationsbeschaffung einer ausgewählten Person der Ansa
!!XING kann man angeben ob man durch google gefunden wird!!!

Die Suchfunktion für eine große Anzahl von Personen kann \textit{hartkodiert} werden und benötigt dadurch keine Textanalyse, da der Aufbau der Webseite im voraus bekannt ist. Das bedeutet, dass das Programm genau weiß wo welche Information auf einer Webseite steht. Auf der Seite "'\textit{www.fupa.net}"' befindet sich beispielsweise der Name einer Person immer an der gleichen Position einer Tabelle. Das bringt den Vorteil mit sich, dass der Text nicht analysiert werden muss und das Programm genau weiß, was mit diesen Daten gemacht werden muss. Zusätzlich entsteht eine sehr performante Methode zur Auslesung von personenbezogenen Daten.

Für die E-Mail-Adressgenerierung wird ein eigener Algorithmus entwickelt. Im Gegensatz zu dem Open Souce-Tool \cite{Bazzell} besteht bei diesem Algorithmus eine höhere Wahrscheinlichkeit, dass die richtige E-Mail-Adresse enthalten ist, da das Geburtsjahr, falls es bekannt ist, mit einbezogen wird. Für eine bessere Laufzeit des Programms wird ein Skript, zur Überprüfung der Adressen auf Verfügbarkeit und Gültigkeit, verwendet.

\section{Generierung der E-Mail-Adressen}
%TODO Anzahl der Möglichen E-Mail-adressen ausrechenen

\section{Erstellung der E-Mail-Muster}

\section{Erzeugung der Phishing-Mail}