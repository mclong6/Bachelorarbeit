
%Kapitel des Hauptteils

\chapter{Bewertung der Lösungsideen anhand der Anforderung}  %Name des Kapitels
\label{cha:Auswahl der Lösung anhand Anforderungen} %Label des Kapitels
Mit der Programmiersprache Python lässt sich das Programm entsprechend den Anforderungen entwickeln und es kann sowohl eine Konsoleneingabe als auch eine Oberflächeneingabe programmiert werden. Es bringt alle Module mit sich um das Projekt mit dem Projekten umzusetzen.

Damit möglichst viele zutreffende Informationen über eine Person im Internet gefunden werden können, wird die Art der Personensuche anhand den eingegebenen Daten angepasst. Falls dennoch keine Information zu der Person gefunden wird kann auf ausgewählten Seiten nach der Person gesucht werden.\\
Wenn die Kontakte der Suchperson in Betracht gezogen werden, kann erkannt werden wann es sich um die gesuchte Person handelt. Darüber hin
Für die optimal Informationsbeschaffung einer ausgewählten Person eignet sich die Methode der Automatic Keyword Extraction um die Information wird bei der Informationsbeschaffung einer ausgewählten Person der Ansa
\section{Programmiersprache/ GUI}
Python
%TODO Begründung  für Python
\section{Informationsbeschaffung von ausgewählten Personen}
	
\section{Informationsbeschaffung von einer großen Menge unbestimmter Personen}
Einfachheitshalber wird CSV verwendet.
	\subsection{Auslesen der Webseite}
	Diese Suchfunktion wird \textit{hartkodiert} und benötigt dadurch keine Textanalyse, da der Aufbau der Webseite im voraus bekannt ist. Das bedeutet, dass das Programm genau weiß wo welche Information auf einer Webseite steht. Beispielsweise befindet sich das Geburtsjahr einer Person, auf der Seite von dem Fußballportal "'FuPa"', immer an der gleichen Position einer Tabelle. Dies bringt den Vorteil mit sich, dass der Text nicht analysiert werden muss und das Programm genau weiß, was mit diesen Daten gemacht werden muss.
\section{Generierung der E-Mail-Adressen}
%TODO Anzahl der Möglichen E-Mail-adressen ausrechenen

\section{Erstellung der E-Mail-Muster}

\section{Erzeugung der Phishing-Mail}