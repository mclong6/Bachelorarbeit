
%Kapitel des Hauptteils

\chapter{Auswahl der Lösung anhand den Anforderungen}  %Name des Kapitels
\label{cha:Auswahl der Lösung anhand Anforderungen} %Label des Kapitels
\section{Programmiersprache/ GUI}
Python
\section{Informationsbeschaffung von ausgewählten Personen}
	
\section{Informationsbeschaffung von einer großen Menge unbestimmter Personen}
Einfachheitshalber wird CSV verwendet.
	\subsection{Auslesen der Webseite}
	Diese Suchfunktion wird \textit{hartkodiert} und benötigt dadurch keine Textanalyse, da der Aufbau der Webseite im voraus bekannt ist. Das bedeutet, dass das Programm genau weiß wo welche Information auf einer Webseite steht. Beispielsweise befindet sich das Geburtsjahr einer Person, auf der Seite von dem Fußballportal "'FuPa"', immer an der gleichen Position einer Tabelle. Dies bringt den Vorteil mit sich, dass der Text nicht analysiert werden muss und das Programm genau weiß, was mit diesen Daten gemacht werden muss.
\section{Generierung der E-Mail-Adressen}
%TODO Anzahl der Möglichen E-Mail-adressen ausrechenen

\section{Erstellung der E-Mail-Muster}

\section{Erzeugung der Phishing-Mail}