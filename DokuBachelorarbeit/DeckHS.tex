%% Deckblatt fuer Studien- und Diplomarbeiten am der
%% Hochschule Weingarten

\thispagestyle{empty}
%~
{
\normalsize\fontfamily{phv} \fontsize{12pt}{10}\selectfont 
\vspace{-1cm}
\begin{minipage}[b]{9.4cm}
{\fontsize{13pt}{13} \selectfont%
Hochschule\\[1ex]
Ravensburg-Weingarten}\\[1ex]
\end{minipage}
}
\begin{minipage}[b]{10cm}
\includegraphics*[height=2.7cm]{bilder/HSLogoWGd}
\end{minipage}


\vspace{10mm}
 
\hrule 
\vspace{1cm}
{
\fontseries{b} \fontsize{20pt}{20}  \selectfont%
\begin{center}
\textcolor{red}{Erstellung eines Programms zur automatisierten Informationsbeschaffung von personenbezogenen Daten in Verbindung mit einem automatisierten Phishing-Mailgenerators} % Titel der Arbeit
\end{center}
}

\begin{center}
\large \textbf{Bachelorarbeit} % Hier Praxisarbeit, Studienarbeit, Bachelorarbeit, Dokumentation zu Seminar, etc. eintragen
\end{center}

\begin{center}
\textbf{Social Engineering} % Hier den Zusatz wie Fach oder Semester eintragen
\end{center}

\vspace{5mm}

\begin{center}
im Studiengang \textcolor{red}{Angewandte Informatik} % Hier den Studiengang eintragen
\end{center}

\begin{center}
an der Hochschule Ravensburg - Weingarten 
\end{center}
\begin{center}

\end{center}
\vspace{5mm}
\begin{center}
von
\end{center}




\begin{center}
{\fontsize{12pt}{12} \selectfont%
\begin{tabular}{ll}
Marco Lang & \textcolor{red}{Matr.-Nr.: 27416}\\[0.5ex] % Hier den Autor und die Matrikelnummer statt xxxxx eintragen
%Autor 2 & \textcolor{red}{Matr.-Nr.: xxxxx}\\[0.5ex] % Für weitere Autoren Zeilen auskommentieren bzw. kopieren und ausfüllen
%Autor 3 & \textcolor{red}{Matr.-Nr.: xxxxx}\\[0.5ex]
Abgabedatum :& \today   % Das Abgabedatum wird gleichgesetzt mit dem Datum der letzten Compilierung. Statt \today kann auch Datum von Hand geschrieben werden
\end{tabular}
}
\end{center}
                               

\vspace{1cm}

\vspace{1cm}
\hrule


%%% Local Variables: 
%%% mode: latex
%%% TeX-master: "Bachelorarbeit"
%%% End: 

