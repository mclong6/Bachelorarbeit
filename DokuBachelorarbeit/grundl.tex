%Kapitel des Hauptteils

\chapter {Grundlagen}  %Name des Kapitels
\label{cha:grundlagen} %Label des Kapitels

\section{Social Engineering} %Unterkapitel
\label {sec:Unterkapitel} %Label des Unterkapitels

Hier befindet sich ein Beispiel (\gls{BSP}), wie ein Bild\index{Bild} in \LaTeX eingebunden wird.
\begin{figure}
 \begin{center}
  \includegraphics*{bilder/HSLogoWGd}
  \caption{Logo der HS -- oder nicht?}
  \label{fig:logo}
 \end{center}
\end{figure}

Und es gibt auch ein Beispiel für eine Tabelle\index{Tabelle}.
\begin{table}
 \begin{center}
 \caption{Verwendete Matrizen}
 \label{matrizen}
  \begin{tabular}{|l|l|l|}
   \hline
   Matrix & Dimension & Symbol \\
   \hline
   Systemmatrix & $n \times n$ & ${\bf A}$  \\
   \hline
   Ausgangsmatrix & $m \times n$ & ${\bf C}$  \\
   \hline
  \end{tabular}
 \end{center}
\end{table}

Man beachte: Bilder haben Bild{\bf unter}schriften, 
Tabellen haben Tabellen{\bf "uber}schriften.

Für jedes Kapitel sollte ein neues \TeX  File erstellt und eingebunden werden. \newline

Ein Symbol wie \gls{symb:Pi} Kann mathematisch korrekt dargestellt werden. Auch \gls{glos:Glossareintrag} zu Abkürzungen wie \gls{AD} können in \LaTeX behandelt werden.
Zum Demonstrieren wird hier noch eine Webseite von Microsoft zitiert\cite{Mid09}, und noch eine Stelle\cite{Mid09b}


 
%%% Local Variables: 
%%% mode: latex
%%% TeX-master: "Bachelorarbeit"
%%% End: 
