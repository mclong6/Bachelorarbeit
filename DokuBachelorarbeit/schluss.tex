\chapter{Schlussbemerkungen und Ausblick}
\label{chap:SchlussUndAusblick}
\section{Wie kann eine Person weiter identifiziert werden?}
Durch die Google Bildersuche ist es möglich, anstatt einem Suchbegriff ein Bild zu verwenden und nach diesem zu suchen. Dabei kann ein zu suchendes Bild selbst hochgeladen oder ein URL angegeben werden. Bei dem Ergebnis kann es sich um ein ähnliches Bild oder eine Webseite, die das Bild enthält, handeln.\\
Als Alternative zur Google-Bildersuche kann eine Bilderkennungssoftware verwendet werden um Personen zu identifizieren bzw. zu unterscheiden. %TODO Bilderkennungssoftware suchen
\section{Keyword Extraction mit Hilfe von Machine Learning}
\label{sec:KeywordExtractionMachine Learning}
In der Theorie ist es möglich, ein Neuronales Netz mit den Begriffen zu trainieren und eine Kategorisierung durchzuführen. Dabei entsteht ein Netz, welches selbst entscheiden würde, in welche Kategorie ein Wort fällt. Das Wort "'Fußball"' müsste dadurch in die Kategorie Hobby eingeordnet werden.

%%% Local Variables: 
%%% mode: latex
%%% TeX-master: "Bachelorarbeit"
%%% End: 
