\chapter{Fazit und Ausblick}
\label{chap:SchlussUndAusblick}
\section{Fazit}
Das Ergebnis der Testfälle ist überwiegend positiv. Dennoch ist nur eins der vier Testergebnisse vollständig korrekt. Demnach ist die Antwort auf die Forschungsfrage, ob es möglich ist, ausschließlich korrekte Opferprofile zu erstellen, nein. Das hat den Grund, dass eine große Anzahl an Elementen in den Wortsammlungen dazu führt, dass viele inkorrekte Schlüsselwörter gefunden werden. Somit steigt die Gefahr von Fehlinformationen. Möglicherweise könnte diese Gefahr durch Methoden, welche unter strengeren Bedingungen die Information von Webseiten herausfiltern, gesenkt werden. Dennoch ist die Mehrzahl der Personenattribute in den meisten Fällen richtig. Der Aufwand zu Erstellung einer Phishing-E-Mail ist durch die Automatisierung verschwindend gering. Lediglich die variierende Laufzeit der Anwendung muss beachtet werden. Diese ist abhängig von der gefundenen Information. Somit können glaubwürdige Phishing-Mails mit allen Kriterien erstellt werden. Dennoch hängt die Glaubwürdigkeit und der Erfolg solch einer Mail von dem Charakter eines Opfers ab.

Die definierten Ziele in Kapitel \ref{sec:Zielsetzung} sind überwiegend erfüllt. Die erstellte Suchfunktion bietet die Möglichkeit bekannte Daten über die Zielperson einzugeben. Diese Daten dienen zur Identifizierung der gesuchten Person und ermöglichen das Auslesen von bedeutender Information. Allerdings ist eine vollständig korrekte Identifizierung der Zielperson nicht möglich.\\ E-Mail-Adressen werden aus den Webseiten herausgelesen. Falls keine übereinstimmende Adresse gefunden wird, generiert ein Algorithmus einen Pool an möglichen Adressen. Um zu Beweisen, dass die Phishing-Mail mit der entwickelten Anwendung versendet werden kann, wurde eine Zieladresse festgelegt. Des Weiteren wird der Inhalt einer Mail abhängig von den gewonnen Informationen ausgewählt und mit den entsprechenden Daten ergänzt.

\section{Ausblick}
Um die Personenidentifikation zu erweitern, könnten Bilderkennungen verwendet werden. Dadurch dienen gleiche Profilbilder auf unterschiedlichen Social-Media-Plattformen als weitere Identifikationsschlüssel. Für diese Methode eignet sich eine Bildererkennungssoftware oder die Google-Bildersuche. Eine weitere Optimierung der Personenidentifizierung kann das Beachten von Zeiträumen sein. Dabei wird erkannt, ob der Inhalt oder das Erstellungsdatum einer Webseite mit dem Alter der Person grundsätzlich übereinstimmt. Hierfür können Jahreszahlen und mögliche Metadaten der Webseite beziehungsweise der Domain ausgelesen werden.

Es stellt sich die Frage, ob die Laufzeit der Anwendung durch die Optimierung der Methode zur Erkennung von wichtigen Informationen verbessert werden kann. Dafür wäre es denkbar, den Vergleich der Schlüsselwörter mit den Elementen der Wortsammlungen zu optimieren. Dazu werden die Datenbanken sortiert und die Schlüsselwörter mit einem angewendeten Suchalgorithmus verglichen. Des Weiteren könnte ein neuronales Netz trainiert werden. Als Trainingsdaten können die Wortsammlungen mit den entsprechenden Kategorien dienen. Das dabei entstehende Netz würde beispielsweise eigenständig das Schlüsselwort "'Fußball"' aus dem Text herauslesen und in die Kategorie Hobby einordnen. \\
Des Weiteren können die gefundene Elemente durch Stemming auf den Wortstamm zurückgeführt werden. Wodurch eine verbesserte Wertung der vorkommenden Elemente entstehen könnte.

Damit die Wahrscheinlichkeit erhöht wird, dass sich die korrekte E-Mail-Adresse in dem erzeugten Adresspool befindet, können weitere Adresse generiert werden. Als Ideengeber könnte hierfür das OSINT-Tool \cite{EmailAssumptions} dienen. Darüber hinaus können dem Adresspool mögliche Firmenadressen hinzugefügt werden. Dazu müsste allerdings die Institution der Zielperson bekannt sein. Der erzeugt Adresspool beinhaltet viele mögliche E-Mail-Adressen der gesuchten Person. Jedoch ist nicht jede dieser Adressen gültig. Aus diesem Grund können die generierten Adressen validiert werden. Möglichkeiten dafür sind bereitgestellte Webseiten oder ein eigenes Skript.

Aktuell wird die Phishing-E-Mail mit der Adresse des gefälschten GMX-Accounts versendet. Dadurch steht diese Adresse als Absender in der entsprechenden Mail. Um die Glaubwürdigkeit der Phishing-Mail zu steigern, kann die Absenderadresse verschleiert werden. Dies ist möglich, indem der E-Mail-Header verändert wird. Im Fall, dass Informationen über Kontakte der Zielperson gefunden werden, können diese Daten zur Generierung einer gefälschten Absenderadresse verwendet werden. Darüber hinaus könnte bei einer bekannter Institution die dazugehörige Domain der realen E-Mail-Adresse herausgefunden werden.

%%% Local Variables: 
%%% mode: latex
%%% TeX-master: "Bachelorarbeit"
%%% End: 
