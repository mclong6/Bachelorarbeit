\chapter{Schlussbemerkungen und Ausblick}
\label{chap:SchlussUndAusblick}
\section{Wie kann eine Person weiter identifiziert werden?}
Durch die Google Bildersuche ist es möglich, anstatt einem Suchbegriff ein Bild zu verwenden und nach diesem zu suchen. Dabei kann ein zu suchendes Bild selbst hochgeladen oder ein URL angegeben werden. Bei dem Ergebnis kann es sich um ein ähnliches Bild oder eine Webseite, die das Bild enthält, handeln.\\
Als Alternative zur Google-Bildersuche kann eine Bilderkennungssoftware verwendet werden um Personen zu identifizieren bzw. zu unterscheiden. %TODO Bilderkennungssoftware suchen

	\subsection{Zeitrahmen wird mit Beachtet}
		\subsubsection{Wie kann Alter der Webseite herausgefunden werden}
		Der Webseitentext kann nach Datums suchen und diese mit dem angegebenen Geburtsjahr verglichen werden. Dabei kann erkannt werden, ob das theoretische Alter des Artikels mit dem Alter der Person übereinstimmen kann. Möglicherweise können Metadaten von der Webseite ausgelesen werden.
		möglicherweise über domain
		
		\subsubsection{Bereits umgesetzt}
		Jahr nach copyright wird ausgelesen, wenn das nicht vorhanden werden alle Jahreszahlen genommen und ein durchschnitt ausgerechnet
	
	\subsection{Zeitraum beachten}
	Eine Methode für das Erkennen von Personen kann das Beachten von Zeiträumen sein. Dabei fließt das Alter der Zielperson mit in die Suche ein. Das bedeutet, dass nach dem Alter der Webseite gesucht wird, indem Jahreszahlen aus dem Webseitentext ausgelesen werden. Dadurch wird erkannt, ob der Zeitrahmen des Artikels oder das Erstellungsdatum einer Webseite mit dem Alter der Person grundsätzlich übereinstimmt.
	%TODO Fraglich ob das aussagekräftig ist
	%TODO jahr heraussuchen und in selben Satz nach dem wort geboren suchen
\section{Adressgenerierung}
	\subsection{Wenn Firma bekannt}
\section{Keyword Extraction mit Hilfe von Machine Learning}
\label{sec:KeywordExtractionMachine Learning}
In der Theorie ist es möglich, ein Neuronales Netz mit den Begriffen zu trainieren und eine Kategorisierung durchzuführen. Dabei entsteht ein Netz, welches selbst entscheiden würde, in welche Kategorie ein Wort fällt. Das Wort "'Fußball"' müsste dadurch in die Kategorie Hobby eingeordnet werden.

\section{Email-adressen}
Adressen von Micheal Bazzel mit verwenden wie ml@web.de

\section{Absender-Adresse}
Spoofing, Kontakte von Fupa oder Instagram nutzen.
%TODO in ausblick

\section{Validität der generierten Mail-Adressen prüfen}

\subsection{Methoden zum Prüfen der Validität}
Die erzeugten Adressen werden anschließend auf Validität geprüft. Hierfür gab es früher eine \textit{VRFY} Anfrage von SMTP. Mit dieser Anfrage konnte eine angegebene E-Mail-Adresse überprüft werden. Allerdings wurde der Dienst von Spammern ausgenutzt und wird dadurch von den meisten SMTP-Servern nicht mehr zu Verfügung gestellt.\cite{balduzzi2010abusing}\\
Demnach muss die Validität auf einem anderen Weg geprüft werden. Eine Möglichkeit zur Prüfung ist die Verwendung bereitgestellter Webseiten, bei der die zu prüfenden E-Mail-Adresse angegeben werden kann. Eine anschließende Rückmeldung verrät dann, ob die Adresse verwendet wird oder nicht. Eine Webseite dafür wäre "'\textit{https://centralops.net/co/}"'. Als Alternative dazu, ist die Entwicklung eines Skriptes, welches die Validität der Adresse prüft.

Im Fall, dass mehrere Adressen von diesem Adresspool gültig sind, kann nach mit Hilfe dieser Mail-Adressen nach Einträgen im Internet gesucht werden. Wenn es eine Übereinstimmung mit der Zielperson gibt, wird diese E-Mail ausgewählt. Andernfalls wird an jede gültige Adresse eine Phishing-Mail gesendet. 

\subsection{Bewertung: Validität Prüfen}
Für eine bessere Laufzeit des Programms, wird ein Skript zur Überprüfung der Adressen auf Verfügbarkeit und Gültigkeit, verwendet.	

%%% Local Variables: 
%%% mode: latex
%%% TeX-master: "Bachelorarbeit"
%%% End: 
