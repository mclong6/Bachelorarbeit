%% Symbole:
%%\newglossaryentry{symb:Name}{name=Symbolname, description={Beschreibung}, sort=alphabetisches Wort für die Einreihung, type=symbolslist}
%\newglossaryentry{symb:Pi}{
%name=$\pi$,
%description={Die Kreiszahl.},
%sort=symbolpi, type=symbolslist
%}
%%Abkürzungen:
%%\newacronym{Referenz}{Abkürzung}{Beschreibung}
\newacronym{BSP}{BSP}{Beispiel}


%%Eine Abkürzung mit Glossareintrag:
%%\newacronym{Referenz}{Abkürzung}{Beschreibung\protect\glsadd{glos:Referenz}}
%\newacronym{AD}{AD}{Active Directory\protect\glsadd{glos:AD}}

%%Glossareintrag:
%%\newglossaryentry{glos:Referenz}{name=Name, description={Beschreibung}}
%\newglossaryentry{glos:AD}{
%name=Active Directory,
%description={Active Directory ist in einem Windows Server 2000, Windows
%Server 2003, oder Windows Server 2008-Netzwerk der Verzeichnisdienst, 
%der die zentrale Organisation und Verwaltung aller Netzwerkressourcen erlaubt. Es
%ermöglicht den Benutzern über eine einzige zentrale Anmeldung den
%Zugriff auf alle Ressourcen und den Administratoren die zentral
%organisierte Verwaltung, transparent von der Netzwerktopologie und
%den eingesetzten Netzwerkprotokollen. Das dafür benötigte
%Betriebssystem ist entweder Windows Server 2000, 
%Windows Server 2003, oder Windows Server 2008, welches auf dem zentralen
%Domänencontroller installiert wird. Dieser hält alle Daten des
%Active Directory vor, wie z.B. Benutzernamen und
%Kennwörter.}
%}

%\newglossaryentry{glos:Glossareintrag}{name=Glossareintrag, description={Erweiterte Informationen zum
%einem Wort oder einer Abkürzung, ähnlich einem Eintrag im Duden.}}




%%% Local Variables: 
%%% mode: latex
%%% TeX-master: "Bachelorarbeit"
%%% End: 