%Kapitel des Hauptteils

\chapter{Lösungsvorschläge}  %Name des Kapitels
\label{cha:} %Label des Kapitels
\section{Informationsbeschaffung} %Unterkapitel

\begin{itemize}
	\item Es soll durch Web Scraping die Internetseite www.fupa.net durchsucht werden. Persönliche Spielerinformationen werden ausgelesen und gespeichert.
	\begin{itemize}
		\item Vorname, Nachname, Spitzname, Geburtsjahr, Verein (must)
		\item Bild (could)
	\end{itemize}
	\item Es soll ein Web Crawler erstellt werden der mit der vorhandenen Information nach weiteren Informationen im Internet sucht.
\end{itemize}
\section{Informationsbeschaffung}
Das Thema Informationsbeschaffung von personenbezogenen Daten lässt sich in zwei Teile gliedern. Erstens in die Informationsbeschaffung von bestimmten bzw. ausgewählten Personen und zweitens die Informationsbeschaffung von vielen unbestimmten Personen.
\subsection{Informationsbeschaffung von bestimmten/ausgewählten Personen}
\subsection{Informationsbeschaffung von unbestimmten Personen}
\section{Datenverwaltung/-speicherung}
\section{Phishing-Mail Erzeugung}