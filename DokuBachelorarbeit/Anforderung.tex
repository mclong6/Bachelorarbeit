%Kapitel des Hauptteils

\chapter{Anforderungsanalyse und Priorisierung}  %Name des Kapitels
\label{cha:} %Label des Kapitels
\section{Anforderungsanalyse} %Unterkapitel
\label{sec:} %Label des Unterkapitels
Wie muss Lösung aussehen? must - should - could

Anforderungen:\\

Informationsbeschaffung:
\begin{itemize}
	\item Mit Hilfe eines Web-Crawlers soll das Internet nach personenbezogenen Daten suchen
	\item Es soll durch web scraping personenbezogen Daten gewonnen werden
\end{itemize}

Datenverwaltung/-speicherung:
\begin{itemize}
\item Die Spieler- bzw. Opferinformationen sollen in einer gut übersichtlichen Struktur gespeichert werden. Informationen müssen erweiterbar sein.
\end{itemize}

E-Mail erzeugung:
\begin{itemize}
\item Es sollen E-Mail-Muster erstellt werden. Diese Muster sollen kategorisiert werden, damit für alle Opferinformationen ein passendes Muster vorhanden ist.
\item Die Phishing-E-Mails sollen automatisiert erstellt werden und die personalisierte Opferinformation verwenden.
\end{itemize}

Unter anderem soll die Arbeit Antworten auf folgende Fragen finden:\\
Wie kann die Webseite www.fupa.net am effizientesten ausgelesen werden?\\
Welche zusätzlichen Webseiten liefern die meisten Informationen zu potentiellen Opfern?\\
Wie soll nach Informationen gesucht werden?\\
Gibt es bereits einen Algorithmus der mit Hilfe von Vorname, Nachname und Geburtsjahr eine E-Mail-Adresse generiert?\\
Wie können die Phishing-E-Mails möglichst auf einzelne Personen zutreffend erstellt werden? Ist es sinnvoll E-Mail-Muster zu erstellen?\\

\subsection{} %Unterunterkapitel
\label{sse:}
\subsubsection{} %Unterkapitel 3. Ordnung
\label{sss:}
%%% Local Variables: 
%%% mode: latex
%%% TeX-master: "Bachelorarbeit"
%%% End: 
\section{Priorisierung} %Unterkapitel
\label{sec:} %Label des Unterkapitels
Priorisierungstabelle.
\FloatBarrier
\begin{table}
	\begin{center}
		\caption{Priorisierung der Anforderungen}
		\label{matrizen}
		\begin{tabular}{|l|l|l|}
			\hline
			Anforderung & Priorisierung (A-C) & must-should-could\\
			\hline
			Anforderung 1 & $ A $ & $must$   \\
			\hline
			Anforderung 2 & $ B $ & $should$  \\
			\hline
		\end{tabular}
	\end{center}
\end{table}
\FloatBarrier
Man beachte: Bilder haben Bild{\bf unter}schriften, 
Tabellen haben Tabellen{\bf "uber}schriften.