%Kapitel des Hauptteils

\chapter{Anforderungsanalyse und Priorisierung}  %Name des Kapitels
\label{cha:Anforderungsanalyse und Prioriesierung} %Label des Kapitels
\section{Anforderungsanalyse} %Unterkapitel
\label{sec:Anforderunsanalyse} %Label des Unterkapitels
Die im Kapitel \ref{sec:Zielsetzung} definierten Ziele sollen mit den folgenden Anforderungen gewährleistet werden.
	\subsection{Anforderung an das Programm bzw. an die Programmiersprache}
	Es sollte eine möglichst übersichtliche und performante Skriptsprache verwendet werden, mit der eine automatisierte Informationsbeschaffung gut möglich ist. Eine Eingabe über die Konsole oder über eine graphische Benutzeroberfläche soll ebenfalls möglich sein. Aus diesem muss die Programmiersprache keine GUI-Programmierung mit sich bringen.
	%TODO Weitere Vorteile von Python herausfinden 
	\subsection{Anforderung an die Informationsbeschaffung}
	Die Anforderung an die Informationsbeschaffung von personenbezogenen Daten lässt sich in zwei Teile gliedern. Der erste Teil beinhaltet die Informationsbeschaffung von ausgewählten Personen und der zweite Teil die Informationsbeschaffung von einer großen Menge unbekannten Personen.
	
		\subsubsection{Informationsbeschaffung von einer ausgewählten Person}
		Bei dieser Informationsbeschaffung soll eine Suchfunktion entwickelt werden, welche Daten zu einer angegeben Person im Internet sucht. Hierbei sollen so viele Daten wie möglich gefunden und gespeichert werden. Dies soll mit Hilfe eines \textit{Web-Crawlers} und mit einem \textit{Web-Scraper} umgesetzt werden.\\
		Das zu entwickelnde Programm soll für die Suche bekannte Daten wie Vorname, Nachname, Geburtsjahr, Ort und Benutzernamen von Social Media Plattformen über eine Konsolen einlesen können.\\
		Die Herausforderung besteht darin, zu erkennen, wann und ob es sich um die Information der gesuchten Person handelt.
	
		\subsubsection{Informationsbeschaffung von unbestimmten Personen}
		Es soll eine Prototyp-Suchfunktion entwickelt werden, die eine komplette Website nach personenbezogenen Daten durchsucht. Dabei sollen möglichst viele Informationen von möglichst vielen Personen herausgefunden werden. Jedoch sind diese Personen dem Programm-Anwender unbekannt. Die Informationen werden aus Webseiten mit einer großen Anzahl von Mitgliedern herausgelesen. Bei dieser Suchfunktion soll den Anwender aus den vorgegebenen Webseiten eine Seite auswählen können. Die ausgewählte Webseite wird daraufhin komplett ausgelesen und nach personenbezogenen Daten durchsucht.\\
		Dabei soll der zu entwickelnde \textit{Web Scraper} möglichst performant arbeiten und kann \textit{hartkodiert} werden. Allerdings müssen E-Mail-Adressen ebenfalls gefunden werden können, obwohl die Position einer E-Mail-Adressen auf einer Webseite variieren kann.
		
	\subsection{Anforderung an die Datenverwaltung/-speicherung}
	Ausgelesene Daten sollen vor dem speichern formatiert und klassifiziert werden, damit die Daten später korrekt in die Phishing-Mails eingesetzt werden können. Die Schwierigkeit besteht darin, zu erkennen, um welche Art von Information es sich handelt. Zusätzlich sollen die Daten in einer gut übersichtlichen Struktur gespeichert werden und müssen beliebig erweiterbar sein.
	
	\subsection{Anforderung an die Generierung der E-Mail-Adressen}
	Da nicht zu jeder Suche eine E-Mail-Adresse im Internet gefunden werden kann, muss die E-Mail-Adresse aus den vorhandenen Informationen generiert werden. Es soll eine größere Anzahl von möglichen E-Mail-Adressen erzeugt werden. Durch den Pool an erzeugten E-Mail-Adressen soll die Wahrscheinlichkeit erhöht werden, dass die richtige E-Mail-Adresse dabei ist. Des Weiteren können die E-Mail-Adresse auf Verfügbarkeit und Gültigkeit geprüft werden.
	
	\subsection{Anforderung an die E-Mail-Muster}
	Die E-Mail-Muster sollen erstellt werden und so klassifiziert sein, dass für jedes gefundene Opferprofil ein passendes Muster vorhanden ist. Des Weiteren soll der E-Mail-Text mit den eingesetzten Informationen Sinn ergeben und eine korrekte Grammatik beinhalten. Weiterführend können SE-Fähigkeiten genutzt werden um die Zielperson tatsächlich zu Manipulieren. Hierfür können beispielsweise Gefühle wie Freude und Angst ausgenützt oder gefälschte E-Mails von bekannten Firmen in Betracht gezogen werden.
	
	\subsection{Anforderung an die Erstellung der Phishing-Mail}
	Die Phishing-Mails sollen automatisiert erstellt werden. Die Auswahl des richtigen E-Mail-Musters zu der gewonnenen Opferinformation soll ebenfalls automatisiert ablaufen.
	
	\subsection{Unter anderem soll die Arbeit Antworten auf folgende Fragen finden}
	\begin{itemize}
		\item Mit welchem Aufwand ist eine Phishing-Mail-Angriff verbunden?
		\item Ist es möglich ein Personenprofil zu erstellen, bei dem ausschließlich korrekte Informationen vorhanden sind?
		\item 
	\end{itemize}
\FloatBarrier

\section{Priorisierung} %Unterkapitel
\label{sec:} %Label des Unterkapitels
Die Tabelle \ref{tab:prio} zeigt die Priorisierung der Anforderungen. Dabei liegt der eindeutige Fokus auf der Informationsbeschaffung von personenbezogenen Daten und der Erstellung von E-Mail-Mustern.

\begin{table}
	\caption{Priorisierung der Anforderungen}
	\label{tab:prio}
	\begin{center} 
		\begin{tabular}{|l|l|}
			\hline
			Anforderung & Priorisierung (A-C) \\
			\hline
			Informationsbeschaffung von ausgewählten Personen & $ A $ \\
			\hline
			Informationsbeschaffung von vielen ubekannten Personen & $ A $ \\
			\hline
			E-Mail-Muster erstellen & $ A $    \\
			\hline
			Phsishing-Mail erzeugen & $ B $   \\
			\hline
			Datenverwaltung/-speicherung & $ B $   \\
			\hline
		\end{tabular}
	\end{center}
\end{table}