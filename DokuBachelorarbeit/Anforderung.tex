%Kapitel des Hauptteils

\chapter{Anforderungsanalyse und Priorisierung}  %Name des Kapitels
\label{cha:Anforderungsanalyse und Prioriesierung} %Label des Kapitels
\section{Anforderungsanalyse} %Unterkapitel
\label{sec:Anforderunsanalyse} %Label des Unterkapitels
Die im Kapitel \ref{sec:Zielsetzung} definierten Ziele sollen mit den folgenden Anforderungen gewährleistet werden.

	\subsection{Anforderung an die Informationsbeschaffung}
	Die Anforderungen an die Informationsbeschaffung von personenbezogenen Daten lässt sich in zwei Teile gliedern. Erstens in die Informationsbeschaffung von bestimmten bzw. ausgewählten Personen und zweitens die Informationsbeschaffung von vielen unbestimmten Personen.
		\subsubsection{Informationsbeschaffung von bestimmten/ausgewählten Personen}
		Bei dieser Informationsbeschaffung soll ein Tool entwickelt werden, welches Informationen zu einer angegeben Person sucht. Dies soll mit Hilfe eines Web-Crawlers und mit einem Web-Scraper umgesetzt werden. Das zu entwickelnde Tool soll bekannte Information/Daten (Name, Geburtsjahr, Ort, Usernames von Social Media Webseiten) über eine Konsolen-Abfrage einlesen können.
	
		\subsubsection{Informationsbeschaffung von unbestimmten Personen}
		Es soll ein Prototyp-Tool entwickelt werden, der möglichst viele Informationen von möglichst vielen Personen herausfindet. Jedoch sind diese Personen dem Tool-Anwender unbekannt. Die Informationen werden aus Webseiten mit einer großen Anzahl von Mitgliedern herausgelesen. Bei dem Prototyp soll es möglich sein, Webseiten auszuwählen, die ausgelesen werden sollen.
	\subsection{Anforderung an die Datenverwaltung/-speicherung}
	Die Spieler- bzw. Opferinformationen sollen in einer gut übersichtlichen Struktur gespeichert werden. Informationen müssen erweiterbar sein.
	
	\subsection{Anforderung an die Phishing-Mail Erzeugung}
	Es sollen E-Mail-Muster erstellt werden. Diese Muster sollen kategorisiert werden, damit für alle Opferinformationen ein passendes Muster vorhanden ist.
	Die Phishing-E-Mails sollen automatisiert erstellt werden und die gewonnene Opferinformation verwenden.
	
	\subsection{Ergänzende Anforderungen}
	Unter anderem soll die Arbeit Antworten auf folgende Fragen finden:\\
	Wie können Webseiten am effizientesten ausgelesen werden?\\
	Welche zusätzlichen Webseiten liefern die meisten Informationen zu potentiellen Opfern?\\
	Wie soll nach Informationen gesucht werden?\\
	Gibt es bereits einen Algorithmus der mit Hilfe von Vorname, Nachname und Geburtsjahr eine E-Mail-Adresse generieren?\\
	Wie können die Phishing-E-Mails möglichst auf einzelne Personen zutreffend erstellt werden? Ist es sinnvoll E-Mail-Muster zu erstellen?\\

\FloatBarrier
\section{Priorisierung} %Unterkapitel
\label{sec:} %Label des Unterkapitels
Die Tabelle \ref{tab:prio} zeigt die Priorisierung der Anforderungen.

\begin{table}
	
	\caption{Priorisierung der Anforderungen}
	\label{tab:prio}
	\begin{center} 
		\begin{tabular}{|l|l|}
			\hline
			Anforderung & Priorisierung (A-C) \\
			\hline
			Informationsbeschaffung von ausgewählten Personen & $ A $ \\
			\hline
			Informationsbeschaffung von vielen ubekannten Personen & $ A $ \\
			\hline
			E-Mail-Muster erstellen & $ A $    \\
			\hline
			Phsishing-Mail erzeugen & $ B $   \\
			\hline
		\end{tabular}
	\end{center}
\end{table}
\FloatBarrier
Man beachte: Bilder haben Bild{\bf unter}schriften, 
Tabellen haben Tabellen{\bf "uber}schriften.