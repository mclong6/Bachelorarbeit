%Kapitel des Hauptteils

\chapter{Anforderungsanalyse und Priorisierung}  %Name des Kapitels
\label{cha:} %Label des Kapitels
\section{Anforderungsanalyse} %Unterkapitel
\label{sec:} %Label des Unterkapitels

Unter anderem soll die Arbeit Antworten auf folgende Fragen finden:\\
Wie kann die Webseite www.fupa.net am effizientesten ausgelesen werden?\\
Welche zusätzlichen Webseiten liefern die meisten Informationen zu potentiellen Opfern?\\
Wie und wo lässt sich ein Opferprofil erstellen bzw. speichern? (z.B. mySQL-Datenbank)\\
Wie soll nach Informationen gesucht werden?\\
Gibt es bereits einen Algorithmus der mit Hilfe von Vorname, Nachname und Geburtsjahr eine E-Mail-Adresse generiert?\\
Wie können die Phishing-E-Mails möglichst auf einzelne Personen zutreffend erstellt werden? Ist es sinnvoll E-Mail-Muster zu erstellen?\\

\subsection{} %Unterunterkapitel
\label{sse:}
\subsubsection{} %Unterkapitel 3. Ordnung
\label{sss:}
%%% Local Variables: 
%%% mode: latex
%%% TeX-master: "Bachelorarbeit"
%%% End: 
\section{Priorisierung} %Unterkapitel
\label{sec:} %Label des Unterkapitels
Und es gibt auch ein Beispiel für eine Tabelle\index{Tabelle}.

\begin{table}
	\begin{center}
		\caption{Verwendete Matrizen}
		\label{matrizen}
		\begin{tabular}{|l|l|l|}
			\hline
			Matrix & Dimension & Symbol \\
			\hline
			Systemmatrix & $n \times n$ & ${\bf A}$  \\
			\hline
			Ausgangsmatrix & $m \times n$ & ${\bf C}$  \\
			\hline
		\end{tabular}
	\end{center}
\end{table}
\FloatBarrier
Man beachte: Bilder haben Bild{\bf unter}schriften, 
Tabellen haben Tabellen{\bf "uber}schriften.