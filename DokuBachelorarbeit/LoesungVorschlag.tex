%Kapitel des Hauptteils

\chapter{Lösungsvorschläge}  %Name des Kapitels
\label{cha:Lösungsideen} %Label des Kapitels
%TODO Struktur wie??
In diesem Kapitel werden die Lösungsideen für die Umsetzung der im Kapitel \ref{sec:Zielsetzung} definierten Ziele beschreiben.


	
\section{Konzept zur Informationsbeschaffung von einer großen Menge unbekannter Personen}
Für die \textit{real-world} Simulation eines Phishing-Mail-Angriffs eine Webseiten mit einer großen Menge von personenbezogenen Daten benötigt. 	Hierfür wird manuell nach einer Webseite gesucht, die eine große Menge an personenbezogenen Daten enthält. Diese wird anschließen als Informationsquelle festgelegt. Möglichkeiten, ausgenommen von den bekannten Social Media Seiten, sind die Webseiten FuPa, Xing und LinkedIn.\\
	\subsection{Methode zur Suche nach Information}
	In diesem Konzept gibt es keine automatisierte Suche nach Informationen, jedoch eine automatisierte Suche nach internen Links. Diese interne Suche kann mit einem Web Crawler realisiert werden. In Vorbereitung darauf wird der Aufbau der Seite analysiert.\\
	
	\subsection{Methode zum Auslesen der Information}
	Zum Auslesen einer großen Menge an Daten wird ein Web Scraper erstellt. Dieser könnte für die ausgewählte Webseite hartkodiert werden. Eine Alternative dazu, wäre die Analyse des Webseitentextes, was dem Ansatz \ref{subsec:ErkennenVonInformation} von der Suchfunktion einer ausgewählten Person entsprechen würde.

%TODO BILDER von Webseite einfügen für groben Überblick oder in Umsetztung
\section{Konzept zur Erstellung einer Phishing-Mail}
Die Generierung einer Phishing-Mail läuft voll automatisch ab. Das bedeutet, dass das Programm eigenständig die E-Mail-Adressen generiert und selbst passende E-Mail-Muster auswählt.
	\subsection{Methoden zur Generierung von E-Mail-Adressen}
	Eine Möglichkeit zur Generierung der E-Mail-Adressen kann das Open Source-Tool von Michael Bazzell \cite{EmailAssumptions} sein, welches mit Hilfe eines automatisierten Webbrowsers verwendet werden kann. Bei diesem Tool werden zuerst über ein Formular, Daten für die E-Mail-Generierung eingetragen. Unter anderem sind das Vorname, Nachname und der E-Mail-Provider. Daraufhin werden die vorgeschlagenen E-Mail-Adressen angezeigt,kopiert und in ein Suchfeld eingefügt. Anschließend kann bei Google, Bing, und Facebook nach Einträgen gesucht und falls ein Eintrag gefunden wurde auch angezeigt werden.

	Eine Weitere Möglichkeit wäre ein Algorithmus zu entwickeln, der alle möglichen E-Mail-Adressen aus den Kombinationen von Vorname, Nachname, Geburtsjahr, Benutzernamen und den Domains von den bekanntesten E-Mail-Providern generiert. Dazu gehören \textit{GMX}, \textit{WEB.DE}, \textit{Gmail}, \textit{T-Online}, \textit{Freenet} und \textit{1\&1}.\cite{AnbieterMail} \\
	Für den Fall, dass der Arbeitgeber der Zielperson bekannt ist, kann auf der Firmenwebseite nach E-Mail-Adressen gesucht werden. Dadurch ist es möglich die Domain einer Firmen-Mailadresse zu bestimmen und eine Anzahl  möglicher Firmenadressen für die Zielperson zu generieren.\\
	Schon bei der Suche von personenbezogenen Daten wird ebenfalls nach E-Mail-Adressen gesucht. Dadurch kann bereits eine bis jetzt unbekannte Anzahl von Adressen gefunden werden.

%TODO Erwähnen dass facbook nach emails suchen konnte

	\subsection{Methode zur Erstellung von E-Mail-Mustern}
	Für die Erstellung der E-Mail-Muster kann eine eigene Klasse erstellt werden, welche für die Erzeugung des Textes zuständig ist. In dieser Klasse werden Strings gespeichert die einem Lückentext ähneln. Abhängig von den gefundenen Daten wird ein Lückentext ausgewählt, welcher anschließend mit den Daten an den passenden Lücken ergänzt wird. Mit dieser Methode muss jedoch für jede Kombination aus gewonnenen Daten ein Lückentext vorhanden sein.\\
	Die Lückentexte werden so kategorisiert, dass für jede gefundene Information ein passender Lückentext vorhanden ist. Eine denkbare Unterteilung wäre in die Kategorien Privat und Geschäftlich.

