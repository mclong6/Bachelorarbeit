%Kapitel des Hauptteils

\chapter{Lösungsideen}  %Name des Kapitels
\label{cha:Lösungsideen} %Label des Kapitels
Für die Umsetzung der im Kapitel \ref{sec:Zielsetzung} definierten Ziele, werden folgende Lösungsideen vorgeschlagen.

\section{Programmiersprache/ GUI}
Für die Auswahl der Programmiersprache gibt es viele Auswahlmöglichkeiten. Dennoch wird in dieser Abschlussarbeit die Sprach Python verwendet, da sie die Nötigen Eigenschaften mit sich bringt.\\
%TODO Warum Python? Antworten finden Welche Eingeschaften
Für die Eingabe von Suchdaten, besteht für beide Informationsbeschaffungen die Möglichkeit eine Grafische-Bedienoberfläche oder eine Konsolen-Eingabe zu verwenden.
\section{Informationsbeschaffung einer ausgewählten Person}	
	\subsection{Wie sieht die Suche nach einer Person im Internet aus?}
	\label{sec:Suche nach Information}
	Die Suche nach einer Person im Internet kann durch mehrere Ansätze erfolgen. Die nachstehenden Ansätze unterscheiden sich in der Art Suche und in dem Umgang der eingeben Daten.
		\subsubsection{Die Art der Personensuche wird anhand den eingegebenen Daten angepasst}
		\label{subsubsec: DieArtderPersonensuchewirdanhanddeneingegebenenDaten angepasst}
		Abhängig von der Anzahl und Art der Daten, die von dem Programm-Anwender eingeben wurden, wird die Art und Reihenfolge der Suche variiert. Die nachfolgenden Fälle sollen diesen Ansatz verdeutlichen.\\
		
		Im Fall, dass der Vorname, Nachname und Wohnort der gesuchten Person eingegeben wird, kann mit der Hilfe von herkömmlichen Suchmaschinen wie Google, Bing, DuckDuckGo, usw. nach Information gesucht werden. Die von den Suchmaschinen vorgeschlagenen Seiten werden anschließend analysiert, interpretiert und gespeichert. Dadurch können weitere Informationen gewonnen werden. Falls Benutzernamen von anderen Webseiten wie Instagram, Facebook oder ähnliches vorgeschlagen werden, kann somit die Suche mit diesen Daten speziell auf den entsprechenden Seiten erweitert werden.\\
		
		Ein weiterer Fall beschreibt das Szenario, wenn ein Benutzername von der gesuchten Person in das Programm eingegeben wird. Hierbei handelt es sich um einen Benutzernamen von Webseiten wie Facebook, Instagram, usw.\\
		Zuallererst, kann die entsprechende Webseite nach Informationen zu dem angegebenen Benutzername durchsucht werden. Dadurch können zusätzliche Daten herausgefunden werden, die bei der weiteren Suche von Vorteil wären. \\
		Nachdem die Webseite nach dem Nutzernamen durchsucht und ausgewertet wurde, kann nun mit herkömmlichen Suchmaschinen die Suche erweitert werden.
		\subsubsection{Es wird unabhängig von den eingegebenen Daten direkt mit einer Suchmaschine nach der Person gesucht}
		Bei diesem Lösungsansatz werden ausschließlich die herkömmlichen Suchmaschinen verwendet. Die Funktion der Suche besteht darin, dass das Programm den vorgeschlagenen Links der Suchmaschinen folgt, wobei die eingegebenen Daten die Art der Suche nicht beeinflussen.
		\subsubsection{Nur ausgewählte Webseiten werden nach der Person durchsucht}
		Unabhängig von den eingegebenen Daten, werden verschiedene Webseiten durchsucht. Allerdings ohne die Verwendung einer Suchmaschine. Vorschläge für die ausgewählten Webseiten sind Facebook, FuPa, Instagram, Xing, LinkedIn und Twitter.
	
	\subsection{Wann handelt es sich um die gleiche Person?}
	Bei jeder einzelnen Suchvariante, besteht die Herausforderung darin, zu erkennen, wann es sich um die gesuchte Person handelt. Durch die große Anzahl an verfügbaren Informationen im Internet, besteht eine hohe Wahrscheinlichkeit, dass Personen mit exakt den gleichen Daten gefunden werden. Um dieses Problem zu umgehen, werden folgende Lösungsideen vorgeschlagen.
		\subsubsection{Die Art der Suche wird anhand den eingegebenen Daten angepasst}	Diese Lösung entspricht dem Ansatz \ref{subsubsec: DieArtderPersonensuchewirdanhanddeneingegebenenDaten angepasst}. Die Suche kann dadurch verfeinert werden und die Anzahl der fehlerhaften Vorschläge wird geringer. Dadurch wird die Wahrscheinlichkeit höher, dass es sich um die richtige Person handelt.
		\subsubsection{Bei keiner perfekten Übereinstimmung wird die Suche erweitert}	Hier kann das Suche erweitert werden, indem auf Verbindungen der Zielperson eingegangen wird. Das heißt bekannte Facebook-Freunden, FuPa-Teammitglieder, oder Xing-Arbeitskollegen gesucht können ebenfalls durchsucht werden.
		\subsubsection{Profilbilder können verglichen werden}
		Entweder mit Bilderkennungssoftware oder Googl-Bildersuche
		\subsubsection{Die Personen-Suche mit Hilfe von korrekten Suchbefehlen verfeinern}	In dem Buch "'Open Source Intelligence Techniques"' \cite{Bazzell}, werden Suchbefehle aufgezeigt, mit denen die Suche mit bekannter Suchmaschinen verbessert und verfeinert werden kann. Auch bei dieser Lösungsidee wird die Wahrscheinlichkeit erhöht, dass es sich um die gesuchte Person handelt.
		
	\subsection{Informationsgewinnung durch Textanlayse}
	Nicht möglich hartkodierung, Dies ist bei der ersten Suchfunktion allerdings nicht möglich. Es können verschiedenste Arten von Webseiten vorgeschlagen werden. Aus diesem Grund muss das Programm "'clever"' sein und die wichtigen Daten aus der Seite herausfiltern können. Dazu gibt es folgende Lösungsideen.
		\subsubsection{Textanalyse mit Hilfe von Python NTLK}
		Mit dem\textit{Natural Language Toolkit} ist es möglich, den vorhandenen Webseitentext zu analysieren. Zu beginn können sogenannte "'stopwords"' aus dem vorgegebenen Text herausgefiltert werden. Stopwords sind Wörter die sehr oft auftreten und keinen großen Informationsgewinn mit sich bringen. Beispiele dafür sind ist, ein, einer, usw. Dadurch verringert sich die Anzahl der gesamten Wörter im Text. Anschließend können Funktionen wie das Zählen des Vorkommens einzelner Wörter angewendet werden, um einen Überblick von dem Text zu bekommen. Des Weiteren kann der Text in Fragmente zerlegt werden um weitere Informationen über den Inhalt zu erlangen. Abschließend können die analysierten Wörter bzw. Fragmente nach Schlüsselbegriffen durchsucht werden.
		\subsubsection{Textanalyse indem nach Schlüsselwörtern gesucht wird}
		Es kann ein Algorithmus entwickelt werden, der nach Schlüsselwörtern in einer Webseite sucht. Hierfür wäre es denkbar, Datenbanken bzw Wortsammlungen zu erstellen, die die zu suchenden Schlüsselwörter beinhalten. Mit diesen Datenbanken kann nun die Webseite nach den Schlüsselwörter durchsucht,analysiert und interpretiert werden. Die Datenbanken werden mit Hilfe von bekannter Listen im Internet befüllt. Beispiele hierfür wären eine aktuelle Liste aller Hochschulen in Deutschland, Berufsbezeichnungen, Studiengänge, Hobbys, Städte, etc...
		\subsubsection{Textanalyse mit Hilfe Machine Learning}
		Möglicherweise irgendeine Python bib. -->	Scikit
		\subsubsection{Mit Hilfe von NLTK Rake den Text interpretieren}
		Rake hat die Aufgabe, einen Text mit vielen Wörten auf eine geringe Anzahl von Schlüsselwörter zu reduzieren und dadurch kann möglicherweise der Inhalt des Textes erkannt werden ohne ihn komplett gelesen zu haben.
	\subsection{Speicherung der gewonnenen Daten}
	Für die Suche einzelner Person, kann ein erweiterbares Personen-Objekt erstellt werden.
	
\section{Informationsbeschaffung von einer großen Menge unbestimmter Personen}
Webseiten mit großen Menge von Daten, ausgenommen von den bekannten Social Media Seiten, sind das Fußballportal FuPa, Xing und LinkedIn.
	\subsection{Informationsgewinnung durch Hartkodierung}
	Die Textanalyse wird nur bei der Suche einer bestimmten Person benötigt. Die  zweite Suchfunktion wird hartkodiert und benötigt dadurch keine Textanalyse, da der Aufbau der Webseite im voraus bekannt ist. Das bedeutet, dass das Programm genau weiß wo welche Information auf der Webseite steht. Beispielsweise befindet das Geburtsjahr bei der Seite von dem Fußballportal FuPa immer an der gleichen Position einer Tabelle.
	\subsection{Speicherung der gewonnenen Daten}
	Für die Speicherung von vielen unbekannten Personen-Daten, könnte eine SQL-Datenbank erstellt werden. Ein weiterer Idee wäre, eine Datei anzulegen, bei der alle Personen gut strukturiert gespeichert werden können. Möglichkeiten dafür wären die Dateiformate CSV und TXT.%TODO ist TXT Dateiformat wirklich eine gute Möglichkeit

\section{Generierung der E-Mail-Adressen}
Es kann das opensource tool von intelligencetechniques mit Hilfe eines automatisierten Webbrowsers verwendet werden. Algorithmus entwickeln, der alle möglichen Mail-Adressen aus den Daten Vorname, Nachname, Geburtsjahr und den bekanntesten Mail-Providern erzeugt.

\section{Erstellung der E-Mail-Muster}
Die Muster können in zwei große Kategorien unterteilt werden. Es gibt einen privaten und geschäftlichen Teil. Der private Teil hat weiter Unterteilungen wie Familie, Hobby/Interessen.
\section{Erzeugung der Phishing-Mail}
