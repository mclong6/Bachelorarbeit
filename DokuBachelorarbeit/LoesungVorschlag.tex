%Kapitel des Hauptteils

\chapter{Lösungsideen}  %Name des Kapitels
\label{cha:Lösungsideen} %Label des Kapitels
Für die Umsetzung der im Kapitel \ref{sec:Zielsetzung} definierten Ziele, werden folgende Lösungsideen vorgeschlagen.

\section{Programmiersprache}
Python
%TODO Warum Python? Antworten finden

\section{Informationsbeschaffung}
Für die Eingabe von Suchdaten, besteht für beide Informationsbeschaffungen die Möglichkeit eine Grafische-Bedienoberfläche oder eine Konsolen-Eingabe zu verwenden.
	\subsection{Informationsbeschaffung von bestimmten/ausgewählten Personen}
		
		\subsubsection{Suche nach Informationen}
		{\bf Ansatz 1} \textit{Die Art der Suche wird anhand den eingegebenen Daten angepasst.}\\\\
		Je nach dem wie viele Daten und was für Daten eingegeben wurden, wird die Art der und die Reihenfolge der Suche variiert. Die nachfolgenden Fälle sollen diesen Ansatz verdeutlichen:\\
		\textit{Fall 1: Vorname, Nachname, Ort wird eingeben:}\\
		In diesem Fall wird mit Hilfe der Suchmaschine von Google nach Information gesucht. Die von Google vorgeschlagenen Seiten werden analysiert, interpretiert und gespeichert. Dadurch können weitere Informationen gewonnen werden. Wenn Benutzernamen von anderen Webseiten wie Instagram, Facebook oder ähnliches vorgeschlagen wird, kann somit die Suche speziell auf der entsprechenden Seite erweitert werden.\\\\
		\textit{Fall 2: Benutzername einer Webseite(Facebook,Instagram,usw.) wird eingeben:}\\
		Hier kann zuallererst auf der entsprechenden Webseite nach Informationen zu dem angegebenen Benutzername gesucht werden. Möglicherweise werden dadurch zusätzliche Daten herausgefunden, die bei der weiteren Suche von Vorteil wären.\\
		Nachdem die Webseite nach dem Nutzernamen durchsucht und ausgewertet wurde, kann nun mit herkömmlichen Suchmaschinen die Suche erweitert werden.\\\\
		{\bf Ansatz 2} \textit{Es wird unabhängig von den eingegebenen Daten direkt mit einer Suchmaschine nach Informationen gesucht.}\\\\
		Man verwendet ausschließlich die herkömmlichen Suchmaschinen und geht anhand den vorgeschlagenen Links auf die Suche nach Informationen.\\\\
		{\bf Ansatz 3} \textit{Es wird nur auf ausgewählten Webseiten nach Informationen gesucht}\\\\
		Verschiedenste Webseiten durchsuchen. Ideen dafür sind Facebook, FuPa, Instagram, Xing, LinkedIn, Google und Twitter.\\
		
		\subsubsection{Wann handelt es sich um die gleiche Person?}
		Damit geschaut werden kann ob es sich um die gleiche Person auf unterschiedlichen Webseiten handelt, können folgenden Ideen angewendet werden:\\
		- je nach vorgegeben Daten kann erst auf den entsprechenden Webseiten gesucht werden (Unsername von Instagram --> dann erste Seite Instagram, Voller Name und Ort --> Xing, LinkedIn, Google, Geburtsjahr und Name --> FuPa)\\
		- bei keiner perfekten Übereinstimmung wird Suche erweitert. D.h. es wird zusätzlich in Verbindung mit Facebook-Freunden, FuPa-Teammitglieder, oder Xing-Arbeitskollegen gesucht)\\
		-Profilbilder können verglichen werden. Entweder mit Bilderkennungssoftware oder Googl-Bildersuche)\\
		-Google Suche verfeinern mit Hilfe von Open Source Intellicence Techniques
		-Andere Suchmaschinen verwenden
		
	\subsection{Informationsbeschaffung von einer großen Menge unbestimmter Personen}
	Webseiten mit großen Menge von Daten, ausgenommen von den bekannten Social Media Seiten, sind das Fußballportal FuPa, Xing und LinkedIn.
	
\section{Datenanalyse/-speicherung}
Für die Datenanalyse kann ein Text-Analyse-Tool verwendet werden, damit die Texte von einer Webseite, vor dem Speichern, korrekt interpretiert und nach Schlüsselwörter untersucht werden können. Des Weiteren kann ein Algorithmus entwickelt werden, der nach Schlüsselwörtern in einer Webseite sucht.%TODO Kümmern um Textanalyse
Bei der Datenspeicherung wird erneut nach der Art der Informationsbeschaffung unterschieden. Für die Suche einzelner Person, kann ein erweiterbares Personen-Objekt erstellt werden. Für die Informationsbeschaffung von vielen unbekannten Personen, könnte eine SQL-Datenbank erstellt werden. Ein weiterer Idee wäre, eine Datei anzulegen, bei der alle Personen gut strukturiert gespeichert werden können. Möglichkeiten dafür wären die Dateiformate CSV und TXT.%TODO ist TXT Dateiformat wirklich eine gute Möglichkeit

\section{Generierung der E-Mail-Adressen}
Es kann das opensource tool von intelligencetechniques mit Hilfe eines automatisierten Webbrowsers verwendet werden. Algorithmus entwickeln, der alle möglichen Mail-Adressen aus den Daten Vorname, Nachname, Geburtsjahr und den bekanntesten Mail-Providern erzeugt.

\section{Erstellung der E-Mail-Muster}
Die Muster können in zwei große Kategorien unterteilt werden. Es gibt einen privaten und geschäftlichen Teil. Der private Teil hat weiter Unterteilungen wie Familie, Hobby/Interessen.
\section{Erzeugung der Phishing-Mail}
