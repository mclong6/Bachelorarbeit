%Kapitel des Hauptteils

\chapter{Lösungsideen}  %Name des Kapitels
\label{cha:Lösungsideen} %Label des Kapitels
Für die Umsetzung der im Kapitel \ref{sec:Zielsetzung} definierten Ziele, werden folgende Lösungsideen vorgeschlagen.

\section{Programmiersprache/ GUI}
Für die Auswahl der Programmiersprache gibt es viele Auswahlmöglichkeiten. Dennoch wird in dieser Abschlussarbeit die Sprach Python verwendet, da sie die Nötigen Eigenschaften mit sich bringt.\\
%TODO Warum Python? Antworten finden Welche Eingeschaften
Für die Eingabe von Suchdaten, besteht für beide Informationsbeschaffungen die Möglichkeit eine Grafische-Bedienoberfläche oder eine Konsolen-Eingabe zu verwenden.
\section{Informationsbeschaffung einer ausgewählten Person}	
	\subsection{Wie sieht die Suche nach einer Person im Internet aus?}
	\label{sec:Suche nach Information}
	Die Suche nach einer Person im Internet kann durch mehrere Ansätze erfolgen. Die nachstehenden Ansätze unterscheiden sich in der Art Suche und in dem Umgang der eingeben Daten.
		\subsubsection{Die Art der Personensuche wird anhand den eingegebenen Daten angepasst}
		\label{subsubsec: DieArtderPersonensuchewirdanhanddeneingegebenenDaten angepasst}
		Abhängig von der Anzahl und Art der Daten, die von dem Programm-Anwender eingeben wurden, wird die Art und Reihenfolge der Suche variiert. Die nachfolgenden Fälle sollen diesen Ansatz verdeutlichen.\\
		
		Im Fall, dass der Vorname, Nachname und Wohnort der gesuchten Person eingegeben wird, kann mit der Hilfe von herkömmlichen Suchmaschinen wie Google, Bing und DuckDuckGo nach Information gesucht werden. Die von den Suchmaschinen vorgeschlagenen Seiten werden anschließend analysiert, interpretiert und gespeichert. Dadurch können weitere Informationen gewonnen werden. Falls Benutzernamen von anderen Webseiten wie Instagram, Facebook oder ähnliches vorgeschlagen werden, kann somit die Suche mit diesen Daten speziell auf den entsprechenden Seiten erweitert werden.\\
		
		Ein weiterer Fall beschreibt das Szenario, wenn ein Benutzername von der gesuchten Person in das Programm eingegeben wird. Hierbei handelt es sich um einen Benutzernamen von Webseiten wie Facebook, Instagram, usw.\\
		Zuallererst, kann die entsprechende Webseite nach Informationen zu dem angegebenen Benutzername durchsucht werden. Dadurch können zusätzliche Daten herausgefunden werden, die bei der weiteren Suche von Vorteil wären. \\
		Nachdem die Webseite nach dem Nutzernamen durchsucht und ausgewertet wurde, kann nun mit herkömmlichen Suchmaschinen die Suche erweitert werden.
		\subsubsection{Es wird unabhängig von den eingegebenen Daten direkt mit einer Suchmaschine nach der Person gesucht}
		Bei diesem Lösungsansatz werden ausschließlich die herkömmlichen Suchmaschinen verwendet. Die Funktion der Suche besteht darin, dass das Programm den vorgeschlagenen Links der Suchmaschinen folgt, wobei die eingegebenen Daten die Art der Suche nicht beeinflussen.
		\subsubsection{Nur ausgewählte Webseiten werden nach einer Person durchsucht}
		Unabhängig von den eingegebenen Daten, werden verschiedene Webseiten durchsucht. Allerdings ohne die Verwendung einer Suchmaschine. Vorschläge für die ausgewählten Webseiten sind Facebook, FuPa, Instagram, Xing, LinkedIn und Twitter.
	
	\subsection{Wann handelt es sich um die gleiche Person?}
	Bei jeder einzelnen Suchvariante, besteht die Herausforderung darin, zu erkennen, wann es sich um die gesuchte Person handelt. Durch die große Anzahl an verfügbaren Informationen im Internet, besteht eine hohe Wahrscheinlichkeit, dass Personen mit exakt den gleichen Daten gefunden werden. Um dieses Problem zu umgehen, werden folgende Lösungsideen vorgeschlagen.
		\subsubsection{Die Art der Suche wird anhand den eingegebenen Daten angepasst}	Diese Lösung entspricht dem Ansatz \ref{subsubsec: DieArtderPersonensuchewirdanhanddeneingegebenenDaten angepasst}. Die Suche kann dadurch verfeinert werden und die Anzahl der fehlerhaften Vorschläge wird geringer. Dadurch wird die Wahrscheinlichkeit höher, dass es sich um die richtige Person handelt.
		\subsubsection{Bei keiner perfekten Übereinstimmung wird die Suche erweitert}	
		Hier kann die Suche erweitert werden, indem auf soziale und berufliche Verbindungen der Zielperson eingegangen wird. Das heißt, dass bekannte Kontakte der gesuchten Person ebenfalls durchsucht werden. In diesem Fall könnten Facebook-Freunden, FuPa-Teammitglieder, Instagram-Follower oder LinkedIn/Xing-Kontakte als Kontaktquelle dienen.
		\subsubsection{Profilbilder können verglichen werden}
		Durch die Google Bildersuche, ist es möglich, anstatt einem Suchbegriff ein Bild zu verwenden und nach diesem zu suchen. Dabei kann ein zu suchendes Bild selbst hochgeladen oder ein URL angegeben werden. Bei dem Ergebnis kann es sich um ein ähnliches Bild oder eine Webseite, die das Bild enthält, handeln.\\
		Des Weiteren kann eine Bilderkennungssoftware verwendet werden um gleiche Personen zu identifizieren. %TODO Bilderkennungssoftware suchen
		\subsubsection{Die Personensuche mit Hilfe von korrekten Suchbefehlen verfeinern}	In dem Buch "'Open Source Intelligence Techniques"' \cite{Bazzell}, werden Suchbefehle für bekannte Suchmaschinen aufgezeigt, mit denen die Suche verbessert und verfeinert werden kann. Dies bedeutet, bei einer Personensuche ist es mit den richtigen Suchbefehlen möglich, die Anzahl der Vorschläge zu verringern. Ein Beispiel in dem Buch von Michael Bazzell zeigt, wie es funktioniert von 8770 Vorschlägen auf lediglich neun Vorschläge zu reduzieren. \cite{Bazzell}Dadurch wird auch bei dieser Lösungsidee die Wahrscheinlichkeit erhöht, dass es sich um die gesuchte Person handelt.
		
	\subsection{Wie erkennt das Programm wenn es sich um wichtige Informationen handelt?}
	
	Für die Suche einer ausgewählten Person können verschiedenste Arten von Webseiten vorgeschlagen werden. Aus diesem Grund muss das Programm eine gewisse Intelligenz beweisen um die wichtigsten Daten aus einer Seite herauszufiltern. Dabei ist es nicht möglich eine \textit{Hartkodierung} zu verwenden und bestimmte Bereiche einer Webseite auszulesen. Die Grundidee zur Lösung diese Problems ist die Analyse des vorliegenden Textes durch verschiedenste Methoden.
		\subsubsection{Automated Keyword Extraction}
		Eine Methode zur Textanalyse ist die automatisierte Schlüsselwort-Gewinnung. Hierbei wird die HTML-Seite zu einem verwendbaren Text umgewandelt, wobei die meisten Sonderzeichen herausgefiltert werden. Sonderzeichen wie "'."' und "'@"' werden dabei nicht herausgefiltert, da sie für die E-Mail-Erkennung wichtig sind.\\
		Die Anzahl der im Text befindenden Wörter werden anschließend mit Hilfe der sogenannten \textit{Stoppwörter} und \textit{Stammformreduktion} um ein großen sehr großen Teil reduziert.\\
		Im darauffolgenden Schritt wird eine Liste  der potentiellen Schlüsselwörter erstellt, welche nach der Häufigkeit des Vorkommens sortiert sind. Zusätzlich können weitere Listen erstellt werden, die N-Gramme des Textes enthalten.\\
		Für die Erkennung wichtiger Schlüsselwörter werden Datenbanken bzw. Wortsammlungen erstellt, welche die zu suchenden Schlüsselwörter beinhalten. Mit diesen Datenbanken kann nun die Liste mit den bereits verarbeiteten Wörter verglichen werden. Die Datenbanken können mit Hilfe von bekannter Listen im Internet befüllt werden. Beispiele hierfür sind eine aktuelle Liste aller Hochschulen in Deutschland, Berufsbezeichnungen, Studiengänge, Hobbys, Städte und Gemeinden.
	%TODO Überarbeiten
		
		\subsubsection{Textanalyse indem nach Schlüsselwörtern gesucht wird}
		Es kann ein Algorithmus entwickelt werden, der nach Schlüsselwörtern in einer Webseite sucht. 
		\subsubsection{Textanalyse mit Hilfe Machine Learning}
		In der Theorie ist es möglich, ein Neuronales Netz mit den Begriffen zu trainieren und eine Kategorisierung durchzuführen. Dabei entsteht ein Netz, welches selbst entscheidet in welche Kategorie ein Wort fällt. Das Wort "'Fußball"' müsste dadurch in die Kategorie Hobby eingeordnet werden.
		%TODO Python schauen nach neuronales Netz möglicherweise Scikit
		\subsubsection{Mit Hilfe von NLTK Rake den Text interpretieren}
		Rake hat die Aufgabe, einen Text mit vielen Wörtern auf eine geringe Anzahl von Schlüsselwörter zu reduzieren. Dadurch kann möglicherweise der Inhalt des Textes verstanden werden ohne ihn komplett gelesen zu haben.
	\subsection{Speicherung der gewonnenen Daten}
	Die gewonnenen Daten können in einem beliebig erweiterbaren Personen-Objekt erstellt werden. Erweiterungen von bekannten Kontakten sind ebenfalls möglich.
	
\section{Informationsbeschaffung von einer großen Menge unbestimmter Personen}
Webseiten mit großen Menge von Daten, ausgenommen von den bekannten Social Media Seiten, sind das Fußballportal FuPa, Xing und LinkedIn.
	\subsection{Informationsgewinnung durch Hartkodierung}
	Diese Suchfunktion wird \textit{hartkodiert} und benötigt dadurch keine Textanalyse, da der Aufbau der Webseite im voraus bekannt ist. Das bedeutet, dass das Programm genau weiß wo welche Information auf einer Webseite steht. Beispielsweise befindet sich das Geburtsjahr einer Person, auf der Seite von dem Fußballportal "'FuPa"', immer an der gleichen Position einer Tabelle. Dies bringt den Vorteil mit sich, dass der Text nicht analysiert werden muss und das Programm genau weiß, was mit diesen Daten gemacht werden muss.
	\subsection{Speicherung der gewonnenen Daten}
	Für die Speicherung von vielen unbekannten Personen-Daten kann eine SQL-Datenbank erstellt werden.\\
	Als Alternative kann eine Datei angelegt werden, bei der alle Daten zu allen Personen gut strukturiert gespeichert werden können. Eine Möglichkeit dafür ist das Dateiformat \textit{CSV} oder \textit{TXT}.%TODO ist TXT Dateiformat wirklich eine gute Möglichkeit

\section{Generierung der E-Mail-Adressen}
Es kann das opensource tool von intelligencetechniques mit Hilfe eines automatisierten Webbrowsers verwendet werden. Algorithmus entwickeln, der alle möglichen Mail-Adressen aus den Daten Vorname, Nachname, Geburtsjahr und den bekanntesten Mail-Providern erzeugt.

\section{Erstellung der E-Mail-Muster}
Die Muster können in zwei große Kategorien unterteilt werden. Es gibt einen privaten und geschäftlichen Teil. Der private Teil hat weiter Unterteilungen wie Familie, Hobby/Interessen.
\section{Erzeugung der Phishing-Mail}
