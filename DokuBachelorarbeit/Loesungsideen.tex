%Kapitel des Hauptteils

\chapter{Lösungsideen}  %Name des Kapitels
\label{cha:Lösungsideen} %Label des Kapitels
%TODO Struktur wie??
In diesem Kapitel werden die Lösungsideen für die Umsetzung der im Kapitel \ref{sec:Zielsetzung} definierten Ziele beschreiben.

\section{OSINT einer ausgewählten Person}
	\subsection{Verwendung von OSINT-Tools}
	Die Personensuche wird durch die Verwendung kostenloser OSINT-Tools durchgeführt.\\ 
	Eine entsprechende Webseite die mehrere OSINT-Methoden bereit stellt, ist unter dem URL \textit{"'https://inteltechniques.com/index.html"'} erreichbar. Sie stellt Methoden zur Suche nach E-Mail-Adressen, Benutzernamen, Social-Media-Profilen, und noch viele mehr zu Verfügung. Allerdings werden nicht nur selbstentwickelt OSINT-Methoden von Michael Bazzell bereitgestellt, sondern auch andere Webseiten mit weiteren OSINT-Tools vorgeschlagen.
	
	\subsection{Algorithmus für OSINT entwickeln}
	Es wird ein Algorithmus für OSINT entwickelt, der aus einem Web Crawler und Web Scraper besteht. Mit diesem ist es möglich eigenständig nach Information zu suchen. Hierfür wird eine Suchmaschine, wie die von Google, verwendet.\\
	Die Suchergebnisse können mit Hilfe des Web Crawlers verfolgt werden. Anschließen wird der Webseitentext, durch den Web Scraper, ausgelesen. Im letzten Schritt, wird der Text analysiert und interpretiert.\\
	All diese Prozesse laufen unabhängig von den vorgeschlagenen Webseiten voll automatisiert ab.


\section{Webseiten für OSINT mehrerer unbekannter Personen}
Für das OSINT mehrere unbekannter Personen stehen die Webseiten von FuPa, Xing und LinkedIn zu Auswahl.
	\subsection{XING}
	XING ist ein soziales Netzwerk für Berufstätige mit über 15 Millionen Mitgliedern. Hier vernetzen sich Kontakte aus allen Branchen um Jobs, Mitarbeiter, Aufträge oder ähnliches zu suchen und zu finden.\cite{WasIstXING}\\
	XING bietet allerdings viele Möglichkeiten zum Schutz der Privatsphäre. So kann ein Nutzer einstellen, ob er von einer Suchmaschine gefunden werden oder nur für Xing-Mitglieder sichtbar sein will. %TODO in die Bewertung
	\begin{figure}[H]
		\centering
		\includegraphics[ scale=0.2]{bilder/XING_profil.png}
		\caption{Profil von der Webseite XING}
		\label{img:XING}
	\end{figure}

	\subsection{LinkedIn}
	LinkedIn ist das weltweit größte soziale Netzwerk für Berufstätige mit hunderten von Millionen Mitgliedern. Es vernetzt berufliche Kontakte der ganzen Welt und stellt ebenfalls Möglichkeiten zum Schutz der Privatsphäre zu Verfügung. \cite{WasIstLinkedIn}
	\begin{figure}[H]
		\centering
		\includegraphics[ scale=0.2]{bilder/LinkedIn_profil.png}
		\caption{Prifl von der Webseite LinkedIn}
		\label{img:Linkedin}
	\end{figure}

		\subsection{Fupa}
		Die Webseite Fupa stellt ein regionales Fußballportal dar, welches zur Berichterstattung des Amateurfußballs vorhanden ist. Allerdings werden nicht nur Berichte sondern auch aussagekräftige Spielerprofile zur Verfügung gestellt.\cite{WasIstFUPA} Außer dem kann FuPa eine Mitgliederzahl von über 200.000 verzeichnet.\cite{FuPaMitglieder}\\
		Das Bild \ref{img:FuPa} zeigt ein Spielerprofil, wie es auf dieser Webseite angezeigt wird. Allerdings kann sich die Vollständigkeit eines Profils variieren.
		\begin{figure}[H]
			\centering
			\includegraphics[ scale=0.2]{bilder/fupa_screenshot.png}
			\caption{Spielerprofil von der Webseite FuPa}
			\label{img:FuPa}
		\end{figure}
		%TODO wie viele Mitglieder hat fupa?
	
	\section{Konzept für die Erstellung einer Phishing-Mail}
	Die Generierung einer realen Phishing-Mail benötigt eine korrekte E-Mail-Adresse der Zielperson und einen sinnvollen Inhalt, der die gewonnenen Informationen verwendet.
	\subsection{E-Mail-Adresse Generierung}
		\subsubsection{Algorithmus entwickeln zum generieren}
		Es kann ein Algorithmus entwickelt werden, der mögliche E-Mail-Adressen aus den gewonnen Daten generiert. Dies ist durch die Kombination aus Vorname, Nachname, Geburtsjahr und den bekanntesten E-Mail-Providern realisierbar. Dabei entsteht ein Adresspool, von dem jede einzelne E-Mail-Adresse auf Validität geprüft werden muss.\\
		\subsubsection{Automatisierbare OSINT-Tools verwenden}
		Für die Generierung der E-Mail-Adressen kann ein kostenloses OSINT-Tools von Michael Bazzel verwendet werden. Diese Tool ermöglicht es, die gewonnenen Informationen über eine Formular einzugeben und anschließend mögliche E-Mail-Adressen zu generieren. Auch hier entsteht ein Adresspool, bei dem die E-Mail-Adressen auf Validität geprüft werden müssen. Allerdings bringt das Tool eine weitere Funktion mit sich. Es wird automatisch nach Einträgen, der generierten E-Mail-Adressen, im Internet gesucht und angezeigt. \cite{EmailAssumptions}
		%TODO OSINT TOOL URL einfügen
	\subsection{E-Mail Inhalt}
		\subsubsection{E-Mail-Muster erstellen}
		Die zu erstellenden E-Mail-Muster entsprechen hier kategorisierten Lückentexten. Abhängig von den gefundenen Daten, wird ein Lückentext ausgewählt und anschließend mit den Daten an den passenden Stellen ergänzt.\\
		Die Lückentexte werden so kategorisiert, dass für jede gefundene Information ein passender Lückentext vorhanden ist. Eine denkbare Unterteilung wären die Kategorien Privat und Geschäftlich.
		\subsubsection{Text aus Fragmenten erzeugen}
		Bei dieser Methode besteht der E-Mail-Text aus zusammengesetzten Fragmenten. Dafür wird zu jeder gefundenen Information ein Fragment erstellt und anschließend werden alle Fragmente zu einem Text zusammengefügt.
		
		
		
