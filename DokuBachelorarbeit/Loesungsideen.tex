%Kapitel des Hauptteils

\chapter{Lösungsideen}  %Name des Kapitels
\label{cha:Lösungsideen} %Label des Kapitels
%TODO Struktur wie??
In diesem Kapitel werden die Lösungsideen für die Umsetzung der im Kapitel \ref{sec:Zielsetzung} definierten Ziele beschreiben.

\section{Konzept für OSINT einer ausgewählten Person}
	\subsection{Algorithmus für OSINT entwickeln}
	Es wird ein Algorithmus für OSINT entwickelt, der aus einem Web Crawler und Web Scraper besteht. Mit diesem ist es möglich eigenständig nach Information zu suchen. Hierfür wird eine Suchmaschine, wie die von Google, verwendet.\\
	Die Suchergebnisse können mit Hilfe des Web Crawlers verfolgt werden. Anschließen wird der Webseitentext, durch den Web Scraper, ausgelesen. Im letzten Schritt, wird der Text analysiert und interpretiert.\\
	All diese Prozesse laufen unabhängig von den vorgeschlagenen Webseiten voll automatisiert ab.
	\subsection{Verwendung von OSINT-Tools}
	Die Personensuche wird durch die Verwendung kostenloser OSINT-Tools durchgeführt.\\ 
	Eine entsprechende Webseite die mehrere OSINT-Methoden bereit stellt, ist unter dem URL \textit{"'https://inteltechniques.com/index.html"'} erreichbar. Sie stellt Methoden zur Suche nach E-Mail-Adressen, Benutzernamen, Social-Media-Profilen, und noch viele mehr zu Verfügung. Allerdings werden nicht nur selbstentwickelt OSINT-Methoden von Michael Bazzell bereitgestellt, sondern auch andere Webseiten mit weiteren OSINT-Tools vorgeschlagen.

\section{Konzept für OSINT mehrerer unbekannter Personen}
	\subsection{Mögliche Webseiten zum Auslesen}
		\subsubsection{Fupa}
		Die Webseite "'www.fupa.net"', stellt ein regionales Fußballportal dar, welches zur Berichterstattung des Amateurfußballs vorhanden ist. Allerdings werden nicht nur Berichte sondern auch aussagekräftige Spielerprofile zur Verfügung gestellt.\cite{WasIstFUPA}
		Das Bild ... zeigt ein Spielerprofil, wie es auf dieser Webseite angezeigt wird.
		%TODO Bild einfügen, wie viele Mitglieder hat fupa?
		\subsubsection{XING}
		XING ist ein soziales Netzwerk für Berufstätige mit über 15 Millionen Mitgliedern. Hier vernetzen sich Kontakte aus allen Branchen um Jobs, Mitarbeiter, Aufträge oder ähnliches zu suchen und zu finden.\cite{WasIstXING}
		%TODO Bild
		\subsubsection{LinkedIn}
		LinkedIn ist das weltweit größte soziale Netzwerk für Berufstätige mit hunderten von Millionen Mitgliedern. Es vernetzt ebenfalls berufliche Kontakte der ganzen Welt. \cite{WasIstLinkedIn}
		
\section{Konzept für die Erstellung einer Phishing-Mail}
	\subsection{E-Mail-Adresse Generierung}
		\subsubsection{Nach E-Mail Adressen suchen/OSINT-Tools verwenden}
		\subsubsection{Algorithmus entwickeln zum generieren}
		\subsubsection{Automatisierbare OSINT-Tools verwenden}
	\subsection{E-Mail Inhalt}
		\subsubsection{E-Mail-Muster erstellen}
		\subsubsection{Text aus wenigen Worten generieren}
		
